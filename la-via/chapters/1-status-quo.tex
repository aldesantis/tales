\chapter{Status quo}
\label{ch:status-quo}

\emph{Londra, anno 2031}
\plainbreak{1}

I passi di Helen risuonavano pesanti sul marciapiede. Si maledisse. Cosa diavolo le era saltato in
mente quella sera? Perché aveva messo gli stivali? Non erano che un pericolo; non potevano portare
che guai. Le strade non erano più sicure da molto tempo. Per una donna giovane e bella come Helen,
poi, erano un suicidio. Cercò di fare meno rumore possibile, ma era un'ardua impresa: le vie di
Londra erano deserte; solo la nebbia le teneva compagnia e, tutto sommato, le dava un sinistro senso
di sicurezza. Pregò di scomparire nel suo abbraccio.

Ora sapeva perché aveva messo gli stivali. Era una sfida. Una sfida lanciata agli altri, alle ombre
nella notte che minacciavano di aggredirla; ma soprattutto una sfida con se stessa. Mentre si stava
vestendo, poche ore prima, si era resa conto che il semplice abbigliarsi secondo i propri dettami, e
non quelli che si convenivano, era un atto di ribellione e di coraggio. E Dio sapeva quanto Helen
avesse bisogno di coraggio in quel momento.

Quando giunse al portone sentiva il cuore battere così forte che credeva le sarebbe esploso nel
petto. Bussò come le era stato insegnato. Due colpi. Poi uno. Ancora due. Dopo un'eternità, si udì
un chiavistello scorrere e la porta si aprì, lasciando entrare un minuscolo spiraglio della luce
artificiale di un lampione. Il viso dall'altra parte non si distingueva nella notte, ma Helen non
aveva bisogno di vederlo: le sarebbe bastato sentirne l'odore per riconoscerlo in una folla.

La porta si aprì completamente ed Helen scivolò dentro l'edificio fatiscente. Il buio era totale.
Faceva quasi più freddo che in strada, ma non le importava. Il legno del pavimento scricchiolava
sotto di loro mentre si dirigevano verso una camera da letto umida e gelata. Helen si strinse nel
cappotto, ma non appena varcarono la soglia della stanza, quelle mani delicate che aveva sognato per
tutto il giorno lo fecero scivolare a terra, e le labbra che aveva immaginato si posarono sulle sue.

Oh, quale gioia! Un brivido percorse la schiena della donna mentre restituiva i baci ardenti. Si
gettarono su un letto vecchio e malmesso facendo un rumore infernale, ma nessuno poteva sentire ciò
che accadeva lì, e anche se qualcuno avesse potuto, nulla sarebbe cambiato. Non c'era rischio che
togliesse piacere allo stare insieme. Semmai era proprio la pericolosità della loro relazione a
renderla ancora più eccitante.

Quando sentì le mani dell'amante farsi strada sotto il suo maglione, Helen sfiorò quei polsi
meravigliosi ed emise il più flebile dei gemiti.

``Aspetta'' sussurrò, ansimante. ``Non così. Voglio vederti. Voglio vederti sorridere.''

Fu allora che la fiamma insignificante di un accendino comparve a rischiarare il buio e si trasferì
a una candela, spargendo una debole luce per tutta la stanza. Nonostante la luce fosse poco più che
sufficiente a raggiungere l'angolo più lontano della camera da letto, gli occhi di Helen impiegarono
qualche secondo per vedere chiaramente.

Quel volto conosciuto si rivelò finalmente a Helen in tutta la sua bellezza. Dinanzi a lei, con le
labbra leggermente incurvate da un sorriso, stava la più bella donna su cui Helen avesse mai posato
lo sguardo.

``Ora va meglio'' sussurrò Helen prima di gettarsi su di lei.

\plainbreak{1}

Aveva conosciuto Catherine quattro mesi prima, alla mostra di un amico pittore che avevano in
comune. Helen aveva deciso di andarsi perché era raro che qualcuno ancora si dedicasse all'arte in
quel mondo devastato. Tutti sembravano aver perso completamente l'amore per il bello, preferendogli
invece l'orrore dell'odio e della violenza. Quella mostra era una rivoluzione. E vi avevano preso
parte non più di quindici persone.

Se chiudeva gli occhi, Helen ancora riusciva a vedere Catherine alla mostra, mentre ammirava un
dipinto astratto. Era così bella. Fisicamente, Catherine ed Helen si completavano in maniera quasi
perfetta. Gli occhi di una erano azzurri, quelli dell'altra verdi. I capelli di una biondi, quelli
dell'altra castani. I movimenti di Catheline erano scattanti e spigolosi, quelli di Helen più
morbidi e pacati.

E quel sorriso\dots Oddio, quel sorriso. Catherine sorrideva sempre come se sapesse precisamente
quando e come il mondo sarebbe finito. Come si può sorridere quando si è circondati da tanta
tristezza? Helen ancora non aveva trovato la forza per farlo; solo con la sua amante le era
possibile lasciarsi andare a un'euforia liberatoria. A tutte le altre situazioni della vita
riservava un'espressione di apatica sufficienza, quella che le avrebbe permesso di sopravvivere più
a lungo. E comunque, anche se avesse potuto trovare la forza per essere felice, non c'era alcun
pretesto per farlo.

Era stato uno di quei amori potenti e dolorosi che riempiono tutto e non lasciano spazio per altro.
Helen non era riuscita a controllarsi; aveva dovuto incontrarla di nuovo. E l'aveva baciata. Senza
pensare ad altro da lei. Non era nemmeno sicura che i gusti di Catheline fossero\dots poco
convenzionali come i suoi. Su quella donna aveva molti dubbi e poche certezze. Una di queste era che
l'amava. Non le serviva sapere altro.

Era passato così tanto tempo da quando Helen aveva baciato qualcuno che aveva dimenticato come ci si
sentisse. Nel nuovo mondo non c'era spazio per quelli come lei. Se si fosse dichiarata al mondo,
nell'arco di una notte sarebbe divenuta un reietto, un'emarginata, un peso per una società
ossessionata. I suoi genitori l'avrebbero probabilmente cacciata di casa, ma quello non era il
problema più grave; chissà cosa le avrebbero fatto gli altri\dots Non sarebbe sopravvissuta a lungo.
L'avrebbero presto trovata in un vicolo, violentata e uccisa. Nessuno si sarebbe interessato a lei.
Sarebbe stata solo un altro numero.

Ma soprattutto, avrebbe corso il rischio di perdere Catheline. E non poteva, non doveva perderla. Ne
aveva bisogno come dell'aria. Catheline la faceva sentire\dots completa. Accanto a lei, non era
solamente una figura grigia in quel clima di terrore. Era qualcuno. Valeva qualcosa. Non poteva
sopravvivere senza quella consapevolezza. Era l'unico fatto sensato a cui ormai si potesse
aggrappare, e lo faceva con quanta forza le rimaneva in corpo.

\plainbreak{1}

Helen avrebbe voluto rimanere in quel letto per sempre, e continuare a sfiorare con il dorso della
mano il corpo nudo di Catheline e continuare ad arricciarle i capelli castani. Era in pace. Era
felice. Non voleva che finisse. Non avrebbe lasciato che finisse. Sembrava che tutta la sua vita, lo
scopo stesso della sua esistenza fosse quello: stare lì, con lei, e\dots guardarla. Non le importava
del freddo, né del letto vecchio e sporco, né di ciò che accadeva al di fuori di quella casa in
rovina. Voleva semplicemente continuare a guardarla. Come si fa con un'opera d'arte, quando ci si
perde nella bellezza di un paesaggio o nella perfezione di una scultura.

Eppure, le pareva che Catheline fosse distante, preoccupata. Helen era irritata dalla sua assenza.
Com'era possibile che non condividesse le sue sensazioni, la sua gioia? Cosa mai poteva esserci al
di fuori di loro e di quel momento così significativo?

``A cosa stai pensando?'' chiese, cercando di far sì che la sua voce non tradisse alcuna emozione.

Catheline, da che era sdraiata, si mise improvvisamente a sedere. Sembrava sollevata, come se non
aspettasse altro che Helen le ponesse quella domanda. Come se dovesse parlarle e non riuscisse a
trovare il coraggio. La guardò intensamente negli occhi per qualche istante ancora, quindi sospirò
pesantemente.

``Ricordi dove ci siamo conosciute?'' chiese Catheline.

Helen sorrise.

``Certo che lo ricordo. Alla mostra del tuo amico. Come si chiamava\dots'' Era certa di ricordarselo,
ma in quel momento il nome le sfuggiva. Un ragazzo poco più giovane di loro, incredibilmente alto e
magro, con il sorriso sincero e la battuta pronta. Come le era capitato con Catheline, Helen era
rimasta colpita dalla sua capacità di ridere di tutto ciò che lo circondava.

``Tom'' suggerì Catheline. "Be', poco dopo quella mostra lo persi di vista. Qualche giorno fa l'ho
incontrato di nuovo."

Per un istante, Helen pensò al peggio. Era possibile che\dots no, decisamente no. Su certe cose non si
cambia idea. A lei non era mai passato per la mente, almeno. Le sue preferenze erano state piuttosto
chiare fin dall'adolescenza. Ma allora non erano state un gran problema. Ora, invece\dots

Catheline sembrava non essersi nemmeno accorta del lampo di terrore che aveva attraversato i suoi
occhi.

``C'è un gruppo di persone. Si chiama la Resistenza. Come nei film, riesci a crederci?
Vogliono\dots''

``No.''

Il volto di Helen era diventato una maschera di pietra. La sua voce non le era mai suonata così
dura, così estranea. Si rese conto che Catheline era rimasta quasi spaventata dalla sua decisione.
Ma cosa diavolo le saltava in mente?! Si rendeva conto di cosa stava parlando?! Di cosa le stava
chiedendo?!

Catheline sembrava improvvisamente dispiaciuta, quasi si pentisse di averglielo chiesto.

``Helen, lo so come sembra\dots''

``No, non lo sai. Non sei nella mia posizione e non hai idea di cosa si provi.''

Sapeva di essere ingiustamente crudele; Catheline aveva da perdere tanto quanto Helen, se non di
più. Eppure, quell'idea così coraggiosa, discussa in maniera così onesta, non poteva che suscitare
sdegno in lei. Per un attimo rabbrividì pensando a cos'era diventata, ma non poteva fare a meno di
avere paura. Tanta, tantissima paura.

``Quindi cosa suggerisci di fare? Dovremmo starcene qui, a coccolarci e guardare mentre il mondo
crolla a pezzi?!''

``Non ti basto io? Non ti bastiamo noi? Che altro ti serve?''

Ora avevano iniziato a urlare.

``Una differenza! Ho bisogno di fare la differenza! Devo sapere di aver fatto la mia parte, anche a
costo di perdere tutto!''

``A costo di perdere me?''

Catheline esitò. Sapeva come sarebbe andata a finire.

``Sì\dots'' disse, la voce ridotta quasi a un sussurro. ``Anche a costo di perdere te.''

Era abbastanza. Helen iniziò a raccogliere i vestiti sparsi per terra. Si aspettava che Catheline
cercasse di fermarla, che si scusasse, che le chiedesse qualcosa, che fornisse una
giustificazione\dots ma non fece nulla di tutto questo. Continuò a fissarla in silenzio, come se
avesse preso la sua decisione molto tempo prima.

Helen uscì dalla stanza trattenendo le lacrime, attraversò come una furia il corridoio buio e uscì
dall'edificio, lasciandosi avvolgere dalla notte.

\plainbreak{1}

Improvvisamente, l'oscurità non sembrava più così pericolosa, e il rumore non nascondeva più tanti
rischi. Helen ce l'aveva con tutti: con Catheline, col suo amico, con la Resistenza, con la sua
famiglia, con chi si nascondeva e con chi aveva scelto di combattere, perché la costringevano a
vivere in quella continua, dilaniante dualità, perché la costringevano ad affrontare i suoi
fantasmi, perché la costringevano a vedere. Quella era la parte più dolorosa: che fosse impossibile
non vedere, voltarsi dall'altra parte. Doveva essere così dolce l'oblio dell'ignoranza che a lei era
negato!

Era triste che, nell'arco di un decennio, l'umanità si potesse ridurre al livello delle bestie.
Eppure era accaduto. La trasformazione era stata completa e, Helen temeva, irreversibile. Non ci
sarebbe stato miglioramento, perché gli uomini avevano conosciuto la pietà e l'avevano consciamente
rifiutata, scegliendo invece la violenza. Come se la violenza potesse curare la Piaga. Come se
qualsiasi cosa potesse curare la Piaga.

Forse era proprio per via dell'inevitabilità della sua fine che la specie umana era regredita a tale
stato. Un uomo che non ha nulla da perdere si lascerà andare a qualunque efferatezza pur di
sopravvivere un giorno in più. Un uomo che non ha nulla da perdere non ha morale o empatia, non ama,
non prova dolore che non sia quello fisico e immediato. Un uomo che non ha nulla da perdere non è un
uomo, perché è stato spogliato di qualunque ambizione, di qualunque speranza. Una volta, Helen aveva
Catheline. Ora non più. E questo la terrorizzava immensamente. Ora che nulla la fermava, cosa
sarebbe successo? Cosa sarebbe diventata?

Non ebbe il tempo di rifletterci, perché dei passi precipitosi per la strada le ricordarono
improvvisamente quanto fosse vulnerabile in quel momento. Si bloccò; i suoi occhi guizzarono
maniacalmente alla ricerca di un nascondiglio. I passi si avvicinavano rapidamente: qualcuno stava
correndo. Le sembrava che fosse proprio dietro di lei. La strada era perfettamente dritta e
continuava per decine di metri, non ce l'avrebbe mai fatta. La sua unica speranza era un'auto
parcheggiata.  Con le gambe tremanti, Helen corse a nascondersi dietro la vecchia Honda dal colore
indecifrabile, e pregò che chiunque si stesse avvicinando non l'avesse vista né sentita.

Helen si affacciò cautamente e scorse una figura nera che si muoveva verso di lei. Quando la figura
si fece più grande, tornò al riparo e si appiattì il più possibile contro l'auto, sperando che non
la notassero nel buio della notte. Nessuno bazzicava a quell'ora pieno di buone intenzioni. L'ombra
scura la superò senza degnarla di un solo sguardo. Helen tirò un sospiro di sollievo, aspettò
qualche secondo e si alzò lentamente per incamminarsi nella direzione opposta. Aveva vissuto
abbastanza emozioni per quella notte.

Fu in quel momento che l'uomo si fermò. Ormai Helen non aveva tempo per nascondersi: restò impotente
a fissare un ragazzo non più vecchio di lei, dai lineamenti delicati e gentili. Sul suo volto,
imbruttito dalla luce gialla di un lampione, c'era un'espressione di terrore che fece rabbrividire
la donna. Fu allora che si rese conto di non avere nulla da temere: le bastò quell'unico sguardo per
capire che anche lui si stava nascondendo. Ma da cosa?

Il ragazzo fece qualche passò verso di lei. Istintivamente, Helen indietreggiò.

``Ti prego'' disse, e la sua voce era a metà tra un urlo d'agonia e un sussurro spavento, un tono
che Helen non aveva mai sentito e che non avrebbe mai più sentito. ``Puoi aiutarmi? Devi aiutarmi.''

Helen non proferì parola. Non avrebbe potuto.

``Stanno arrivando. Devo nascondermi. Dove posso nascondermi?'' continuò quello, ma parlava più a se
stesso che a lei.

Proprio come aveva fatto Helen qualche momento prima, guardò tutt'intorno a sé cercando un rifugio
sicuro, ma non pensò alla Honda o non la trovò di suo gradimento, perché lo prese una disperazione
folle. Non riprese a correre; semplicemente rimase lì, al centro della strada, continuando a vagare
di qua e di là, borbottando frasi sconnesse.

Helen si accorse di non aver mosso un muscolo negli ultimi due minuti. Cosa ci faceva ancora lì?
Doveva andarsene!

Proprio quando credeva di aver ritrovato il senno, un furgone nero la superò a velocità folle,
mancandola di pochi centimetri e fermandosi con uno stridore infernale qualche metro oltre lo
sconosciuto. Ne scesero velocemente quattro uomini; indossavano il passamontagna e una divisa
grigia. Sul petto era cucito un piccolo logo che Helen non riuscì distinguere. Era impossibile che
non l'avessero vista. Prima che questo potesse reagire, si lanciarono contro l'uomo. Il suono dei
manganelli che colpivano le ossa e i muscoli era sordo, sterile, quasi ovattato, ma doloroso alle
orecchie di Helen tanto quanto al corpo di quel ragazzo. Cercò senza riuscirci di distogliere lo
sguardo.

Tutto finì rapidamente com'era iniziato. Lo tirarono su e letteralmente lo lanciarono nel furgone,
come se fosse un sacco, quindi risalirono per andarsene. Prima che il mezzo ripartisse, uno degli
uomini guardò Helen e posò l'indice sulle labbra. I suoi occhi ridevano sotto il passamontagna.
