\chapter{}
\label{ch:2}

\emph{Londra, anno 2031}
\plainbreak{1}

``Il gruppo di ricerca istituito da Padre Frey non ha ancora trovato una cura alla Piaga. Gli
scienziati, guidati dal dottor Bishop, che negli ultimi quarant'anni ha risolto con successo
centinaia di casi di infertilità, non hanno potuto fare nulla per rimediare all'inspiegabile
azzeramento del tasso di natalità che da otto anni ormai affligge il mondo intero. Si mostrano
tuttavia fiduciosi riguardo le potenzialità di una tecnica di fecondazione assistita recentemente
ideata la cui sperimentazione dovrebbe iniziare nel 2033. Una conferenza stampa è stata organizzata
per\dots''

Lo schermo divenne improvvisamente nero; il viso del giornalista scomparve e nel salone di casa
Wright calò il silenzio. Helen si voltò e si accorse che suo padre era tornato dal lavoro. Era stato
lui a spegnere il televisore.

``Non dovresti ascoltare quella roba. Ti mette ansia'' le disse, mentre posava sul tavolo pistola e
distintivo.

``Preferisco essere informata e ansiosa che ignorante e felice.''

Lui sospirò. Lo faceva spesso quando parlavano. Piuttosto che rispondere a un argomento scomodo,
preferiva sospirare. In questo modo risparmiava molte energie, ma mostrava ugualmente il suo
disappunto. Helen sapeva che suo padre non la capiva; non capiva perché la figlia di vent'anni, nel
fiore della sua giovinezza, dovesse interessarsi di quell'attualità angosciante.

E se avesse saputo delle sue uscite notturne\dots Allora l'avrebbe capita ancora meno. Forse
l'avrebbe fatta arrestare. Era quello che faceva la Polizia della Vita, no? Arrestava chiunque
ostacolasse il Ripopolamento. Compresi quelli come lei e Catheline. Figure inutili nel grande schema
delle cose, niente più che parassiti, consumatori di risorse che potevano essere meglio impiegate.
Di tanto in tanto, Helen si chiedeva se suo padre si sarebbe davvero spinto tanto in là. In quanto
Capo della Polizia doveva dare l'esempio, ma avrebbe davvero avuto il coraggio di arrestare la
propria figlia, di interrogarla, di rinchiuderla, di ucciderla? Era difficile dirlo.

Peter Wright era stato un giusto, un tutore della legge integerrimo ma umano. Quando Padre Frey era
andato al potere e aveva istituito la Polizia della Vita, Peter aveva deciso di dare le dimissioni,
sapendo bene a cosa sarebbe andato incontro. Purtroppo non aveva fatto in tempo: la moglie era stata
uccisa dalla Resistenza per rappresaglia contro la cattura, la tortura e l'esecuzione di cinquanta
omosessuali qualche settimana prima. Le avevano tagliato la gola mentre tornava da casa e l'avevano
lasciata morire dissanguata in mezzo alla strada, quindi avevano rivendicato l'atto con dei
volantini sparsi per le strade.

Era stato allora che Peter, accecato dall'odio, aveva strappato la lettera di dimissioni sulla sua
scrivania e aveva abbracciato con entusiasmo il piano di Ripopolamento, mantenendo il suo posto di
Capo della Polizia e giurando fedeltà a Padre Frey. Helen capiva il gesto, eppure non riusciva a
perdonarlo. Come si può lasciare che la propria rabbia si riversi sugli innocenti? In che modo Peter
era migliore degli uomini e delle donne che avevano ucciso sua moglie? Non faceva forse anche lui lo
stesso?

Quando Helen aveva scoperto la propria omosessualità, ne era stata così felice. Era una rivincita
segreta, una sfida. Suo padre non l'avrebbe mai saputo, ma non era importante; importava solo che
lei lo sapesse. Che mantenesse una propria identità era sufficiente a non farla impazzire.

Dopo aver assistito al pestaggio di quel ragazzo, Helen riusciva a contenere a stento il proprio
disprezzo. Sapeva che senza dubbio il padre aveva organizzato tutto nei minimi dettagli, e questo la
disgustava. Voleva colpirlo, ferirlo fino a farlo sanguinare, gridargli il proprio odio. Voleva
pretendere che fosse coerente e che uccidesse anche lei, perché non era giusto che lei potesse
vivere e il ragazzo no. Voleva capire\dots capire perché fosse diventato così, quale logica malata
avesse spinto a passare dalla parte dei carnefici un uomo che aveva giurato di difendere le vittime.

Eppure le mancava il coraggio, e anche questo la disgustava. Era schifata dalla propria codardia.
Rimpiangeva il rifiuto dato a Catheline, ripudiava il suo silenzio durante il pestaggio, odiava la
propria complicità. Disprezzava se stessa, ciò che era diventata.

Decise che quella notte le cose sarebbero cambiate. Quella notte era per lei, per redimersi dai
propri peccati. Quella notte avrebbe fatto un salto nel vuoto, avrebbe affrontato l'ignoto che
l'aveva terrorizzata e affascinata per tutta la vita. Quella notte avrebbe combattuto per ciò in cui
credeva, e non importava l'esito della guerra, solo avervi preso parte, poter dire di non essere
rimasta a guardare mentre il mondo sprofondava e l'Inferno si elevava a Paradiso.

Quella notte Helen si sarebbe unita alla Resistenza.

\plainbreak{1}

Non poteva essere.

Helen era andata a trovare Catheline, ma Catheline non c'era. Il loro luogo di incontro pullulava di
agenti della Polizia della Vita. Sbarravano l'ingresso, interrogavano i passanti, entravano e
uscivano da quella casa che per Helen era sacra, sicura, inviolabile. Helen dovette lottare per non
urlare, per non gettarsi contro di loro. Con il petto infiammato dalla paura e dalla rabbia, li
osservava di nascosto mentre mettevano a soqquadro il posto. Non dovevano. Non potevano.

Come e cosa avevano scoperto di Catheline? E soprattutto, cosa le avevano fatto? Era anche lei in
pericolo? Era riuscita a fuggire in tempo? E se era fuggita, dov'era? Sapevano della sua relazione
con Helen? Sarebbero andati a prenderla? Helen non poteva rispondere a nessuna delle mille domande
che la tormentavano, e ciò la gettava nella disperazione totale. Non sapeva neanche quale sarebbe
stato il suo prossimo passo. A chi poteva chiedere aiuto? Al padre? Era escluso. Avrebbe fatto
arrestare anche lei, se ancora non aveva emesso un mandato di cattura.

Era assurdo. Se la situazione non fosse stata così tragica, ci sarebbe stato da ridere della sua
assurdità. Catheline era riuscita a eludere la polizia per anni, e ora che finalmente anche Helen
decideva di unirsi alla Resistenza, Catheline veniva scoperta, forse catturata e forse uccisa. Era
tutto così surreale, un meccanismo così ridicolo nella sua spietata precisione e infallibilità.

Qualcuno afferrò il braccio di Helen; solo la consapevolezza che urlare sarebbe stato intuile la
trattenne, all'ultimo momento, dal farlo.

``Sono un amico.''
