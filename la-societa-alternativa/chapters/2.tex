\chapter{}
\label{ch:2}

\emph{Fuori dal teatro, stasera, alle nove.}

Il messaggio giungeva inaspettato, ma sapevo chi l'aveva mandato.

Per tutto il giorno pensai a cosa sarebbe stato giusto fare: avrei potuto ignorarlo, e l'orgoglio mi
spingeva a farlo. In quel modo però avrei perso un'occasione. Oppure potevo mettere da parte
l'Achille che era in me e incontrare L. per vedere cos'altro aveva da offrirmi.

Il dubbio mi strusse fino alle otto e quaranta, quando decisi che il mio animo ferito poteva essere
barattato con una vita felice. Ero ancora indeciso quando la vidi proprio davanti al teatro,
pensierosa; sembrò non accorgersi nemmeno del mio arrivo. Nel momento in cui mi avvicinai mi afferrò
per il braccio, dirigendosi verso l'auto dalla quale ero appena sceso.

«Sei in ritardo».

Decisi che quello era il momento di sfogarmi.

«Meravigliati che sia venuto dopo quello che hai combinato l'altra sera con la macchina! È stato
«veramente\dots{}»

«Lascia perdere la macchina, dobbiamo andare».

La mia spietata determinazione si spense come la fiamma di una candela muore a causa di
un'improvvisa folata di vento.

«Andare\dots{} dove?»

«Hai detto che volevi qualcosa di più, no? Questa è la tua occasione. Guidi tu».

Immaginavo che mi avrebbe portato ancora più lontano dalla città, invece mi indicò una via del
centro. Non avevo idea di cosa sarebbe successo ma sapevo che qualunque cosa fosse avrebbe cambiato
la mia vita; ero dunque nervoso, emozionato, terrorizzato, ma allo stesso tempo sicuro di me stesso,
perché sapevo che L. non mi avrebbe lasciato cadere.

O lo avrebbe fatto al solo scopo di afferrarmi per i capelli.

Non disse nulla per tutto il viaggio e non rispose alle mie continue e insistenti domande.

Dopo mezz'ora circa ci fermammo sotto un blocco di lussuosi appartamenti. Il portone si aprì non
appena ci avvicinammo e L. lo tenne aperto per me. Mentre eravamo in ascensore mi guardò a fondo e
disse qualcosa che mi turbò ulteriormente.

«Stasera ti sarà chiesto di prendere una scelta. Non è importante cosa decidi, almeno non per me. In
caso dovessi rifiutare, però, è di fondamentale importanza che non parli con nessuno di ciò che hai
visto e sentito. Se lo farai potrebbe esserci tolta la possibilità di aiutare altre persone, persone
come te. Hai capito?».

Allucinato, la guardavo.

«Hai capito?» chiese di nuovo scuotendomi.

Mossi leggermente la testa, prima su, poi giù, come se volessi dire sì; in realtà, però, non era
proprio un sì. Non troppo convinto, almeno.

\plainbreak{1}

È curioso come il nostro cervello ricordi soprattutto l'odore dei luoghi che visitiamo. Ma se
dovessi sforzarmi di far tornare alla mente l'immagine di quell'appartamento al dodicesimo piano non
ricorderei l'odore, perché non ne aveva; rammenterei bene, invece, il pavimento di marmo sul quale
mi specchiai appena entrato. L'immagine che vidi era quella di un uomo indeciso, impaurito ma
felice; un'espressione che non avevo mai vista dipinta sulla faccia di nessuno fino a quel momento.

Il posto sembrava deserto. L. però si diresse con sicurezza verso un altro ambiente, uno studio con
un'immensa vetrata che permetteva di ammirare la città nel momento in cui il sole cala e le prime
luci si accendono. Davanti a questa, in piedi, stava un uomo non più giovane, di sessant'anni circa,
vestito elegantemente. Aveva l'aria di essere una persona ricca ma sobria, due qualità non
facilmente conciliabili.

«Sei qui» disse L. senza muoversi dalla soglia.

«Oh, finalmente» rispose quello seccato. «Temevo vi foste persi».

Non si voltò per guardarci, ma potevo vedere il riflesso del suo viso nel vetro; mi colpirono più di
ogni altro dettaglio i suoi occhi, stanchi, distratti, indici di una mente rivolta altrove. In mano
teneva un calice di vino bianco.

L. mise una mano sulla mia schiena e mi spinse avanti, ma ella non si mosse; mi inquietava quella
nervosa immobilità di lei che era sempre a proprio agio. Era come se il vecchio fosse tanto
importante o pericoloso da non potercisi permettere alcun errore in sua presenza.

«Questo è l'amico di cui ti avevo parlato» mi presentò.

«Pensi che sia pronto?»

«Sì. Credo di sì».

Mi osservò a lungo. Quando i nostri occhi si incrociarono abbassai lo sguardo, a disagio.

«Venite di là: qui mi sento osservato».

Ci guidò verso una terza stanza, ancora più grande di quella in cui ci trovavamo prima. C'era una
scrivania di mogano tanto grande quanto costosa. Accanto a questa stava una libreria anch'essa di
mogano nella quale erano contenute decine di volumi; pur essendo un buon lettore riconobbi pochi
titoli, perché erano quasi tutti in lingue straniere, per la maggior parte orientali.

L'uomo si sedette su una costosissima poltrona in pelle mentre io e L. rimanemmo al di qua della
scrivania, così che finalmente ebbi modo di vederlo bene. Era vestito completamente di nero ed
effettivamente incuteva un certo timore. Con i suoi lunghi capelli bianchi e la barba ispida
sembrava un dottore o un alchimista, non so bene quale delle due gli si confacesse meglio.

Egli prese una penna e un foglio dalla risma davanti a sé e me li porse.

«Firma».

«Ma non c'è scritto niente!» protestai.

Guardò accigliato L., che aveva l'aria costernata.

«Ti prego, firma» mi implorò anche lei.

Il colmo per un avvocato dev'essere firmare un foglio completamente bianco.

Eppure io lo feci, e non perché fossi convinto, ma perché la pressione era troppa e non volevo
deludere L.

Allora l'uomo prese dalle mie mani carta e penna e iniziò a scrivere qualcosa in bella grafia
proprio sopra la mia firma.

Si alzò di scatto facendomi trasalire.

«Possiamo andare. Hai portato il necessario per il viaggio?»

«Viaggio? Che viaggio?»

Tutta la faccenda iniziava a preoccuparmi.

L. posò le mani sulle mie spalle e mi guardò dolcemente.

«Ti fidi di me?»

Esitai parecchio prima di rispondere: mi fidavo ciecamente di lei ma avevo paura di quali sarebbero
state le conseguenze delle mie parole. Nonostante tutto, però, mai e poi mai avrei rischiato di
perderla.

«Sì».

«Allora non fare altre domande; lo dico per te, è questo che volevi: rischiare tutto per una nuova
vita. Non è così? Devi scegliere usando l'istinto, e ciò sarebbe impossibile se ti spiegassi ogni
dettaglio. Del resto anche volendo non potrei farlo: il futuro non mi è molto più chiaro di quanto
lo sia a te in questo momento. Tuttavia sappi che qualunque sia la tua decisione non potrai
assolutamente tornare indietro».

Tentai di deglutire, ma avevo la gola completamente secca.

«Va bene».

«Va bene cosa?»

«Verrò con voi».

Avevo cercato di rimandare il più possibile quel momento, ma ormai il passo era fatto. E forse era
un passo troppo lungo per la mia gamba.

L. e l'uomo, animandosi all'improvviso, uscirono dall'appartamento, trascinando me dietro. Salimmo
su un Suv parcheggiato poco lontano dall'edificio.

«Ma che ne sarà della mia macchina? E della mia casa?».

«Certo che questa macchina non ti dà pace!» rise L.

«Non ne avrai bisogno dove stiamo andando» aggiunse l'uomo.

«Perché? Dov'è che stiamo andando?»

Nessuno mi rispose, ma non me ne preoccupai troppo: avevo accettato l'invito di due sconosciuti a
iniziare una nuova vita con loro\dots{} Che importava se ancora non sapevo dove? Non era che un
fattarello di poco conto.

Mi appoggiai comodamente al sedile tentando di rilassarmi, e stranamente la cosa sembrava
funzionare. Pensai che il mio destino dipendeva ormai da forze di gran lunga superiori alla mia, e
reazioni che io stesso avevo messo in moto e non potevano più essere fermate. Ero come una spiga di
grano in balia del vento.

Ormai era sera inoltrata e la città era completamente illuminata. Iniziava a piovere; i colori delle
insegne visti attraverso le gocce d'acqua formavano strani giochi di luce. L'uomo guidava tenendo
gli occhi fissi sulla strada, mentre L. sul sedile del passeggero rifletteva silenziosa.

Si avvertiva nell'aria un sentimento pesante: nostalgia e malinconia, il tutto mischiato a una buona
dose di eccitazione. Era il clima ideale per una partenza.

L. si girò e allungò la mano chiara verso di me: sul palmo stava una compressa lunga un centimetro
circa.

«Prendila».

«Cos'è?».

Silenzio.

«Devi\dots{}».

«\dots{}fidarmi di te. Sì, ho capito».

Raccolsi la compressa da quella mano deicata, la misi in bocca e deglutii. Pochi minuti dopo fui
pervaso da un improvviso torpore e chiusi gli occhi, stanchissimo.

L'ultima cosa che ricordo è il suo sguardo vivace e accorto nello specchietto retrovisore.
