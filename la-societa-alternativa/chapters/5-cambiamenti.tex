\chapter{Cambiamenti}
\label{ch:cambiamenti}

«Sono incinta».

Lo disse tutto d'un tratto, senza preavviso, proprio come accade nei film smielati che io non guardo
perché trovo troppo banali.

Quella, effettivamnete, \emph{era} una situazione banale. Un fatto fisiologico, naturale: lei era
una donna sana, io un uomo sano, facevamo l'amore da diverso tempo e ora lei aspettava un bambino.
Io non avevo mai pensato di parlarle della questione.

Non dissi nulla. Non sapevo cosa dire. Andai alla finestra del mio appartamento, e improvvisamente
ricordai che L. non c'era mai stata. Avrei dovuto capire che qualcosa non andava quando mi aveva
chiesto di passare per un caffè.

Ora pioveva, e l'odore di quel caffè riempiva l'aria, e quella donna mi aveva appena comunicato una
notizia che era la causa del mio improvviso tremore.

«Mi rendo conto che non te l'aspettavi e, onestamente, neanch'io. Non ti sto chiedendo niente; non
voglio che ti senta costretto a prendere una decisione, ma mi sembrava giusto che lo sapessi. Se hai
qualcosa da dire, ti prego, fallo».

Un altro lungo silenzio. Nessuno dei due guardava l'altro. Il rumore della pioggia era assordante.

L. si alzò.

«Bene, allora vado».

«Cosa farai?».

«Non lo so ancora».

«E tuo marito? Cosa gli dirai?».

«La verità. È l'unica cosa che merita. Sono certa che capirà.

«Ho paura. Scusa. Non ero pronto. Non sono pronto».

Lei mi abbracciò. Era spaventata. Lo sentivo.

«Non preoccuparti, andrà bene» disse sorridendo. Ma era un sorriso amaro, non sereno come sempre.
Probabilmente L. già intuiva che quella gravidanza avrebbe scatenato una lunga catena di eventi.

Sapeva che le sue certezze stavano per crollare.

\plainbreak{1}

Ovviamente quella notte non riuscii a chiudere occhio. Non capitava dai miei primi giorni nella
Società. Era ormai un anno che ogni sera mi addormentavo tranquillo come un bambino, senza angosce o
dubbi nel cuore.

Credo fossero circa le tre del mattino quando udii il trambusto. Le urla si placarono
immediatamente, e lasciarono il loro posto a un costante rumore di umanità, come se molte persone si
fossero unite a me nella veglia notturna. Ero ancora vestito, così uscii e chiesi a un passante cosa
stesse accadendo.

«Qualcuno vuole andarsene».

«E qual è il problema? Se vuole andarsene, che se ne vada».

L'uomo guardò a terra, chiaramente a disagio.

«Non è esattamente così che funzionano le cose qui, ragazzo: solo i membri anziani possono entrare e
uscire liberamente; tutti gli altri devono dare una motivazione, e ``Mi sono stufato'' non è
accettata. E\dots{} in ogni caso, a nessuno è permesso mollare: una volta che sei dentro, sei dentro
per sempre».

Quella conversazione mi permise di comprendere gli avvertimenti che L. mi aveva lanciato quando ci
vedevamo al teatro. Ma perché non era stata più esplicita? Perché tutti conoscevano quella regola
tranne me?

Poiché in quel momento lei era l'ultima persona a cui volessi pensare, decisi di rimandare a un
altro momento le riflessioni.

Mi incamminai seguendo gli altri. Le strade non erano illuminate --- nessuno si sarebbe sognato di
aggredire un proprio pari nella Società --- così inciampai diverse volte.

Qualcuno si trovava davanti al pesante cancello verde, che era saldamente chiuso. Di fronte a lui
stava una donna, appoggiata alle sbarre, con le braccia conserte e un malvagio sorriso di scherno
sul viso. Certamente ricopriva una posizione influente: si comportava con la sicurezza di chi è nel
proprio ambiente ed è consapevole del proprio potere.

«Lasciami passare, Nadia!» intimava lui, visiblmente agitato e irritato.

«Mi dispiace, ma davvero non posso farlo. Sapevi quali erano le regole, le hai accettate; ora devi
convivere con la scelta che hai fatto».

Ero riuscito ad avvicinarmi abbastanza da riconoscerlo: si trattava del marito di L., e con lui
stava il figlio. Quando lo vidi una fitta allo stomaco mi tolse il respiro, come se qualcuno mi
avesse colpito con violenza. Temevo di conoscere il motivo di quell'improvviso desiderio di libertà:
L. era stata sincera, e ora lui, disgustato, voleva andare via.

«È inaccettabile! Questo è un sequestro di persona! Vi farò arrestare tutti!».

Nadia rise crudelmente.

«Per farlo dovrai prima uscire di qui».

Lui la guardò con aria di sfida.

«Ci riuscirò».

La donna si fece subito seria. Si allontanò dal cancello e andò via.

Quando gli passò davanti disse con voce roca: «Per il tuo bene e quello di tuo figlio, spero
veramente di no».

Quindi scomparve nell'oscurità.

L'uomo corse verso le sbarre di acciaio, vi si aggrappò e le scosse con violenza, provocando un
tremendo rumore e urlando: «Fatemi uscire di qui, bastardi!». Ma assolutamente nulla accadde.

Il bambino sembrava molto spaventato, non l'avevo mai visto così. Provai pena per lui.

La folla si faceva sempre meno numerosa: i membri della Società Alternativa tornavano alle loro
perfette vite, colme di ispirazione e arte.

Rimanevamo ormai io e pochi altri.

Lui impiegò qualche minuto per arrendersi. Sospirò e decise di lasciar perdere. Quando i nostri
occhi si incrociarono mi guardò con odio. Non era l'odio di un adolescente lasciato dalla fidanzata,
ma quello di un uomo maturo ed esperto cui è stata tolta la propria ragione di vita.

Abbassai lo sguardo, vergognandomi come un assassino; ora che ci penso, non avrei dovuto: non avevo
nulla di cui sentirmi in colpa.

E poi, se non avessi guardato a terra, forse avrei visto arrivare il pugno. Mi colpì dal basso,
subdolamente, sul sopracciglio. Fu abbastanza forte da farmi cadere, sbattendo violentemente la
testa. L'ultima cosa che vidi fu il cielo stellato.

\plainbreak{1}

«Ahia».

«Scusa».

Ero steso su un divano, e L. temponava la ferita al sopracciglio. Non l'avevo vista al cancello; ciò
non poteva che essere un bene.

Mi alzai di scatto e avvertii un forte dolore alla testa.

«Non fare movimenti bruschi. Il dottore sarà qui tra poco».

«Dove sono tuo marito e tuo figlio?».

Scosse la testa.

«Non lo so, ma non possono essere andati lontano. Non credo che torneranno a casa, però». Sospirò.
«Mi dispiace tanto: non hai colpa in tutto questo».

«Neanche tu».

«Invece sì. La colpa è mia e di nessun altro. La gravidanza è stata la goccia di troppo, ma già da
molto tempo mettevo alla prova la devozione di mio marito, per esempio con le mie lunghe assenze».

«Ma non mi hai detto che amare significa non aspettarsi nulla in cambio?».

«Sì, ma vuol dire anche rispettare l'altro. Lui ha accettato il mio stile di vita perché mi amava,
ma non è mai stato d'accordo. Ha sbagliato nel non dirmelo, e io sono stata una pessima moglie
perché non me ne sono accorta».

Il dottore disse che la ferita non era grave, ma sarei dovuto stare a riposo per qualche giorno.

L'indomani appresi che il marito di L. era tornato alla sua vita comune: stranamente Nadia l'aveva
lasciato andare via quella stessa mattina.

A L. non chiesi nulla in merito.

Lei non mi cercò.
