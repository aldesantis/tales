\chapter{Il rito}
\label{ch:il-rito}

Conobbi L. durante una fase particolare della mia vita: ero un avvocato di discreto successo, eppure
non mi sentivo affatto felice; mi sembrava che le mie giornate fossero insopportabili, soffocanti
nella loro perfezione. Ero una persona realizzata: facevo ciò che avevo sempre sognato, avevo
trovato l'amore, riuscivo negli sport, ma mancava quella scintilla che mi permettesse di andare
avanti, che rendesse ogni istante della mia esistenza unico e speciale.

Non auguro a nessuno di provare una sensazione del genere: si sente che c'è qualcosa che non va ma
non si riesce a individuare esattamente il problema. Non si può fare niente per migliorare ciò che
si è, non importa quanto ci si sforzi di pensare a una vita migliore. È un dolore così opprimente
che neanche piangere sembra appropriato di fronte a cotanta sofferenza.

Un giorno venne da me questa donna. Non era la stessa che mi sarei abituato a vedere dopo diverso
tempo: più ordinaria, meno potente. Anche così però si portava dietro un'aura di energia ovunque
andasse. ``Energia''\dots{} era quello l'elemento mancante nella mia vita; era a causa della sua
assenza che stavo tanto male. Non mi resi subito conto di chi avevo di fronte, perché lei stava ben
attenta a fare in modo che solo chi era pronto si accorgesse della sua grandezza. Solo quando fu
sicura che fossi la persona adatta si lasciò andare e io la potei ammirare in tutta la sua
maestosità.

All'inizio si trattava di piccole consulenze. Mi chiedevo perché una donna tanto bella si
interessasse a un argomento noioso come il diritto. Un giorno mi azzardai a chiederle che lavoro
facesse.

«Io?» chiese sorpresa. «Be', si potrebbe dire che sono un'artista, credo».

Non feci altre domande, perché capii che ancora non voleva spingersi tanto oltre. Ma più tempo
passavo con lei e più ero intossicato dai suoi atteggiamenti, dalla sua esistenza.

\plainbreak{1}

Poco dopo il nostro primo incontro iniziai a scrivere. Non narravo apertamente di lei, forse perché
mi vergognavo all'idea che qualcuno che conoscevo appena potesse avere un ascendente così forte su
di me.

Lasciavo le cartelle in ufficio perché così potevo uccidere quello sprazzo di noia che esisteva tra
un cliente e l'altro, quando il lavoro non mi impegnava e non potevo pensare a qualcosa che non
fosse l'imperfezione della mia vita, un po' meno imperfetta da quando quella donna vi era entrata.

Quel pomeriggio dovevo incontrare L. ed ero in ritardo. Sembrava arrivare sempre al momento giusto,
ovunque ci fosse da attuare un cambiamento.

Entrai di corsa e incrociai per un istante il suo sguardo.

Aveva in mano una delle mie cartelle.

«Scrivi molto bene» osservò.

Non mi aveva mai chiesto il permesso di darmi del tu; lei non chiedeva mai il permesso di fare
qualcosa. Lo faceva e basta.

La ringraziai arrossendo.

Tentai di parlare di lavoro ma non me lo permise.

«Vedi,» mi interruppe «io faccio parte di un gruppo di artisti, e mi piacerebbe che ti unissi a noi.
Sarebbe un enorme contributo».

«Veramente non saprei\dots{} ho parecchio da fare in questo periodo».

Tentai di evitare il confronto diretto: avevo paura in un certo modo di L., di ciò che sarebbe
potuto accadere se fossi entrato nel suo strano mondo; era come non mi sentivo all'altezza. Lei era
così potente e decisa e io così infinitamente debole e nudo.

Tuttavia l'idea di essere parte di qualcosa di più grande mi affascinava: mi sentivo come un bambino
che gioca col fuoco e sa che in ogni momento potrebbe bruciarsi, e ha paura di provare dolore ma
ancora più paura di rimanere ignorante.

«Avanti,» disse con voce suadente «sono sicura che non te ne pentirai».

Alla fine accettai e non controvoglia, solo con un po' di timore.

«Benissimo» rispose L. «Ora possiamo parlare di lavoro».

\plainbreak{1}

Mi lasciò un indirizzo e una data. Si rifiutò di dirmi dove sarei andato.

I giorni si succedettero in fretta, e il tanto atteso --- e temuto! --- momento arrivò prima di
quanto mi aspettassi. Un'ora prima dell'incontro ero nervoso come non mai, ed enormemente tentato di
chiamarla e disdire tutto. Non lo feci, forse, solo perché avevo paura di sentire la sua voce dopo
tanto tempo: era un po' infatti che L. non si faceva vedere allo studio, e credevo si fosse
dimenticata di me.

Ormai rassegnato, ma eccitato al tempo stesso, mi avviai in macchina verso quel luogo. Era piuttosto
lontano dal centro e sulle mappe non era segnato assolutamente nulla.

Giunsi a un teatro; non era più attivo da qualche anno. Tuttavia era ben curato da una mano attenta.
Mi parve di riconoscere il suo tocco.

All'interno l'aria sapeva di poltrona eppure era familiare e rassicurante. Nonostante ci fossero
diverse sale seppi subito dove andare: forse fu perché ormai ero così assuefatto a L. che l'avrei
trovata ovunque. Forse, più semplicemente, seguii le voci.

All'interno della stanza si trovavano lei e un'altra decina di persone. Proprio in quel momento una
donna bionda e grassoccia in piedi sul palco stava leggendo quella che presumetti essere una sua
poesia. Rimasi in piedi ad ascoltarla, ma L. mi fece segno di sedermi accanto a lei in prima fila.
La salutai; sorrise e posò il dito sulle labbra, chiedendomi di fare silenzio.

Quando ebbe terminato tutti applaudirono. Uno dopo l'altro ognuno si esibì nella propria arte.
Credevo fossero tutti scrittori ma mi sbagliavo: c'era chi cantava, chi ballava e chi mostrava le
sue ultime fotografie.

Infine arrivò il suo turno. Salì sul palco, meravigliosa nel suo abito, e accompagnata da
un'orchestra nell'angolo della sala cantò con una delle più belle voci che avessi mai ascoltato. Le
parole non possono descrivere le sensazioni che quei suoni mi hanno lasciato, dunque mi limiterò a
dire che quando finì provai la stessa tristezza che si prova svegliandosi e interrompendo a metà uno
stupendo sogno.

Nessuno la applaudì: sembravano tutti stupiti quanto me, sebbene, immaginai, non fosse la prima
volta che la ascoltavano. Al termine L. non scese dal palco ma vi indugiò qualche momento, tenendo
gli occhi chiusi e la testa bassa, le labbra incurvate in un impercettibile sorriso.

«Oggi, \emph{amici},» chissà perché, quella parola mi colpì «si è unito a noi un nuovo artista; uno
scrittore. L'ho conosciuto per caso, come è accaduto con la maggior parte di voi, e per caso ho
scoperto il suo talento. Lo vorrei qui accanto a me, sul palco, con una delle sue opere che sono
sicura avrà portato».

Tutti si voltarono a guardarmi e proprio come accadde con L. in ufficio, arrossii. Abbozzai un
sorriso ebete, mi alzai e con le mani sudate presi dalla tasca un foglietto, piegato tante volte da
essere quasi illeggibile, sul quale era riportato uno dei miei ultimi testi.

«Prego» disse L. sorridendomi rassicurante e facendosi da parte. Avevo l'impressione che conoscesse
il mio stato d'animo in quel momento e avesse deliberatamente deciso di mettermi in imbarazzo;
quello che non capivo, però, era il motivo: qual era il suo scopo? Che cosa sperava di ottenere?

Così, con voce incrinata dall'emozione lessi quelle poche ma interminabili righe. Man mano che
procedevo l'aria si faceva sempre più calda e pesante. Quando infine la tortura cessò mi sembrò di
liberarmi di un pesante fardello, qualcosa che mi portavo dietro da anni senza neanche saperlo,
qualcosa che era venuto a galla nel mare della mia anima solo quel pomeriggio.

«Molto bene» disse L. Sembrava fiera di me, come una madre di suo figlio.

Più tardi, quando gli altri se ne furono andati e ci trovammo da soli, le parlai dei problemi che
avevo avuto. «Non capisco,» dissi «sono abituato a parlare in aula, sotto pressione, davanti a
tutti. Eppure rendere gli altri partecipi di ciò che ho scritto mi è costato un'enorme fatica».

«Forse sei un buon avvocato che sa come usare le prove a proprio favore. Forse sei abituato a
esporre i fatti in aula. Ma non sei abituato a comunicare i tuoi sentimenti più profondi. È proprio
qui il punto: è per questo che ho voluto che leggessi davanti a noi, stasera. Ci siamo passati
tutti, non preoccupartene troppo».

«Tutti tranne te, a quanto pare. Sembri così naturale quando sei sul palco e canti\dots{}».

«Ho dovuto lavorare anch'io per avere una tale dimestichezza con le persone; le persone sono
complicate, ma meravigliose a loro modo».

Parlammo ancora per un po', quindi ci salutammo e ognuno andò per la propria strada.

\plainbreak{1}

A quell'incontro ne seguirono altri, simili ma mai uguali: a volte veniva qualcuno di nuovo e
anch'egli doveva passare quel rito di iniziazione del quale ero stato partecipe solo poche settimane
prima. Vedendo come erano impacciati i nuovi arrivati capivo dove avevo sbagliato fino a quel
momento. Nonostante fossi ancora lontano dal raggiungere i livelli di L., mi sembrava in qualche
modo più facile vivere, vedere, respirare.

Mi stavo sensibilizzando: iniziavo a cogliere la bellezza, più spesso la bruttezza del mondo intorno
a me, quando prima ero apatico. Mi sembrava di poter capire meglio i sentimenti altrui, le
motivazioni dietro alle azioni e le idee ancora più dietro.

C'era qualcosa, però, che mi angosciava terribilmente: andare agli incontri stava diventando
un'abitudine, e dunque iniziavo a perdere interesse. Pensai di essere io quello sbagliato e
incontentabile, sempre alla ricerca di stimoli fuori dalla mia portata. Ero terrorizzato all'idea di
stufarmi di L., dei suoi sorrisi e della sua voce.

A un incontro le parlai delle mie paure. Non sembrava sorpresa.

«Mi dispiace che ti senti così» disse.

Le dissi che a mio parere non era lei il problema, che pensavo di non essere adatto alle azioni
ripetitive: il gruppo era una novità all'inizio ma ormai iniziava a essere un obbligo.

«L'uomo ha bisogno di azioni ripetitive. Ciò di cui non ha bisogno è la routine: respiri, altrimenti
soffocheresti, ma non ne soffri. Mangi, perché moriresti di fame non facendolo, eppure non te ne
lamenti; non lo fai perché queste sono \emph{necessità}. Ma annoiarsi non è una necessità,
tutt'altro! È quando l'azione smette di essere dettata dall'istinto di sopravvivenza che diventa
routine, capisci?»

Non ne ero troppo sicuro, ma risposi di sì.

«Adesso cosa farai?»

«Non lo so, ma non smetterò di venire agli incontri».

«Non avevo dubbi, ma io intendevo adesso, ora, in questo preciso momento. Salirai in macchina e
«poi?»

«Be', è tardi\dots{}»  risposi guardando l'orologio. «Andrò a casa, farò una doccia e andrò a
dormire. Domani devo alzarmi presto».

«Benissimo, allora non hai fretta: andiamo».

Ebbi il sospetto che non avesse sentito la mia ultima frase.

«Dove?»

«A fare una passeggiata; il mare non è lontano».

\plainbreak{1}

Poco oltre il teatro, in effetti, si trovava una bellissima spiaggia di cui non mi ero mai accorto
in tutto quel tempo.

Giungemmo sul limitare della strada, là dove finiva l'asfalto e iniziava la sabbia.

L. tolse le scarpe e mi invitò a fare altrettanto. Esitante, la imitai.

Mi prese sottobraccio e camminammo così per un'ora almeno, avanti e indietro su quella spiaggia che
sembrava non aver ancora subito il triste intervento dell'uomo. Parlammo di moltissime cose:
religione, amore, filosofia, e anche delle piccole sciocchezze di ogni giorno. Non capivo cosa
stesse cercando di ottenere e non mi importava affatto: avrei solo voluto che quel momento non
finisse mai.

Al ritorno però mi aspettava una spiacevole sorpresa: la mia auto era sparita. L. sembrò non
accorgersene, tanto che stava per andare via, lasciandomi solo.

«Mi hanno rubato la macchina!» urlai verso di lei, già lontana.

Si voltò stupita e tornò sui suoi passi.

«Come?» disse ridendo.

«La mia macchina\dots{} è sparita!»

«Sì, lo so. L'ho fatta portare via io».

Sentii le guance avvampare per lo stupore e l'irritazione.

«Cos'è, uno scherzo?».

La mia voce si faceva più alta.

«Tutt'altro» rispose, dimostrando una calma sconcertante.

Boccheggiai, incapace di dire qualunque cosa, e allora lei iniziò ad allontanarsi.

«Aspetta! Come torno a casa?».

«Arrangiati» la sentii dire.

Non ebbi la prontezza di inseguirla: ero troppo sconcertato dal suo atteggiamento. Sapevo che L. non
mi avrebbe mai ferito. Non senza avere un fine più grande, almeno. Ma allora cosa voleva fare?
Perché metteva a così dura prova la mia dedizione?

Ebbi modo di pensarci mentre facevo l'autostop. Quando, dopo circa due ore --- in pochi passavano di
lì, in pochissimi andavano nella mia direzione, e solo uno era disposto a dare un passaggio a uno
sconosciuto --- tornai a casa, trovai l'auto parcheggiata fuori dal cancello, come se non fossi mai
uscito. Come immaginavo, non mancava assolutamente nulla.

Mi sdraiai sul divano, e il sonno mi trovò vestito, mentre ancora riflettevo, e dentro di me covavo
un sentimento che non era amore ma neanche odio.
