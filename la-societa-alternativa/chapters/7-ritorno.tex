\chapter{Ritorno}
\label{ch:ritorno}

Nei tre giorni successivi L. sfruttò tutte le sue conoscenze per tentare di evadere da quel luogo,
ma senza successo. Alcuni non potevano aiutarla, altri avevano paura, altri ancora, addirittura,
minacciarono di denunciarla al Consiglio.

La notte del terzo giorno, finalmente, qualcuno decise di darci una mano. Qualcuno da cui davvero
non pensavo di riceverla.

Eravamo a casa di L. e fuori un brutto temporale stava contribuendo a peggiorare il nostro umore. Un
tuono squarciò l'aria proprio nel momento in cui il campanello suonò. Entrambi non ci aspettavamo
che qualcuno venisse a trovarci: la voce si era sparsa rapidamente e ora, per tutti, eravamo dei
traditori.

«Tutti ti vogliono bene finché tifi per la loro squadra» aveva detto lei quando si era accorta che
nessuno si degnava più neanche di salutarci.

Per qualche secondo non ci muovemmo.

«Forse vogliono controllare se siamo in casa?» suggerii.

«Ti assicuro che non hanno bisogno di suonare il campanello per saperlo» commentò.

Pigramente raggiunse il citofono e chiese chi fosse. La risposta dovette sorprenderla, perché cambiò
subito espressione. Esclamò solo «oh».

Quando la porta si spalancò, e fece il suo ingresso il vecchio che L. sembrava rispettare tanto,
rimasi a bocca aperta. Era vestito proprio come la prima volta che l'avevo visto: completamente di
nero. Nera la camicia, neri i pantaloni, nere le scarpe. I capelli, questi invece bianchissimi,
erano elegantemente pettinati.

«Dobbiamo muoverci, non c'è tempo».

L. aveva già raccolto le poche cose che voleva portare, compresa la foto di sé con il marito che
teneva sulla libreria. Io mi avvicinai subito a lui.

«Credevo che lei non stesse qui».

«È che non mi piace farmi vedere in giro» rispose mentre dava un'ultima occhiata alla casa e
chiudeva la porta alle nostre spalle. «Piacere,» aggiunse tendendomi la mano «sono Aaron».

Uno dei quattro uomini che lo accompagnavano --- erano divisi in due macchine, e in una terza,
immaginai, dovevamo salire noi --- si avvicinò e lo prese per il braccio.

«Signore, bisogna andare. Rimandate le presentazioni».

Quando sistemò la giacca notai che era armato: alla cintura portava una pistola. Non mi sorpresi:
Aaron, chiunque fosse, era certamente il genere di individuo che aveva bisogno di guardie del corpo.
E ora che ci aveva aiutati la sua incolumità sarebbe stata ulteriormente a rischio.

«Sì, certo. Forza, salite in macchina».

Lui si mise alla guida.

Pioveva quando ero arrivato, e pioveva ora che me ne stavo andando. Che ce ne stavamo andando.

Il cancello si aprì senza problemi. Era assurdo come fosse semplice entrare e uscire da quel posto,
se solo ci si accompagnava alle persone giuste.

Dopo quasi mezz'ora abbandonammo il cumulo di nuvole. Il complesso era alla nostra sinistra,
lontano, e spaventosi lampi lo illuminavano di tanto in tanto, facendo luce sugli oscuri segreti che
la Società celava.

Giungemmo infine alla pista d'atterraggio costruita nel bel mezzo del nulla. Un piccolo aereo ci
aspettava, forse lo stesso che ci aveva portati lì.

Mentre L. saliva a bordo, Aaron la guardò dritta negli occhi e disse: «Mi dispiace per tuo marito.
Era un brav'uomo».

Lei fece solo un cenno con la testa.

\plainbreak{1}

Aaron e i suoi uomini partirono con noi. L. si addormentò poco dopo il decollo, sfinita dagli
avvenimenti e dalle emozioni di quei giorni. Io volevo sapere di più sull'uomo che sembrava tanto
importante quanto pericoloso e aveva deciso di aiutarci.

Fui diretto.

«Chi è lei?».

«Chi sono? È una domanda vaga. Ho dedicato la mia vita a cercare la risposta, e penso di non averla
ancora trovata. Immagino però che non sia questo che vuoi sapere».

«No, infatti. Quello che voglio sapere è qual è il suo ruolo nella Società Alternativa».

L'uomo sospirò profondamente, come se rispondere gli costasse una immensa fatica.

«Io l'ho fondata, molto tempo fa».

Certamente era l'ultima risposta che mi aspettavo.

«Se l'ha fondata, perché ha deciso di aiutarci? Non è stato proprio lei a organizzare questo
complotto?» chiesi, improvvisamente sospettoso.

«Assolutamente no» disse sdegnato. «Non era questo che volevo quando le ho dato vita. La Società mi
ha tradito tanto quanto ha tradito voi due. Il suo fine rimane quello di creare un mondo ideale, ma
i mezzi sono cambiati. Come possiamo creare un mondo dove non esista l'omicidio tramite
l'omicidio?».

«È per questo che ci ha salvati?».

«Per questo, e anche perché conosco la tua amica,» e indicò L. che dormiva tranquilla «da quando era
una bambina. Quando ho fondato la Società è stata una delle prime persone a cui ho chiesto di
entrare. Non potevo lasciare che accadesse qualcosa a lei o alla sua famiglia».

Immaginai L. che si lasciava guidare e istruire da qualcuno. Anch'ella, dunque, aveva avuto bisogno
di un maestro!

Pensai di essere in buone mani. Mi sentivo, per la prima volta da quando avevo scoperto la verità,
al sicuro: Aaron e i suoi amici avrebbero pensato a tutto.

«La ringrazio».

Lui scosse la testa.

«Non ringraziarmi ancora. Fallo quando sarete tutti lontani, compreso il bambino».

«Teme che possano raggiungerlo?».

«Sicuramente l'hanno già raggiunto,» e quella prima frase mi fece gelare il sangue nelle vene «ma
ancora non hanno fatto nulla: è ben custodito da certi miei conoscenti. Però non posso proteggerlo
in eterno: dovete andare via finché siete in tempo».

\plainbreak{1}

Quando l'aereo atterrò chiesi ad Aaron dove ci trovassimo.

«In Inghilterra; troppo vicino a una delle sedi della Società. Siamo qui solo per prendere il
«bambino e andarcene».

«Una delle sedi? Dunque ce ne sono diverse?».

«Quattro che io sappia, e probabilmente altre della cui esistenza sono tenuto all'oscuro».

L. era ormai sveglia e comprensibilmente nervosa, anche se cercava di non farlo vedere. Il
portellone si aprì con lentezza estenuante, e un centimetro alla volta compariva suo figlio,
accompagnato dagli uomini di Aaron. Appena le fu possibile gli corse incontro e lo abbracciò,
baciandolo più volte sui capelli e sulla fronte. Il bambino era confuso, sembrava non rendersi
nemmeno conto della situazione.

Aaron guardava la scena sorridendo.

«Come avete fatto a prenderlo?» gli domandò L.

«Abbiamo dovuto rapirlo. Mi dispiace per il trauma che abbiamo causato ai tuoi genitori, ma è meglio
che non sappiano nulla. Dovete sparire».

Ci chiese di allontanarci, in modo che potesse parlare sola con il figlio.

Dopo qualche minuto tornarono da noi.

«Sono pronta. Andiamo».

\plainbreak{1}

L'aeroporto era affollato: persone che andavano e venivano, uomini d'affari, famiglie in gita\dots{}

Faceva freddo, come in tutti gli aeroporti.

Aaron comprò due biglietti per la prima destinazione che fosse abbastanza lontana. Con noi non
avevamo nulla, a parte i passaporti e la voglia di lasciarci tutto alle spalle.

Solo noi e il resto del mondo. Finalmente.

Arrivò il momento di salutarsi. E il freddo parve farsi più intenso.

L. strinse Aaron con quanta forza aveva in corpo.

«Grazie di tutto. Se non fosse stato per te, ora non so dove mi troverei, né dove si troverebbe mio
figlio. Ho condotto così tante persona alla rovina,» continuò «e loro non se ne rendono neanche
conto. Quando accadrà, potranno mai perdonarmi?».

«Tu stessa sei stata a un passo dalla rovina. Non hai nulla da farti perdonare. Vai, e concentrati
«su tuo figlio».

L'uomo mi strinse la mano.

«Giurami di proteggerla, qualunque cosa accada. Io ho fatto lo stesso molto tempo fa, e farò del mio
meglio per mantenere il giuramento, ma non sarò più così vicino».

«Lo giuro» dissi. «Perché non viene anche lei?».

«Vorrei tanto, ma devo finire quello che ho iniziato. La Società è un mostro che è nato solo grazie
alle mie idee. L'ho creata, e posso anche distruggerla. È solo una questione di tempo. Ora andate.
Buona fortuna».

Non era tipo da guardarci partire: si voltò e andò via.

Quando presentammo i passaporti che ci aveva dato poco prima, la donna dietro il banco ci guardò
sorridendo.

«Che bella famiglia! Andate in vacanza?» chiese.

«Al contrario,» disse L. «stiamo tornando solo ora».
