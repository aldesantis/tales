\chapter{Dio è amore}
\label{ch:dio-e-amore}

Il teatro era simile a quello nel quale si riuniva il gruppo di L., ma c'erano molte più persone;
sembrava quasi che lo spettacolo avesse luogo in una metropoli, tanto la sala era piena. Eravamo
tutti seduti --- io in prima fila, perché mi era stato tenuto un posto d'onore --- e pronti ad
assistere. Ero nervoso, agitato al pensiero che lei avrebbe nuovamente cantato: la sua voce mi
permetteva di vedere altri mondi, provando sensazioni con cui non ero mai entrato in contatto prima.

Poi lei comparve sul palco, anche quella volta vestita come una dea, e si fece il silenzio. Una mano
invisibile abbassò le luci proprio nel momento in cui L. iniziava a cantare, con l'orchestra che la
accompagnava meravigliosamente. Insieme a lei c'era un coro; forse una decina di persone che
contribuivano con le loro voci, belle ma mai quanto la sua, a rendere il tutto un'esperienza magica.

Pensai a quelle persone: ognuna di loro aveva una diversa storia da raccontare. Forse alcuni avevano
dei figli, altri erano divorziati, o addirittura non avevano mai conosciuto le gioie dell'amore.
Eppure in quel momento erano lì, come una sola, grande entità, e cantavano all'unisono per me. Il
pensiero, nonostante la sua ovvietà, mi fece venire i brividi.

Lo spettacolo andò avanti per poco più di un'ora. L. pareva non essere mai stanca. Io avevo
reclinato la testa e chiuso gli occhi, cercando di concentrarmi sulla musica, sulla sua meravigliosa
voce ora incredibilmente acuta, ora sensazionalmente grave. Passavo dalla risata al pianto con una
velocità impressionante, e mi resi conto di cosa intendesse dire L. il giorno prima, quando parlava
di ``mostrare come ci si sente''. Io, lei e tutti gli altri spettatori eravamo una cosa sola,
proprio come i membri del coro.

L. sembrava assente: ogni tanto mi guardava negli occhi, permettendomi di ammirare quelle splendide
iridi azzurre e profonde, ma la sua attenzione era rivolta altrove, probabilmente ai mondi e ai
personaggi di cui narrava.

Infine, rapido come era iniziato, tutto terminò. Le luci diventarono lentamente più brillanti,
finché tornarono alla loro originaria vividezza. Per pochi secondi il tempo sembrò fermarsi: nessuno
osava neanche respirare. All'improvviso il pubblico esplose in un rumorosissimo applauso, al quale
io mi unii con piacere. L., sorridente, fece un inchino, quindi scomparve dietro al palcoscenico.

Uscì da una porticina che dava sul pubblico, così che molti le corsero incontro per complimentarsi,
mentre altri, che evidentemente avevano già assistito alla scena diverse volte, se ne andavano
scuotendo la testa e ridacchiando alla vista dei suoi ammiratori. Anch'io volevo parlarle, ma in
privato, per raccontarle ciò che avevo provato in quell'ora. Così aspettai che la folla si fosse
diradata. Rimaneva ormai una decina di persone, dunque mi feci avanti.

Ero a pochi passi e L., che mi aveva visto, stava per venirmi incontro. Ma in quel momento un
bambino --- avrà avuto cinque o sei anni --- spuntò dal nulla, correndo verso di lei. Lo prese
prontamente in braccio, mentre lui urlava: «Mamma! Sei stata bravissima!».

«Grazie, tesoro» replicò con affetto, baciandolo su una guancia e scompigliandogli i capelli.

Un uomo dai lineamenti simili a quelli del bambino, di qualche anno più vecchio di L., si avvicinò
poco dopo. Lei lo abbracciò a lungo e infine gli regalò un bacio sulle labbra.

«Mi siete mancati» sussurrò.

I tre si accorsero improvvisamente della mia presenza.

«Spero ti sia piaciuto il concerto» disse L. «Ti presento mio marito e mio figlio».

Lui mi porse la mano, che mi sforzai di stringere con vigore, per dimostrargli che non ero
intimorito.

«Piacere, ragazzo» mi salutò con voce profonda.

Non ero intimorito. Ero terrorizzato.

In primo luogo perché L. era una persona dalle risorse potenti e illimitate, e dunque doveva esserlo
anche suo marito.

Inoltre --- e questa era la mia paura maggiore --- temevo che, ora che si trovava accanto alla
famiglia, lei si sarebbe allontanata da me. La paura dell'abbandono mi avrebbe portato a essere
ostile nei loro confronti, e l'ostilità sarebbe certamente stata percepita da L., che allora mi
avrebbe abbandonato sul serio.

Ero un'egoista.

Ma io non potevo, non dovevo perderla.

\plainbreak{1}

Accadde invece l'esatto contrario. L. non mi fu mai vicina come in quei giorni: ogni volta che le
era possibile, mi chiedeva di farle visita, a pranzo, a cena, o semplicemente per un tè pomeridiano.
E ogni volta il marito e il figlio erano presenti, benché il primo si tenesse in disparte,
limitandosi a salutarmi quando arrivavo e quando andavo via. Come sempre, mi era impossibile
decifrarla e capire se il suo obiettivo fosse mostrarmi che non avevo da temere il marito, o
mostrare al marito che non aveva da temere me.

Del tutto particolare era il suo rapporto con il figlio, che, come avevo intuito durante il nostro
primo, breve incontro, aveva poco più di sei anni. Non ho mai visto una madre tanto paziente, così
come mai ho avuto a che fare con un bambino tanto curioso e, allo stesso tempo, obbediente. A volte
giocava nell'immenso giardino di fronte alla villa o in camera --- e se faceva qualcosa di
pericoloso, o dannoso, L. lo chiamava e gli spiegava cosa non doveva fare e perché, senza ira o
disappunto, e lui annuiva e andava a fare altro --- altre rimaneva con noi, seduto in terra, a gambe
incrociate, e ci ascoltava, ponendo di tanto in tanto domande che dimostravano una sconcertante
maturità intellettuale.

Ricordo che una volta, con la sua innocente voce domandò alla madre chi fosse Dio. Lei lo guardò,
pensando non alla risposta corretta, ma a quella meno sbagliata.

«Non chi, ma cosa. Dio è amore. È quella forza che permette all'uomo di creare e vivere cose
meravigliose, che fa sì che i pianeti continuino a girare intorno al sole, che permette ancora la
nascita e lo sviluppo di relazioni sincere tra gli esseri umani. Da solo, Dio non avrebbe
significato: sono gli uomini che, tramite la loro sola azione, gli conferiscono il potere di
plasmare il mondo. Perché Dio è l'Universo, e l'uomo è Dio, e chiunque pensi che Dio sia superiore
all'uomo --- perché così gli è sempre stato detto, o perché dopo aver studiato molto è giunto a
questa conclusione --- ha perso di vista i propri obiettivi. Dio non chiede altro che lo spazio per
esprimersi nelle nostre vite: se glielo concediamo, esse saranno giustificate, altrimenti non avremo
motivo di esistere, indipendentemente da ciò di cui ci convinciamo».

«Quindi per dare un senso alla propria vita bisogna necessariamente lavorare?» intervenni io.

«Se il tuo lavoro è un'attività che ami praticare e che porta un contributo all'umanità, allora è
\emph{possibile} raggiungere questo stato anche lavorando, sì. Tuttavia ci sono altri modi: pregare,
per esempio».

«Ma la preghiera non aiuta nessuno!».

«Al contrario: la preghiera porta tranquillità e pace, e con esse la capacità di pensare
lucidamente, e dunque la creatività».

«Tutte le persone che conosco --- che conoscevo --- pregavano per parlare con Dio».

«Sbagliavano. La preghiera non è un modo per comunicare con Dio, ma ci pone in uno stato di
rilassamento nel quale è più semplice entrare in contatto con Esso».

«Tu preghi?».

«Lo facevo, quando non ero parte della Società,» e alzò gli occhi al cielo perché aveva di nuovo
detto ``la Società'' «perché era l'unico modo per trovare la pace. Ma quando si è circondati dalla
pace, non ce n'è bisogno. Prega pure, se ne senti il bisogno, ma non pensare che ciò darà un motivo
alla tua esistenza: il passo successivo alla contemplazione è la creazione».

«Dunque Dio è nell'azione. Non Lo si può raggiungere senza fare nulla: bisogna dipingere, scrivere,
«cantare\dots{}»

«\dots{}o pensare» completò lei. «La mente è uno strumento potente, capace di cose che non
immagineremmo mai. L'azione non è necessariamente qualcosa di fisico: chi lavora dodici ore al
giorno, e passa le altre dodici stressato perché si preoccupa del domani, vive nell'azione, eppure è
un'azione statica, una situazione in cui l'anima stagna e non è possibile parlare con Dio».

Ero confuso, ma sapevo che, come accadeva con ogni insegnamento di L., con il tempo tutto mi sarebbe
stato più chiaro.

Mi voltai verso il bambino per vedere se lui avesse capito --- nel qual caso mi sarei sentito un
idiota --- ma era già andato via.

\plainbreak{1}

Pochi giorni più tardi accadde l'inaspettato e --- ora che ripenso a quel momento --- l'inevitabile.

L. mi invitò per la solita conversazione, e stavolta eravamo soli: marito e figlio erano fuori per
una passeggiata. Almeno, questo è quello che disse lei.

Preparò un tè, e parlammo per poco di cose che non ricordo. Poi si interruppe, posò la tazza, mi
guardò intensamente. Era solita fare così quando rifletteva. Quella volta, però, a occupare la sua
mente non erano quesiti esistenziali, ma un essere umano. Io.

Si avvicinò --- eravamo seduti su un divano --- tanto che potei percepire il suo odore di donna
forte, indipendente, quasi selvatica. Mi parve di poter leggere nella sua mente: anche se per pochi
secondi, L. non era più un mistero.

Posò le labbra sulle mie, con leggerezza. Tremai --- non so se per la paura, l'euforia,
l'eccitazione, o tutte e tre le cose insieme --- e lei si fece più insistente, come per
incoraggiarmi a fare quel passo.

Fu allora che mi resi conto di una verità tanto idiota quanto sconcertante: L. amava il sesso. Fino
a quel punto non avevo mai pensato a lei come si fantastica su una donna di eguale grazia. Era
bella, affascinante, sensuale. Forse volubile. Ma era anche la persona più intelligente e sensibile
che avessi mai incontrato, e dunque mi sembrava assurdo che scegliesse di partecipare a un'attività
tanto sporca.

Quel giorno scoprii che non solo L. era la persona più intelligente che avessi mai incontrato, ma
anche la più brava amante. Ci spostammo nella camera da letto, quella dove, immaginai, si concedeva
anche al marito. A quel pensiero mi bloccai per un'istante, affranto e spaventato, ma lei fu svelta
a coinvolgermi nuovamente.

Presto fummo nel letto, e ognuno ansimava nell'orecchio dell'altro, e gemeva esprimendo piacere
intenso.

Raggiungemmo un orgasmo che sconquassò e provò duramente i nostri corpi.

Ricaddi sul materasso, esausto. Lei si sdraiò delicatamente accanto a me, incrociò le braccia sul
cuscino e vi appoggiò la testa. Era così vicina che nei suoi occhi mi potevo specchiare. Aveva
un'espressione serena. Quando qualcosa mi turba, mi capita spesso di pensare a quel volto. Il suo
sorriso la prima volta che facemmo l'amore. Non ostinato, come di chi vuole sguaiatamente dimostrare
al mondo la propria felicità, ma discreto, dolce, caloroso.

«Ti è piaciuto?». Lo sussurrò.

«Molto. Perché l'hai fatto?».

«Cos'hai provato?».

«Ti ho visto. Come sei veramente, intendo. Non ti avevo mai vista così\dots{} concreta».

«Allora ci sono riuscita».

«A fare cosa?».

«Qualche giorno fa mi sono accorta di una cosa: tu hai smesso di dubitare di me. Non c'è più
discussione, il dialogo tra noi due è unidirezionale: se ti dicessi di lanciarti da un dirupo,
probabilmente lo faresti, convinto che io ne sappia molto più di te. Be', non è vero. Io sono qui
per discutere con te, non per insegnarti qualcosa. Se pensi che stia sbagliando, contraddicimi. Non
mi importa che le tue idee siano uguali alle mie. Voglio solo che tu \emph{abbia} delle idee».

«E il sesso cosa c'entra?».

«Davanti al sesso siamo tutti uguali, maestri e allievi. Mostrandoti che puoi avermi, ti ho fatto
capire che tra noi due non c'è quell'oceano che hai sempre immaginato».

«Ti ringrazio».

«Non ringraziarmi: ora pesa su di te la grossa responsabilità di pensare, analizzare, giungere alle
conclusioni. Credevi di saperlo già fare, ma eri imprigionato dai preconcetti, schiavo della
normalità».

«Dunque era solo questo il tuo scopo? È stata la prima e ultima volta che siamo stati a letto
«insieme?».

Sembrò sorpresa per un attimo, come se non si aspettasse l'audacia che ella stessa aveva
incoraggiato un attimo prima.

Infine sorrise, nuovamente rilassata.

«Ovviamente no. Anche a me è piaciuto».

\plainbreak{1}

Le nuvole si ergevano scure, gonfie e minacciose sopra le nostre teste. Io e L. eravamo sdraiati sul
prato, e non dicevamo nulla. Ero stato io a chiederle di incontrarci quella volta, ma non riuscivo a
trovare le parole, e così ella attendeva pazientemente, silenziosa quanto quel pomeriggio d'inizio
inverno. Inspirai profondamente, lasciando che l'umidità penetrasse nei polmoni.

«Questa storia deve finire».

«Quale storia?».

«Sai quale storia. Io e te. Dobbiamo smettere di vederci, o almeno di andare a letto insieme».

Adesso lei sembrava veramente sorpresa e attenta.

«E perché mai? Io sto bene con te».

«Anch'io. Ma ogni volta che facciamo l'amore, continuo a vedere tuo marito e tuo figlio. È una cosa
che non posso ignorare. Immagino che per te sia diverso».

Da ormai due mesi eravamo amanti, e quei pensieri mi avevano angosciato fin dalla prima volta.
Inoltre in quell'arco di tempo avevo incominciato a incontrare anche suo marito, la cui amicizia nei
miei confronti sembrava sincera. Come potevo pugnalarlo alle spalle in quel modo? Come poteva farlo
lei, che parlava tanto di rispetto?

«Tu sei un idiota».

Lo disse sorridendo, come se qualcosa di ovvio mi stesse sfuggendo in quella situazione.

«Scusa?».

«Pensi davvero che mio marito non sappia di noi? Che sarei così crudele? Credevo che avessi più di
«fiducia in me».

«Sa del tempo che abbiamo passato sul \emph{vostro} letto? E gli sta bene?».

«Mio marito ha una parte del mio cuore. Non il mio corpo».

«Ma siete sposati!» protestai.

«No, non lo siamo. Mi hai mai visto portare la fede? Si è guadagnato il titolo di ``marito'' perché
mi è accanto da dieci anni, e insieme abbiamo affrontato molte, moltissime situazioni, compreso il
nostro ingresso qui».

Quella era forse la stranezza più grande con cui mi fossi mai confrontato, e proprio per questo era
anche la prova più importante. E io l'avevo fallita. Mi vergognai della mia stupidità.

«Se qualcuno di là sentisse questa storia\dots{}» sussurrai.

«\dots{}penserebbe che siamo malati». Si mise a sedere sull'erba, e io feci altrettanto. «Ma da
quando l'amore ha smesso di essere piacere di stare accanto a una persona ed è diventato desiderio
di possederla? È assurdo che le coppie si sposino! La stessa volontà di sposarsi implica che non ci
sia fiducia, e dunque amore!».

«Sposarsi significa giurare di amarsi davanti a Dio».

«Già sai come la penso su Dio: non esiste. O meglio, non esiste nella forma in cui lo intendono la
maggior parte delle persone. E anche se esitesse, non vorrei che fosse lui a governare i miei
sentimenti. Amare significa dare senza aspettarsi nulla in cambio: si sposa chi ha paura di essere
abbandonato dalla propria metà. Ma se mio marito non mi volesse più vicino a sé, io non vorrei che
soffrisse e fosse costretto a rimanere con me, proprio perché continuerei ad amarlo!».

Era qualcosa a cui avevo sempre pensato anch'io, ma non avevo mai avuto il coraggio di affermarlo
pubblicamente, anche perché non ero sicuro del mio stesso ragionamento. Tutti si sposavano, dunque
quell'azione doveva avere un senso! Possibile che nel XXI secolo si continuasse a praticare
un'attività così palesemente sbagliata?

«È per questo che esistono i divorzi» tentai di protestare, perché non volevo lasciarmi convincere
«troppo facilmente.

Lei esplose in una fragorosa, sana risata.

«Peggio ancora! Non vedi l'ipocrisia in tutto questo? Ci si sposa giurando amore eterno, ma poi si
ricorre alla legge per annullare quel giuramento! In questo modo un rituale già privo di significato
perde ulteriormente valore!».

«Dunque tu cosa suggerisci? Che tutti abbiano un rapporto aperto come il tuo?».

«No, mi rendo conto che non a tutti può andare bene. Ma in tutti i rapporti dovrebbe essere elemento
fondamentale la sincerità. Il matrimonio, per sua stessa natura, non crea un ambiente favorevole al
dialogo».

Un'altra lezione. Perché con quella donna ogni conversazione si trasformava in un dibattito sulle
cose della vita?

«Dopo quello che ti ho detto, vuoi che continuiamo a vederci? Oppure urta il tuo orgoglio di uomo?».

Lo disse con tono provocatorio e ironico. Io non risposi.

«Bene,» proseguì «vorrà dire che tornerò tra le braccia del mio comprensivo marito». E si alzò,
andandosene.

Rimasi a riflettere per qualche minuto. L. era la migliore amica e compagna che potessi desiderare.
Sarebbe stato terribilmente stupido lasciarla andare così. Non importava che amasse anche un altro
uomo: solo la consapevolezza di essere nel suo cuore era immensamente gratificante.

Amare senza chiedere nulla.

Mi alzai e inseguii quella figura ormai piccola. Quando la raggiunsi, le cinsi la vita. Lei mi baciò
su una guancia e sorrise.

«Lo sapevo» disse, e in quel momento le prime goccie di pioggia caddero sulle nostre teste.

Corremmo verso la sua casa. La nostra casa.
