\chapter{Oltre il cancello}
\label{ch:oltre-il-cancello}

Motori. Furono la prima cosa che udii agli albori della mia nuova vita: motori che si spegnevano.
Mentre aprivo gli occhi, lentamente perché la luce passava da ogni spiraglio e sembrava farli
sanguinare, giunsero alle mie orecchie anche alcune voci.

«Si sta svegliando» disse qualcuno.

«Giusto in tempo» ridacchiò un altro.

Non avevo idea di cosa mi fosse successo e non riuscivo a pensare né a ricordare nulla per via del
mal di testa; mi sembrava di essere un neonato espulso dal ventre materno che si affaccia sul mondo
con curiosità e timore.

«Chi siete? Dove sono?».

Un immenso afroamericano dall'aria minacciosa mi si avvicinò.

«Quanto alla prima domanda,» rispose con voce profonda «non hai bisogno né sei tenuto a saperlo. Per
ciò che riguarda la seconda invece la risposta è ``su un aereo'', ma non credo che ti soddisfi
molto. In questo momento non sei in grado di ragionare con lucidità: tra qualche minuto ti sarà
spiegato tutto».

Mi accorsi di trovarmi su quello che, a giudicare dagli interni, era un jet privato, appena
atterrato nel mezzo del nulla. Ovunque fossimo pioveva anche lì: potevo sentire il ticchettio
dell'acqua contro il metallo. Il portellone era aperto e dalla scaletta vidi scendere pilota e
copilota, entrambi elegantemente vestiti, dopo avermi lanciato un'occhiata furtiva.

«Lei dov'è?».

Stavolta fu un altro a rispondermi; sedeva di fronte a me e teneva le gambe accavallate, una mano a
sostenere il mento, un leggero sorriso dovuto probabilmente al mio stato di evidente smarrimento.

«Di chi parli?».

Non ebbi l'occasione di rispondere perché in quel momento L. si materializzò alla mia sinistra; era
in piedi e mi posò affettuosamente una mano sulla spalla. I due uomini la guardavano con rispetto.

«Bentornato!» rise. «Vieni, dobbiamo andare; senza fretta, ormai».

Slacciai la cintura e mi alzai lentamente, perché avevo paura che le gambe cedessero. Non c'era
traccia dell'uomo che ci aveva condotto fino all'aereo; pensai che non fosse salito con noi.

«Dove siamo?» chiesi nuovamente.

«Pazienta ancora qualche attimo: allora potrai vedere con i tuoi occhi».

Una volta fuori dall'aereo, mentre scendevo lentamente gli scalini sostenuto da L., qualcuno aprì un
ombrello sulle nostre teste. In fondo ci attendevano due fuoristrada che vennero occupati dagli
uomini e in mezzo una terza auto nella quale stavamo noi due.

Il viaggio durò più di mezz'ora, eppure non parlammo granché, complice anche la mia tremenda
stanchezza. Le chiesi cosa mi avesse dato per farmi stare così, anche se intuivo già la risposta.

«Un sonnifero».

Sospirai.

«Immaginavo».

«Non è nulla di personale, solo non volevamo che vedessi la strada».

«La strada per arrivare fin qui? Come avrei potuto?».

«La strada per arrivare all'aeroporto. E devi ringraziare me se ora non te ne abbiamo dato un
altro\dots{} I miei ``colleghi'' non sono del tutto d'accordo con questa scelta, ma io mi fido di
te».

«Ma perché tanta segretezza?».

«Lo scoprirai stando con noi. Per ora sappi che quello che cerchiamo di introdurre è un grande
cambiamento, e non a tutti piacciono i cambiamenti. I nostri nemici sono potenti, politicamente
parlando, e potrebbero distruggerci se non fossimo così prudenti».

Fu la fine della nostra conversazione, perché avevo bisogno di silenzio per pensare ai guai in cui
mi ero cacciato.

Il paesaggio non era molto vario ma piacevole: eravamo entrati in un bosco e seguivamo un perfetto
sentiero che lo attraversava da parte a parte. Gli alberi finivano ma la strada proseguiva per
svariati chilometri curvando dolcemente.

L. guidò ancora per qualche minuto, finché giungemmo a un cancello.

La vista mi incutè del timore: sembrava che pochi uscissero di lì una volta entrati, e di certo lo
facevano con l'intento o l'obbligo di tornare presto. Il cancello si aprì quando eravamo a venti
metri; evidentemente ci stavano aspettando.

Al di là facevano ombra sulla strada altri alberi, alti e dalle foglie larghe e verdi; la pioggia
aveva portato fino a noi l'odore della loro corteccia mischiato a quello della terra, così piacevole
e rassicurante. Quando l'ultima auto passò, il cancello si chiuse alle nostre spalle. Ancora non
sapevo se l'avrei mai rivisto, o se avrei desiderato di farlo.

Ma ciò che ancora mi aspettava era qualcosa a cui, nonostante tutto quello che mi era successo in
quegli ultimi giorni, non ero minimamente preparato.

Oltre il cancello vidi case, parchi, fontane, ma soprattutto persone: ai margini della strada ci
attendeva una folla sorridente che salutava con la mano: uomini, donne, bambini\dots{} talvolta
intere famiglie erano lì e ci fissavano con degli sguardi pieni di benevolenza e amore.

Oltre il cancello c'era una piccola città, colorata e meravigliosa.

L. entrò nel vialetto di una delle case e si fermò lì, mentre le altre due auto proseguirono.
Scendemmo ed ella frugò nella borsa per poi tirarne fuori un mazzo di chiavi.

«È la tua nuova casa,» disse «spero che tutto sia di tuo gradimento. Se hai qualche problema puoi
chiedere in giro: le persone qui non aspettano altro che darti una mano. Io vado a cambiarmi; sarò
qui tra due ore circa e ti accompagnerò a casa mia dove cenerai per stasera. Ti consiglio di fare
una doccia: hai un aspetto orrendo».

Detto ciò tornò in macchina.

«Ah, dimenticavo» aggiunse sorridendo prima di andare via. «Benvenuto».

\plainbreak{1}

Seguii il saggio consiglio ricevuto e feci una doccia, quindi indossai una delle camicie
nell'armadio che, come tutti i nuovi abiti, non solo era della giusta taglia ma anche di mio
gradimento, quasi fosse stata scelta da qualcuno che conosceva perfettamente i miei gusti.

Quell'alternanza di momenti di felicità e tremenda inquietudine continuava a scombussolarmi. Sì, ero
contento di essere finalmente accanto a L. e lontano da quel mondo che mi rendeva tanto apatico, ma
avvertivo anche una presenza oscura, qualcosa di inspiegabile per una persona razionale come me.
Forse era il mio stesso animo a mettermi in guardia. Mi promisi di non pensarci più fino
all'indomani, quando mi sarei ripreso dal viaggio e avrei potuto ragionare con lucidità.

Come mi aspettavo L. fu lì esattamente due ore dopo; venne a piedi e a piedi raggiungemmo la sua
abitazione, che non distava molto. Era notevolmente più grande rispetto alle altre.

«Vedo che anche qui ci sono i raccomandati» scherzai.

«È una questione di esigenze» sorrise lei. «Se hai bisogno di più spazio non devi che chiedere».

La cena fu lunga: posi tutte le domande che in macchina ero stato troppo stanco per formulare. Così
molte cose mi furono più chiare e fui in parte rasserenato, giacché mi convinsi di essere in ottime
mani.

«Dunque,» iniziò L. «ti starai chiedendo dove sei capitato».

«Proprio così».

«È difficile descriverci. Alcuni dicono che siamo un'organizzazione, altri una setta; tu non dovrai
lasciarti influenzare. Noi ci consideriamo prima di tutto e soprattutto una enorme famiglia, una
piccola società. Qui, al di qua del cancello troviamo tutto ciò che ci serve per condurre una vita
dignitosa, rispettosa e soddisfacente. La maggior parte di noi è composta da artisti, come te e me,
mentre alcuni si occupano di altre faccende, più terrene».

«Ma come fate a vivere qui? Da dove viene il cibo?».

«Abbiamo le nostre coltivazioni e i nostri allevamenti. La Società ha le sue risorse».

«Come l'hai chiamata? La Società?»

Era seccata.

«L'ho fatto? Non avrei dovuto: non siamo soliti darci un nome, anche se spesso riferendosi a noi
parlano di Società Alternativa».

«Quindi è così che dovrò dire a qualcuno quando mi presento? ``Piacere, faccio parte della Società
«Alternativa''?»

L. mi guardò, terribilmente seria.

«Tu non dovrai mai dire a nessuno di cosa fai parte. Mai».

\plainbreak{1}

Quella notte mi fu impossibile mantenere la promessa fatta a me stesso. ``Non pensare, non
riflettere, non farti domande'' continuavo a ripetermi, ma le parole di L. insistevano nel
risuonarmi gravi e minacciose in testa. ``Non dire mai a nessuno di cosa fai parte''. Perché? Cosa
c'era di tanto terribile o segreto nella Società Alternativa?

Mi addormentai con un peso sullo stomaco che mi rendeva complicato deglutire e anche solo respirare.
Mi svegliai allo stesso modo, arrabbiato con me stesso per il mio essere incontentabile.

``Probabilmente sono impazzito del tutto. Di cosa ho bisogno per essere in pace? Possibile che
neanche qui sia felice?''

Io cercavo, cercavo di non pensare, non elaborare, ma non potevo! Per quanto sarei riuscito a vivere
in quella condizione?

L. mi raggiunse per la colazione.

«Ti presento gli altri» disse. E poi guardandomi meglio aggiunse preoccupata: «Ieri ti ho detto che
avevi un aspetto orribile, ma mi rendo conto che era nulla rispetto a oggi. Che ti è successo?»

Temetti che avesse capito. Non volevo mentire perché mi avrebbe scoperto subito, così optai per una
mezza verità.

«Sì, ho dormito solo poche ore. Ero nervoso».

«Mi sembra comprensibile. Hai fatto una scelta coraggiosa; solo un idiota non avrebbe ripensamenti.
Tra poco ti sentirai meglio, vedrai».

L., mia ancora di salvezza! Mia luce nell'oscurità! Per quanto stessi male riusciva sempre ad
allietare il mio spirito ferito, perché mi conosceva meglio di me stesso.

Mi condusse a una sorta di capannone all'interno del quale era già presente una moltitudine di
persone, alcune sedute ai tavolini circolari sparsi ovunque, altre in fila per servirsi. Noi ci
dirigemmo verso quest'ultimo gruppo, e una volta fatto -- ossia, una volta prese le tre tazze di
caffè nero necessarie e indispensabili per svegliarmi -- ci sedemmo con due giovani coppie.

L. me le presentò ma in questo momento ho solo un vago ricordo di loro, perché mi colpirono, più dei
loro volti e dei loro nomi, i loro atteggiamenti: nessuno sembrava inquieto; erano anzi così
rilassati e felici, così appagati\dots{} Come se nessun pensiero negativo potesse scalfire la loro
perfetta esistenza. Avevano raggiunto la tranquillità che io cercavo da tempo.

Mi fecero sentire a casa, e ciò mi rincuorò immensamente.

\plainbreak{1}

Iniziò così la mia nuova vita. Ebbi l'occasione di conoscere moltissime persone che facevano parte
della Società, non perché fossi io a cercarle --- nonostante gli insegnamenti di L., infatti, sono
sempre rimasto piuttosto timido --- ma perché erano loro a fermare me non appena mi vedevano
passare. Chiedevano come mi trovavo, se mi serviva qualcosa. Quando rispondevo di no, iniziava una
piacevole conversazione sulle cose importanti di questo mondo, e terminava quasi sempre con
l'espressione ``La mia porta è sempre aperta'', pronunciata dall'una o dall'altra parte.

Tramite gli altri membri cercai di apprendere qualcosa sul conto di L., ma riuscii a ottenere
pochissime informazioni e spesso discordanti: alcuni dicevano che fosse una cantante di successo,
altri che avesse acquistato fama solo all'interno della Società, altri ancora che prima lavorasse
nel mondo dell'informatica\dots{}

Quando gli impegni glielo permettevano --- ora che era tornata nel suo mondo, infatti, era una donna
molto occupata --- L. passava a trovarmi. Ciò avveniva solitamente al tramonto, per entrambi grande
fonte d'ispirazione. Per me era come rivivere ogni volta la nostra esperienza sulla spiaggia:
passeggiavamo chiacchierando di argomenti senza senso e argomenti importanti come se avessero lo
stesso peso, e guardavamo il sole scendere tra due montagne.

Le montagne di cui non ho mai scoperto i nomi.

Glieli chiesi più volte, ma lei mi ignorò sempre, finché un giorno mi disse: «Ha davvero importanza?
In questo momento tutto si crea e tutto si distrugge, e un giorno queste montagne non esisteranno
più. Allora l'unica cosa che varrà la pena di ricordare saranno i momenti che hai passato
guardandole insieme a me e traendo ispirazione da questo spettacolo».

In realtà non mi era dato di conoscere i loro nomi perché avrei potuto intuire dove mi trovavo, ma
L. era riuscita a dare un aspetto filosofico alla questione, e tanto mi bastava.

A quel periodo appartengono le mie opere più emozionali e meno oggettive. Qualche giorno fa le ho
rilette, e mi sono reso conto che, benché fossero cariche di sensazioni meravigliose, mancavano
quasi del tutto di coerenza. Questo accadeva perché accanto a L. la coerenza era inutile, elemento
superfluo e del tutto irrilevante ai fini dell'opera.

«L'arte» diceva sempre lei «deve trasmettere emozioni, non idee. Le emozioni sono uniche e
irripetibili, mentre per riflettere su un dato argomento l'essere umano ha decine di occasioni che
butta via. Perché dovrebbe essere l'artista a risolvere questo problema? Non descrivere come ti
senti, \emph{mostra} come ti senti».

Io obiettavo che nel canto è semplice; quando si ha a che fare con un racconto, però, è spesso
impossibile mostrare senza descrivere.

«Non è così» replicava sorridendo. «Con il tempo ci riuscirai, ne sono convinta. Perché non vieni al
mio prossimo concerto? Lo vedrai con i tuoi occhi».

Accettai con entusiasmo. Avevo udito L. cantare una sola volta, ma era stata un'esperienza
meravigliosa, ed ero ansioso di ripeterla.

Dopotutto, perché accadesse, non mi sarei neanche dovuto spostare molto.
