\chapter{}
\label{ch:6}

Un pomeriggio L. venne da me. Era adirata e confusa.

«L'hanno ucciso,» ripeteva «mio Dio, l'hanno ucciso».

«Calmati. Chi hanno ucciso?».

«Mio marito. L'hanno ucciso. Nadia dice che è stato lui ma non è vero! Lei ha organizzato tutto! E
io lo amavo. Lo amo ancora! L'hanno ucciso perché voleva parlare, e ora uccideranno mio figlio per
paura di quello che potrebbe dire!».

Pensai che fosse impazzita.

«Sei sconvolta. Non pensi che dovresti dormire e tornare sull'argomento quando sarai più lucida?
Voglio dire, è quasi un anno che sono qui, e mi sembra che tutti i membri della Società si ispirino
a principi di fratellanza e amore».

«Non è così. Non hai idea di chi sia dietro a quest'organizzazione. E per tutto questo tempo io li
ho aiutati. L'ho ucciso io!».

Non avevo mai visto nessuno piangere in quel modo. Accucciata sul divano, con la testa appoggiata al
mio petto, L. iniziò a singhiozzare disperatamente. Io non parlavo.

Dopo più di due ore terminarono le lacrime e le andò via la voce; lei però continuò a tremare.
Quando la vidi così desiderai la morte: non riuscivo a smettere di pensare che ero la causa di
tutto.

Le accarezzavo i capelli e sussurravo parole dolci, ma nulla sembrava poter placare la sua
sofferenza. Impiegò un'altra ora per addormentarsi.

Tra i singulti riuscii a dirmi che, secondo Nadia, il marito si era ucciso due settimane dopo
essersene andato, lasciando un biglietto dove le attribuiva la colpa di quel gesto. Due settimane in
cui L. non ne aveva mai parlato, come se lo avesse rimosso dalla propria memoria.

Il figlio era stato affidato ai nonni, giacché la madre era irrintracciabile.

\plainbreak{1}

Quando la sera ci colse eravamo ancora in quella posizione. Ogni muscolo del mio corpo era
addormentato, ma non osavo muovermi per non svegliarla.

Riflettevo sulle sue parole. Chi c'era dietro la Società Alternativa? Una setta? La Massoneria? Gli
Illuminati? Avevo sempre sentito raccontare molte cose sul loro conto, alcune vere, altre totali
idiozie.

E che cos'avevo firmato prima di entrare nella Società? Anch'io mi ero impegnato a restare lì per il
resto dei miei giorni? Ero prigioniero di quel luogo come tutti gli altri?

E ancora, se la Società aveva in realtà altri scopi, quali erano? Certo non il lucro: mantenere
tante persone non era che un costo. Un costo enorme, e dunque il burattinaio, chiunque fosse, doveva
essere molto ricco.

Pensai a tutte le stranezze di quel posto, alla realtà in cui vivevano i suoi abitanti, convinti che
il mondo esterno non li riguardasse affatto. Solo allora mi resi conto di tutte le allusioni che
avevo ignorato, gli atteggiamenti che avevo finto di non vedere. Pensai alla segretezza che ci era
richiesta: non ci era dato sapere dove ci trovassimo precisamente, né potevamo parlare con qualcuno
della nostra affiliazione.

L. dormì poco; quando si svegliò era calma.

Continuava a piovere. Ormai la dolce primavera che dominava al mio arrivo aveva fatto spazio per un
malinconico inverno, che spegneva ogni entusiasmo e faceva venir voglia di passare le proprie
giornate nel letto.

«Mi è sempre piaciuta la pioggia» disse, con la bellissima voce ormai ridotta a un sussurro. «Quando
fuori c'è un temporale e io sono dentro casa, al caldo, mi sento protetta e immortale. Tuttavia sono
anche consapevole che, se uscissi, mi bagnerei fino alle ossa».

«Dove vuoi arrivare?».

«Finora io ho vissuto dentro casa mentre fuori imperversava la tempesta. Avevo paura di bagnarmi,
perché mi hanno insegnato a evitare l'acqua. Ma non mi sono accorta che l'abitazione nella quale mi
rifugiavo era divorata da un incendio che, centimetro dopo centimetro, avrebbe distrutto tutto.
Temevo un raffreddore, ma stavo andando incontro all'incenerimento».

«Cos'hai intenzione di fare?».

«Devo andare via di qui».

«Vengo con te».

La frase venne fuori istintiva. I suoi occhi si illuminarono.

«Davvero? Lo faresti?».

Esitai --- non sarebbe stato semplice riadattarsi --- ma il dubbio fu subito vinto.

«Questo posto non è quello che sembra. E io ti accompagnerei ovunque».

Mi abbracciò.

«Dobbiamo fare in fretta,» disse poi «il cerchio intorno a mio figlio si stringe sempre di più. Se
gli succedesse qualcosa a causa delle scelte che ho fatto non potrei mai perdonarmelo».

Così L., che mi aveva introdotto nella Società Alternativa, ora chiedeva il mio aiuto per uscirne.

O forse era lei ad aiutare me.

\plainbreak{1}

Il piano era semplice quanto ambizioso: uscire dal cancello principale, chiedendo il permesso, per
non tornare mai più. L. avrebbe addotto una motivazione qualunque, per esempio la ricerca di nuovi
talenti. Quanto a me, poteva dire che voleva insegnarmi a vedere la ricchezza nell'animo delle
persone, in modo che potessi seguire le sue orme. Pensava che nessuno avrebbe protestato, dato che
era un membro anziano e aveva quel diritto.

Si sbagliava. L'uscita le venne negata dall'uomo cui si era rivolta.

«È assurdo! Sono un membro del Consiglio da diversi anni! Dove vuoi che vada?».

«Dato che ne fai parte sai anche che il Consiglio può, per una valida ragione, limitare le libertà
di alcuni membri per proteggere gli altri».

«E quale sarebbe questa ragione?».

«Temiamo la fuga di informazioni».

«Se la temete vuol dire che avete qualcosa da nascondere. Siete stati voi, vero?».

Quello allargò le braccia. «A fare cosa?».

«A uccidere mio marito! Abbiate almeno il coraggio di ammetterlo, ipocriti vigliacchi!».

La sua voce si incrinò. Le strinsi la mano.

«Ecco di cosa parlavo» disse, colmo di finta compassione. «Sei sconvolta per la sua morte; potresti
dire qualunque cosa».

Lei scosse la testa, amareggiata.

«Credevo che avresti capito. Ci conosciamo da molto tempo».

«Sai che la decisione non spetta a me. Proprio perché siamo vecchi amici, non rendere tutto più
«penoso».

«Va bene. Tanto per sapere, tu hai votato a favore o contro il mio imprigionamento?».

«Sai che il voto è segreto».

Ma stavamo già andando via quando lo disse.

Senza neanche voltarsi L. rispose: «Come immaginavo».

\plainbreak{1}

«Non sapevo che ci fosse un Consiglio» dissi mentre tornavamo a casa sua.

Camminava in fretta. Per il bene di suo figlio era passata dalla disperazione alla risolutezza.

«Il Consiglio è l'organo che prende tutte le decisioni. È composto da dodici membri, sei uomini e
«sei donne».

«E quella Nadia che ho visto la sera in cui tuo marito voleva andarsene? Anche lei ne fa parte?».

«Sì. Ed è mia sorella».

Ecco cosa c'era di tanto familiare in lei! Aveva gli stessi occhi penetranti, solo neri come la
notte invece del marrone autunnale di L.

«Mi sembra una persona crudele».

«Ricordi il fuoco di cui parlavo? Lei si è scottata».

Apprezzai la metafora.

«Adesso cosa facciamo?».

«Evitiamo di fare sciocchezze. Io ho ancora qualche asso da giocare: non tutti mi hanno voltato le
spalle. Almeno spero».
