\documentclass[a4paper,oneside,11pt]{memoir}

\usepackage[utf8]{inputenc}
\usepackage[italian]{babel}
\usepackage[T1]{fontenc}
\usepackage{XCharter}
\usepackage{import}
\usepackage{microtype}
\DisemulatePackage{setspace}
\usepackage{setspace}

\chapterstyle{dash}
\pagestyle{plain}
\frenchspacing
\raggedbottom
\sloppy
\onehalfspacing

\title{La Società Alternativa}
\author{Alessandro Desantis}
\date{}

\begin{document}

    \begin{titlingpage}
        \maketitle
    \end{titlingpage}

    \import{../_utils/}{dedication.tex}
    \begin{dedicationpage}
        \begin{verse}
            \itshape{
            ``Tigre! Tigre! Divampante fulgore\\
            Nelle foreste della notte,\\
            Quale fu l'immortale mano o l'occhio\\
            Ch'ebbe la forza di formare la tua agghiacciante simmetria?

            In quali abissi o in quali cieli\\
            Accese il fuoco dei tuoi occhi?\\
            Sopra quali ali osa slanciarsi?\\
            E quale mano afferra il fuoco?

            Quali spalle, quale arte\\
            Poté torcerti i tendini del cuore?\\
            E quando il tuo cuore ebbe il primo palpito,\\
            Quale tremenda mano? Quale tremendo piede?

            Quale mazza e quale catena?\\
            Il tuo cervello fu in quale fornace?\\
            E quale incudine?\\
            Quale morsa robusta osò serrarne i terrori funesti?

            Mentre gli astri perdevano le lance tirandole alla terra\\
            e il paradiso riempivano di pianti?\\
            Fu nel sorriso che ebbe osservando compiuto il suo lavoro,\\
            Chi l'Agnello creò, creò anche te?

            Tigre! Tigre! Divampante fulgore\\
            Nelle foreste della notte,\\
            Quale mano, quale immortale spia\\
            Osò formare la tua agghiacciante simmetria?''
            \/}
        \end{verse}

        \begin{flushright}
            --- W. Blake
        \end{flushright}
    \end{dedicationpage}

    \pagenumbering{arabic}

    \import{chapters/}{1-il-rito.tex}
    \import{chapters/}{2-partenza.tex}
    \import{chapters/}{3-oltre-il-cancello.tex}
    \import{chapters/}{4-dio-e-amore.tex}
    \import{chapters/}{5-cambiamenti.tex}
    \import{chapters/}{6-liberta-vigilata.tex}

    \backmatter

    \import{chapters/}{8-epilogo.tex}

\end{document}
