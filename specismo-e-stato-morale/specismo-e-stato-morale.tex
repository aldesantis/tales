\documentclass[a4paper,11pt,oneside,article]{memoir}

\usepackage[italian]{babel}
\usepackage[T1]{fontenc}
\usepackage[utf8]{inputenc}
\usepackage{multicol}

\frenchspacing

\title{Specismo e stato morale}
\author{Alessandro Desantis}
\date{}

\begin{document}

\begin{titlingpage}
\maketitle

\begin{abstract}

\noindent Alcuni filosofi e associazioni animaliste sostengono la necessità di
rinunciare alla discriminazione tra specie ed estendere anche agli animali non
umani alcuni diritti fondamentali come quello alla vita e alla libertà. In
questo testo cerco di individuare un criterio oggettivo, razionale e replicabile
con cui si possa valutare la sofferenza di un individuo in seguito al proprio
sfruttamento e dimostro come tale criterio discrimini tra animali umani e non
umani.

\end{abstract}
\end{titlingpage}

\chapter{Introduzione}

Gli esseri umani hanno raggiunto un tale livello di sviluppo da potersi
permettere il lusso di preoccuparsi e impegnarsi attivamente per il benessere
non solo proprio, ma anche delle altre specie e dell'ecosistema. Ne è un'ovvia
testimonianza la nascita di organizzazioni ambientaliste e animaliste che da
decenni ormai si battono per un giusto trattamento nei confronti della Terra e
degli animali non umani.

Sebbene in alcuni casi questa battaglia possa essere considerata giusta (è
nell'interesse comune, infatti, porre un freno al folle sfruttamento delle
risorse che stiamo praticando dalla rivoluzione industriale), alcune di queste
associazioni giungono talvolta a conclusioni piuttosto discutibili, arrivando ad
affermare l'uguaglianza di tutti gli animali, umani e non, e pretendendo che
l'uomo inizi a considerarsi alla pari di tutte le altre specie. Mi rendo conto
di non poter considerare, in un solo articolo, la varietà di posizioni e
opinioni riscontrabili nel mondo animalista, dunque mi limiterò a discutere le
idee estreme, i cui sostenitori chiedono che l'uomo cessi subito lo sfruttamento
di ogni specie indipendentemente dallo scopo per cui viene impiegata
(alimentazione, ricerca biomedica etc.).

A mio parere, si tratta di una posizione fondamentalmente incompatibile con il
proseguimento della specie umana, oltre a rappresentare una palese resistenza
all'istinto naturale di conservazione. Inoltre, è solo per via del progresso
ottenuto grazie allo sfruttamento delle specie non umane che possiamo
permetterci, in primo luogo, il lusso di considerare il rifiuto del suddetto
sfruttamento. Dunque, pur essendo auspicabile una limitazione del nostro impatto
sull'ecosistema, pensare di poter annullare quest'impatto senza trotterellare
mano nella mano verso l'estinzione, è quantomeno utopico, se non addirittura
ipocrita e ingenuo.

\chapter{Cos'è lo specismo?}

Lo specismo è, per definizione, la tendenza a preferire i membri di una
determinata specie esclusivamente sulla base della loro appartenenza a quella
specie. Nel caso degli esseri umani, lo specismo si presenta comunemente in due
forme:

\begin{itemize}

\item la discriminazione tra diverse specie animali non umane (nella cultura
occidentale, per esempio, il consumo di carne bovino, ovina e suina è comune, ma
il consumo di carne canina è considerato un tabù);

\item la discriminazione tra specie umane e non umane (per esempio, nella
ricerca biomedica vengono impiegati continuamente animali non umani, ma
impiegare esseri umani sarebbe considerato crudele e immorale).

\end{itemize}

In quest'articolo ho intenzione di prendere in considerazione la seconda forma
di specismo, ossia la discriminazione tra animali umani e non umani.

\chapter{Lo specismo in natura}

In natura è assolutamente normale favorire la propria specie rispetto alle
altre. È solo una delle tante forme sotto cui si manifesta l'istinto di
conservazione della specie. I leoni non cacciano altri leoni in primo luogo
perché sarebbe poco pratico, e in secondo luogo perché non sarebbe una strategia
di sopravvivenza efficace. Attaccano i propri simili solo per garantire la
preservazione del proprio bagaglio genetico, ossia quando l'istinto di
sopravvivenza individuale entra in conflitto con quello di conservazione della
specie (è una prassi comune per il maschio uccidere i cuccioli della femmina in
modo che questa torni in calore). L'istinto suggerisce dunque che anche gli
animali non umani pratichino, in una forma meno raffinata della nostra (in
natura non esiste, del resto, il concetto di diritto) lo specismo.

Inoltre, trattare la natura e l'uomo come se fossero due entità separate non ha
alcun senso. Non c'è motivo per cui non dovremmo considerarci parte della natura
insieme a tutte le altre specie. Dunque, se anche quella umana fosse l'unica
specie a praticare lo specismo, questo sarebbe comunque naturale. Ma supponiamo
che i comportamenti appartenenti esclusivamente all'uomo siano contrari
all'ordine delle cose o ``contronatura'': dovremmo allora smettere di vestirci?
di guidare automobili? di praticare la medicina? Dovremmo, insomma, rinunciare a
millenni di progresso solo perché le altre specie non hanno progredito? La
risposta è chiaramente no. Possiamo quindi dire con relativa tranquillità che la
tendenza delle altre specie a certe pratiche è irrilevante nella determinazione
della correttezza delle stesse.

Tuttavia, il fatto che lo specismo sia naturale non implica che sia anche
moralmente accettabile. Questo perché, a un certo punto della nostra storia
evolutiva, abbiamo deciso di creare un contratto sociale, ovvero di
autolimitarci, rinunciando ad alcuni dei nostri istinti animaleschi, al fine di
garantire a tutti gli umani una migliore convivenza. L'infanticidio è un crimine
ed è moralmente condannabile; eppure, come ho scritto sopra, i leoni e molte
altre specie lo praticano quotidianamente.

La ``naturalezza'' di una pratica non è, per questo, un fattore determinante
nella scelta di adottare o meno tale pratica. Questo non significa, di nuovo,
che non siamo parte della natura; solo che in alcuni aspetti ci differenziamo da
tutte le altre specie. La nostra stessa coscienza è naturale, pur essendo
un'esclusiva umana.

\chapter{Il giusto criterio}

Ci sono stati diversi tentativi nell'individuazione di un criterio che sia
universalmente e coerentemente applicabile a tutti gli individui e che ci
permetta di determinare il valore di un individuo e, di conseguenza, la
possibilità di sfruttarlo per il bene comune.

Jeremy Bentham, il padre dell'utilitarismo, riteneva\footnotemark[1] che la sola
capacità di soffrire fosse sufficiente per condannare lo sfruttamento di altre
specie:

\begin{quotation}

[…] un cavallo o un cane adulti sono senza paragone animali più razionali, e più
comunicativi, di un bambino di un giorno, o di una settimana, o persino di un
mese. Ma anche ammesso che fosse altrimenti, cosa importerebbe? La domanda non è
Possono ragionare?, né Possono parlare?, ma Possono soffrire?

\end{quotation}

In risposta, si sente spesso dire che gli esseri umani possiedano un proprio
valore e dignità intrinsechi. Tuttavia, senza un'ulteriore elaborazione, si
tratta di un'affermazione dogmatica e priva di fondamento, creata a posteriori
per giustificare una superiorità provata inconsciamente. Perché solo gli esseri
umani dovrebbero avere valore intrinseco? Cosa li distingue, cosa li rende unici
tra tutte le specie animali?

Sebbene tutti i vertebrati e alcuni invertebrati siano in grado di provare
dolore, non tutti gli individui soffrono allo stesso modo. Gli esseri umani (e,
in maniera limitata, alcuni altri grandi primati) sono dotati di ciò che viene
definita ``autocoscienza''; sono cioè consapevoli dell'esistenza di un io e di
come quest'io si relazioni con l'ambiente circostante e con altri individui.

Per questo motivo, negli individui dotati di autocoscienza il dolore non è
semplicemente un dolore fisico ma anche emozionale. Per esempio, quando un cane
viene investito, lo stress e la paura che prova non sono che una risposta
dell'istinto di sopravvivenza al pericolo di morire. Un uomo investito invece
prova, oltre alla naturale paura della morte, rimpianto e tristezza all'idea di
non essere riuscito a realizzarsi appieno, di non poter più rivedere le persone
care, di non avere più tempo per portare il proprio contributo nel mondo e i
propri piani a compimento. Pensare di poter paragonare queste due dimensioni del
dolore, l'una semplice risposta istintiva, l'altra frutto di riflessioni ben più
profonde, significa abbracciare una posizione irrazionale e antiscientifica.

Le affermazioni di Bentham, però, non sono del tutto prive di fondamento. Così
come noi, infatti, gli animali fuggono dal dolore fisico. È quindi un nostro
preciso dovere ridurre al minimo indispensabile la sofferenza a cui sono
sottoposti. Usare un topo per l'avanzamento della ricerca biomedica è lecito
perché ha come risultato quello di ridurre la sofferenza umana che, come abbiamo
visto, ha più valore rispetto a quella delle altre specie. Torturare un topo
(ovvero infliggere sofferenza senza alcuno scopo) è invece un'azione immorale
perché non beneficia alcun individuo al di fuori del torturatore.

Questa linea di ragionamento, però, solleva un'ulteriore questione: se il dolore
fisico è l'unica condizione da evitare negli animali non umani, è moralmente
accettabile ucciderne uno senza causare sofferenza? Di certo una mucca non ha
piani per il futuro, né dei cari che soffriranno per la sua mancanza o di cui
soffrirà la mancanza, né l'idea di un aldilà. Che viva o che muoia in maniera
indolore, indipendentemente dall'esistenza di uno scopo per la sua morte,
quindi, dovrebbe essere del tutto indifferente. Eppure guarderemmo con orrore, e
in alcuni casi arresteremmo, qualcuno che uccidesse animali, seppur usando ogni
premura necessaria per evitarne la sofferenza, per semplice passatempo.

In tutta onestà, è un dilemma che io stesso non sono certo di aver risolto
completamente. Si potrebbe obiettare, come nel caso del torturatore di topi, che
uccidendo animali per proprio piacere personale si causi un danno inutile
all'ecosistema, i cui interessi sono più a cuore, alla comunità, rispetto a
quelli del singolo individuo che trae piacere dall'uccisione.

\section{L'argomento dei casi marginali}

Prendiamo un altro esempio, uno che viene citato spesso dagli animalisti per
dimostrare la fallacia della logica specista e presentato per la prima volta da
Peter Singer\footnotemark[2].

Supponiamo che un uomo nasca con un deficit cognitivo talmente esteso da poter
essere paragonato, sul piano dell'intelletto, a un animale non umano. Cosa ci
impedirebbe di trattarlo proprio come tutte le altre specie, ritenendoci dunque
in diritto di sfruttarlo per i nostri scopi (es. per il progresso della
ricerca)? Se infatti istintivamente siamo portati a tutelarlo proprio come
faremmo con un individuo normodotato, razionalmente dobbiamo trovare un
argomento che possa giustificare questa tutela.

Ci sono diverse considerazioni da fare. In primo luogo, se quest'uomo avesse dei
cari, il suo sfruttamento causerebbe loro del dolore. Questo dolore, che può
essere più o meno razionale, è pur sempre dolore; danneggiando un caso marginale
dunque, danneggeremmo chiunque lo avesse a cuore. Poiché c'è un'alternativa che
provoca meno sofferenza (usare un animale non umano), siamo moralmente obbligati
ad adottarla.

Inoltre, c'è sempre la possibilità che la scienza trovi una cura prima della
morte dell'uomo. Appare invece improbabile che riusciamo a inventare, in tempi
ragionevoli, un metodo per dotare gli animali non umani di autocoscienza. E
anche se lo inventassimo, le implicazioni etiche di una simile azione
meriterebbero un tomo a parte.

Infine, quest'argomento è utilizzabile solo per via dell'esistenza di
(pochissimi) casi marginali. Se questi casi non esistessero e tutti gli esseri
umani fossero superiori in capacità cognitive agli animali non umani, allora
saremmo giustificati nel negare loro i privilegi che gli animalisti invece
pretendono. Appare quantomeno curioso che un fattore completamente indipendente
dalla natura delle specie non umane (l'esistenza dei casi marginali tra gli
esseri umani) possa influire in qualche modo sul loro stato morale. Questo fatto
suggerisce che il vizio potrebbe trovarsi non nel trattamento che riserviamo
agli animali non umani quanto piuttosto in quello che riserviamo ai casi
marginali, ferme restando le ragioni di cui sopra per cui sfruttarli sarebbe
immorale.

\section{L'argomento non egalitario raffinato}

Singer sostiene anche\footnotemark[3] che non si possano assegnare diritti agli
individui in base alle loro capacità (e, nel caso specifico, della capacità di
distinguere l'io) perché, applicando questo ragionamento in maniera coerente, si
getterebbero le basi per una discriminazione tra gli stessi esseri umani. Si
potrebbe infatti obiettare che gli esseri umani con un quoziente intellettivo
particolarmente elevato soffrirebbero per via del proprio sfruttamento più di
quanto ne soffrirebbe un normodotato.

Ma che un quoziente intellettivo elevato ``amplifichi'' la percezione del dolore
non è vero. Si potrebbe anzi dire che non esista alcun fattore oggettivo che ci
permetta di giudicare quale essere umano potrebbe soffrire di più per via del
proprio sfruttamento (a parte, ovviamente, i casi marginali che ho discusso
prima), essendo il dolore una sensazione estremamente soggettiva.

Ma se anche tale fattore esistesse e alcuni individui provassero il dolore più
intensamente di altri, sarebbe comunque immorale il loro sfruttamento in quanto
tali individui avrebbero dei diritti. I diritti, infatti, non possono essere
negati in modo arbitrario: ha dei diritti chiunque possa rispettare dei doveri,
indipendentemente da considerazioni di altra natura.

\section{La razionalità delle emozioni}

% Le emozioni hanno una base razionale.

\chapter{Il diritto di avere diritti}

Gli animalisti parlano spesso di diritti degli animali. Ma come vedremo, gli
animali non hanno diritto ad avere diritti. Per poter avere dei diritti,
infatti, come ho già scritto nel paragrafo precedente, bisogna essere in grado
di rispettare dei doveri. L'unico motivo per cui io, in quanto essere umano, ho
il diritto alla vita, è che posso rispettare il dovere di non uccidere.

Gli animali non umani non dispongono delle capacità cognitive necessarie per
rispettare un dovere e dunque non possono avere dei diritti. Sono gli uomini che
hanno regolato autonomamente la libertà con cui possono disporre delle altre
specie: uccidendo un cane non ho violato il suo diritto alla vita; ho infranto
una legge umana che mi proibiva di uccidere il cane.

Anche questa concezione del sistema diritti-doveri si presta all'argomento del
caso marginale, ma le obiezioni sono le stesse che ho già presentato sopra,
dunque non mi dilungherò più di tanto.

C'è però un'altra interessante situazione, più un mero esercizio della ragione
che un'eventualità concreta. Se una scimmia uguale all'uomo per capacità
cognitive si presentasse alla società e pretendesse i nostri stessi diritti,
saremmo moralmente (ma non legalmente, almeno per ora) obbligati a garantirli.
La scimmia, potendo rispettare tutti i doveri umani, può e deve anche
beneficiare di tutti i diritti.

Infine, il ragionamento che ho descritto non giustifica in alcun modo
l'introduzione della pena di morte. Non rispettando un dovere altrui, infatti,
non perdo il diritto corrispondente perché sono comunque potenzialmente in grado
di rispettarlo. Per questo uccidere un omicida è un'azione immorale e si
dovrebbe procedere invece alla sua rieducazione: l'assassino ha un'intera vita
davanti a sé per rientrare nel sistema diritti-doveri creato dalla società;
ucciderlo significherebbe privarlo di questa opportunità.

\chapter{Rinunciare allo specismo}

% Possiamo rinunciare allo specismo?
% Dobbiamo rinunciare allo specismo?

\chapter{Innocenza e libero arbitrio}

% Per essere innocenti bisogna poter scegliere.
% Gli animali non sono più innocenti di quanto siano colpevoli.

\chapter{La moralità dell'inazione}

% È immorale lasciar morire un bambino che annega.
% Perché dovrebbe esserlo lasciar morire un bambino malato?

\footnotetext[1]{ J. Bentham, \emph{Introduzione ai princìpi della morale e della legislazione}, cap. 17}
\footnotetext[2]{ P. Singer, \emph{Speciesism and moral status} in \emph{Metaphilosophy}, luglio 2009, 3-4, pp. 568-571}
\footnotetext[3]{ P. Singer, op. cit., pp. 572-573}

\end{document}
