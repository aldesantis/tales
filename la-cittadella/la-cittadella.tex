\documentclass[a4paper,oneside,12pt]{memoir}

\usepackage[italian]{babel}
\usepackage[T1]{fontenc}
\usepackage[utf8]{inputenc}

\title{La Cittadella}
\author{Alessandro Desantis}

\chapterstyle{dash}
\pagestyle{plain}
\frenchspacing

\begin{document}

\begin{titlingpage}
\maketitle
\end{titlingpage}

\chapter{}

Stephan amava guidare. Era l'unica attività che ancora gli desse un po' di
gioia, perché era del tutto normale, durante la guida, concentrarsi
esclusivamente sulla strada e non pensare a null'altro. Si preoccupava solo del
percorso, senza che la sua mente vagasse verso la destinazione. Ogni volta che
si distraeva e la lasciava libera di correre per conto suo, quella maledetta non
faceva che tormentarlo con le sue assurde e infantili preoccupazioni: per
esempio, dove stava andando? ``A casa'', era la risposta ovvia. Ma poi? Che cosa
avrebbe fatto una volta a casa? Sarebbe andato a dormire, perché sua moglie
probabilmente era già al letto, troppo stanca per parlare o fare l'amore. E
l'indomani cosa sarebbe accaduto? Sarebbe andato di nuovo al lavoro. E così per
il resto dei suoi miserabili giorni.

Ah, diamine, era successo un'altra volta. Riportò la propria attenzione sul
volante, troncando quell'infelice flusso di pensiero. A dir la verità era
diventato piuttosto bravo a sopprimere alla radice quelle domande; erano un
lusso che non poteva permettersi alla sua età. Aveva passato anni a combattere
con le aspettative che la società aveva per lui e si era convinto, infine, che
fosse una lotta impari. Non aveva alcuna speranza di vincere; era molto più
dolce la resa di quella continua, infruttuosa sofferenza. Non era poi così
importante cosa egli volesse, né era importante quale fosse il suo scopo
nell'universo: aveva scelto di prendere tutto così come veniva, perché gli
rendeva le cose così infinitamente sopportabili.

Ma poiché ogni uomo nasce libero, per poi essere incatenato da altri, schiavi
a loro volta, è forse doveroso soffermarsi un istante a parlare di \emph{come}
Stephan sia diventato così disilluso, se non per un simpatico esercizio di
etica, almeno per una cinica analisi del sistema in cui tutti gli esseri umani
si muovono, più o meno volontariamente. Si potrà così notare come, in fin dei
conti, la libertà sia solo un'illusione, perché ogni scelta è resa possibile, e
talvolta preferibile, solo dalle scelte che altri hanno compiuto in passato, a
loro volta fatte solo a causa delle scelte di chi è venuto prima ancora, e così
via fino al primo motore.

\end{document}
