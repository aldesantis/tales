\documentclass[a4paper,12pt]{book}

\usepackage[italian]{babel}
\usepackage[utf8]{inputenc}
\usepackage{verse}

\frenchspacing

\title{La Società Alternativa}
\author{Alessandro Desantis}
\date{}

\begin{document}

\maketitle

\begin{flushright}
A L., astro della notte ed eclissi del giorno.
\end{flushright}

\clearpage

\begin{verse}
\itshape{
``Tigre! Tigre! Divampante fulgore\\
Nelle foreste della notte,\\
Quale fu l'immortale mano o l'occhio\\
Ch'ebbe la forza di formare la tua agghiacciante simmetria?

In quali abissi o in quali cieli\\
Accese il fuoco dei tuoi occhi?\\
Sopra quali ali osa slanciarsi?\\
E quale mano afferra il fuoco?\\
Quali spalle, quale arte\\
Poté torcerti i tendini del cuore?\\
E quando il tuo cuore ebbe il primo palpito,\\
Quale tremenda mano? Quale tremendo piede?

Quale mazza e quale catena?\\
Il tuo cervello fu in quale fornace?\\
E quale incudine?\\
Quale morsa robusta osò serrarne i terrori funesti?

Mentre gli astri perdevano le lance tirandole alla terra\\
e il paradiso empivano di pianti?\\
Fu nel sorriso che ebbe osservando compiuto il suo lavoro,\\
Chi l'Agnello creò, creò anche te?

Tigre! Tigre! Divampante fulgore\\
Nelle foreste della notte,\\
Quale mano, quale immortale spia\\
Osa formare la tua agghiacciante simmetria?''
\/}
\end{verse}

\begin{flushright}
--- \scshape{W. Blake}
\end{flushright}

\chapter{Il rito}

\paragraph{}
Conobbi L. durante una fase particolare della mia vita: ero un avvocato di
discreto successo, eppure non mi sentivo affatto felice; mi sembrava che le mie
giornate fossero insopportabili, soffocanti nella loro perfezione. Ero una
persona realizzata: facevo ciò che avevo sempre sognato, avevo trovato l'amore,
riuscivo negli sport, ma mancava quella scintilla che mi permettesse di andare
avanti, che rendesse ogni istante della mia esistenza unico e speciale.

Non auguro a nessuno di provare una sensazione del genere: si sente che c'è
qualcosa che non va ma non si riesce a individuare esattamente il problema. Non
si può fare niente per migliorare ciò che si è, non importa quanto ci si
sforzi di pensare a una vita migliore. È un dolore così opprimente che neanche
piangere sembra appropriato di fronte a cotanta sofferenza.

Un giorno venne da me questa donna. Non era la stessa che mi sarei abituato a
vedere dopo diverso tempo: più ordinaria, meno potente. Anche così però si
portava dietro un'aura di energia ovunque andasse. ``Energia''... era quello
l'elemento mancante nella mia vita; era a causa della sua assenza che stavo
tanto male. Non mi resi subito conto di chi avevo di fronte, perché lei stava
ben attenta a fare in modo che solo chi era pronto si accorgesse della sua
grandezza. Solo quando fu sicura che fossi la persona adatta si lasciò andare
e io la potei ammirare in tutta la sua maestosità.

All'inizio si trattava di piccole consulenze. Mi chiedevo perché una donna
tanto bella si interessasse a un argomento noioso come il diritto. Un giorno mi
azzardai a chiederle che lavoro facesse.

«Io?» chiese sorpresa. «Be', si potrebbe dire che sono un'artista, credo».

Non feci altre domande, perché capii che ancora non voleva spingersi tanto
oltre. Ma più tempo passavo con lei e più ero intossicato dai suoi
atteggiamenti, dalla sua esistenza.

\paragraph{}
Poco dopo il nostro primo incontro iniziai a scrivere. Non narravo apertamente
di lei, forse perché mi vergognavo all'idea che qualcuno che conoscevo appena
potesse avere un ascendente così forte su di me.

Lasciavo le cartelle in ufficio perché così potevo uccidere quello sprazzo di
noia che esisteva tra un cliente e l'altro, quando il lavoro non mi impegnava
e non potevo pensare a qualcosa che non fosse l'imperfezione della mia vita,
un po' meno imperfetta da quando quella donna vi era entrata.

Quel pomeriggio dovevo incontrare L. ed ero in ritardo. Sembrava arrivare sempre
al momento giusto, ovunque ci fosse da attuare un cambiamento.

Entrai di corsa e incrociai per un istante il suo sguardo.

Aveva in mano una delle mie cartelle.

«Scrivi molto bene» osservò.

Non mi aveva mai chiesto il permesso di darmi del tu; lei non chiedeva mai
il permesso di fare qualcosa. Lo faceva e basta.

La ringraziai arrossendo.

Tentai di parlare di lavoro ma non me lo permise.

«Vedi,» mi interruppe «io faccio parte di un gruppo di artisti, e mi
piacerebbe che ti unissi a noi. Sarebbe un enorme contributo».

«Veramente non saprei... ho parecchio da fare in questo periodo».

Tentai di evitare il confronto diretto: avevo paura in un certo modo di L., di
ciò che sarebbe potuto accadere se fossi entrato nel suo strano mondo; era come
non mi sentivo all'altezza. Lei era così potente e decisa e io così
infinitamente debole e nudo.

Tuttavia l'idea di essere parte di qualcosa di più grande mi affascinava: mi
sentivo come un bambino che gioca col fuoco e sa che in ogni momento potrebbe
bruciarsi, e ha paura di provare dolore ma ancora più paura di rimanere
ignorante.

«Avanti,» disse con voce suadente «sono sicura che non te ne pentirai».

Alla fine accettai e non controvoglia, solo con un po' di timore.

«Benissimo» rispose L. «Ora possiamo parlare di lavoro».

\paragraph{}
Mi lasciò un indirizzo e una data. Si rifiutò di dirmi dove sarei andato.

I giorni si succedettero in fretta, e il tanto atteso --- e temuto! --- momento
arrivò prima di quanto mi aspettassi. Un'ora prima dell'incontro ero nervoso
come non mai, ed enormemente tentato di chiamarla e disdire tutto. Non lo feci,
forse, solo perché avevo paura di sentire la sua voce dopo tanto tempo: era un
po' infatti che L. non si faceva vedere allo studio, e credevo si fosse
dimenticata di me.

Ormai rassegnato, ma eccitato al tempo stesso, mi avviai in macchina verso
quel luogo. Era piuttosto lontano dal centro e sulle mappe non era segnato
assolutamente nulla.

Giunsi a un teatro; non era più attivo da qualche anno. Tuttavia era ben curato
da una mano attenta. Mi parve di riconoscere il suo tocco.

All'interno l'aria sapeva di poltrona eppure era familiare e rassicurante.
Nonostante ci fossero diverse sale seppi subito dove andare: forse fu perché
ormai ero così assuefatto a L. che l'avrei trovata ovunque. Forse, più
semplicemente, seguii le voci.

All'interno della stanza si trovavano lei e un'altra decina di persone. Proprio
in quel momento una donna bionda e grassoccia in piedi sul palco stava leggendo
quella che presumetti essere una sua poesia. Rimasi in piedi ad ascoltarla, ma
L. mi fece segno di sedermi accanto a lei in prima fila. La salutai; sorrise e
posò il dito sulle labbra, chiedendomi di fare silenzio.

Quando ebbe terminato tutti applaudirono. Uno dopo l'altro ognuno si esibì nella
propria arte. Credevo fossero tutti scrittori ma mi sbagliavo: c'era chi
cantava, chi ballava e chi mostrava le sue ultime fotografie.

Infine arrivò il suo turno. Salì sul palco, meravigliosa nel suo abito, e
accompagnata da un'orchestra nell'angolo della sala cantò con una delle più
belle voci che avessi mai ascoltato. Le parole non possono descrivere le
sensazioni che quei suoni mi hanno lasciato, dunque mi limiterò a dire che
quando finì provai la stessa tristezza che si prova svegliandosi e interrompendo
a metà uno stupendo sogno.

Nessuno la applaudì: sembravano tutti stupiti quanto me, sebbene, immaginai,
non fosse la prima volta che la ascoltavano. Al termine L. non scese dal palco
ma vi indugiò qualche momento, tenendo gli occhi chiusi e la testa bassa, le
labbra incurvate in un impercettibile sorriso.

«Oggi, \emph{amici},» chissà perché, quella parola mi colpì «si è unito a noi un
nuovo artista; uno scrittore. L'ho conosciuto per caso, come è accaduto con la
maggior parte di voi, e per caso ho scoperto il suo talento. Lo vorrei qui
accanto a me, sul palco, con una delle sue opere che sono sicura avrà portato».

Tutti si voltarono a guardarmi e proprio come accadde con L. in ufficio,
arrossii. Abbozzai un sorriso ebete, mi alzai e con le mani sudate presi dalla
tasca un foglietto, piegato tante volte da essere quasi illeggibile, sul quale
era riportato uno dei miei ultimi testi.

«Prego» disse L. sorridendomi rassicurante e facendosi da parte. Avevo
l'impressione che conoscesse il mio stato d'animo in quel momento e avesse
deliberatamente deciso di mettermi in imbarazzo; quello che non capivo, però,
era il motivo: qual era il suo scopo? Che cosa sperava di ottenere?

Così, con voce incrinata dall'emozione lessi quelle poche ma interminabili
righe. Man mano che procedevo l'aria si faceva sempre più calda e pesante.
Quando infine la tortura cessò mi sembrò di liberarmi di un pesante fardello,
qualcosa che mi portavo dietro da anni senza neanche saperlo, qualcosa che era
venuto a galla nel mare della mia anima solo quel pomeriggio.

«Molto bene» disse L. Sembrava fiera di me, come una madre di suo figlio.

Più tardi, quando gli altri se ne furono andati e ci trovammo da soli, le
parlai dei problemi che avevo avuto. «Non capisco,» dissi «sono abituato a
parlare in aula, sotto pressione, davanti a tutti. Eppure rendere gli altri
partecipi di ciò che ho scritto mi è costato un'enorme fatica».

«Forse sei un buon avvocato che sa come usare le prove a proprio favore. Forse
sei abituato a esporre i fatti in aula. Ma non sei abituato a comunicare i tuoi
sentimenti più profondi. È proprio qui il punto: è per questo che ho voluto che
leggessi davanti a noi, stasera. Ci siamo passati tutti, non preoccupartene
troppo».

«Tutti tranne te, a quanto pare. Sembri così naturale quando sei sul palco e
canti...».

«Ho dovuto lavorare anch'io per avere una tale dimestichezza con le persone; le
persone sono complicate, ma meravigliose a loro modo».

Parlammo ancora per un po', quindi ci salutammo e ognuno andò per la propria
strada.

\paragraph{}
A quell'incontro ne seguirono altri, simili ma mai uguali: a volte veniva
qualcuno di nuovo e anch'egli doveva passare quel rito di iniziazione del
quale ero stato partecipe solo poche settimane prima. Vedendo come erano
impacciati i nuovi arrivati capivo dove avevo sbagliato fino a quel momento.
Nonostante fossi ancora lontano dal raggiungere i livelli di L., mi sembrava in
qualche modo più facile vivere, vedere, respirare.

Mi stavo sensibilizzando: iniziavo a cogliere la bellezza, più spesso la
bruttezza del mondo intorno a me, quando prima ero apatico. Mi sembrava di
poter capire meglio i sentimenti altrui, le motivazioni dietro alle azioni e le
idee ancora più dietro.

C'era qualcosa, però, che mi angosciava terribilmente: andare agli incontri
stava diventando un'abitudine, e dunque iniziavo a perdere interesse. Pensai di
essere io quello sbagliato e incontentabile, sempre alla ricerca di stimoli
fuori dalla mia portata. Ero terrorizzato all'idea di stufarmi di L., dei suoi
sorrisi e della sua voce.

A un incontro le parlai delle mie paure. Non sembrava sorpresa.

«Mi dispiace che ti senti così» disse.

Le dissi che a mio parere non era lei il problema, che pensavo di non essere
adatto alle azioni ripetitive: il gruppo era una novità all'inizio ma ormai
iniziava a essere un obbligo.

«L'uomo ha bisogno di azioni ripetitive. Ciò di cui non ha bisogno è la
routine: respiri, altrimenti soffocheresti, ma non ne soffri. Mangi, perché
moriresti di fame non facendolo, eppure non te ne lamenti; non lo fai perché
queste sono \emph{necessità}. Ma annoiarsi non è una necessità, tutt'altro! È
quando l'azione smette di essere dettata dall'istinto di sopravvivenza che
diventa routine, capisci?»

Non ne ero troppo sicuro, ma risposi di sì.

«Adesso cosa farai?»

«Non lo so, ma non smetterò di venire agli incontri».

«Non avevo dubbi, ma io intendevo adesso, ora, in questo preciso momento.
Salirai in macchina e poi?»

«Be', è tardi...»  risposi guardando l'orologio. «Andrò a casa, farò una doccia
e andrò a dormire. Domani devo alzarmi presto».

«Benissimo, allora non hai fretta: andiamo».

Ebbi il sospetto che non avesse sentito la mia ultima frase.

«Dove?»

«A fare una passeggiata; il mare non è lontano».

\paragraph{}
Poco oltre il teatro, in effetti, si trovava una bellissima spiaggia di cui non
mi ero mai accorto in tutto quel tempo.

Giungemmo sul limitare della strada, là dove finiva l'asfalto e iniziava la
sabbia.

L. tolse le scarpe e mi invitò a fare altrettanto. Esitante, la imitai.

Mi prese sottobraccio e camminammo così per un'ora almeno, avanti e indietro
su quella spiaggia che sembrava non aver ancora subito il triste intervento
dell'uomo. Parlammo di moltissime cose: religione, amore, filosofia, e anche
delle piccole sciocchezze di ogni giorno. Non capivo cosa stesse cercando di
ottenere e non mi importava affatto: avrei solo voluto che quel momento non
finisse mai.

Al ritorno però mi aspettava una spiacevole sorpresa: la mia auto era sparita.
L. sembrò non accorgersene, tanto che stava per andare via, lasciandomi solo.

«Mi hanno rubato la macchina!» urlai verso di lei, già lontana.

Si voltò stupita e tornò sui suoi passi.

«Come?» disse ridendo.

«La mia macchina... è sparita!»

«Sì, lo so. L'ho fatta portare via io».

Sentii le guance avvampare per lo stupore e l'irritazione.

«Cos'è, uno scherzo?».

La mia voce si faceva più alta.

«Tutt'altro» rispose, dimostrando una calma sconcertante.

Boccheggiai, incapace di dire qualunque cosa, e allora lei iniziò ad
allontanarsi.

«Aspetta! Come torno a casa?».

«Arrangiati» la sentii dire.

Non ebbi la prontezza di inseguirla: ero troppo sconcertato dal suo
atteggiamento. Sapevo che L. non mi avrebbe mai ferito. Non senza avere un fine
più grande, almeno. Ma allora cosa voleva fare? Perché metteva a così dura prova
la mia dedizione?

Ebbi modo di pensarci mentre facevo l'autostop. Quando, dopo circa due ore ---
in pochi passavano di lì, in pochissimi andavano nella mia direzione, e solo
uno era disposto a dare un passaggio a uno sconosciuto --- tornai a casa, trovai
l'auto parcheggiata fuori dal cancello, come se non fossi mai uscito. Come
immaginavo, non mancava assolutamente nulla.

Mi sdraiai sul divano, e il sonno mi trovò vestito, mentre ancora riflettevo, e
dentro di me covavo un sentimento che non era amore ma neanche odio.

\chapter{Partenza}

\paragraph{}
\emph{Fuori dal teatro, stasera, alle nove.}

Il messaggio giungeva inaspettato, ma sapevo chi l'aveva mandato.

Per tutto il giorno pensai a cosa sarebbe stato giusto fare: avrei potuto
ignorarlo, e l'orgoglio mi spingeva a farlo. In quel modo però avrei perso
un'occasione. Oppure potevo mettere da parte l'Achille che era in me e
incontrare L. per vedere cos'altro aveva da offrirmi.

Il dubbio mi strusse fino alle otto e quaranta, quando decisi che il mio animo
ferito poteva essere barattato con una vita felice. Ero ancora indeciso quando
la vidi proprio davanti al teatro, pensierosa; sembrò non accorgersi nemmeno
del mio arrivo. Nel momento in cui mi avvicinai mi afferrò per il braccio,
dirigendosi verso l'auto dalla quale ero appena sceso.

«Sei in ritardo».

Decisi che quello era il momento di sfogarmi.

«Meravigliati che sia venuto dopo quello che hai combinato l'altra sera con
la macchina! È stato veramente...»

«Lascia perdere la macchina, dobbiamo andare».

La mia spietata determinazione si spense come la fiamma di una candela muore a
causa di un'improvvisa folata di vento.

«Andare... dove?»

«Hai detto che volevi qualcosa di più, no? Questa è la tua occasione. Guidi tu».

Immaginavo che mi avrebbe portato ancora più lontano dalla città, invece mi
indicò una via del centro. Non avevo idea di cosa sarebbe successo ma sapevo
che qualunque cosa fosse avrebbe cambiato la mia vita; ero dunque nervoso,
emozionato, terrorizzato, ma allo stesso tempo sicuro di me stesso, perché
sapevo che L. non mi avrebbe lasciato cadere.

O lo avrebbe fatto al solo scopo di afferrarmi per i capelli.

Non disse nulla per tutto il viaggio e non rispose alle mie continue e
insistenti domande.

Dopo mezz'ora circa ci fermammo sotto un blocco di lussuosi appartamenti. Il
portone si aprì non appena ci avvicinammo e L. lo tenne aperto per me. Mentre
eravamo in ascensore mi guardò a fondo e disse qualcosa che mi turbò
ulteriormente.

«Stasera ti sarà chiesto di prendere una scelta. Non è importante cosa
decidi, almeno non per me. In caso dovessi rifiutare, però, è di fondamentale
importanza che non parli con nessuno di ciò che hai visto e sentito. Se lo farai
potrebbe esserci tolta la possibilità di aiutare altre persone, persone come te.
Hai capito?».

Allucinato, la guardavo.

«Hai capito?» chiese di nuovo scuotendomi.

Mossi leggermente la testa, prima su, poi giù, come se volessi dire sì; in
realtà, però, non era proprio un sì. Non troppo convinto, almeno.

\paragraph{}
È curioso come il nostro cervello ricordi soprattutto l'odore dei luoghi che
visitiamo. Ma se dovessi sforzarmi di far tornare alla mente l'immagine di
quell'appartamento al dodicesimo piano non ricorderei l'odore, perché non ne
aveva; rammenterei bene, invece, il pavimento di marmo sul quale mi specchiai
appena entrato. L'immagine che vidi era quella di un uomo indeciso, impaurito ma
felice; un'espressione che non avevo mai vista dipinta sulla faccia di nessuno
fino a quel momento.

Il posto sembrava deserto. L. però si diresse con sicurezza verso un altro
ambiente, uno studio con un'immensa vetrata che permetteva di ammirare la
città nel momento in cui il sole cala e le prime luci si accendono. Davanti a
questa, in piedi, stava un uomo non più giovane, di sessant'anni circa, vestito
elegantemente. Aveva l'aria di essere una persona ricca ma sobria, due qualità
non facilmente conciliabili.

«Sei qui» disse L. senza muoversi dalla soglia.

«Oh, finalmente» rispose quello seccato. «Temevo vi foste persi».

Non si voltò per guardarci, ma potevo vedere il riflesso del suo viso nel vetro;
mi colpirono più di ogni altro dettaglio i suoi occhi, stanchi, distratti,
indici di una mente rivolta altrove. In mano teneva un calice di vino bianco.

L. mise una mano sulla mia schiena e mi spinse avanti, ma ella non si mosse; mi
inquietava quella nervosa immobilità di lei che era sempre a proprio agio. Era
come se il vecchio fosse tanto importante o pericoloso da non potercisi
permettere alcun errore in sua presenza.

«Questo è l'amico di cui ti avevo parlato» mi presentò.

«Pensi che sia pronto?»

«Sì. Credo di sì».

Mi osservò a lungo. Quando i nostri occhi si incrociarono abbassai lo sguardo,
a disagio.

«Venite di là: qui mi sento osservato».

Ci guidò verso una terza stanza, ancora più grande di quella in cui ci trovavamo
prima. C'era una scrivania di mogano tanto grande quanto costosa. Accanto a
questa stava una libreria anch'essa di mogano nella quale erano contenute decine
di volumi; pur essendo un buon lettore riconobbi pochi titoli, perché erano
quasi tutti in lingue straniere, per la maggior parte orientali.

L'uomo si sedette su una costosissima poltrona in pelle mentre io e L.
rimanemmo al di qua della scrivania, così che finalmente ebbi modo di vederlo
bene. Era vestito completamente di nero ed effettivamente incuteva un certo
timore. Con i suoi lunghi capelli bianchi e la barba ispida sembrava un dottore
o un alchimista, non so bene quale delle due gli si confacesse meglio.

Egli prese una penna e un foglio dalla risma davanti a sé e me li porse.

«Firma».

«Ma non c'è scritto niente!» protestai.

Guardò accigliato L., che aveva l'aria costernata.

«Ti prego, firma» mi implorò anche lei.

Il colmo per un avvocato dev'essere firmare un foglio completamente bianco.

Eppure io lo feci, e non perché fossi convinto, ma perché la pressione era
troppa e non volevo deludere L.

Allora l'uomo prese dalle mie mani carta e penna e iniziò a scrivere qualcosa in
bella grafia proprio sopra la mia firma.

Si alzò di scatto facendomi trasalire.

«Possiamo andare. Hai portato il necessario per il viaggio?»

«Viaggio? Che viaggio?»

Tutta la faccenda iniziava a preoccuparmi.

L. posò le mani sulle mie spalle e mi guardò dolcemente.

«Ti fidi di me?»

Esitai parecchio prima di rispondere: mi fidavo ciecamente di lei ma avevo
paura di quali sarebbero state le conseguenze delle mie parole. Nonostante
tutto, però, mai e poi mai avrei rischiato di perderla.

«Sì».

«Allora non fare altre domande; lo dico per te, è questo che volevi: rischiare
tutto per una nuova vita. Non è così? Devi scegliere usando l'istinto, e
ciò sarebbe impossibile se ti spiegassi ogni dettaglio. Del resto anche
volendo non potrei farlo: il futuro non mi è molto più chiaro di quanto lo
sia a te in questo momento. Tuttavia sappi che qualunque sia la tua decisione
non potrai assolutamente tornare indietro».

Tentai di deglutire, ma avevo la gola completamente secca.

«Va bene».

«Va bene cosa?»

«Verrò con voi».

Avevo cercato di rimandare il più possibile quel momento, ma ormai il passo era
fatto. E forse era un passo troppo lungo per la mia gamba.

L. e l'uomo, animandosi all'improvviso, uscirono dall'appartamento, trascinando
me dietro. Salimmo su un Suv parcheggiato poco lontano dall'edificio.

«Ma che ne sarà della mia macchina? E della mia casa?».

«Certo che questa macchina non ti dà pace!» rise L.

«Non ne avrai bisogno dove stiamo andando» aggiunse l'uomo.

«Perché? Dov'è che stiamo andando?»

Nessuno mi rispose, ma non me ne preoccupai troppo: avevo accettato l'invito
di due sconosciuti a iniziare una nuova vita con loro... Che importava se ancora
non sapevo dove? Non era che un fattarello di poco conto.

Mi appoggiai comodamente al sedile tentando di rilassarmi, e stranamente la
cosa sembrava funzionare. Pensai che il mio destino dipendeva ormai da forze di
gran lunga superiori alla mia, e reazioni che io stesso avevo messo in moto e
non potevano più essere fermate. Ero come una spiga di grano in balia del
vento.

Ormai era sera inoltrata e la città era completamente illuminata. Iniziava a
piovere; i colori delle insegne visti attraverso le gocce d'acqua formavano
strani giochi di luce. L'uomo guidava tenendo gli occhi fissi sulla strada,
mentre L. sul sedile del passeggero rifletteva silenziosa.

Si avvertiva nell'aria un sentimento pesante: nostalgia e malinconia, il tutto
mischiato a una buona dose di eccitazione. Era il clima ideale per una partenza.

L. si girò e allungò la mano chiara verso di me: sul palmo stava una compressa
lunga un centimetro circa.

«Prendila».

«Cos'è?».

Silenzio.

«Devi...».

«...fidarmi di te. Sì, ho capito».

Raccolsi la compressa da quella mano deicata, la misi in bocca e deglutii. Pochi
minuti dopo fui pervaso da un improvviso torpore e chiusi gli occhi,
stanchissimo.

L'ultima cosa che ricordo è il suo sguardo vivace e accorto nello specchietto
retrovisore.

\chapter{Oltre il cancello}

\paragraph{}
Motori. Furono la prima cosa che udii agli albori della mia nuova vita: motori
che si spegnevano. Mentre aprivo gli occhi, lentamente perché la luce passava
da ogni spiraglio e sembrava farli sanguinare, giunsero alle mie orecchie anche
alcune voci.

«Si sta svegliando» disse qualcuno.

«Giusto in tempo» ridacchiò un altro.

Non avevo idea di cosa mi fosse successo e non riuscivo a pensare né a ricordare
nulla per via del mal di testa; mi sembrava di essere un neonato espulso dal
ventre materno che si affaccia sul mondo con curiosità e timore.

«Chi siete? Dove sono?».

Un immenso afroamericano dall'aria minacciosa mi si avvicinò.

«Quanto alla prima domanda,» rispose con voce profonda «non hai bisogno né sei
tenuto a saperlo. Per ciò che riguarda la seconda invece la risposta è ``su un
aereo'', ma non credo che ti soddisfi molto. In questo momento non sei in grado
di ragionare con lucidità: tra qualche minuto ti sarà spiegato tutto».

Mi accorsi di trovarmi su quello che, a giudicare dagli interni, era un jet
privato, appena atterrato nel mezzo del nulla. Ovunque fossimo pioveva anche lì:
potevo sentire il ticchettio dell'acqua contro il metallo. Il portellone era
aperto e dalla scaletta vidi scendere pilota e copilota, entrambi elegantemente
vestiti, dopo avermi lanciato un'occhiata furtiva.

«Lei dov'è?».

Stavolta fu un altro a rispondermi; sedeva di fronte a me e teneva le gambe
accavallate, una mano a sostenere il mento, un leggero sorriso dovuto
probabilmente al mio stato di evidente smarrimento.

«Di chi parli?».

Non ebbi l'occasione di rispondere perché in quel momento L. si materializzò
alla mia sinistra; era in piedi e mi posò affettuosamente una mano sulla spalla.
I due uomini la guardavano con rispetto.

«Bentornato!» rise. «Vieni, dobbiamo andare; senza fretta, ormai».

Slacciai la cintura e mi alzai lentamente, perché avevo paura che le gambe
cedessero. Non c'era traccia dell'uomo che ci aveva condotto fino all'aereo;
pensai che non fosse salito con noi.

«Dove siamo?» chiesi nuovamente.

«Pazienta ancora qualche attimo: allora potrai vedere con i tuoi occhi».

Una volta fuori dall'aereo, mentre scendevo lentamente gli scalini sostenuto
da L., qualcuno aprì un ombrello sulle nostre teste. In fondo ci attendevano due
fuoristrada che vennero occupati dagli uomini e in mezzo una terza auto nella
quale stavamo noi due.

Il viaggio durò più di mezz'ora, eppure non parlammo granché, complice anche la
mia tremenda stanchezza. Le chiesi cosa mi avesse dato per farmi stare così,
anche se intuivo già la risposta.

«Un sonnifero».

Sospirai.

«Immaginavo».

«Non è nulla di personale, solo non volevamo che vedessi la strada».

«La strada per arrivare fin qui? Come avrei potuto?».

«La strada per arrivare all'aeroporto. E devi ringraziare me se ora non te ne
abbiamo dato un altro... I miei ``colleghi'' non sono del tutto d'accordo con
questa scelta, ma io mi fido di te».

«Ma perché tanta segretezza?».

«Lo scoprirai stando con noi. Per ora sappi che quello che cerchiamo di
introdurre è un grande cambiamento, e non a tutti piacciono i cambiamenti. I
nostri nemici sono potenti, politicamente parlando, e potrebbero distruggerci se
non fossimo così prudenti».

Fu la fine della nostra conversazione, perché avevo bisogno di silenzio per
pensare ai guai in cui mi ero cacciato.

Il paesaggio non era molto vario ma piacevole: eravamo entrati in un bosco e
seguivamo un perfetto sentiero che lo attraversava da parte a parte. Gli alberi
finivano ma la strada proseguiva per svariati chilometri curvando dolcemente.

L. guidò ancora per qualche minuto, finché giungemmo a un cancello.

La vista mi incutè del timore: sembrava che pochi uscissero di lì una volta
entrati, e di certo lo facevano con l'intento o l'obbligo di tornare presto. Il
cancello si aprì quando eravamo a venti metri; evidentemente ci stavano
aspettando.

Al di là facevano ombra sulla strada altri alberi, alti e dalle foglie larghe e
verdi; la pioggia aveva portato fino a noi l'odore della loro corteccia
mischiato a quello della terra, così piacevole e rassicurante. Quando l'ultima
auto passò, il cancello si chiuse alle nostre spalle. Ancora non sapevo se
l'avrei mai rivisto, o se avrei desiderato di farlo.

Ma ciò che ancora mi aspettava era qualcosa a cui, nonostante tutto quello che
mi era successo in quegli ultimi giorni, non ero minimamente preparato.

Oltre il cancello vidi case, parchi, fontane, ma soprattutto persone: ai margini
della strada ci attendeva una folla sorridente che salutava con la mano: uomini,
donne, bambini... talvolta intere famiglie erano lì e ci fissavano con degli
sguardi pieni di benevolenza e amore.

Oltre il cancello c'era una piccola città, colorata e meravigliosa.

L. entrò nel vialetto di una delle case e si fermò lì, mentre le altre due
auto proseguirono. Scendemmo ed ella frugò nella borsa per poi tirarne fuori un
mazzo di chiavi.

«È la tua nuova casa,» disse «spero che tutto sia di tuo gradimento. Se hai
qualche problema puoi chiedere in giro: le persone qui non aspettano altro che
darti una mano. Io vado a cambiarmi; sarò qui tra due ore circa e ti
accompagnerò a casa mia dove cenerai per stasera. Ti consiglio di fare una
doccia: hai un aspetto orrendo».

Detto ciò tornò in macchina.

«Ah, dimenticavo» aggiunse sorridendo prima di andare via. «Benvenuto».

\paragraph{}
Seguii il saggio consiglio ricevuto e feci una doccia, quindi indossai una delle
camicie nell'armadio che, come tutti i nuovi abiti, non solo era della giusta
taglia ma anche di mio gradimento, quasi fosse stata scelta da qualcuno che
conosceva perfettamente i miei gusti.

Quell'alternanza di momenti di felicità e tremenda inquietudine continuava a
scombussolarmi. Sì, ero contento di essere finalmente accanto a L. e lontano da
quel mondo che mi rendeva tanto apatico, ma avvertivo anche una presenza oscura,
qualcosa di inspiegabile per una persona razionale come me. Forse era il mio
stesso animo a mettermi in guardia. Mi promisi di non pensarci più fino
all'indomani, quando mi sarei ripreso dal viaggio e avrei potuto ragionare con
lucidità.

Come mi aspettavo L. fu lì esattamente due ore dopo; venne a piedi e a piedi
raggiungemmo la sua abitazione, che non distava molto. Era notevolmente più
grande rispetto alle altre.

«Vedo che anche qui ci sono i raccomandati» scherzai.

«È una questione di esigenze» sorrise lei. «Se hai bisogno di più spazio non
devi che chiedere».

La cena fu lunga: posi tutte le domande che in macchina ero stato troppo stanco
per formulare. Così molte cose mi furono più chiare e fui in parte rasserenato,
giacché mi convinsi di essere in ottime mani.

«Dunque,» iniziò L. «ti starai chiedendo dove sei capitato».

«Proprio così».

«È difficile descriverci. Alcuni dicono che siamo un'organizzazione, altri una
setta; tu non dovrai lasciarti influenzare. Noi ci consideriamo prima di tutto
e soprattutto una enorme famiglia, una piccola società. Qui, al di qua del
cancello troviamo tutto ciò che ci serve per condurre una vita dignitosa,
rispettosa e soddisfacente. La maggior parte di noi è composta da artisti, come
te e me, mentre alcuni si occupano di altre faccende, più terrene».

«Ma come fate a vivere qui? Da dove viene il cibo?».

«Abbiamo le nostre coltivazioni e i nostri allevamenti. La Società ha le sue
risorse».

«Come l'hai chiamata? La Società?»

Era seccata.

«L'ho fatto? Non avrei dovuto: non siamo soliti darci un nome, anche se spesso
riferendosi a noi parlano di Società Alternativa».

«Quindi è così che dovrò dire a qualcuno quando mi presento? ``Piacere, faccio
parte della Società Alternativa''?»

L. mi guardò, terribilmente seria.

«Tu non dovrai mai dire a nessuno di cosa fai parte. Mai».

\paragraph{}
Quella notte mi fu impossibile mantenere la promessa fatta a me stesso. ``Non
pensare, non riflettere, non farti domande'' continuavo a ripetermi, ma le
parole di L. insistevano nel risuonarmi gravi e minacciose in testa. ``Non dire
mai a nessuno di cosa fai parte''. Perché? Cosa c'era di tanto terribile o
segreto nella Società Alternativa?

Mi addormentai con un peso sullo stomaco che mi rendeva complicato deglutire e
anche solo respirare. Mi svegliai allo stesso modo, arrabbiato con me stesso
per il mio essere incontentabile.

``Probabilmente sono impazzito del tutto. Di cosa ho bisogno per essere in pace?
Possibile che neanche qui sia felice?''

Io cercavo, cercavo di non pensare, non elaborare, ma non potevo! Per quanto
sarei riuscito a vivere in quella condizione?

L. mi raggiunse per la colazione.

«Ti presento gli altri» disse. E poi guardandomi meglio aggiunse preoccupata:
«Ieri ti ho detto che avevi un aspetto orribile, ma mi rendo conto che era nulla
rispetto a oggi. Che ti è successo?»

Temetti che avesse capito. Non volevo mentire perché mi avrebbe scoperto subito,
così optai per una mezza verità.

«Sì, ho dormito solo poche ore. Ero nervoso».

«Mi sembra comprensibile. Hai fatto una scelta coraggiosa; solo un idiota non
avrebbe ripensamenti. Tra poco ti sentirai meglio, vedrai».

L., mia ancora di salvezza! Mia luce nell'oscurità! Per quanto stessi male
riusciva sempre ad allietare il mio spirito ferito, perché mi conosceva meglio
di me stesso.

Mi condusse a una sorta di capannone all'interno del quale era già presente una
moltitudine di persone, alcune sedute ai tavolini circolari sparsi ovunque,
altre in fila per servirsi. Noi ci dirigemmo verso quest'ultimo gruppo, e una
volta fatto -- ossia, una volta prese le tre tazze di caffè nero necessarie e
indispensabili per svegliarmi -- ci sedemmo con due giovani coppie.

L. me le presentò ma in questo momento ho solo un vago ricordo di loro, perché
mi colpirono, più dei loro volti e dei loro nomi, i loro atteggiamenti: nessuno
sembrava inquieto; erano anzi così rilassati e felici, così appagati... Come se
nessun pensiero negativo potesse scalfire la loro perfetta esistenza. Avevano
raggiunto la tranquillità che io cercavo da tempo.

Mi fecero sentire a casa, e ciò mi rincuorò immensamente.

\paragraph{}
Iniziò così la mia nuova vita. Ebbi l'occasione di conoscere moltissime persone
che facevano parte della Società, non perché fossi io a cercarle --- nonostante
gli insegnamenti di L., infatti, sono sempre rimasto piuttosto timido --- ma
perché erano loro a fermare me non appena mi vedevano passare. Chiedevano come
mi trovavo, se mi serviva qualcosa. Quando rispondevo di no, iniziava una
piacevole conversazione sulle cose importanti di questo mondo, e terminava
quasi sempre con l'espressione ``La mia porta è sempre aperta'', pronunciata
dall'una o dall'altra parte.

Tramite gli altri membri cercai di apprendere qualcosa sul conto di L., ma
riuscii a ottenere pochissime informazioni e spesso discordanti: alcuni dicevano
che fosse una cantante di successo, altri che avesse acquistato fama solo
all'interno della Società, altri ancora che prima lavorasse nel mondo
dell'informatica...

Quando gli impegni glielo permettevano --- ora che era tornata nel suo mondo,
infatti, era una donna molto occupata --- L. passava a trovarmi. Ciò avveniva
solitamente al tramonto, per entrambi grande fonte d'ispirazione. Per me era
come rivivere ogni volta la nostra esperienza sulla spiaggia: passeggiavamo
chiacchierando di argomenti senza senso e argomenti importanti come se avessero
lo stesso peso, e guardavamo il sole scendere tra due montagne.

Le montagne di cui non ho mai scoperto i nomi.

Glieli chiesi più volte, ma lei mi ignorò sempre, finché un giorno mi disse:
«Ha davvero importanza? In questo momento tutto si crea e tutto si distrugge,
e un giorno queste montagne non esisteranno più. Allora l'unica cosa che varrà
la pena di ricordare saranno i momenti che hai passato guardandole insieme a me
e traendo ispirazione da questo spettacolo».

In realtà non mi era dato di conoscere i loro nomi perché avrei potuto intuire
dove mi trovavo, ma L. era riuscita a dare un aspetto filosofico alla questione,
e tanto mi bastava.

A quel periodo appartengono le mie opere più emozionali e meno oggettive.
Qualche giorno fa le ho rilette, e mi sono reso conto che, benché fossero
cariche di sensazioni meravigliose, mancavano quasi del tutto di coerenza.
Questo accadeva perché accanto a L. la coerenza era inutile, elemento superfluo
e del tutto irrilevante ai fini dell'opera.

«L'arte» diceva sempre lei «deve trasmettere emozioni, non idee. Le emozioni
sono uniche e irripetibili, mentre per riflettere su un dato argomento l'essere
umano ha decine di occasioni che butta via. Perché dovrebbe essere l'artista a
risolvere questo problema? Non descrivere come ti senti, \emph{mostra} come ti
senti».

Io obiettavo che nel canto è semplice; quando si ha a che fare con un racconto,
però, è spesso impossibile mostrare senza descrivere.

«Non è così» replicava sorridendo. «Con il tempo ci riuscirai, ne sono convinta.
Perché non vieni al mio prossimo concerto? Lo vedrai con i tuoi occhi».

Accettai con entusiasmo. Avevo udito L. cantare una sola volta, ma era stata
un'esperienza meravigliosa, ed ero ansioso di ripeterla.

Dopotutto, perché accadesse, non mi sarei neanche dovuto spostare molto.

\chapter{Dio è amore}

\paragraph{}
Il teatro era simile a quello nel quale si riuniva il gruppo di L., ma c'erano
molte più persone; sembrava quasi che lo spettacolo avesse luogo in una
metropoli, tanto la sala era piena. Eravamo tutti seduti --- io in prima fila,
perché mi era stato tenuto un posto d'onore --- e pronti ad assistere. Ero
nervoso, agitato al pensiero che lei avrebbe nuovamente cantato: la sua voce mi
permetteva di vedere altri mondi, provando sensazioni con cui non ero mai
entrato in contatto prima.

Poi lei comparve sul palco, anche quella volta vestita come una dea, e si fece
il silenzio. Una mano invisibile abbassò le luci proprio nel momento in cui L.
iniziava a cantare, con l'orchestra che la accompagnava meravigliosamente.
Insieme a lei c'era un coro; forse una decina di persone che contribuivano con
le loro voci, belle ma mai quanto la sua, a rendere il tutto un'esperienza
magica.

Pensai a quelle persone: ognuna di loro aveva una diversa storia da raccontare.
Forse alcuni avevano dei figli, altri erano divorziati, o addirittura non
avevano mai conosciuto le gioie dell'amore. Eppure in quel momento erano lì,
come una sola, grande entità, e cantavano all'unisono per me. Il pensiero,
nonostante la sua ovvietà, mi fece venire i brividi.

Lo spettacolo andò avanti per poco più di un'ora. L. pareva non essere mai
stanca. Io avevo reclinato la testa e chiuso gli occhi, cercando di concentrarmi
sulla musica, sulla sua meravigliosa voce ora incredibilmente acuta, ora
sensazionalmente grave. Passavo dalla risata al pianto con una velocità
impressionante, e mi resi conto di cosa intendesse dire L. il giorno prima,
quando parlava di ``mostrare come ci si sente''. Io, lei e tutti gli altri
spettatori eravamo una cosa sola, proprio come i membri del coro.

L. sembrava assente: ogni tanto mi guardava negli occhi, permettendomi di
ammirare quelle splendide iridi azzurre e profonde, ma la sua attenzione era
rivolta altrove, probabilmente ai mondi e ai personaggi di cui narrava.

Infine, rapido come era iniziato, tutto terminò. Le luci diventarono lentamente
più brillanti, finché tornarono alla loro originaria vividezza. Per pochi
secondi il tempo sembrò fermarsi: nessuno osava neanche respirare.
All'improvviso il pubblico esplose in un rumorosissimo applauso, al quale io mi
unii con piacere. L., sorridente, fece un inchino, quindi scomparve dietro al
palcoscenico.

Uscì da una porticina che dava sul pubblico, così che molti le corsero incontro
per complimentarsi, mentre altri, che evidentemente avevano già assistito alla
scena diverse volte, se ne andavano scuotendo la testa e ridacchiando alla vista
dei suoi ammiratori. Anch'io volevo parlarle, ma in privato, per raccontarle ciò
che avevo provato in quell'ora. Così aspettai che la folla si fosse diradata.
Rimaneva ormai una decina di persone, dunque mi feci avanti.

Ero a pochi passi e L., che mi aveva visto, stava per venirmi incontro. Ma in
quel momento un bambino --- avrà avuto cinque o sei anni --- spuntò dal nulla,
correndo verso di lei. Lo prese prontamente in braccio, mentre lui urlava:
«Mamma! Sei stata bravissima!».

«Grazie, tesoro» replicò con affetto, baciandolo su una guancia e
scompigliandogli i capelli.

Un uomo dai lineamenti simili a quelli del bambino, di qualche anno più vecchio
di L., si avvicinò poco dopo. Lei lo abbracciò a lungo e infine gli regalò un
bacio sulle labbra.

«Mi siete mancati» sussurrò.

I tre si accorsero improvvisamente della mia presenza.

«Spero ti sia piaciuto il concerto» disse L. «Ti presento mio marito e mio
figlio».

Lui mi porse la mano, che mi sforzai di stringere con vigore, per dimostrargli
che non ero intimorito.

«Piacere, ragazzo» mi salutò con voce profonda.

Non ero intimorito. Ero terrorizzato.

In primo luogo perché L. era una persona dalle risorse potenti e illimitate, e
dunque doveva esserlo anche suo marito.

Inoltre --- e questa era la mia paura maggiore --- temevo che, ora che si
trovava accanto alla famiglia, lei si sarebbe allontanata da me. La paura
dell'abbandono mi avrebbe portato a essere ostile nei loro confronti, e
l'ostilità sarebbe certamente stata percepita da L., che allora mi avrebbe
abbandonato sul serio.

Ero un'egoista.

Ma io non potevo, non dovevo perderla.

\paragraph{}
Accadde invece l'esatto contrario. L. non mi fu mai vicina come in quei giorni:
ogni volta che le era possibile, mi chiedeva di farle visita, a pranzo, a cena,
o semplicemente per un tè pomeridiano. E ogni volta il marito e il figlio erano
presenti, benché il primo si tenesse in disparte, limitandosi a salutarmi quando
arrivavo e quando andavo via. Come sempre, mi era impossibile decifrarla e
capire se il suo obiettivo fosse mostrarmi che non avevo da temere il marito, o
mostrare al marito che non aveva da temere me.

Del tutto particolare era il suo rapporto con il figlio, che, come avevo intuito
durante il nostro primo, breve incontro, aveva poco più di sei anni. Non ho mai
visto una madre tanto paziente, così come mai ho avuto a che fare con un bambino
tanto curioso e, allo stesso tempo, obbediente. A volte giocava nell'immenso
giardino di fronte alla villa o in camera --- e se faceva qualcosa di
pericoloso, o dannoso, L. lo chiamava e gli spiegava cosa non doveva fare e
perché, senza ira o disappunto, e lui annuiva e andava a fare altro --- altre
rimaneva con noi, seduto in terra, a gambe incrociate, e ci ascoltava, ponendo
di tanto in tanto domande che dimostravano una sconcertante maturità
intellettuale.

Ricordo che una volta, con la sua innocente voce domandò alla madre chi fosse
Dio. Lei lo guardò, pensando non alla risposta corretta, ma a quella meno
sbagliata.

«Non chi, ma cosa. Dio è amore. È quella forza che permette all'uomo di creare
e vivere cose meravigliose, che fa sì che i pianeti continuino a girare intorno
al sole, che permette ancora la nascita e lo sviluppo di relazioni sincere tra
gli esseri umani. Da solo, Dio non avrebbe significato: sono gli uomini che,
tramite la loro sola azione, gli conferiscono il potere di plasmare il mondo.
Perché Dio è l'Universo, e l'uomo è Dio, e chiunque pensi che Dio sia superiore
all'uomo --- perché così gli è sempre stato detto, o perché dopo aver studiato
molto è giunto a questa conclusione --- ha perso di vista i propri obiettivi.
Dio non chiede altro che lo spazio per esprimersi nelle nostre vite: se glielo
concediamo, esse saranno giustificate, altrimenti non avremo motivo di esistere,
indipendentemente da ciò di cui ci convinciamo».

«Quindi per dare un senso alla propria vita bisogna necessariamente lavorare?»
intervenni io.

«Se il tuo lavoro è un'attività che ami praticare e che porta un contributo
all'umanità, allora è \emph{possibile} raggiungere questo stato anche lavorando,
sì. Tuttavia ci sono altri modi: pregare, per esempio».

«Ma la preghiera non aiuta nessuno!».

«Al contrario: la preghiera porta tranquillità e pace, e con esse la capacità
di pensare lucidamente, e dunque la creatività».

«Tutte le persone che conosco --- che conoscevo --- pregavano per parlare con
Dio».

«Sbagliavano. La preghiera non è un modo per comunicare con Dio, ma ci pone in
uno stato di rilassamento nel quale è più semplice entrare in contatto con
Esso».

«Tu preghi?».

«Lo facevo, quando non ero parte della Società,» e alzò gli occhi al cielo
perché aveva di nuovo detto 'la Società'' «perché era l'unico modo per trovare
la pace. Ma quando si è circondati dalla pace, non ce n'è bisogno. Prega pure,
se ne senti il bisogno, ma non pensare che ciò darà un motivo alla tua
esistenza: il passo successivo alla contemplazione è la creazione».

«Dunque Dio è nell'azione. Non Lo si può raggiungere senza fare nulla: bisogna
dipingere, scrivere, cantare...»

«...o pensare» completò lei. «La mente è uno strumento potente, capace di cose
che non immagineremmo mai. L'azione non è necessariamente qualcosa di fisico:
chi lavora dodici ore al giorno, e passa le altre dodici stressato perché si
preoccupa del domani, vive nell'azione, eppure è un'azione statica, una
situazione in cui l'anima stagna e non è possibile parlare con Dio».

Ero confuso, ma sapevo che, come accadeva con ogni insegnamento di L., con il
tempo tutto mi sarebbe stato più chiaro.

Mi voltai verso il bambino per vedere se lui avesse capito --- nel qual caso mi
sarei sentito un idiota --- ma era già andato via.

\paragraph{}
Pochi giorni più tardi accadde l'inaspettato e --- ora che ripenso a quel
momento --- l'inevitabile.

L. mi invitò per la solita conversazione, e stavolta eravamo soli: marito e
figlio erano fuori per una passeggiata. Almeno, questo è quello che disse lei.

Preparò un tè, e parlammo per poco di cose che non ricordo. Poi si interruppe,
posò la tazza, mi guardò intensamente. Era solita fare così quando rifletteva.
Quella volta, però, a occupare la sua mente non erano quesiti esistenziali, ma
un essere umano. Io.

Si avvicinò --- eravamo seduti su un divano --- tanto che potei percepire il suo
odore di donna forte, indipendente, quasi selvatica. Mi parve di poter leggere
nella sua mente: anche se per pochi secondi, L. non era più un mistero.

Posò le labbra sulle mie, con leggerezza. Tremai --- non so se per la paura,
l'euforia, l'eccitazione, o tutte e tre le cose insieme --- e lei si fece più
insistente, come per incoraggiarmi a fare quel passo.

Fu allora che mi resi conto di una verità tanto idiota quanto sconcertante:
L. amava il sesso. Fino a quel punto non avevo mai pensato a lei come si
fantastica su una donna di eguale grazia. Era bella, affascinante, sensuale.
Forse volubile. Ma era anche la persona più intelligente e sensibile che avessi
mai incontrato, e dunque mi sembrava assurdo che scegliesse di partecipare a
un'attività tanto sporca.

Quel giorno scoprii che non solo L. era la persona più intelligente che avessi
mai incontrato, ma anche la più brava amante. Ci spostammo nella camera da
letto, quella dove, immaginai, si concedeva anche al marito. A quel pensiero mi
bloccai per un'istante, affranto e spaventato, ma lei fu svelta a coinvolgermi
nuovamente.

Presto fummo nel letto, e ognuno ansimava nell'orecchio dell'altro, e gemeva
esprimendo piacere intenso.

Raggiungemmo un orgasmo che sconquassò e provò duramente i nostri corpi.

Ricaddi sul materasso, esausto. Lei si sdraiò delicatamente accanto a me,
incrociò le braccia sul cuscino e vi appoggiò la testa. Era così vicina che nei
suoi occhi mi potevo specchiare. Aveva un'espressione serena. Quando qualcosa mi
turba, mi capita spesso di pensare a quel volto. Il suo sorriso la prima volta
che facemmo l'amore. Non ostinato, come di chi vuole sguaiatamente dimostrare al
mondo la propria felicità, ma discreto, dolce, caloroso.

«Ti è piaciuto?». Lo sussurrò.

«Molto. Perché l'hai fatto?».

«Cos'hai provato?».

«Ti ho visto. Come sei veramente, intendo. Non ti avevo mai vista così...
concreta».

«Allora ci sono riuscita».

«A fare cosa?».

«Qualche giorno fa mi sono accorta di una cosa: tu hai smesso di dubitare di me.
Non c'è più discussione, il dialogo tra noi due è unidirezionale: se ti dicessi
di lanciarti da un dirupo, probabilmente lo faresti, convinto che io ne sappia
molto più di te. Be', non è vero. Io sono qui per discutere con te, non per
insegnarti qualcosa. Se pensi che stia sbagliando, contraddicimi. Non mi importa
che le tue idee siano uguali alle mie. Voglio solo che tu \emph{abbia} delle
idee».

«E il sesso cosa c'entra?».

«Davanti al sesso siamo tutti uguali, maestri e allievi. Mostrandoti che puoi
avermi, ti ho fatto capire che tra noi due non c'è quell'oceano che hai sempre
immaginato».

«Ti ringrazio».

«Non ringraziarmi: ora pesa su di te la grossa responsabilità di pensare,
analizzare, giungere alle conclusioni. Credevi di saperlo già fare, ma eri
imprigionato dai preconcetti, schiavo della normalità».

«Dunque era solo questo il tuo scopo? È stata la prima e ultima volta che siamo
stati a letto insieme?».

Sembrò sorpresa per un attimo, come se non si aspettasse l'audacia che ella
stessa aveva incoraggiato un attimo prima.

Infine sorrise, nuovamente rilassata.

«Ovviamente no. Anche a me è piaciuto».

\paragraph{}
Le nuvole si ergevano scure, gonfie e minacciose sopra le nostre teste. Io e L.
eravamo sdraiati sul prato, e non dicevamo nulla. Ero stato io a chiederle di
incontrarci quella volta, ma non riuscivo a trovare le parole, e così ella
attendeva pazientemente, silenziosa quanto quel pomeriggio d'inizio inverno.
Inspirai profondamente, lasciando che l'umidità penetrasse nei polmoni.

«Questa storia deve finire».

«Quale storia?».

«Sai quale storia. Io e te. Dobbiamo smettere di vederci, o almeno di andare a
letto insieme».

Adesso lei sembrava veramente sorpresa e attenta.

«E perché mai? Io sto bene con te».

«Anch'io. Ma ogni volta che facciamo l'amore, continuo a vedere tuo marito e tuo
figlio. È una cosa che non posso ignorare. Immagino che per te sia diverso».

Da ormai due mesi eravamo amanti, e quei pensieri mi avevano angosciato fin
dalla prima volta. Inoltre in quell'arco di tempo avevo incominciato a
incontrare anche suo marito, la cui amicizia nei miei confronti sembrava
sincera. Come potevo pugnalarlo alle spalle in quel modo? Come poteva farlo lei,
che parlava tanto di rispetto?

«Tu sei un idiota».

Lo disse sorridendo, come se qualcosa di ovvio mi stesse sfuggendo in quella
situazione.

«Scusa?».

«Pensi davvero che mio marito non sappia di noi? Che sarei così crudele? Credevo
che avessi più di fiducia in me».

«Sa del tempo che abbiamo passato sul \emph{vostro} letto? E gli sta bene?».

«Mio marito ha una parte del mio cuore. Non il mio corpo».

«Ma siete sposati!» protestai.

«No, non lo siamo. Mi hai mai visto portare la fede? Si è guadagnato il titolo
di ``marito'' perché mi è accanto da dieci anni, e insieme abbiamo affrontato
molte, moltissime situazioni, compreso il nostro ingresso qui».

Quella era forse la stranezza più grande con cui mi fossi mai confrontato, e
proprio per questo era anche la prova più importante. E io l'avevo fallita. Mi
vergognai della mia stupidità.

«Se qualcuno di là sentisse questa storia...» sussurrai.

«...penserebbe che siamo malati». Si mise a sedere sull'erba, e io feci
altrettanto. «Ma da quando l'amore ha smesso di essere piacere di stare accanto
a una persona ed è diventato desiderio di possederla? È assurdo che le coppie si
sposino! La stessa volontà di sposarsi implica che non ci sia fiducia, e dunque
amore!».

«Sposarsi significa giurare di amarsi davanti a Dio».

«Già sai come la penso su Dio: non esiste. O meglio, non esiste nella forma in
cui lo intendono la maggior parte delle persone. E anche se esitesse, non vorrei
che fosse lui a governare i miei sentimenti. Amare significa dare senza
aspettarsi nulla in cambio: si sposa chi ha paura di essere abbandonato dalla
propria metà. Ma se mio marito non mi volesse più vicino a sé, io non vorrei che
soffrisse e fosse costretto a rimanere con me, proprio perché continuerei ad
amarlo!».

Era qualcosa a cui avevo sempre pensato anch'io, ma non avevo mai avuto il
coraggio di affermarlo pubblicamente, anche perché non ero sicuro del mio stesso
ragionamento. Tutti si sposavano, dunque quell'azione doveva avere un senso!
Possibile che nel XXI secolo si continuasse a praticare un'attività così
palesemente sbagliata?

«È per questo che esistono i divorzi» tentai di protestare, perché non volevo
lasciarmi convincere troppo facilmente.

Lei esplose in una fragorosa, sana risata.

«Peggio ancora! Non vedi l'ipocrisia in tutto questo? Ci si sposa giurando amore
eterno, ma poi si ricorre alla legge per annullare quel giuramento! In questo
modo un rituale già privo di significato perde ulteriormente valore!».

«Dunque tu cosa suggerisci? Che tutti abbiano un rapporto aperto come il tuo?».

«No, mi rendo conto che non a tutti può andare bene. Ma in tutti i rapporti
dovrebbe essere elemento fondamentale la sincerità. Il matrimonio, per sua
stessa natura, non crea un ambiente favorevole al dialogo».

Un'altra lezione. Perché con quella donna ogni conversazione si trasformava in
un dibattito sulle cose della vita?

«Dopo quello che ti ho detto, vuoi che continuiamo a vederci? Oppure urta il tuo
orgoglio di uomo?».

Lo disse con tono provocatorio e ironico. Io non risposi.

«Bene,» proseguì «vorrà dire che tornerò tra le braccia del mio comprensivo
marito». E si alzò, andandosene.

Rimasi a riflettere per qualche minuto. L. era la migliore amica e compagna che
potessi desiderare. Sarebbe stato terribilmente stupido lasciarla andare così.
Non importava che amasse anche un altro uomo: solo la consapevolezza di essere
nel suo cuore era immensamente gratificante.

Amare senza chiedere nulla.

Mi alzai e inseguii quella figura ormai piccola. Quando la raggiunsi, le cinsi
la vita. Lei mi baciò su una guancia e sorrise.

«Lo sapevo» disse, e in quel momento le prime goccie di pioggia caddero sulle
nostre teste.

Corremmo verso la sua casa. La nostra casa.

\chapter{Cambiamenti}

\paragraph{}
«Sono incinta».

Lo disse tutto d'un tratto, senza preavviso, proprio come accade nei film
smielati che io non guardo perché trovo troppo banali.

Quella, effettivamnete, \emph{era} una situazione banale. Un fatto fisiologico,
naturale: lei era una donna sana, io un uomo sano, facevamo l'amore da
diverso tempo e ora lei aspettava un bambino. Io non avevo mai pensato di
parlarle della questione.

Non dissi nulla. Non sapevo cosa dire. Andai alla finestra del mio appartamento,
e improvvisamente ricordai che L. non c'era mai stata. Avrei dovuto capire che
qualcosa non andava quando mi aveva chiesto di passare per un caffè.

Ora pioveva, e l'odore di quel caffè riempiva l'aria, e quella donna mi aveva
appena comunicato una notizia che era la causa del mio improvviso tremore.

«Mi rendo conto che non te l'aspettavi e, onestamente, neanch'io. Non ti sto
chiedendo niente; non voglio che ti senta costretto a prendere una decisione,
ma mi sembrava giusto che lo sapessi. Se hai qualcosa da dire, ti prego, fallo».

Un altro lungo silenzio. Nessuno dei due guardava l'altro. Il rumore della
pioggia era assordante.

L. si alzò.

«Bene, allora vado».

«Cosa farai?».

«Non lo so ancora».

«E tuo marito? Cosa gli dirai?».

«La verità. È l'unica cosa che merita. Sono certa che capirà.

«Ho paura. Scusa. Non ero pronto. Non sono pronto».

Lei mi abbracciò. Era spaventata. Lo sentivo.

«Non preoccuparti, andrà bene» disse sorridendo. Ma era un sorriso amaro, non
sereno come sempre. Probabilmente L. già intuiva che quella gravidanza avrebbe
scatenato una lunga catena di eventi.

Sapeva che le sue certezze stavano per crollare.

\paragraph{}
Ovviamente quella notte non riuscii a chiudere occhio. Non capitava dai miei
primi giorni nella Società. Era ormai un anno che ogni sera mi addormentavo
tranquillo come un bambino, senza angosce o dubbi nel cuore.

Credo fossero circa le tre del mattino quando udii il trambusto. Le urla si
placarono immediatamente, e lasciarono il loro posto a un costante rumore di
umanità, come se molte persone si fossero unite a me nella veglia notturna. Ero
ancora vestito, così uscii e chiesi a un passante cosa stesse accadendo.

«Qualcuno vuole andarsene».

«E qual è il problema? Se vuole andarsene, che se ne vada».

L'uomo guardò a terra, chiaramente a disagio.

«Non è esattamente così che funzionano le cose qui, ragazzo: solo i membri
anziani possono entrare e uscire liberamente; tutti gli altri devono dare una
motivazione, e ``Mi sono stufato'' non è accettata. E... in ogni caso, a nessuno
è permesso mollare: una volta che sei dentro, sei dentro per sempre».

Quella conversazione mi permise di comprendere gli avvertimenti che L. mi aveva
lanciato quando ci vedevamo al teatro. Ma perché non era stata più esplicita?
Perché tutti conoscevano quella regola tranne me?

Poiché in quel momento lei era l'ultima persona a cui volessi pensare, decisi di
rimandare a un altro momento le riflessioni.

Mi incamminai seguendo gli altri. Le strade non erano illuminate --- nessuno si
sarebbe sognato di aggredire un proprio pari nella Società --- così inciampai
diverse volte.

Qualcuno si trovava davanti al pesante cancello verde, che era saldamente
chiuso. Di fronte a lui stava una donna, appoggiata alle sbarre, con le braccia
conserte e un malvagio sorriso di scherno sul viso. Certamente ricopriva una
posizione influente: si comportava con la sicurezza di chi è nel proprio
ambiente ed è consapevole del proprio potere.

«Lasciami passare, Nadia!» intimava lui, visiblmente agitato e irritato.

«Mi dispiace, ma davvero non posso farlo. Sapevi quali erano le regole, le hai
accettate; ora devi convivere con la scelta che hai fatto».

Ero riuscito ad avvicinarmi abbastanza da riconoscerlo: si trattava del marito
di L., e con lui stava il figlio. Quando lo vidi una fitta allo stomaco mi
tolse il respiro, come se qualcuno mi avesse colpito con violenza. Temevo di
conoscere il motivo di quell'improvviso desiderio di libertà: L. era stata
sincera, e ora lui, disgustato, voleva andare via.

«È inaccettabile! Questo è un sequestro di persona! Vi farò arrestare tutti!».

Nadia rise crudelmente.

«Per farlo dovrai prima uscire di qui».

Lui la guardò con aria di sfida.

«Ci riuscirò».

La donna si fece subito seria. Si allontanò dal cancello e andò via.

Quando gli passò davanti disse con voce roca: «Per il tuo bene e quello di tuo
figlio, spero veramente di no».

Quindi scomparve nell'oscurità.

L'uomo corse verso le sbarre di acciaio, vi si aggrappò e le scosse con
violenza, provocando un tremendo rumore e urlando: «Fatemi uscire di qui,
bastardi!». Ma assolutamente nulla accadde.

Il bambino sembrava molto spaventato, non l'avevo mai visto così. Provai pena
per lui.

La folla si faceva sempre meno numerosa: i membri della Società Alternativa
tornavano alle loro perfette vite, colme di ispirazione e arte.

Rimanevamo ormai io e pochi altri.

Lui impiegò qualche minuto per arrendersi. Sospirò e decise di lasciar perdere.
Quando i nostri occhi si incrociarono mi guardò con odio. Non era l'odio di un
adolescente lasciato dalla fidanzata, ma quello di un uomo maturo ed esperto cui
è stata tolta la propria ragione di vita.

Abbassai lo sguardo, vergognandomi come un assassino; ora che ci penso, non
avrei dovuto: non avevo nulla di cui sentirmi in colpa.

E poi, se non avessi guardato a terra, forse avrei visto arrivare il pugno.
Mi colpì dal basso, subdolamente, sul sopracciglio. Fu abbastanza forte da farmi
cadere, sbattendo violentemente la testa. L'ultima cosa che vidi fu il cielo
stellato.

\paragraph{}
«Ahia».

«Scusa».

Ero steso su un divano, e L. temponava la ferita al sopracciglio. Non l'avevo
vista al cancello; ciò non poteva che essere un bene.

Mi alzai di scatto e avvertii un forte dolore alla testa.

«Non fare movimenti bruschi. Il dottore sarà qui tra poco».

«Dove sono tuo marito e tuo figlio?».

Scosse la testa.

«Non lo so, ma non possono essere andati lontano. Non credo che torneranno a
casa, però». Sospirò. «Mi dispiace tanto: non hai colpa in tutto questo».

«Neanche tu».

«Invece sì. La colpa è mia e di nessun altro. La gravidanza è stata la goccia di
troppo, ma già da molto tempo mettevo alla prova la devozione di mio marito, per
esempio con le mie lunghe assenze».

«Ma non mi hai detto che amare significa non aspettarsi nulla in cambio?».

«Sì, ma vuol dire anche rispettare l'altro. Lui ha accettato il mio stile di
vita perché mi amava, ma non è mai stato d'accordo. Ha sbagliato nel non
dirmelo, e io sono stata una pessima moglie perché non me ne sono accorta».

Il dottore disse che la ferita non era grave, ma sarei dovuto stare a riposo per
qualche giorno.

L'indomani appresi che il marito di L. era tornato alla sua vita comune:
stranamente Nadia l'aveva lasciato andare via quella stessa mattina.

A L. non chiesi nulla in merito.

Lei non mi cercò.

\chapter{Libertà vigilata}

\paragraph{}
Un pomeriggio L. venne da me. Era adirata e confusa.

«L'hanno ucciso,» ripeteva «mio Dio, l'hanno ucciso».

«Calmati. Chi hanno ucciso?».

«Mio marito. L'hanno ucciso. Nadia dice che è stato lui ma non è vero! Lei ha
organizzato tutto! E io lo amavo. Lo amo ancora! L'hanno ucciso perché voleva
parlare, e ora uccideranno mio figlio per paura di quello che potrebbe dire!».

Pensai che fosse impazzita.

«Sei sconvolta. Non pensi che dovresti dormire e tornare sull'argomento quando
sarai più lucida? Voglio dire, è quasi un anno che sono qui, e mi sembra che
tutti i membri della Società si ispirino a principi di fratellanza e amore».

«Non è così. Non hai idea di chi sia dietro a quest'organizzazione. E per tutto
questo tempo io li ho aiutati. L'ho ucciso io!».

Non avevo mai visto nessuno piangere in quel modo. Accucciata sul divano, con la
testa appoggiata al mio petto, L. iniziò a singhiozzare disperatamente. Io non
parlavo.

Dopo più di due ore terminarono le lacrime e le andò via la voce; lei però
continuò a tremare. Quando la vidi così desiderai la morte: non riuscivo a
smettere di pensare che ero la causa di tutto.

Le accarezzavo i capelli e sussurravo parole dolci, ma nulla sembrava poter
placare la sua sofferenza. Impiegò un'altra ora per addormentarsi.

Tra i singulti riuscii a dirmi che, secondo Nadia, il marito si era ucciso due
settimane dopo essersene andato, lasciando un biglietto dove le attribuiva la
colpa di quel gesto. Due settimane in cui L. non ne aveva mai parlato, come se
lo avesse rimosso dalla propria memoria.

Il figlio era stato affidato ai nonni, giacché la madre era irrintracciabile.

\paragraph{}
Quando la sera ci colse eravamo ancora in quella posizione. Ogni muscolo del
mio corpo era addormentato, ma non osavo muovermi per non svegliarla.

Riflettevo sulle sue parole. Chi c'era dietro la Società Alternativa? Una setta?
La Massoneria? Gli Illuminati? Avevo sempre sentito raccontare molte cose sul
loro conto, alcune vere, altre totali idiozie.

E che cos'avevo firmato prima di entrare nella Società? Anch'io mi ero impegnato
a restare lì per il resto dei miei giorni? Ero prigioniero di quel luogo come
tutti gli altri?

E ancora, se la Società aveva in realtà altri scopi, quali erano? Certo non il
lucro: mantenere tante persone non era che un costo. Un costo enorme, e dunque
il burattinaio, chiunque fosse, doveva essere molto ricco.

Pensai a tutte le stranezze di quel posto, alla realtà in cui vivevano i suoi
abitanti, convinti che il mondo esterno non li riguardasse affatto. Solo allora
mi resi conto di tutte le allusioni che avevo ignorato, gli atteggiamenti che
avevo finto di non vedere. Pensai alla segretezza che ci era richiesta: non ci
era dato sapere dove ci trovassimo precisamente, né potevamo parlare con
qualcuno della nostra affiliazione.

L. dormì poco; quando si svegliò era calma.

Continuava a piovere. Ormai la dolce primavera che dominava al mio arrivo aveva
fatto spazio per un malinconico inverno, che spegneva ogni entusiasmo e faceva
venir voglia di passare le proprie giornate nel letto.

«Mi è sempre piaciuta la pioggia» disse, con la bellissima voce ormai ridotta
a un sussurro. «Quando fuori c'è un temporale e io sono dentro casa, al caldo,
mi sento protetta e immortale. Tuttavia sono anche consapevole che, se uscissi,
mi bagnerei fino alle ossa».

«Dove vuoi arrivare?».

«Finora io ho vissuto dentro casa mentre fuori imperversava la tempesta. Avevo
paura di bagnarmi, perché mi hanno insegnato a evitare l'acqua. Ma non mi sono
accorta che l'abitazione nella quale mi rifugiavo era divorata da un incendio
che, centimetro dopo centimetro, avrebbe distrutto tutto. Temevo un raffreddore,
ma stavo andando incontro all'incenerimento».

«Cos'hai intenzione di fare?».

«Devo andare via di qui».

«Vengo con te».

La frase venne fuori istintiva. I suoi occhi si illuminarono.

«Davvero? Lo faresti?».

Esitai --- non sarebbe stato semplice riadattarsi --- ma il dubbio fu subito
vinto.

«Questo posto non è quello che sembra. E io ti accompagnerei ovunque».

Mi abbracciò.

«Dobbiamo fare in fretta,» disse poi «il cerchio intorno a mio figlio si stringe
sempre di più. Se gli succedesse qualcosa a causa delle scelte che ho fatto non
potrei mai perdonarmelo».

Così L., che mi aveva introdotto nella Società Alternativa, ora chiedeva il mio
aiuto per uscirne.

O forse era lei ad aiutare me.

\paragraph{}
Il piano era semplice quanto ambizioso: uscire dal cancello principale,
chiedendo il permesso, per non tornare mai più. L. avrebbe addotto una
motivazione qualunque, per esempio la ricerca di nuovi talenti. Quanto a me,
poteva dire che voleva insegnarmi a vedere la ricchezza nell'animo delle
persone, in modo che potessi seguire le sue orme. Pensava che nessuno avrebbe
protestato, dato che era un membro anziano e aveva quel diritto.

Si sbagliava. L'uscita le venne negata dall'uomo cui si era rivolta.

«È assurdo! Sono un membro del Consiglio da diversi anni! Dove vuoi che vada?».

«Dato che ne fai parte sai anche che il Consiglio può, per una valida ragione,
limitare le libertà di alcuni membri per proteggere gli altri».

«E quale sarebbe questa ragione?».

«Temiamo la fuga di informazioni».

«Se la temete vuol dire che avete qualcosa da nascondere. Siete stati voi,
vero?».

Quello allargò le braccia. «A fare cosa?».

«A uccidere mio marito! Abbiate almeno il coraggio di ammetterlo, ipocriti
vigliacchi!».

La sua voce si incrinò. Le strinsi la mano.

«Ecco di cosa parlavo» disse, colmo di finta compassione. «Sei sconvolta per la
sua morte; potresti dire qualunque cosa».

Lei scosse la testa, amareggiata.

«Credevo che avresti capito. Ci conosciamo da molto tempo».

«Sai che la decisione non spetta a me. Proprio perché siamo vecchi amici, non
rendere tutto più penoso».

«Va bene. Tanto per sapere, tu hai votato a favore o contro il mio
mprigionamento?».

«Sai che il voto è segreto».

Ma stavamo già andando via quando lo disse.

Senza neanche voltarsi L. rispose: «Come immaginavo».

\paragraph{}
«Non sapevo che ci fosse un Consiglio» dissi mentre tornavamo a casa sua.

Camminava in fretta. Per il bene di suo figlio era passata dalla disperazione
alla risolutezza.

«Il Consiglio è l'organo che prende tutte le decisioni. È composto da dodici
membri, sei uomini e sei donne».

«E quella Nadia che ho visto la sera in cui tuo marito voleva andarsene? Anche
lei ne fa parte?».

«Sì. Ed è mia sorella».

Ecco cosa c'era di tanto familiare in lei! Aveva gli stessi occhi penetranti,
solo neri come la notte invece del marrone autunnale di L.

«Mi sembra una persona crudele».

«Ricordi il fuoco di cui parlavo? Lei si è scottata».

Apprezzai la metafora.

«Adesso cosa facciamo?».

«Evitiamo di fare sciocchezze. Io ho ancora qualche asso da giocare: non tutti
mi hanno voltato le spalle. Almeno spero».

\chapter{Ritorno}

\paragraph{}
Nei tre giorni successivi L. sfruttò tutte le sue conoscenze per tentare di
evadere da quel luogo, ma senza successo. Alcuni non potevano aiutarla, altri
avevano paura, altri ancora, addirittura, minacciarono di denunciarla al
Consiglio.

La notte del terzo giorno, finalmente, qualcuno decise di darci una mano.
Qualcuno da cui davvero non pensavo di riceverla.

Eravamo a casa di L. e fuori un brutto temporale stava contribuendo a peggiorare
il nostro umore. Un tuono squarciò l'aria proprio nel momento in cui il
campanello suonò. Entrambi non ci aspettavamo che qualcuno venisse a trovarci:
la voce si era sparsa rapidamente e ora, per tutti, eravamo dei traditori.

«Tutti ti vogliono bene finché tifi per la loro squadra» aveva detto lei quando
si era accorta che nessuno si degnava più neanche di salutarci.

Per qualche secondo non ci muovemmo.

«Forse vogliono controllare se siamo in casa?» suggerii.

«Ti assicuro che non hanno bisogno di suonare il campanello per saperlo»
commentò.

Pigramente raggiunse il citofono e chiese chi fosse. La risposta dovette
sorprenderla, perché cambiò subito espressione. Esclamò solo «oh».

Quando la porta si spalancò, e fece il suo ingresso il vecchio che L. sembrava
rispettare tanto, rimasi a bocca aperta. Era vestito proprio come la prima volta
che l'avevo visto: completamente di nero. Nera la camicia, neri i pantaloni,
nere le scarpe. I capelli, questi invece bianchissimi, erano elegantemente
pettinati.

«Dobbiamo muoverci, non c'è tempo».

L. aveva già raccolto le poche cose che voleva portare, compresa la foto di sé
con il marito che teneva sulla libreria. Io mi avvicinai subito a lui.

«Credevo che lei non stesse qui».

«È che non mi piace farmi vedere in giro» rispose mentre dava un'ultima occhiata
alla casa e chiudeva la porta alle nostre spalle. «Piacere,» aggiunse tendendomi
la mano «sono Aaron».

Uno dei quattro uomini che lo accompagnavano --- erano divisi in due macchine,
e in una terza, immaginai, dovevamo salire noi --- si avvicinò e lo prese per
il braccio.

«Signore, bisogna andare. Rimandate le presentazioni».

Quando sistemò la giacca notai che era armato: alla cintura portava una pistola.
Non mi sorpresi: Aaron, chiunque fosse, era certamente il genere di individuo
che aveva bisogno di guardie del corpo. E ora che ci aveva aiutati la sua
incolumità sarebbe stata ulteriormente a rischio.

«Sì, certo. Forza, salite in macchina».

Lui si mise alla guida.

Pioveva quando ero arrivato, e pioveva ora che me ne stavo andando. Che ce ne
stavamo andando.

Il cancello si aprì senza problemi. Era assurdo come fosse semplice entrare e
uscire da quel posto, se solo ci si accompagnava alle persone giuste.

Dopo quasi mezz'ora abbandonammo il cumulo di nuvole. Il complesso era alla
nostra sinistra, lontano, e spaventosi lampi lo illuminavano di tanto in tanto,
facendo luce sugli oscuri segreti che la Società celava.

Giungemmo infine alla pista d'atterraggio costruita nel bel mezzo del nulla.
Un piccolo aereo ci aspettava, forse lo stesso che ci aveva portati lì.

Mentre L. saliva a bordo, Aaron la guardò dritta negli occhi e disse: «Mi
dispiace per tuo marito. Era un brav'uomo».

Lei fece solo un cenno con la testa.

\paragraph{}
Aaron e i suoi uomini partirono con noi. L. si addormentò poco dopo il decollo,
sfinita dagli avvenimenti e dalle emozioni di quei giorni. Io volevo sapere di
più sull'uomo che sembrava tanto importante quanto pericoloso e aveva deciso di
aiutarci.

Fui diretto.

«Chi è lei?».

«Chi sono? È una domanda vaga. Ho dedicato la mia vita a cercare la risposta,
e penso di non averla ancora trovata. Immagino però che non sia questo che vuoi
sapere».

«No, infatti. Quello che voglio sapere è qual è il suo ruolo nella Società
Alternativa».

L'uomo sospirò profondamente, come se rispondere gli costasse una immensa
fatica.

«Io l'ho fondata, molto tempo fa».

Certamente era l'ultima risposta che mi aspettavo.

«Se l'ha fondata, perché ha deciso di aiutarci? Non è stato proprio lei a
organizzare questo complotto?» chiesi, improvvisamente sospettoso.

«Assolutamente no» disse sdegnato. «Non era questo che volevo quando le ho dato
vita. La Società mi ha tradito tanto quanto ha tradito voi due. Il suo fine
rimane quello di creare un mondo ideale, ma i mezzi sono cambiati. Come possiamo
creare un mondo dove non esista l'omicidio tramite l'omicidio?».

«È per questo che ci ha salvati?».

«Per questo, e anche perché conosco la tua amica,» e indicò L. che dormiva
tranquilla «da quando era una bambina. Quando ho fondato la Società è stata una
delle prime persone a cui ho chiesto di entrare. Non potevo lasciare che
accadesse qualcosa a lei o alla sua famiglia».

Immaginai L. che si lasciava guidare e istruire da qualcuno. Anch'ella, dunque,
aveva avuto bisogno di un maestro!

Pensai di essere in buone mani. Mi sentivo, per la prima volta da quando avevo
scoperto la verità, al sicuro: Aaron e i suoi amici avrebbero pensato a tutto.

«La ringrazio».

Lui scosse la testa.

«Non ringraziarmi ancora. Fallo quando sarete tutti lontani, compreso il
bambino».

«Teme che possano raggiungerlo?».

«Sicuramente l'hanno già raggiunto,» e quella prima frase mi fece gelare il
sangue nelle vene «ma ancora non hanno fatto nulla: è ben custodito da certi
miei conoscenti. Però non posso proteggerlo in eterno: dovete andare via finché
siete in tempo».

\paragraph{}
Quando l'aereo atterrò chiesi ad Aaron dove ci trovassimo.

«In Inghilterra; troppo vicino a una delle sedi della Società. Siamo qui solo
per prendere il bambino e andarcene».

«Una delle sedi? Dunque ce ne sono diverse?».

«Quattro che io sappia, e probabilmente altre della cui esistenza sono tenuto
all'oscuro».

L. era ormai sveglia e comprensibilmente nervosa, anche se cercava di non farlo
vedere. Il portellone si aprì con lentezza estenuante, e un centimetro alla
volta compariva suo figlio, accompagnato dagli uomini di Aaron. Appena le fu
possibile gli corse incontro e lo abbracciò, baciandolo più volte sui capelli e
sulla fronte. Il bambino era confuso, sembrava non rendersi nemmeno conto della
situazione.

Aaron guardava la scena sorridendo.

«Come avete fatto a prenderlo?» gli domandò L.

«Abbiamo dovuto rapirlo. Mi dispiace per il trauma che abbiamo causato ai tuoi
genitori, ma è meglio che non sappiano nulla. Dovete sparire».

Ci chiese di allontanarci, in modo che potesse parlare sola con il figlio.

Dopo qualche minuto tornarono da noi.

«Sono pronta. Andiamo».

\paragraph{}
L'aeroporto era affollato: persone che andavano e venivano, uomini d'affari,
famiglie in gita...

Faceva freddo, come in tutti gli aeroporti.

Aaron comprò due biglietti per la prima destinazione che fosse abbastanza
lontana. Con noi non avevamo nulla, a parte i passaporti e la voglia di
lasciarci tutto alle spalle.

Solo noi e il resto del mondo. Finalmente.

Arrivò il momento di salutarsi. E il freddo parve farsi più intenso.

L. strinse Aaron con quanta forza aveva in corpo.

«Grazie di tutto. Se non fosse stato per te, ora non so dove mi troverei, né
dove si troverebbe mio figlio. Ho condotto così tante persona alla rovina,»
continuò «e loro non se ne rendono neanche conto. Quando accadrà, potranno mai
perdonarmi?».

«Tu stessa sei stata a un passo dalla rovina. Non hai nulla da farti perdonare.
Vai, e concentrati su tuo figlio».

L'uomo mi strinse la mano.

«Giurami di proteggerla, qualunque cosa accada. Io ho fatto lo stesso molto
tempo fa, e farò del mio meglio per mantenere il giuramento, ma non sarò più
così vicino».

«Lo giuro» dissi. «Perché non viene anche lei?».

«Vorrei tanto, ma devo finire quello che ho iniziato. La Società è un mostro che
è nato solo grazie alle mie idee. L'ho creata, e posso anche distruggerla. È
solo una questione di tempo. Ora andate. Buona fortuna».

Non era tipo da guardarci partire: si voltò e andò via.

Quando presentammo i passaporti che ci aveva dato poco prima, la donna dietro
il banco ci guardò sorridendo.

«Che bella famiglia! Andate in vacanza?» chiese.

«Al contrario,» disse L. «stiamo tornando solo ora».

\backmatter

\chapter{Epilogo}

\paragraph{}
Harris osservò di soppiatto la sua nuova collega: era giovane, bella e piena
di vita. Perché aveva scelto quel mestiere orribile? L'avrebbe presto logorata,
stancata, annoiata. Non ce l'avrebbe fatta. Alla sua età doveva occuparsi di ben
altre faccende.

«Mio Dio,» sussurrò la donna, guardando attraverso il vetro unidirezionale «quel
tizio è lì dentro da otto ore. Possibile che si possa scrivere per così tanto
tempo senza mai fermarsi?».

«A quanto pare sì. Non ha nemmeno chiesto un bicchiere d'acqua o qualcosa da
mangiare. Sembra posseduto».

L'uomo nella stanza si avvicinò a quello che sapeva non essere uno specchio e
bussò piano.

«Finalmente! Dici che vuole fare una pausa?».

Lo guardò --- Harris ne fu certo --- dritto negli occhi.

«Mi serve un'altra penna. Questa è finita».

\tableofcontents

\end{document}
