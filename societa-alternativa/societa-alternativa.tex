\documentclass[a4paper,12pt]{book}

\usepackage[italian]{babel}
\usepackage[utf8]{inputenc}
\usepackage{verse}

\frenchspacing

\newenvironment{dedication}
{
    \thispagestyle{empty}
    \begin{flushright}
}%
{
    \end{flushright}
}

\title{La Società Alternativa}
\author{Alessandro Desantis}

\begin{document}

\pagenumbering{roman}
\maketitle

\begin{dedication}
A L., che mi ha aiutato ad accettare l'oscurità.
\end{dedication}

\clearpage

\begin{verse}
\itshape{
``Tigre! Tigre! Divampante fulgore\\
Nelle foreste della notte,\\
Quale fu l'immortale mano o l'occhio\\
Ch'ebbe la forza di formare la tua agghiacciante simmetria?\\

``In quali abissi o in quali cieli\\
Accese il fuoco dei tuoi occhi?\\
Sopra quali ali osa slanciarsi?\\
E quale mano afferra il fuoco?\\
Quali spalle, quale arte\\
Poté torcerti i tendini del cuore?\\
E quando il tuo cuore ebbe il primo palpito,\\
Quale tremenda mano? Quale tremendo piede?

``Quale mazza e quale catena?\\
Il tuo cervello fu in quale fornace?\\
E quale incudine?\\
Quale morsa robusta osò serrarne i terrori funesti?

``Mentre gli astri perdevano le lance tirandole alla terra\\
e il paradiso empivano di pianti?\\
Fu nel sorriso che ebbe osservando compiuto il suo lavoro,\\
Chi l'Agnello creò, creò anche te?

``Tigre! Tigre! Divampante fulgore\\
Nelle foreste della notte,\\
Quale mano, quale immortale spia\\
Osa formare la tua agghiacciante simmetria?''
\/}
\end{verse}
\begin{flushright}
\emph{--- W. Blake}
\end{flushright}

\chapter{}

\pagenumbering{arabic}

\paragraph{}
Conobbi L. durante una fase particolare della mia vita: ero un avvocato di
discreto successo, eppure non mi sentivo affatto felice; mi sembrava che le mie
giornate fossero insopportabilmente perfette, quasi soffocanti. Ero una persona
realizzata: facevo ciò che avevo sempre sognato, ma mancava quella scintilla
che mi permettesse di andare avanti, che rendesse ogni istante della mia
esistenza unico e speciale.

Non auguro a nessuno di provare una sensazione del genere: si sente che c'è
qualcosa che non va ma non si riesce a individuare esattamente il problema. Non
si può fare niente per migliorare ciò che si è, non importa quanto ci si
sforzi di pensare a una vita migliore. È un dolore così opprimente che neanche
piangere sembra appropriato di fronte a cotanta sofferenza.

Dicevo, ero un avvocato di successo (nonostante la mia giovane età) e un
giorno venne da me questa donna. Non era la stessa che mi sarei abituato a
vedere dopo diverso tempo: più ordinaria, meno potente. Anche così però si
portava dietro un'aura di energia ovunque andasse. ``Energia''... era quello
l'elemento mancante nella mia vita; era a causa della sua assenza che stavo
tanto male. Non me ne resi subito conto, ovviamente, perché lei stava ben
attenta a fare in modo che solo chi era pronto si accorgesse della sua
grandezza. Solo quando si accorse che ero la persona adatta si lasciò andare
e io la potei ammirare in tutta la sua maestosità.

All'inizio si trattava di piccole consulenze. Mi chiedevo perché una donna
tanto bella si interessasse di un argomento noioso come il diritto. Un giorno mi
azzardai a chiederle che lavoro facesse.

«Io?» chiese sorpresa. «Beh, si potrebbe dire che sono un'artista, credo.»

Non feci altre domande, perché capii che ancora non voleva spingersi tanto
oltre. Ma più tempo passavo con lei e più ero intossicato dai suoi
atteggiamenti, dalla sua esistenza.

\paragraph{}
Poco dopo il nostro primo incontro iniziai a scrivere. Non narravo apertamente
di lei, forse perché mi vergognavo all'idea che qualcuno che conoscevo appena
potesse avere un ascendente così forte su di me che ero un professionista. Ma
rileggendo i miei scritti capivo che si trattava di L. E di me, ovviamente.

Lasciavo le cartelle in ufficio perché così potevo uccidere quello sprazzo di
noia che esisteva tra un cliente e l'altro, quando il lavoro non mi impegnava
e non potevo pensare a qualcosa che non fosse l'imperfezione della mia vita
un po' meno imperfetta da quando quella donna vi era entrata.

Quel pomeriggio dovevo incontrare L. per l'ennesimo consiglio ed ero
terribilmente in ritardo; lei invece era già nello studio, puntuale come
sempre. Sembrava arrivare sempre al momento giusto: lei era presente ovunque ci
fosse da attuare un cambiamento.

Entrai di corsa e incrociai per un istante il suo sguardo.

Aveva in mano una delle mie cartelle.

«Scrivi molto bene» osservò.

Non mi aveva mai chiesto il permesso di darmi del ``tu''; lei non chiedeva mai
il permesso di fare qualcosa: lo faceva e basta. Era dolcemente arrogante.

Ovviamente la cosa non mi infastidì: non potevo che essere contento che il
nostro rapporto stesse diventando più intimo.

La ringraziai arrossendo.

Tentai di parlare di lavoro ma non me lo permise.

«Vedi,» mi interruppe «io faccio parte di un gruppo di artisti, e mi
piacerebbe che ti unissi a noi. Sarebbe un enorme contributo.»

«Veramente non saprei... ho parecchio da fare in questo periodo» tentai di
evitare il confronto diretto: avevo paura in un certo modo di L., di ciò che
sarebbe potuto accadere se fossi entrato nel suo strano mondo; era come se non
mi sentissi all'altezza. Lei era così potente e decisa e io così infinitamente
debole e nudo. Tuttavia l'idea di essere parte di qualcosa di più grande mi
affascinava: mi sentivo come un bambino che gioca col fuoco e sa che in ogni
momento potrebbe bruciarsi, e ha paura di provare dolore ma ancora più paura di
rimanere ignorante.

«Avanti, avvocato,» disse con voce suadente «le prometto che non se ne
pentirà.»

Notai che aveva ricominciato a darmi del ``lei''.

Alla fine accettai e non controvoglia, solo con un po' di timore.

«Benissimo» rispose L. «Ora possiamo parlare di lavoro.»

\paragraph{}
Alla fine di quell'incontro mi lasciò un indirizzo e una data. Si rifiutò di
dirmi dove sarei andato. «Lo vedrai quando sarà il momento» disse.

I giorni si succedettero in fretta, e il tanto atteso (e temuto!) momento
arrivò più presto di quanto mi aspettassi. Un'ora prima dell'incontro ero
nervoso come non mai, ed enormemente tentato di chiamarla e disdire tutto. Non
lo feci forse solo perché avevo paura di sentire la sua voce dopo tanto tempo:
era un po' infatti che L. non si faceva federe allo studio, e credevo si fosse
dimenticata di me.

Ormai rassegnato, ma eccitato al tempo stesso, mi avviai in macchina verso
l'indirizzo datomi. Era piuttosto lontano dal centro e sulle mappe non era
segnato assolutamente nulla in quel luogo. Per un istante mi balenò nella mente
la folle idea che quella fosse solo una truffa: già mi vedevo tramortito e
derubato di tutto, una volta giunto sul posto. Mi resi conto subito di quanto
fosse idiota un simile pensiero: una donna bella come L. non poteva certo
rendersi complice di simili nefandezze.

Grande fu il mio stupore quando finalmente arrivato mi trovai a parcheggiare
sotto un teatro. Sembrava vecchio e in disuso, ma ben curato da una mano attenta
e precisa. Mi parve di riconoscere il tocco di L. fermo e delicato al tempo
stesso.

All'interno l'aria sapeva di poltrona eppure era familiare, rassicurante,
invitante. Nonostante ci fossero diverse sale seppi subito dove andare: forse fu
perché ormai ero così assuefatto a L. che l'avrei trovata ovunque, come un
segugio. Forse più semplicemente seguii le voci.

All'interno della stanza si trovavano lei e un'altra decina di persone.

Proprio in quel momento una donna bionda e grassoccia in piedi sul palco stava
leggendo quella che presumetti essere una sua poesia. Rimasi in piedi ad
ascoltarla, ma L. mi fece segno di sedermi accanto a lei in prima fila. La
salutai; lei sorrise e posò il dito sulle labbra, chiedendomi di fare silenzio.

Quando ebbe terminato tutti applaudirono. Uno dopo l'altro ognuno, alla destra
di L. si esibì nella propria arte. Credevo fossero tutti scrittori ma mi
sbagliavo: c'era chi cantava, chi ballava e chi mostrava le sue ultime
fotografie.

Infine anche lei si esibì. Salì sul palco, meravigliosa nel suo abito, e
accompagnata da un'orchestra nell'angolo della sala cantò con una delle
più meravigliose voci che avessi mai ascoltato. Le parole non possono
descrivere le sensazioni che quei suoni mi hanno lasciato, dunque mi limiterò a
dire che quando ella finì provai la stessa tristezza che si prova svegliandosi e
interrompendo a metà uno stupendo sogno.

Nessuno la applaudì: sembravano tutti stupiti quanto me, sebbene immaginai
non fosse la prima volta che ascoltavano quella voce soave. Al termine L. non
scese dal palco ma vi indugiò qualche momento, tenendo gli occhi chiusi e la
testa bassa, le labbra incurvate in un impercettibile sorriso che pure tutti
sentivamo grande come l'Universo.

«Oggi, amici,» chissà perché, quella parola mi colpì «si è unito a noi un
nuovo artista; uno scrittore per la precisione. L'ho conosciuto per caso,
come è accaduto con la maggior parte di voi, e per caso ho scoperto il suo
talento. Lo vorrei qui accanto a me, sul palco, con uno dei suoi scritti che
sono sicura avrà portato.»

Tutti si voltarono a guardarmi e proprio come accadde con L. in ufficio,
arrossii. Perché mi trovavo in quella situazione assurda? Quasi non ricordavo
neanche come ci ero finito! Abbozzai un sorriso ebete, mi alzai e con le mani
sudate presi dalla tasca un foglietto, piegato tante volte da sembrare una
particella subatomica, sul quale era riportato uno dei miei ultimi racconti.

«Prego» disse L. sorridendomi rassicurante e facendosi da parte. Avevo
l'impressione che conoscesse il mio stato d'animo in quel momento e avesse
deliberatamente deciso di mettermi in imbarazzo; quello che non capivo però
era il motivo: qual era il suo scopo? Che cosa sperava di ottenere?

Così, con voce incrinata dall'imbarazzo e dall'emozione lessi quelle
poche, interminabili righe. Man mano che procedevo l'aria si faceva sempre
più calda e pesante. Quando infine la tortura cessò mi sembrò di liberarmi
di un pesante fardello, qualcosa che mi portavo dietro da anni senza neanche
saperlo, qualcosa che era venuto a galla nel mare della mia anima solo quel
pomeriggio.

«Molto bene» disse L. Sembrava fiera di me, come una madre lo è del proprio
figlio.

Più tardi, quando gli altri se ne furono andati e ci trovammo da soli, le
parlai dei problemi che avevo avuto. «Non capisco,» dissi «sono abituato a
parlare in aula, sotto pressione, davanti a tutti. Eppure rendere gli altri
partecipi di ciò che ho scritto mi è costato un'enorme fatica.»

«Forse sei un buon avvocato che sa come usare le prove a proprio favore. Forse
come dici tu sei abituato a esporre i fatti in aula. Ma non sei abituato a
esporre la tua anima. È proprio qui il punto: è per questo che ho voluto che tu
leggessi davanti a noi, stasera. Ora, o mai più. Ci siamo passati tutti, non
preoccupartene troppo.»

«Tutti tranne te, a quanto pare. Sembri così naturale quando sei sul palco e
canti...»

«Ho dovuto lavorare anch'io per avere una tale... dimestichezza con le
persone; le persone sono complicate, ma meravigliose a loro modo.»

«Forse hai ragione» annuì. «Sei tu a capo di questo gruppo?»

«Non c'è un capo, io sono solo la persona di riferimento.»

Parlammo ancora per un po', quindi ci salutammo e ognuno andò per la propria
strada.

\paragraph{}
A quell'incontro ne seguirono altri, simili ma mai uguali: a volte veniva
qualcuno di nuovo e anch'egli doveva passare quel rito di iniziazione del
quale ero stato partecipe solo qualche settimana prima. Vedendo come erano
impacciati capivo dove avevo sbagliato fino a quel momento. Non che ora fossi
diventato una stella del cinema: ero ancora ben lontano dal raggiungere i
livelli di L.; tuttavia mi sembrava, in qualche modo, infinitamente più facile
vivere, vedere, respirare.

Mi stavo sensibilizzando: iniziavo a cogliere la bellezza, più spesso la
bruttezza del mondo intorno a me quando prima ero apatico. Mi sembrava di
poter capire meglio i sentimenti altrui, le motivazioni dietro alle azioni e le
idee ancor più dietro.

C'era qualcosa però che mi angosciava terribilmente: anche gli incontri col
gruppo stavano diventando un'abitudine, e dunque iniziavo a perdere
interesse. Pensai di essere io quello sbagliato e incontentabile, sempre alla
ricerca di stimoli fuori dalla mia portata. Ero terrorizzato all'idea di
stufarmi di L., dei suoi sorrisi e della sua voce.

Ad un incontro le parlai delle mie paure.

Non sembrava sorpresa, e io non ero sorpreso che lei non lo fosse.

«Mi dispiace che ti senti così» disse.

Le dissi che a mio parere non era lei il problema, che pensavo di non essere
adatto alle azioni ripetitive: il gruppo era una novità all'inizio ma ormai
iniziava a essere un obbligo, uno statico imponimento.

«L'uomo ha bisogno di azioni ripetitive. Ciò di cui non ha bisogno è la
routine: respiri, altrimenti soffocheresti, ma non ne soffri. Mangi, perché
altrimenti moriresti di fame, eppure non te ne lamenti; non lo fai perché
queste sono necessità. Ma annoiarsi non è una necessità, tutt'altro! È
quando l'azione smette di essere dettata dall'istinto di sopravvivenza che
diventa routine, capisci?»

Non ne ero troppo sicuro, ma risposi di sì.

«Adesso cosa farai?»

«Non lo so ma non smetterò di venire agli incontri.»

«Non avevo dubbi, ma io intendevo adesso, ora, in questo preciso momento.
Salirai in macchina e poi?»

«Beh, andrò a casa, cenerò guardando la televisione, e andrò a dormire.»

«Benissimo, allora non hai fretta: andiamo.»

«Dove?»

«A fare una passeggiata; il mare non è lontano.»

\paragraph{}
Poco oltre il teatro, in effetti, si trovava una bellissima spiaggia, di cui io,
stranamente, non mi ero mai accorto in tutto quel tempo. Glielo dissi, e lei
sorrise. «Nessuno se ne accorge mai, finché non ne ha bisogno» commentò.

Giungemmo sul limitare della strada, là dove finiva il duro e pungente cemento
e iniziava invece la morbida e candida sabbia.

L. si tolse le scarpe e mi invitò a fare altrettanto. Esitante, la imitai.

Mi prese sottobraccio e camminammo così per un'ora almeno, avanti e indietro
su quella spiaggia che sembrava non aver ancora subito il triste intervento
dell'uomo. Parlammo di molte, moltissime cose: religione, amore, filosofia, e
anche delle piccole sciocchezze di ogni giorno. Non capivo cosa stesse cercando
di ottenere e non mi importava affatto: avrei solo voluto che quel momento non
finisse mai.

Al ritorno però mi aspettava una spiacevole sorpresa: la mia auto era sparita.
L. sembrò non accorgersene, tanto che stava per andare via, lasciandomi solo.

«Mi hanno rubato la macchina!» urlai verso di lei, già lontana.

Si voltò stupita e tornò sui suoi passi.

«Come?» disse ridendo.

«La mia macchina... è sparita!»

«Ah, lo vedo». Sembrava infinitamente esilarata. «Beh, va bene così.»

«Va bene così?! Stai scherzando, vero? E io come ci torno a casa?»

«Vai a piedi, non può che farti bene.»

«Ma io abito a venticinque chilometri da qui!»

«Beh, allora trova un passaggio.»

«Ho lasciato il telefono in macchina!»

«Come ho già detto: trova un passaggio.»

E voltandosi si allontanò.

La chiamai a gran voce più volte ma lei finse di non sentirmi.

Quella sera tornai a casa con un'allegra famigliola costituita da padre, madre
e due figli. Ero piuttosto irritato per come L. mi aveva trattato, sebbene
sapessi che non l'aveva fatto solo per danneggiarmi. All'improvviso ero
stufo dei suoi giochi di potere e di quei suoi continui manipolamenti; l'idea
di non avere la situazione in pugno mi dava la nausea. Tuttavia non avrei
smesso di vederla neanche se avesse tentato di uccidermi; volevo solo farle
capire quanto il suo gesto mi avesse fatto adirare.

Pareva però che i colpi di scena non fossero ancora finiti: quando dopo quelle
estenuanti fatiche (non tanto fisiche quanto mentali) riuscii a vedere la mia
tanto amata casa, trovai la macchina parcheggiata nel vialetto. Dovetti
strofinarmi gli occhi più e più volte prima di convincermi che non si trattava
di un'allucinazione. Dunque qualcuno aveva rubato la macchina al teatro
quindi l'aveva riportata a casa mia. Per quale motivo? Dall'auto non mancava
niente: la giacca con portafoglio e cellulare non era stata toccata e nessuno
dei documenti nella mia valigetta. L'unica ragione che poteva spingere
qualcuno a commettere quell'imbroglio era... obbligarmi a chiedere aiuto, a
fidarmi del prossimo per tornare a casa.

Ormai era ovvio che si trattava di lei: potevo quasi vederla mentre architettava
il suo piano, probabilmente ben prima di quell'incontro, e lo spiegava al suo
complice.

Li immaginavo all'opera.

\paragraph{}
``Fuori dal teatro, questa sera. Alle nove.''

L'Sms giungeva inaspettato, ma sapevo chi l'aveva mandato.

A lungo pensai a cosa sarebbe stato giusto fare: avrei potuto ignorarlo, e
l'orgoglio mi spingeva a farlo. In quel modo però avrei perso
un'occasione. Oppure potevo mettere da parte l'Achille che era in me e
incontrare L. per vedere cos'altro aveva da offrirmi.

Il dubbio mi strusse fino alle otto e quaranta, quando decisi che il mio animo
ferito poteva essere barattato con una vita felice. Ero ancora indeciso quando
la vidi proprio davanti al teatro, pensierosa; sembrò non accorgersi nemmeno
del mio arrivo. Nel momento in cui mi avvicinai si svegliò da un sogno: mi
afferrò per il braccio dirigendosi verso l'auto dalla quale ero appena
sceso.

«Sei in ritardo.»

Decisi che quello era il momento di sfogarmi. Ora o mai più, mi dissi.

«Meravigliati che sia venuto dopo quello che hai combinato l'altra sera con
la macchina!»

«Lascia perdere la macchina, dobbiamo andare.»

La mia spietata determinazione si spense come la fiamma di una candela muore a
causa di un'improvvisa folata di vento.

«Andare dove?»

«Hai detto che volevi qualcosa di più, no? Questa è la tua occasione. Guidi
tu.»

Immaginavo che mi avrebbe portato ancora più lontano dalla città, invece mi
indicò una via del centro. Non avevo idea di cosa sarebbe successo ma sapevo
che qualunque cosa fosse avrebbe cambiato la mia vita; ero dunque nervoso,
emozionato, terrorizzato, ma allo stesso tempo sicuro di me stesso, perché
sapevo che L. non mi avrebbe lasciato cadere.

O lo avrebbe fatto al solo scopo di afferrarmi per i capelli.

Non disse nulla per tutto il viaggio e non rispose alle mie continue e
insistenti domande.

Dopo mezz'ora circa ci fermammo sotto un blocco di lussuosi appartamenti: vi
abitavano personaggi del mondo dello spettacolo, così come alcuni tra i
principali esponenti politici del momento. Il portone si aprì non appena ci
avvicinammo e L. lo tenne aperto per me. Mentre eravamo in ascensore mi guardò a
fondo e disse qualcosa che mi turbò ulteriormente.

«Stasera ti sarà chiesto di prendere una scelta. Non è importante cosa
decidi, almeno non per me. È di fondamentale importanza però che non parli
con nessuno di ciò che hai sentito. Se lo farai potrebbe esserci tolta la
possibilità di aiutare altre persone, persone come te. Hai capito?»

Allucinato, la guardavo.

«Hai capito?» chiese di nuovo scuotendomi.

Mossi leggermente la testa, prima su, poi giù, come se volessi dire sì; in
realtà però non era proprio un sì. Non troppo convinto, almeno.

\paragraph{}
È curioso come il nostro cervello ricordi i luoghi che visita per il loro
odore. Ma se dovessi sforzarmi di far tornare alla mente l'immagine di
quell'immenso appartamento al dodicesimo piano non ricorderei l'odore,
perché non ne aveva; ricorderei bene però il pavimento di marmo sul quale
mi specchiai appena entrato. L'immagine che vidi era quella di un uomo
indeciso, impaurito ma felice; un'espressione che non avevo mai vista dipinta
sulla faccia di nessuno fino a quel momento.

Il posto sembrava deserto. L. però si diresse con sicurezza verso un altro
ambiente, uno studio con un'immensa vetrata che permetteva di ammirare la
città nel momento in cui il sole cala e le prime luci si accendono.

Davanti a questa, in piedi, stava un uomo non più giovane, di sessant'anni
circa, vestito elegantemente. Aveva l'aria di essere una persona ricca ma
sobria, due qualità non facilmente conciliabili.

«Sei qui» disse L. senza muoversi dalla soglia della porta.

«Oh, finalmente» rispose quello seccato. «Temevo vi foste persi.»

Non si voltò per guardarci ma potevo vedere il riflesso del suo viso nel
vetro: mi colpirono più di ogni altro dettaglio i suoi occhi, stanchi,
distratti, indici di una mente rivolta altrove. In mano teneva un calice di vino
bianco o forse spumante.

L. mise una mano sulla mia schiena e mi spinse avanti, ma ella non si mosse; mi
inquietava quella nervosa immobilità di lei che era sempre a proprio agio. Era
come se il vecchio fosse tanto importante o pericoloso da non potercisi
permettere alcun errore in sua presenza.

«Questo è l'amico di cui ti avevo parlato» mi presentò.

«Pensi che sia pronto?»

«Sì. Credo di sì.»

«Venite di là: qui mi sento osservato.»

Ci guidò verso un terzo ambiente, ancora più grande di quello in cui ci
trovavamo prima. C'era una scrivania di mogano tanto grande quanto costosa
posizionata proprio di fronte alla vetrata, la quale copriva probabilmente tutto
l'appartamento. Accanto a questa stava una libreria anch'essa di mogano
nella quale erano contenute decine di volumi; pur essendo un buon lettore
riconobbi pochi titoli perché erano quasi tutti in lingue straniere, per la
maggior parte orientali.

L'uomo si sedette su una costosissima poltrona in pelle mentre io e L.
rimanemmo al di qua della scrivania, così che finalmente ebbi modo di vederlo
bene. Era vestito completamente di nero ed effettivamente incuteva un certo
timore. Con i suoi lunghi capelli bianchi e la barba ispida sembrava un dottore
o un alchimista, non so bene quale delle due gli si confacesse meglio.

Egli prese una penna e un foglio dalla risma davanti a sé e me li porse.

«Firma.»

«Ma non c'è scritto niente!» protestai.

Guardò accigliato L., che sembrava costernata.

«Ti prego, firma» mi implorò anche lei.

Il colmo per un avvocato dev'essere firmare un foglio completamente bianco.

Eppure io lo feci, e non perché fossi convinto, ma perché la pressione era
troppa e non volevo deludere L.

Allora l'uomo prese dalle mie mani penna e foglio e iniziò a scrivere
qualcosa in bella grafia proprio sopra la mia firma.

``Almeno ora so che non è un medico'' pensai divertito.

Si alzò di scatto facendomi trasalire.

«Molto bene. Possiamo andare. Hai portato il necessario per il viaggio?»

«Viaggio? Che viaggio?»

Tutta la faccenda iniziava a preoccuparmi.

L. posò le mani sulle mie spalle e mi guardò dolcemente.

«Ti fidi di me?»

Esitai parecchio prima di rispondere: mi fidavo ciecamente di lei ma avevo
paura di quali sarebbero state le conseguenze delle mie parole. Nonostante
tutto però mai e poi mai avrei rischiato di rattristarla.

«Sì.»

«Allora non fare altre domande; lo dico per te, è questo che volevi: rischiare
tutto per una nuova vita. Non è così? Devi scegliere usando l'istinto, e
ciò sarebbe impossibile se ti spiegassi ogni dettaglio. Del resto anche
volendo non potrei farlo: il futuro non mi è molto più chiaro di quanto lo
sia a te in questo momento. Tuttavia sappi che qualunque sia la tua decisione
stasera non potrai assolutamente tornare indietro.»

Tentai di deglutire, ma avevo la gola completamente secca.

«Va bene.»

«Va bene... cosa?»

«Verrò con voi.»

Avevo cercato di rimandare il più possibile quel momento ma ormai il passo era
fatto. E forse era un passo decisamente troppo lungo per la mia gamba.

L. e l'uomo, animandosi all'improvviso, uscirono dall'appartamento,
trascinando me dietro. Salimmo su un lussuosissimo Suv parcheggiato poco lontano
dall'edificio.

«Ma che ne sarà della mia macchina? E della mia casa?»

«Certo che questa macchina non ti dà pace!» rise L.

«Non ne avrai bisogno dove stiamo andando» aggiunse l'uomo.

«Perché? Dov'è che stiamo andando?»

Nessuno mi rispose, ma non me ne preoccupai troppo: avevo accettato l'invito
di due (quasi) sconosciuti a iniziare una nuova vita con loro... Che importava
se ancora non sapevo dove? Non era che un fattarello di poco conto.

Mi appoggiai comodamente al sedile tentando di rilassarmi e curiosamente la
cosa sembrava funzionare. Pensai che il mio destino dipendeva ormai da forze di
gran lunga superiori alla mia, e reazioni che io stesso avevo messo in moto e
non potevano più essere fermate. Ero come una spiga di grano in balia del
vento. Dolcemente in ostaggio.

Ormai era sera inoltrata e la città era completamente illuminata. Feci un
rapido bilancio della mia vita fino a quel momento: avevo un lavoro, avevo molti
soldi, ero stimato e rispettato. Eppure non ero soddisfatto e stavo scappando
da tutto.

Iniziava a piovere; i colori delle insegne visti attraverso le gocce d'acqua
formavano strani giochi di luce. L'uomo guidava tenendo gli occhi fissi sulla
strada, mentre L. sul sedile del passeggero rifletteva silenziosa, forse
pensando al mio destino.

Si avvertiva nell'aria un sentimento pesante: nostalgia e malinconia, il tutto
mischiato a una buona dose di eccitazione. Era il clima ideale per una partenza.

L. si girò e allungò la mano chiara verso di me: sul palmo stava una compressa
lunga un centimetro circa.

«Prendila.»

«Cos'è?»

Silenzio.

«Devi...»

«...fidarmi di te. Sì, ho capito.»

Raccolsi la compressa da quel palmo delicato, la misi in bocca e deglutii. Pochi
minuti dopo fui pervaso da un improvviso torpore e chiusi gli occhi,
stanchissimo.

L'ultima cosa che ricordo è lo sguardo vivace e accorto di L. nello
specchietto retrovisore.

\paragraph{}
Motori. Furono la prima cosa che udii agli albori della mia nuova vita: motori
che si spegnevano. Mentre aprivo gli occhi, lentamente perché la luce passava
da ogni spiraglio e sembrava farli sanguinare, giunsero alle mie orecchie anche
alcune voci.

«Si sta svegliando» disse qualcuno.

«Giusto in tempo» ridacchiò un altro.

Non avevo idea di cosa mi fosse successo e non riuscivo a pensare né a
ricordare nulla per via del mal di testa; mi sembrava di essere come un neonato
espulso dal ventre materno che si affaccia sul mondo con curiosità e timore.

«Chi siete? Dove sono?»

Un immenso uomo di colore dall'aria minacciosa mi si avvicinò.

«Quanto alla prima domanda,» rispose con voce profonda «non hai bisogno né
sei tenuto a saperlo. Per quanto riguarda la seconda invece la risposta è
``su un aereo'', ma non credo che ti soddisfi molto. In questo momento non sei
in grado di ragionare con lucidità: tra qualche minuto ti sarà spiegato tutto
meglio.»

Mi accorsi di trovarmi su quello che a giudicare dagli interni era un jet
privato, appena atterrato nel bel mezzo del nulla. Ovunque fossimo atterrati
pioveva anche lì. Adesso però era giorno, anche se le nuvole grigie dense e
minacciose coprivano il cielo. Potevo sentire il ticchettio dell'acqua contro
il metallo e il vetro. Il portellone era aperto e dalla scaletta vidi scendere
pilota e copilota, entrambi elegantemente vestiti, dopo avermi lanciato
un'occhiata furtiva.

«Lei dov'è?»

Stavolta fu un altro a rispondermi; sedeva di fronte a me e teneva le gambe
accavallate, una mano a sostenere il mento, un leggero sorriso dovuto
probabilmente al mio stato di smarrimento.

«Di chi parli?»

Non ebbi l'occasione di rispondere perché in quel momento L. si
materializzò alla mia sinistra; era in piedi e mi posò affettuosamente una
mano sulla spalla. I due uomini la guardavano con rispetto.

«Bentornato tra noi!» rise. «Vieni, dobbiamo andare; senza fretta, ormai.»

Slacciai la cintura e mi alzai lentamente, perché avevo paura che le gambe
cedessero da un momento all'altro. Non c'era traccia dell'uomo che ci
aveva condotti fino all'aereo; pensai che non fosse salito con noi.

«Dove siamo?» chiesi nuovamente.

«Ormai non ho più segreti da nasconderti, ma ti prego di pazientare ancora
qualche attimo: allora potrai vedere con i tuoi occhi.»

Una volta fuori dall'aereo, mentre scendevo lentamente gli scalini sostenuto
da L., qualcuno aprì un ombrello sulle nostre teste. In fondo ci attendevano due
fuoristrada che vennero occupati dagli uomini e in mezzo una terza auto,
anch'essa costosissima, nella quale stavamo io e L.

Il viaggio durò più di mezz'ora, eppure io e L. non parlammo granché,
complice anche la mia tremenda stanchezza. Le chiesi cosa mai mi avesse dato per
farmi stare così, anche se intuivo già la risposta.

«Un sonnifero.»

«Immaginavo.»

«Non è nulla di personale, solo non volevamo che vedessi la strada.»

«La strada per arrivare fin qui? Come avrei potuto?»

«La strada per arrivare all'aeroporto. E devi ringraziare me se ora non te ne
abbiamo dato un altro... I miei ``colleghi'' non sono del tutto d'accordo
con questa mia scelta, ma io mi fido di te.»

«Ma perché tanta segretezza?»

«Lo scoprirai stando con noi. Per ora sappi che quello che cerchiamo di
introdurre è un grande cambiamento, e non a tutti piacciono i cambiamenti. I
nostri nemici sono potenti, politicamente parlando, e potrebbero distruggerci se
non fossimo così prudenti.»

Fu la fine della nostra conversazione perché avevo bisogno di silenzio per
pensare ai guai in cui mi ero cacciato.

Il paesaggio non era molto vario ma piacevole: dal nulla cosmico ora eravamo
entrati in un bosco e seguivamo un perfetto sentiero che lo attraversava da
parte a parte. Gli alberi finivano ma la strada proseguiva curvando dolcemente
per svariati chilometri.

L. guidò ancora per qualche minuto, poi si fermò davanti ad un cancello.

Quando lo vidi mi mise un po' di timore: sembrava che pochi uscissero di lì
una volta entrati, e di certo lo facevano con l'intento o l'obbligo di
tornare presto. Il cancello si aprì poco prima che la prima auto del corteo
passasse senza neanche accennare a frenare; evidentemente ci stavano aspettando.

Al di là facevano ombra sulla strada altri alberi, alti e dalle foglie larghe,
verdi e grasse; la pioggia aveva portato fino a noi l'odore della loro corteccia
mischiato a quello della terra così piacevole e rassicurante. Quando l'ultima
auto passò il cancello si chiuse alle nostre spalle. Ancora non sapevo se
l'avrei mai rivisto o se il destino mi avrebbe portato a desiderare di farlo.

Ma ciò che mi aspettava oltre gli alberi era qualcosa a cui, nonostante tutto
ciò che mi era successo in quegli ultimi giorni, non ero minimamente preparato.

Vidi case, parchi, fontane, ma soprattutto persone: ai margini della strada ci
attendeva una folla sorridente che salutava con la mano: uomini, donne, bambini,
animali... talvolta intere famiglie erano lì e ci fissavano con quegli sguardi
pieni di benevolenza.

Oltre quel cancello c'era un'intera piccola città, colorata e meravigliosa.

L. entrò nel vialetto di una delle case e si fermò lì mentre le altre due
auto proseguirono. Scendemmo ed ella frugò nella borsa per poi tirarne fuori un
mazzo di chiavi.

«È la tua nuova casa,» disse «spero che tutto sia di tuo gradimento. È già
pronta e abbiamo messo dei vestiti della tua taglia nei cassetti e negli
armadi. Se hai qualche problema puoi chiedere in giro: le persone qui non
aspettano altro che darti una mano. Io vado a cambiarmi; sarò qui tra due ore
circa e ti accompagnerò a casa mia dove cenerai per stasera. Ti consiglio di
fare una doccia: hai un aspetto orrendo.»

Detto ciò tornò in macchina.

«Ah, dimenticavo» aggiunse sorridendo prima di andare via. «Benvenuto in
Irlanda.»

%\tableofcontents

\end{document}
