\chapter{Rinunciare allo sfruttamento}
\label{ch:rinunciare-allo-sfruttamento}

Tutte le considerazioni fatte sopra portano a una conclusione: è necessario, se degli individui
devono essere sfruttati per il bene della nostra specie, che questi individui siano non umani. Non
ci siamo ancora fermati a considerare, però, se questo sfruttamento sia effettivamente necessario o
se possiamo farne a meno grazie al progresso che abbiamo raggiunto. Se infatti potessimo, sarebbe
nostro preciso dovere rinunciarvi per limitare la sofferenza degli animali non umani.

Oggi, purtroppo, gli animali vengono ancora impiegati (per necessità) in diversi campi. Oltre che a
fine alimentare, ne facciamo uso per la sperimentazione di farmaci e per il progresso della ricerca
di base. Queste attività non riducono solo la sofferenza umana ma anche quella animale: i farmaci
per uso veterinario, per esempio, vengono sperimentati sugli animali. Poiché dal sacrificio di pochi
individui deriva il benessere di molti, è preferibile continuare sulla strada che stiamo seguendo.

Fortunatamente esistono molti metodi complementari alla sperimentazione animale che ci permettono di
ridurre al minimo l'impiego degli animali nella ricerca. I ricercatori sono in effetti obbligati,
legalmente e moralmente, a usare un metodo complementare ovunque sia possibile, in accordo col
principio delle tre R (\emph{replacement}, \emph{reduction}, \emph{refinement})\footnote{Russell,
W.M.S., Burch, R.L. \emph{The Principles of Humane Experimental Technique}. Methuen, London 1959.}.

Non esiste, però, alcun metodo alternativo. Sembra difficile se non impossibile, peraltro, che si
possa sviluppare un modello di sperimentazione in tutto uguale all'organismo animale: se così fosse,
infatti, avremmo di fatto creato un animale e saremmo in balìa delle stesse complicazioni etiche che
ci troviamo ad affrontare oggi. Di contro, un modello dissimile dall'originale non è sempre utile ai
fini della sperimentazione.

Coloro che chiedono l'abolizione della sperimentazione animale, oltre a considerare i danni che
questo avrebbe sul benessere degli animali, dovrebbero anche fermarsi a riflettere sulla moralità
dell'inazione. Se mi rifiutassi di aiutare un uomo colto da un infarto sul ciglio della strada,
difficilmente potrei essere ritenuto innocente. Perché allora dovrebbe essere morale rinunciare alla
ricerca che potrebbe portare alla cura di patologie gravi?
