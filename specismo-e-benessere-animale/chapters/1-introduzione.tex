\chapter{Introduzione}
\label{ch:introduzione}

Gli esseri umani hanno raggiunto un tale livello di sviluppo da potersi permettere il lusso di
preoccuparsi e impegnarsi attivamente non solo per il proprio benessere, ma anche per quello delle
altre specie e dell'ecosistema. Ne è una chiara prova la nascita di organizzazioni ambientaliste e
animaliste, così come la crescente quantità di filosofi contemporanei e non che si sono battuti e si
battono per un migliore trattamento degli animali non umani.\footnote{Peter Singer, Jeremy Bentham e
Paola Cavalieri sono tra i più importanti.}

Sebbene in alcuni casi questa battaglia possa essere considerata giusta (è nell'interesse comune
porre un freno al folle sfruttamento delle risorse che stiamo praticando dagli anni della
rivoluzione industriale), alcune di queste associazioni giungono talvolta a conclusioni piuttosto
discutibili, arrivando ad affermare l'uguaglianza di tutti gli animali e pretendendo che l'uomo si
consideri alla pari delle altre specie e ne cessi dunque lo sfruttamento.

A mio parere, si tratta di una posizione fondamentalmente incompatibile con il proseguimento della
specie umana e di un'improduttiva resistenza al naturale istinto di conservazione. Pur essendo
auspicabile una limitazione del nostro impatto sull'ecosistema, pensare di poter annullare
quest'impatto è quantomeno utopico, se non addirittura ipocrita e ingenuo.
