\chapter{Lo specismo in natura}
\label{ch:lo-specismo-in-natura}

In natura è assolutamente normale favorire la propria specie rispetto alle altre e lavorare perché
questa prosperi. È solo una delle tante forme sotto cui si manifesta l'istinto di conservazione. I
leoni non cacciano altri leoni in primo luogo perché sarebbe poco pratico e in secondo luogo perché
non sarebbe una strategia di sopravvivenza efficace. Attaccano i propri simili solo per garantire la
preservazione del proprio bagaglio genetico, ossia quando l'istinto di sopravvivenza individuale
entra in conflitto con quello di conservazione della specie\footnote{È prassi comune per il maschio
uccidere i cuccioli della femmina in modo che questa torni in calore. Il maschio riesce così a
preservare il proprio bagaglio genetico}. Anche gli animali non umani dunque praticano lo specismo.

Inoltre, trattare la natura e l'uomo come se fossero due entità distinte non ha alcun senso. Non c'è
motivo per cui non dovremmo considerarci parte della natura insieme a tutte le altre specie. Dunque,
se anche quella umana fosse l'unica specie a praticare lo specismo, questo sarebbe comunque
naturale.

Ma supponiamo che i comportamenti appartenenti esclusivamente all'uomo siano contrari all'ordine
delle cose o ``contronatura'': dovremmo allora smettere di vestirci? di guidare automobili? di
esercitare la medicina? Dovremmo, insomma, rinunciare a millenni di progresso solo perché le altre
specie non hanno progredito? La risposta è chiaramente no. Possiamo quindi affermare che la tendenza
delle altre specie ad adottare certe pratiche è irrilevante se vogliamo derminare la correttezza
delle stesse.

Tuttavia, il fatto che lo specismo sia naturale non implica che sia anche moralmente accettabile.
Questo perché a un certo punto nella storia della nostra specie abbiamo deciso di autolimitarci,
rinunciando ad alcuni dei nostri istinti animaleschi, al fine di garantire a tutti gli esseri umani
una migliore convivenza. L'infanticidio è considerato crudele da buona parte dell'umanità; eppure
molte specie lo praticano quotidianamente.

La presenza in natura di una pratica non è, per questo, un fattore determinante nella scelta di
adottare o meno tale pratica, perché l'etica morale si configura, a mio parere, a un livello
superiore rispetto alla natura.
