\chapter{Conclusioni}
\label{ch:conclusioni}

È nostro preciso dovere, io credo, ridurre la sofferenza di tutti gli individui in grado di soffrire
al minimo indispensabile. Ma non tutti soffrono nella stessa maniera: l'autocoscienza fa sì che gli
esseri umani sperimentino il dolore in maniera più intensa rispetto agli animali non umani. Perciò è
preferibile sfruttare un animale non umano rispetto a un essere umano.

Ciò nonostante siamo obbligati, laddove possibile, a impiegare tutti i mezzi di cui disponiamo per
evitare tale sfruttamento. Queste cautele vengono già adottate nell'impiego degli animali ai fini
della sperimentazione e, parzialmente, nel loro impiego a fini alimentari. Quello che possiamo fare
è continuare questo meticoloso lavoro di riduzione e rifinitura nella speranza che un giorno potremo
cessare del tutto lo sfruttamento.

Rinunciare oggi all'impiego di animali non umani è invece impossibile: significherebbe fermare lo
sviluppo, portare alla sofferenza umana e animale. Dal benessere di un ristretto gruppo di individui
deriverebbe il dolore di molti altri, umani e non umani.
