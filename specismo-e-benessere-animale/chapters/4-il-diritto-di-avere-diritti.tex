\chapter{Il diritto di avere diritti}
\label{ch:il-diritto-di-avere-diritti}

Gli animalisti parlano spesso di diritti degli animali. Ma perché un individuo abbia il diritto alla
vita, è necessario che tutti gli altri individui rispettino il dovere di non
ucciderlo.\footnote{Hohfeld, W.N. \emph{Fundamental legal conceptions as applied in judicial
reasoning}. In \emph{The Yale Law Journal}. Giugno 1917 (vol. 26, n. 8), pp. 710-770.} Allo stesso
tempo, questi individui rispettano i propri doveri per ottenerne in cambio dei
diritti.\footnote{Questo ragionamento non giustifica la pena di morte. Non rispettando un dovere
altrui, infatti, non si perde il diritto corrispondente perché si è comunque potenzialmente in grado
di rispettarlo. Uccidere un omicida è immorale perché questo ha un'intera vita per rientrare nel
sistema diritti- doveri; ucciderlo significherebbe privarlo di questa opportunità.} Se nessuno
rispettasse i propri doveri e pretendesse esclusivamente il riconoscimento dei propri diritti, non
ci sarebbe più un equilibrio e l'intero sistema crollerebbe.

Se, per ipotesi, una scimmia uguale all'uomo per capacità cognitive pretendesse gli stessi diritti
dell'uomo, saremmo moralmente obbligati a garantirglieli. La scimmia infatti, potendo rispettare
tutti i doveri umani, potrebbe e dovrebbe beneficiare di tutti i diritti.

Ma in generale gli animali non umani non dispongono delle capacità cognitive necessarie per
rispettare un dovere e dunque non possono pretendere dei diritti.\footnote{Non è un caso che le
minoranze a cui in passato furono negati dei diritti combatterono in prima persona le battaglie per
ottenere tali diritti: quello degli animali non umani è l'unico caso in cui non sia la categoria
discriminata ad avanzare delle pretese.} Sono gli uomini che hanno regolato autonomamente la libertà
con cui possono disporre delle altre specie: uccidere un animale non è una violazione del suo
diritto alla vita ma un'infrazione della legge. All'animale, in altre parole, è stata assegnata
arbitrariamente la libertà\footnote{Per la definizione formale di ``libertà'' si veda \emph{Hohfeld,
W.N., op. cit.}} di vivere: può vivere senza rispettare alcun dovere. Questa libertà, però, ha delle
limitazioni: laddove l'animale sia necessario al progresso medico o comunque a ridurre la sofferenza
umana, può essere soppressa la sua libertà di vivere.
