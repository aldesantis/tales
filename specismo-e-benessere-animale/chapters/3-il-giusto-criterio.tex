\chapter{Il giusto criterio}
\label{ch:il-giusto-criterio}

Ci sono stati diversi tentativi nell'individuazione di un criterio che fosse universalmente e
coerentemente applicabile a tutti gli individui e che ci permettesse di determinare il valore di un
individuo e, di conseguenza, la possibilità di sfruttarlo per il bene comune.

Jeremy Bentham, il padre dell'utilitarismo, riteneva che la sola capacità di soffrire di un
individuo fosse l'unico fattore da tenere in considerazione nell'equipararlo ad altri individui:

\begin{quotation}

    The day has been, I am sad to say in many places it is not yet past, in which the greater part
    of the species, under the denomination of slaves, have been treated by the law exactly upon the
    same footing, as, in England for example, the inferior races of animals are still. The day may
    come when the rest of the animal creation may acquire those rights which never could have been
    witholden from them but by the hand of tyranny. The French have already discovered that the
    blackness of the skin is no reason a human being should be abandoned without redress to the
    caprice of a tormentor. It may one day come to be recognised that the number of the legs, the
    villosity of the skin, or the termination of the os sacrum are reasons equally insufficient for
    abandoning a sensitive being to the same fate. What else is it that should trace the insuperable
    line? Is it the faculty of reason or perhaps the faculty of discourse? But a full-grown horse or
    dog, is beyond comparison a more rational, as well as a more conversable animal, than an infant
    of a day or a week or even a month, old. But suppose the case were otherwise, what would it
    avail? The question is not, Can they reason? nor, Can they talk? but, Can they
    suffer?\footnote{Bentham, J. \emph{An Introduction to the Principles of Morals and Legislation}.
    1823, cap. 17, note.}

\end{quotation}

In risposta, si sente spesso dire che gli esseri umani possiedano un proprio valore e una propria
dignità intrinsechi. Tuttavia, senza ulteriore elaborazione, si tratta di un'affermazione dogmatica
e priva di fondamento, creata a posteriori per giustificare una superiorità provata inconsciamente.
Perché solo gli esseri umani dovrebbero avere valore intrinseco? Cosa li distingue, cosa li rende
unici tra tutte le specie animali?

Sebbene tutti i vertebrati e alcuni invertebrati siano in grado di provare dolore, non tutti gli
individui soffrono allo stesso modo. Gli esseri umani (e, in maniera limitata, alcuni altri grandi
primati\footnote{Patterson, F., Gordon, W. \emph{The Case for the Personhood of Gorillas}. In
Cavalieri, P., Singer, P. \emph{The Great Ape Project}. St. Martin's Griffin, pp. 58-77.}) sono
dotati di ciò che viene definita ``autocoscienza''; sono cioè consapevoli dell'esistenza di un io e
di come quest'io si relazioni con l'ambiente circostante e con gli altri individui, concepiscono e
fanno piani per il futuro.

L'autocoscienza ha a che fare anche con la percezione del dolore. Infatti, studi
neurologici\footnote{Murray, M. \emph{Nature Red in Tooth and Claw: Theism and the Problem of Animal
Suffering}. Oxford: Oxford University Press, 2008.} hanno dimostrato l'esistenza di tre diversi
livelli di percezione del dolore: (1) gli organismi di livello 1 hanno semplicemente una reazione di
fuga agli stimoli dannosi; (2) gli organismi di livello 2 sono in grado di percepire il dolore come
sensazione fisica; (3) gli organismi di livello 3 sono in grado di accorgersi che stanno provando
dolore.

La maggior parte degli animali non umani rientra nel livello 2: sebbene siano in grado di percepire
fisicamente il dolore, non concepiscono l'idea del dolore. Negli individui dotati di autocoscienza
invece il dolore non è semplicemente uno stimolo fisico: questi si \emph{rendono conto} di provare
dolore. Per esempio, un uomo investito da una macchina prova rimpianto e tristezza all'idea di non
essere riuscito a realizzarsi appieno, di non poter più rivedere le persone care, di non avere più
tempo per portare il proprio contributo al mondo. Non si può pensare di poter paragonare queste due
dimensioni del dolore: gli organismi di livello 3 soffrono manifestamente di più rispetto a quelli
di livello 1 e 2.

Le affermazioni di Bentham, però, non sono del tutto prive di fondamento. Così come noi, infatti,
gli animali fuggono dal dolore fisico. È quindi nostro dovere ridurre al minimo indispensabile la
sofferenza a cui sono sottoposti. Usare un topo per l'avanzamento della ricerca biomedica è lecito
perché riduce la sofferenza umana che, come abbiamo visto, ha più valore rispetto a quella delle
altre specie. Torturare il topo è invece un'azione immorale perché non beneficia alcun individuo al
di fuori del torturatore.

Questa linea di ragionamento, però, solleva un'ulteriore questione: se il dolore fisico è l'unica
condizione da evitare negli animali non umani, è moralmente accettabile ucciderne uno senza causare
sofferenza? Quest'idea è alla base dei movimenti \emph{happy meat}. Persino Peter Singer, il padre
dell'antispecismo moderno, suggerì in un'intervista che potesse trattarsi di un'azione moralmente
giustificabile:

\begin{quote}

    If it is the infliction of suffering that we are concerned about, rather than killing, then I
    can also imagine a world in which people mostly eat plant foods, but occasionally treat
    themselves to the luxury of free range eggs, or possibly even meat from animals who live good
    lives under conditions natural for their species, and are then humanely killed on the
    farm.\footnote{Singer, P. In \emph{The Vegan}. 2007.}

\end{quote}

Del resto, di certo una mucca non ha piani per il futuro, né dei cari che soffriranno per la sua
mancanza o di cui soffrirà la mancanza, né l'idea di un aldilà. Che viva o che muoia in maniera
indolore, indipendentemente dall'esistenza di uno scopo per la sua morte, quindi, dovrebbe essere
indifferente.

Sebbene sia difficile trovare un'obiezione all'uccisione indolore, a fini alimentari, di un animale
che ha condotto una vita senza sofferenza, guarderemmo con orrore l'uccisione dello stesso animale
per puro divertimento. Ma si potrebbe ribattere che l'uccisione ingiustificata causi un danno
all'essere umano che avrebbe potuto invece usare l'animale per fini più nobili.

\section{L'argomento dei casi marginali}

L'argomento dei casi marginali viene spesso citato dagli antispecisti per dimostrare la fallacia
della logica specista. È stato presentato per la prima volta da Singer in un
articolo\footnote{Singer, P. \emph{Speciesism and moral status}. In \emph{Metaphilosophy}. Luglio
2009 (vol. 3-4), pp. 568-571.} ed è poi divenuto un famoso esempio di come la logica specista sia
incoerente e priva di fondamento.

Supponiamo che un uomo nasca con un deficit cognitivo talmente esteso da poter essere paragonato,
sul piano intellettuale, a un animale non umano. Cosa ci impedirebbe di trattarlo non come un essere
umano ma alla pari delle altre specie, ritenendoci dunque in diritto di sfruttarlo per i nostri
scopi, come il progresso della ricerca? Se infatti istintivamente siamo portati a tutelarlo proprio
come faremmo con un individuo normodotato, razionalmente dobbiamo trovare un argomento che possa
giustificare questa tutela.

Ci sono diverse considerazioni da fare. In primo luogo, se quest'uomo avesse dei cari, il suo
sfruttamento causerebbe loro del dolore. Danneggiando un caso marginale dunque, danneggeremmo
chiunque lo avesse a cuore. Poiché c'è un'alternativa che provoca meno sofferenza (usare un animale
non umano), siamo moralmente obbligati ad adottarla.

Inoltre, c'è sempre la possibilità che la scienza trovi una cura prima della morte dell'uomo, e che
dunque sfruttandolo gli negheremmo il futuro. Appare invece improbabile che riusciamo a inventare,
in tempi ragionevoli, un metodo per dotare gli animali non umani di autocoscienza; e anche se lo
inventassimo, le implicazioni etiche del suo impiego meriterebbero una discussione a parte.

Infine, quest'argomento è utilizzabile solo per via dell'esistenza di un ristretto numero di casi
marginali. Se questi casi non esistessero e tutti gli esseri umani fossero superiori in capacità
cognitive agli animali non umani, allora saremmo giustificati nel negare loro i privilegi che gli
animalisti invece pretendono. Appare dunque piuttosto curioso che un fattore completamente
indipendente dalla natura delle specie non umane possa influire in qualche modo sul loro stato
morale.

\section{L'argomento non egalitario raffinato}

Singer sostiene anche\footnote{Singer, P. op. cit., pp. 572-573} che non si possano negare dei
diritti fondamentali agli individui in base alle loro capacità (e, nel caso specifico,
all'autocoscienza) perché, applicando questo ragionamento in maniera coerente, si getterebbero le
basi per una discriminazione tra gli stessi esseri umani. Si potrebbe infatti obiettare che gli
esseri umani con un quoziente intellettivo particolarmente elevato soffrirebbero per via del proprio
sfruttamento più di quanto ne soffrirebbe un normodotato.

Ma che un quoziente intellettivo elevato ``amplifichi'' la percezione del dolore non è vero: non
esiste alcun fattore oggettivo che ci permetta di giudicare quale essere umano soffrirebbe di più
per via del proprio sfruttamento (a parte ovviamente i casi marginali che ho discusso prima),
essendo la percezione del dolore estremamente soggettiva.

E se anche tale fattore esistesse e alcuni individui provassero il dolore più intensamente di altri,
il loro sfruttamento sarebbe comunque immorale, in quanto tali individui avrebbero dei diritti. Tali
diritti non potrebbero e non devono essere negati in modo arbitrario: ha dei diritti chiunque possa
rispettare dei doveri, indipendentemente da considerazioni di altra natura.

\section{La razionalità delle emozioni}

Si potrebbe obiettare che le emozioni provate in seguito a una determinata azione siano un argomento
valido nella discussione etica riguardante l'azione. Curiosamente, questo ragionamento può essere
adottato sia per sostenere l'infondatezza dell'argomento dei casi marginali (il dolore provato in
seguito all'uccisione o allo sfruttamento di un altro essere umano sarebbe sufficiente, in tal caso,
a evitare simili azioni indipendentemente dalla capacità del caso marginale di percepirne le
conseguenze), sia per supportare l'abolizione dello specismo (il dolore che lo sfruttamento degli
animali provoca all'uomo è indice che questo debba essere abolito). Si tratta, a mio parere, di un
argomento fallace sotto diversi aspetti.

In primo luogo, così come quello dei casi marginali si regge in piedi solo grazie all'esistenza di
pochi individui con capacità cognitive estremamente ridotte, allo stesso modo l'argomento emozionale
esiste solo perché gli esseri umani sono naturalmente empatici. La maggior parte dei nazisti, per
esempio, non ha provato rimorso in seguito all'uccisione di milioni di ebrei nei campi di sterminio:
era anzi convinta di fare il bene dell'umanità. Se dunque non ci fosse stata un'etica morale a cui
appoggiarsi, o se non ci fossero stati giudici al di fuori dei perpetratori, non ci sarebbe stato
disgusto e l'azione sarebbe stata perfettamente lecita.

In secondo luogo, è errato credere che le emozioni non abbiano nulla a che fare con la razionalità;
molte prove portano invece a credere che le emozioni siano scatenate dalla ragione, anche se a prima
vista può non sembrare così\footnote{Załuski, W. \emph{Emotions and Rationality}. In LeDoux, J. et
al., \emph{The Emotional Brain Revisited}. Copernicus Center Press, Cracovia 2014, pp. 303-318.}.
Prendendo come esempio quello del genocidio ebreo appare chiaro che il resto del mondo non si sia
ribellato semplicemente perché l'azione era disgustosa; semmai è stata l'eclatante violazione dei
diritti umani a suscitare una reazione emotiva. In questo caso l'emozione ha agito da catalizzatore,
coadiuvando il ragionamento e l'azione che ne è seguita.

Allo stesso modo la felicità, la tristezza, la paura e tutte le altre emozioni sono scatenate dalla
ragione, sia che lo realizziamo razionalmente sia che lo ignoriamo. La felicità non è che una
reazione al piacere ovvero a una situazione favorevole; la tristezza è una risposta alla sofferenza
ovvero a una situazione sfavorevole; la paura ci mette al riparo dai pericoli. Tutto questo fa
pensare che le emozioni non siano che un'evoluzione darwiniana dei nostri istinti e che funzionino
dunque secondo dinamiche ben precise. In particolare, tutte le emozioni si riconducono all'istinto
di sopravvivenza. \footnote{Quella sul fine ultimo e dunque sulla razionalità della vita è una
discussione più adatta ad altre sedi. Il fine di questo articolo non è individuare una motivazione
per l'esistenza.}

Di tanto in tanto proviamo emozioni scatenate da motivazioni illogiche, emozioni esagerate che ci
portano a danneggiarci o a danneggiare gli altri, ma questo non è certo un segno dell'irrazionalità
di \emph{tutte} le emozioni; è semplicemente frutto della natura limitata dell'essere umano, che
tende all'errore e al pregiudizio nelle emozioni così come nell'etica morale.

Le emozioni non razionali non sono dunque accettabili in una discussione etica. Se lo fossero,
sarebbe impossibile assegnare la ragione o il torto con sufficiente certezza, poiché non ci sarebbe
alcun criterio oggettivo di cui discutere e qualunque dibattito perderebbe significato.
