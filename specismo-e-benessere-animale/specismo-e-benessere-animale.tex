\documentclass[a4paper,oneside,11pt,article]{memoir}

\usepackage[utf8]{inputenc}
\usepackage[italian]{babel}
\usepackage[T1]{fontenc}
\usepackage{XCharter}
\usepackage{import}
\usepackage{microtype}
\DisemulatePackage{setspace}
\usepackage{setspace}

\pagestyle{plain}
\frenchspacing
\raggedbottom
\sloppy
\onehalfspacing

\title{
\textsc{Specismo e benessere animale}\\
\vspace{1.5mm}
\small{La ricerca di un compromesso accettabile.}
}
\author{Alessandro Desantis}
\date{}

\begin{document}

    \begin{titlingpage}

        \maketitle

        \begin{abstract}
            \begin{center}
                Alcuni filosofi e associazioni animaliste sostengono la necessità di
                rinunciare alla discriminazione tra specie ed estendere anche agli animali non
                umani alcuni diritti fondamentali come quello alla vita e alla libertà. In
                questo testo cerco di individuare un criterio oggettivo, razionale e replicabile
                con cui si possa valutare la sofferenza di un individuo in seguito al proprio
                sfruttamento e dimostro come tale criterio debba necessariamente discriminare
                tra animali umani e non umani.
            \end{center}
        \end{abstract}

    \end{titlingpage}

    \pagenumbering{arabic}

    \import{chapters/}{1-introduzione.tex}
    \import{chapters/}{2-lo-specismo-in-natura.tex}
    \import{chapters/}{3-il-giusto-criterio.tex}
    \import{chapters/}{4-il-diritto-di-avere-diritti.tex}
    \import{chapters/}{5-rinunciare-allo-sfruttamento.tex}
    \import{chapters/}{6-conclusioni.tex}

\end{document}
