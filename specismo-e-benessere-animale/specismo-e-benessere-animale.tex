\documentclass[a4paper,11pt,oneside,article]{memoir}

\usepackage{cmap}
\usepackage[italian]{babel}
\usepackage[T1]{fontenc}
\usepackage[utf8]{inputenc}

\frenchspacing

\title{\textsc{Specismo e benessere animale} \\ \vspace{1mm} \small{La ricerca di un compromesso accettabile.}}
\author{Alessandro Desantis}
\date{}

\begin{document}

\maketitle

\begin{abstract}

\noindent Alcuni filosofi e associazioni animaliste sostengono la necessità di
rinunciare alla discriminazione tra specie ed estendere anche agli animali non
umani alcuni diritti fondamentali come quello alla vita e alla libertà. In
questo testo cerco di individuare un criterio oggettivo, razionale e replicabile
con cui si possa valutare la sofferenza di un individuo in seguito al proprio
sfruttamento e dimostro come tale criterio debba necessariamente discriminare
tra animali umani e non umani.

\end{abstract}

\chapter{Introduzione}

Gli esseri umani hanno raggiunto un tale livello di sviluppo da potersi
permettere il lusso di preoccuparsi e impegnarsi attivamente per il benessere
non solo proprio, ma anche delle altre specie e dell'ecosistema. Ne è un'ovvia
testimonianza la nascita di organizzazioni ambientaliste e animaliste così come
la crescente quantità di filosofi contemporanei e non che si sono battuti e si
battono per un migliore trattamento degli animali non umani.\footnote{Peter
Singer e William Bentham sono tra i più importanti.}

Sebbene in alcuni casi questa battaglia possa essere considerata giusta (è
nell'interesse comune porre un freno al folle sfruttamento delle risorse che
stiamo praticando dalla rivoluzione industriale), alcune di queste associazioni
giungono talvolta a conclusioni piuttosto discutibili, arrivando ad affermare
l'uguaglianza di tutti gli animali e pretendendo che l'uomo si considerari alla
pari delle altre specie e ne cessi dunque lo sfruttamento.

A mio parere, si tratta di una posizione fondamentalmente incompatibile con il
proseguimento della specie umana, oltre a rappresentare una palese resistenza
all'istinto naturale di conservazione. Inoltre, è solo per via del progresso
ottenuto grazie allo sfruttamento delle specie non umane che possiamo
permetterci, in primo luogo, il lusso di considerare il rifiuto del suddetto
sfruttamento. Dunque, pur essendo auspicabile una limitazione del nostro impatto
sull'ecosistema, pensare di poter annullare quest'impatto è quantomeno utopico,
se non addirittura ipocrita e ingenuo.

\chapter{Lo specismo in natura}

In natura è assolutamente normale favorire la propria specie rispetto alle altre
e lavorare perché questa prosperi. È solo una delle tante forme sotto cui si
manifesta l'istinto di conservazione. I leoni non cacciano altri leoni in primo
luogo perché sarebbe poco pratico e in secondo luogo perché non sarebbe una
strategia di sopravvivenza efficace. Attaccano i propri simili solo per
garantire la preservazione del proprio bagaglio genetico, ossia quando l'istinto
di sopravvivenza individuale entra in conflitto con quello di conservazione
della specie\footnote{È prassi comune per il maschio uccidere i cuccioli della
femmina in modo che questa torni in calore. Il maschio riesce così a preservare
il proprio bagaglio genetico}. Anche gli animali non umani dunque praticano lo
specismo.

Inoltre, trattare la natura e l'uomo come se fossero due entità separate non ha
alcun senso. Non c'è motivo per cui non dovremmo considerarci parte della natura
insieme a tutte le altre specie. Dunque, se anche quella umana fosse l'unica
specie a praticare lo specismo, questo sarebbe comunque naturale. Ma supponiamo
che i comportamenti appartenenti esclusivamente all'uomo siano contrari
all'ordine delle cose o ``contronatura'': dovremmo allora smettere di vestirci?
di guidare automobili? di praticare la medicina? Dovremmo, insomma, rinunciare a
millenni di progresso solo perché le altre specie non hanno progredito? La
risposta è chiaramente no. Possiamo quindi affermare che la tendenza delle altre
specie ad adottare certe pratiche è irrilevante nella determinazione della
correttezza delle stesse.

Tuttavia, il fatto che lo specismo sia naturale non implica che sia anche
moralmente accettabile. Questo perché, a un certo punto della nostra storia
evolutiva, abbiamo deciso di stipulare un contratto sociale, ovvero di
autolimitarci, rinunciando ad alcuni dei nostri istinti animaleschi, al fine di
garantire a tutti gli umani una migliore convivenza. L'infanticidio è un crimine
ed è moralmente condannabile; eppure, come ho scritto sopra, i leoni e molte
altre specie lo praticano quotidianamente.

La presenza in natura di una pratica non è, per questo, un fattore determinante
nella scelta di adottare o meno tale pratica, perché l'etica morale si configura
a un livello superiore.

\chapter{Il giusto criterio}

Ci sono stati diversi tentativi nell'individuazione di un criterio che sia
universalmente e coerentemente applicabile a tutti gli individui e che ci
permetta di determinare il valore di un individuo e, di conseguenza, la
possibilità di sfruttarlo per il bene comune.

Jeremy Bentham, il padre dell'utilitarismo, riteneva che la sola capacità di
soffrire fosse sufficiente per condannare lo sfruttamento di altre specie:

\begin{quotation}

[…] un cavallo o un cane adulti sono senza paragone animali più razionali, e più
comunicativi, di un bambino di un giorno, o di una settimana, o persino di un
mese. Ma anche ammesso che fosse altrimenti, cosa importerebbe? La domanda non è
``Possono ragionare?'', né ``Possono parlare?'', ma ``Possono soffrire?''
\footnote{J. Bentham, ``Introduzione ai princìpi della morale e della
legislazione'', cap. 17.}

\end{quotation}

In risposta, si sente spesso dire che gli esseri umani possiedano un proprio
valore e dignità intrinsechi. Tuttavia, senza un'ulteriore elaborazione, si
tratta di un'affermazione dogmatica e priva di fondamento, creata a posteriori
per giustificare una superiorità provata inconsciamente. Perché solo gli esseri
umani dovrebbero avere valore intrinseco? Cosa li distingue, cosa li rende unici
tra tutte le specie animali?

Sebbene tutti i vertebrati e alcuni invertebrati siano in grado di provare
dolore, non tutti gli individui soffrono allo stesso modo. Gli esseri umani (e,
in maniera limitata, alcuni altri grandi primati\footnote{F. Patterson e W.
Gordon, ``The case for personhood of gorillas'' in P. Cavalieri e P. Singer,
``The Great Ape Project'', St. Martin's Griffin, pp. 58-77.}) sono dotati di ciò
che viene definita ``autocoscienza''; sono cioè consapevoli dell'esistenza di un
io e di come quest'io si relazioni con l'ambiente circostante e con altri
individui, concepiscono e fanno piani per il futuro.

L'autocoscienza ha a che fare anche con la percezione del dolore. Infatti, studi
neurologici\footnote{M. Murray, ``Nature Red in Tooth and Claw: Theism and the
Problem of Animal Suffering'', Oxford: Oxford University Press 2008.} hanno
dimostrato che esistono tre diversi livelli di percezione del dolore: gli
organismi di livello 1 hanno semplicemente una reazione di fuga agli stimoli
fastidiosi; gli organismi di livello 2 sono in grado di percepire il dolore come
sensazione fisica; gli organismi di livello 3 sono in grado di accorgersi che
stanno provando dolore.

La maggior parte degli animali non umani rientra nel livello 2: sebbene siano in
grado di percepire fisicamente il dolore, non concepiscono l'idea del dolore.
Negli individui dotati di autocoscienza invece il dolore non è semplicemente uno
stimolo fisico: questi si \emph{rendono conto} di provare dolore. Per esempio,
un uomo investito da una macchina prova rimpianto e tristezza all'idea di non
essere riuscito a realizzarsi appieno, di non poter più rivedere le persone
care, di non avere più tempo per portare il proprio contributo nel mondo e i
propri piani a compimento. Non si può pensare di poter paragonare queste due
dimensioni del dolore: gli organismi di livello 3 soffrono manifestamente più
di quelli di livello 1 e 2.

Le affermazioni di Bentham, però, non sono del tutto prive di fondamento. Così
come noi, infatti, gli animali fuggono dal dolore fisico. È quindi un nostro
dovere ridurre al minimo indispensabile la sofferenza a cui sono sottoposti.
Usare un topo per l'avanzamento della ricerca biomedica è lecito perché riduce
la sofferenza umana che, come abbiamo visto, ha più valore rispetto a quella
delle altre specie. Torturare un topo è invece un'azione immorale perché non
beneficia alcun individuo al di fuori del torturatore.

Questa linea di ragionamento, però, solleva un'ulteriore questione: se il dolore
fisico è l'unica condizione da evitare negli animali non umani, è moralmente
accettabile ucciderne uno senza causare sofferenza? Quest'idea è alla base dei
movimenti \emph{happy meat}. Persino Peter Singer, il padre dell'antispecismo
moderno, suggerì in un'intervista che si tratti di un'azione moralmente
giustificabile:

\begin{quote}

If it is the infliction of suffering that we are concerned about, rather than
killing, then I can also imagine a world in which people mostly eat plant foods,
but occasionally treat themselves to the luxury of free range eggs, or possibly
even meat from animals who live good lives under conditions natural for their
species, and are then humanely killed on the farm.\footnote{P. Singer in ``The
Vegan'' 2007.}

\end{quote}

Del resto, di certo una mucca non ha piani per il futuro, né dei cari che
soffriranno per la sua mancanza o di cui soffrirà la mancanza, né l'idea di un
aldilà. Che viva o che muoia in maniera indolore, indipendentemente
dall'esistenza di uno scopo per la sua morte, quindi, dovrebbe essere
indifferente.

Sebbene sia difficile trovare un'obiezione all'uccisione indolore, a fini
alimentari, di un animale che ha condotto una vita senza sofferenza, guarderemmo
con orrore l'uccisione dello stesso animale per puro divertimento. Si potrebbe
ribattere che l'uccisione ingiustificata causi un danno all'essere umano che
avrebbe potuto invece usare l'animale per fini più nobili. In tutta onesta, è un
dilemma che io stesso non ho risolto appieno.

\section{L'argomento dei casi marginali}

L'argomento dei casi marginali viene citato spesso dagli antispecisti per
dimostrare la fallacia della logica specista. È stato presentato per la prima
volta da Singer in un articolo\footnote{P. Singer, ``Speciesism and moral
status'' in ``Metaphilosophy'', luglio 2009, 3-4, pp. 568-571.} poi divenuto
famoso.

Supponiamo che un uomo nasca con un deficit cognitivo talmente esteso da poter
essere paragonato, sul piano dell'intelletto, a un animale non umano. Cosa ci
impedirebbe di trattarlo al pari le altre specie, ritenendoci dunque in diritto
di sfruttarlo per i nostri scopi (es. per il progresso della ricerca)? Se
infatti istintivamente siamo portati a tutelarlo proprio come faremmo con un
individuo normodotato, razionalmente dobbiamo trovare un argomento che possa
giustificare questa tutela.

Ci sono diverse considerazioni da fare. In primo luogo, se quest'uomo avesse dei
cari, il suo sfruttamento causerebbe loro del dolore. Danneggiando un caso
marginale dunque, danneggiamo chiunque lo abbia a cuore. Poiché c'è
un'alternativa che provoca meno sofferenza (usare un animale non umano), siamo
moralmente obbligati ad adottarla.

Inoltre, c'è sempre la possibilità che la scienza trovi una cura prima della
morte dell'uomo. Appare invece improbabile che riusciamo a inventare, in tempi
ragionevoli, un metodo per dotare gli animali non umani di autocoscienza. E
anche se lo inventassimo, le implicazioni etiche del suo utilizzo meriterebbero
una discussione a parte.

Infine, quest'argomento è utilizzabile solo per via dell'esistenza di un
ristretto numero casi marginali. Se questi casi non esistessero e tutti gli
esseri umani fossero superiori in capacità cognitive agli animali non umani,
allora saremmo giustificati nel negare loro i privilegi che gli animalisti
invece pretendono. Appare quantomeno curioso che un fattore completamente
indipendente dalla natura delle specie non umane (l'esistenza dei casi marginali
tra gli esseri umani) possa influire in qualche modo sul loro stato morale.

\section{L'argomento non egalitario raffinato}

Singer sostiene anche\footnote{P. Singer, op. cit., pp. 572-573} che non si
possano negare dei diritti fondamentali agli individui in base alle loro
capacità (e, nel caso specifico, all'autocoscienza) perché, applicando questo
ragionamento in maniera coerente, si getterebbero le basi per una
discriminazione tra gli stessi esseri umani. Si potrebbe infatti obiettare che
gli esseri umani con un quoziente intellettivo particolarmente elevato
soffrirebbero per via del proprio sfruttamento più di quanto ne soffrirebbe un
normodotato.

Ma che un quoziente intellettivo elevato ``amplifichi'' la percezione del dolore
non è vero. Non esiste alcun fattore oggettivo che ci permetta di giudicare
quale essere umano potrebbe soffrire di più per via del proprio sfruttamento (a
parte ovviamente i casi marginali che ho discusso prima), essendo la percezione
del dolore estremamente soggettiva.

Ma se anche tale fattore esistesse e alcuni individui provassero il dolore più
intensamente di altri, sarebbe comunque immorale il loro sfruttamento in quanto
tali individui avrebbero dei diritti. I diritti, infatti, non possono essere
negati in modo arbitrario: ha dei diritti chiunque possa rispettare dei doveri,
indipendentemente da considerazioni di altra natura.

\section{La razionalità delle emozioni}

Si potrebbe obiettare che le emozioni provate in seguito a una determinata
azione siano un argomento valido nella discussione etica riguardante
quell'azione. Curiosamente, questo ragionamento può essere adottato sia per
sostenere l'infondatezza dell'argomento dei casi marginali (il dolore provato in
seguito all'uccisione o allo sfruttamento di un altro essere umano sarebbe
sufficiente, in tal caso, a evitare simili azioni indipendentemente dalla
capacità del caso marginale di percepirne le conseguenze), sia per supportare
l'abolizione dello specismo (il dolore che lo sfruttamento degli animali provoca
all'uomo è segno che questo debba essere abolito). Si tratta, a mio parere, di
un argomento fallace sotto diversi aspetti.

In primo luogo, così come quello dei casi marginali si regge in piedi solo
grazie alla loro esistenza, allo stesso modo l'argomento emozionale esiste solo
perché ci sono gli esseri umani sono naturalmente empatici. La maggior parte dei
nazisti non ha provato rimorso in seguito all'uccisione di milioni di ebrei nei
campi di sterminio: erano anzi convinti di fare il bene dell'umanità. Se dunque
non ci fosse stata un'etica morale a cui appoggiarsi, o se non ci fossero stati
dei giudici al di fuori dei perpetratori, non ci sarebbe stato disgusto e
l'azione sarebbe stata perfettamente lecita.

In secondo luogo, è errato credere che le emozioni non abbiano nulla a che fare
con la razionalità; molte prove portano invece a credere che le emozioni siano
scatenate dalla ragione, anche se a prima vista può non sembrare
così\footnote{W. Załuski, ``Emotions and Rationality'' in J. LeDoux et al.,
``The Emotional Brain Revisited'', Copernicus Center Press, Krakow 2014, pp.
303-318.}. Prendendo come esempio quello del genocidio ebreo appare chiaro che
il resto del mondo non si sia ribellato semplicemente perché l'azione era
disgustosa; semmai è stata l'eclatante violazione dei diritti umani a suscitare
una reazione emotiva. In questo caso l'emozione ha agito da catalizzatore,
coadiuvando il ragionamento e l'azione che ne è seguita.

Allo stesso modo la felicità, la tristezza, la paura, l'amore e tutte le altre
emozioni sono scatenate dalla ragione, sia che lo realizziamo razionalmente sia
che lo ignoriamo. La felicità non è che una reazione al piacere ovvero a una
situazione favorevole; la tristezza è una risposta alla sofferenza ovvero a una
situazione sfavorevole; la paura ci mette al riparo dai pericoli. L'amore è
forse la più razionale delle emozioni: ci permette di trovare il partner
migliore e di preservare il nostro bagaglio genetico.\footnote{Quella sul fine
ultimo e dunque sulla razionalità della vita è una discussione più adatta ad
altre sedi. Il fine di questo articolo non è individuare una motivazione per
l'esistenza.} Tutto questo fa pensare che le emozioni non siano che
un'evoluzione darwiniana dei nostri istinti e che funzionino dunque secondo
dinamiche ben precise. In particolare, tutte le emozioni si riconducono
all'istinto di sopravvivenza.

Di tanto in tanto proviamo emozioni scatenate da motivazioni illogiche o
esagerate, emozioni che ci portano a danneggiarci o a danneggiare gli altri, ma
questo non è certo un segno dell'irrazionalità di \emph{tutte} le emozioni; è
semplicemente frutto della natura limitata dell'essere umano, che tende
all'errore e al pregiudizio in ogni campo, nelle emozioni così come nell'etica
morale.

Le emozioni non razionali non sono dunque accettabili in una discussione etica.
Se lo fossero, sarebbe impossibile assegnare la ragione o il torto con relativa
certezza poiché non ci sarebbe alcun criterio oggettivo di cui discutere e
qualunque dibattito perderebbe di significato. Potremmo anzi dire che le
emozioni non siano semplicemente accettabili, in quanto si tratta semplicemente
di uno strumento della ragione.

\chapter{Il diritto di avere diritti}

Gli animalisti parlano spesso di diritti degli animali. Ma perché un individuo
abbia il diritto alla vita, è necessario che tutti gli altri individui
rispettino il dovere di non ucciderlo.\footnote{W. N. Hohfeld, ``Fundamental
Legal Conceptions as Applied in Judicial Reasoning'', Yale University Press
1946.} Allo stesso tempo, questi individui rispettano i propri doveri per
ottenerne in cambio dei diritti. Se nessuno rispettasse i propri doveri e
pretnedesse esclusivamente il riconoscimento dei propri diritti, non ci sarebbe
più un equilibrio e l'intero sistema crollerebbe.

Gli animali non umani non dispongono delle capacità cognitive necessarie per
rispettare un dovere e dunque non possono pretendere dei diritti. Sono gli
uomini che hanno regolato autonomamente la libertà con cui possono disporre
delle altre specie: uccidere un animale non è una violazione del suo diritto
alla vita ma un'infrazione della legge. All'animale, in altre parole, è stata
assegnata arbitrariamente la libertà\footnote{Per la definizione formale di
``libertà'' si veda W. N. Hohfeld, op. cit.} di vivere: può vivere senza
rispettare alcun dovere. Questa libertà, però, ha delle limitazioni: laddove
l'animale sia necessario al progresso medico o comunque a ridurre la sofferenza
umana, può essere soppressa la sua libertà di vivere.

Inoltre, se una scimmia uguale all'uomo per capacità cognitive pretendesse gli
stessi diritti dell'uomo, saremmo moralmente obbligati a garantirlieli. La
scimmia infatti, potendo rispettare tutti i doveri umani, potrebbe e dovrebbe
beneficiare di tutti i diritti.\footnote{Questo ragionamento non giustifica la
pena di morte. Non rispettando un dovere altrui, infatti, non si perde il
diritto corrispondente perché si è comunque potenzialmente in grado di
rispettarlo. Uccidere un omicida immorale perché questo ha un'intera vita per
rientrare nel sistema diritti-doveri; ucciderlo significherebbe privarlo di
questa opportunità.}

% \chapter{Rinunciare allo specismo}

% Possiamo rinunciare allo specismo?
% Dobbiamo rinunciare allo specismo?

% \chapter{Innocenza e libero arbitrio}

% Per essere innocenti bisogna poter scegliere.
% Gli animali non sono più innocenti di quanto siano colpevoli.

% \chapter{La moralità dell'inazione}

% È immorale lasciar morire un bambino che annega.
% Perché dovrebbe esserlo lasciar morire un bambino malato?

\end{document}
