\chapter{}
\label{ch:2}

Era stato il caso a farli incontrare molti anni addietro, quando Thomas aveva appena fondato la sua
compagnia e ancora viveva con un amico in attesa di potersi permettere un appartamento. Entrambi
erano stati trascinati da qualche loro conoscente a una festa, ed entrambi avevano di meglio da
fare. Per Thomas era dedicarsi ai suoi progetti, ovviamente; per Kate, che aveva alle spalle una
relazione disastrata, era restare a casa a rimuginare su quale menzogna fosse l'amore e quale valle
di lacrime fosse il mondo.

Tuttora, se le avessero chiesto che cosa le fosse piaciuto quella sera in Thomas, avrebbe risposto:
il silenzio. Mentre tutti intorno a lei conversavano, urlavano le inutilità che credevano
indispensabile comunicare al resto del mondo, egli taceva. Ogni tanto rivolgeva qualche sorriso di
circostanza a un passante, o rispondeva con frasi di cortesia a domande di cortesia, ma per lo più
si limitava a osservare, taciturno, pensieroso. Era come se fosse invisibile a tutti tranne che a
lei.

Per la prima volta nella propria vita, Kate aveva avvertito l'urgenza. L'urgenza di agire, di osare,
di vivere. Non poteva perderlo; sapeva che non ci sarebbe stato rimedio. Si compiaceva e si
vergognava di quell'ambiziosa intraprendenza. Si sentiva bloccata dalla paura di un'altra delusione,
ma mossa dalla volontà di dimostrare a se stessa che la felicità non le era preclusa, che una vita
serena era ancora possibile. Per tutta la sera aveva danzato intorno a Thomas, senza però osarsi mai
avvicinare troppo. Aveva riscoperto la meraviglia di essere stretta in una morsa e non volerne
uscire.

Se non fosse stato per una sua conoscente, una di quelle fastidiosissime donne che sembrano voler
mettere ordine nella vita di tutti tranne che nella propria, e si sentono per questo in dovere di
dirigere e dare consigli a tutti, probabilmente Kate non avrebbe mai avuto modo di parlargli. Questa
ragazza, dunque, con cui Kate aveva conversato in rarissime occasioni -- traendone ben poco piacere
-- le si era avvicinata.

«C'è una persona che devi \emph{assolutamente} conoscere!» aveva squittito, trascinandola verso un
punto tra la folla. Kate aveva tentato di opporre resistenza, urlando per coprire le altre voci, ma
ogni protesta era cessata non appena si era accorta di essere condotta verso Thomas. Benché fosse
terrorizzata, non avrebbe mai osato rifiutare quell'occasione.

Pochi minuti più tardi conversavano come vecchi amici, ogni traccia di imbarazzo sparita dal volto
di lei, il carattere taciturno di lui sostituito da quello di un intelligente e spiritoso
osservatore. Kate non riusciva a capire molto del progetto di Thomas, ma sapeva, anche ora, anche
senza averci mai parlato prima, che era destinato a un grande futuro. Quell'uomo le dava speranza, e
la speranza era esattamente ciò di cui aveva bisogno. Si vedeva accanto a lui, che sfidava le
intemperie della sorte, provata ma felice, consapevole di essere guidata dalla mano ferma e precisa
di una persona geniale.

Era tornata a casa con uno strano ma piacevole peso sul petto. In mano stringeva un biglietto su cui
egli aveva scritto il proprio numero di telefono. Thomas Westford. Persino il nome suonava
importante. Senza dubbio si sarebbe sentito molte volte negli anni a seguire. Lo sapeva. Credeva in
lui. Lo amava.

Per circa due settimane, ogni sera, aveva preso quel biglietto e lo aveva guardato con ansia,
tenendo il telefono in mano. Proprio per via dell'importanza che l'incontro con Thomas aveva avuto,
era ora attanagliata da un oscuro e irritante terrore: che egli non si rivelasse all'altezza delle
sue aspettative. Si rendeva conto di come la sua fantasia avesse contribuito a proiettare su Thomas
il suo uomo ideale. Se si fosse sbagliata, non sarebbe sopravvissuta a un'altra delusione.

C'era poi un'altra possibilità, che la preoccupava ancora di più: quella di rovinare tutto. Non
erano passati nemmeno sei mesi dalla sua ultima relazione, ma già le sembravano così lontani i tempi
in cui era ancora capace di amare. Odiandosi per il modo in cui si era lasciata plasmare da un uomo,
aveva promesso a se stessa che non sarebbe mai più stata così ingenua. E proprio ora che le era
indispensabile, le pareva di aver perso quella pericolosa capacità di lasciarsi completamente
andare, di affidarsi a un altro essere umano.

Infine ronzava nella sua testa, senza che potesse trovare risposta, questa domanda: perché egli non
l'aveva ancora chiamata? Aveva il suo numero, cosa lo tratteneva? Non era plausibile che l'avesse
smarrito: ella teneva quel foglietto, ormai indecentemente stropicciato, come una reliquia, e dunque
l'unica ragione poteva essere che non gli fosse piaciuta; che tutti quei sorrisi, quelle battute,
quella complicità fossero solo una farsa, magari un modo per passare il tempo; che ella fosse stata
una stupida a farsi tante illusioni, e che fosse destinata a\dots

Il telefono squillava da almeno dieci secondi, ma Kate non se n'era accorta, persa nei propri
angosciosi pensieri. Quando finalmente aveva realizzato, lo stupore era stato tale che un sussulto
aveva scosso tutto il suo corpo. Aveva messo a fuoco il numero sullo schermo e le era parso
famigliare: i suoi occhi erano andati velocemente al biglietto, sul tavolo lì accanto; leggendo
cifra dopo cifra, quasi ad alta voce, sentiva uno strano sentimento occupare il posto della
precedente sorpresa. Avrebbe voluto piangere, ma non sapeva bene il motivo. Rendendosi
improvvisamente conto che stava per perdere la telefonata, aveva premuto il tasto di risposta con
molta più foga del dovuto.

\plainbreak{1}

Quando l'uomo aveva aperto la porta, il suo grande e nervoso sorriso si era spento. Aveva sgranato
gli occhi, cercando di capire se quello che le stava davanti fosse Thomas o il suo coinquilino. Il
viso era pallidissimo, tranne per le occhiaie gonfie e violacee che cerchiavano gli occhi rossi,
forse per il pianto. Nemmeno un capello sembrava voler stare al proprio posto, e sulla fronte questi
si appiccicavano per via del sudore che la imperlava. Le palpebre erano socchiuse, come se fosse
infastidito dalla luce. Indossava una felpa pesante, decisamente troppo per la stagione, pantaloni e
scarpe da ginnastica.

«Ciao» aveva detto, con una voce quasi impercettibile. Aveva chiuso la porta dietro di lei, con
lentezza esasperante. Quando gli era passata accanto, un terribile odore di sudore e disperazione
era salito su per le narici di Kate, dove si era posato e contribuiva, piano piano, a convincerla
che tutto quello non fosse che un sogno. No, non poteva essere Thomas. Non era possibile che l'uomo
appassionato ed elegante che aveva incontrato meno di venti giorni prima si fosse ridotto in quello
stato. Non poteva esserci evento che lo abbattesse in quella maniera.

Lo aveva guardato ancora: le spalle ricurve, lo sguardo spento, il naso che gocciolava come quello
di un neonato\dots La casa era schifosamente sporca, il pavimento macchiato e coperto di fazzoletti
usati, e impregnata dello stesso odore di Thomas. Era troppo: non poteva sopportarlo; Kate era
tornata sui propri passi con una furia che aveva stupito anche lei, quasi travolgendo Thomas. Aveva
aperto la porta ed era uscita sul pianerottolo, decisa a fuggire da quell'appartamento per non
tornarvi mai più.

«Che fai?» le aveva gridato l'uomo. «Aspetta!»

Aveva cercato di fermarla, trattenendola per un braccio, ma ella si era divincolata con tanta forza
da farlo inciampare e cadere a terra.

«Kate, non andare! Ti prego!» continuava a urlare.

Aveva già sceso una rampa di scale quando i singhiozzi l'avevano raggiunta. Thomas piangeva
sommessamente, quasi avesse paura di disturbare. Piangeva sdraiato a terra, proprio dove l'aveva
lasciato. Piangeva tenendo la testa fra le braccia, come i bambini puniti dai genitori. E anche
Thomas era stato punito: punito dal fato, dalla crudele ironia di un destino che sembrava prendersi
gioco di lui, che lo privava, in pochi giorni, del suo più grande male e del più grande bene.

Kate si era voltata e fissava l'uomo, interdetta e spaventata, incapace di distinguere tra incubo e
realtà.

\plainbreak{1}

Questa era stata la reazione di Thomas alla morte della madre. Dopo aver riso della debolezza di
quella donna, incapace di vivere senza il proprio aguzzino, era tornato a casa per confidarsi con
l'amico con cui viveva: voleva raccontargli come realmente si sentisse al riguardo, parlargli
dell'ingiustizia e della tristezza dell'esistenza; ma il ragazzo era partito per una vacanza in
Europa, e sarebbe tornato solamente il mese successivo. Per qualche giorno Thomas era stato in grado
di mantenere il controllo, ma la depressione lo assaliva lentamente, e sentiva di esserne sempre più
schiavo ogni secondo che passava.

Aveva smesso di mangiare, di lavarsi, di vestirsi, di respirare. Aveva accarezzato l'ipotesi del
suicidio; non gli sembrava poi così male l'idea del sonno eterno. Non ci sarebbe stata più
sofferenza, né morte, né ingiustizia. Solo il nulla. Ma soprattutto, avrebbe pagato la propria
colpa: l'indifferenza. Per tutti quegli anni aveva ignorato la situazione della madre: l'aveva
considerata una debole perché, pur avendo avuto più volte l'occasione di sottrarsi a quella tortura,
aveva scelto di non farlo; quale persona sana di mente si sarebbe mai comportata così? Ora Thomas
sentiva che la propria morte sarebbe stata una giusta pena, un modo adeguato per espiare il proprio
immenso, imperdonabile peccato.

Ma non aveva la forza nemmeno per quell'ultimo, estremo gesto; o forse era il coraggio a mancargli.
Così, piuttosto che farla finita in un breve istante, aveva scelto di morire giorno dopo giorno,
lasciando che le forze lo abbandonassero lentamente ma irrimediabilmente. Aveva continuato a vagare
per la casa, e quando anche spostarsi gli era sembrato troppo difficoltoso, si era seduto in un
angolo senza più muoversi. Oramai aveva perso ogni sensibilità: il suo corpo non era più suo, ed era
meraviglioso lasciarsi cullare dalla morte, sentire di non avere più il controllo, di non avere più
responsabilità. Ormai nulla dipendeva più da lui: si affidava al dio sulla cui esistenza era sempre
stato scettico, e che ora gli sussurrava all'orecchio.

Ma quel dio non voleva che morisse, o forse era il suo istinto di sopravvivenza a ingannarlo. E così
una scintilla si era accesa in lui: la morte aveva perso tutto il suo fascino, portava con sé solo
la prospettiva dell'annientamento totale, l'incapacità di agire, e dunque di rimediare. La morte gli
era preclusa, riservata a chi aveva vissuto la vita dei giusti. Non ci sarebbe stata alcuna dignità
nella \emph{sua} morte. E poi, maledizione, aveva paura: non voleva morire! Sapeva di poter ancora
fare la differenza, di poter migliorare la vita di altre persone per riscattare quella di sua madre,
e l'idea di privarsi di quell'opportunità gli era assolutamente odiosa: non era da lui rinunciare.

Ma non poteva farcela da solo. Aveva chiamato a raccolta le poche forze che ancora gli rimanevano e
aveva chiamato Kate, cercando di suonare il più naturale e sano possibile. Sapeva di non poterla
ingannare a lungo, ma non avrebbe mai ammesso la propria debolezza al telefono. Quando era arrivata
e Thomas aveva visto il suo viso, aveva potuto intravedere, lontana, una pallida luce di speranza.
Ma poi aveva notato anche lo schifo dipingersi sul volto della donna,

Così aveva pianto. Di rabbia o di tristezza o di vergogna, neanch'egli lo sapeva bene. Non riusciva
nemmeno a immaginare quanto piccolo, debole e patetico potesse apparire agli occhi di lei. Era
finita, lo sentiva: quella era l'ultima occasione che aveva, ed era riuscito a sprecarla; oltre, non
ci sarebbe stato più nulla. Tanto valeva tornare dentro casa e tagliarsi le vene, o impiccarsi, o
cadere in un sonno senza risveglio come aveva scelto di fare sua madre. Era quello il suo destino,
il progetto che c'era per lui; era sempre stato quello, e pensare di sfuggirgli era solamente
l'illusione di un folle incapace di accettare la morte. Non avrebbe più pregato, non avrebbe più
lottato: non sarebbe servito a niente.

Ma Kate si era fermata. Aveva sentito il suo sguardo su di sé per qualche minuto, poi i suoi passi e
infine le sue mani, delle mani così delicate, candide e sincere che incontravano le mani sporche di
un codardo e di un assassino. Quelle mani l'avevano costretto a tirarsi su, l'avevano portato dentro
casa. Thomas non si muoveva, non parlava, non protestava. Osava a malapena respirare: era
completamente in balìa di lei e delle sue cure amorevoli. Perché lo stava aiutando? Non c'era
motivo: non era nessuno, non meritava niente.

Kate lo aveva fatto sedere e, mentre attendeva pazientemente che smettesse di singhiozzare, aveva
pulito la casa al meglio delle proprie possibilità. Thomas non si azzardava a incrociare il suo
sguardo, ma poteva sentirla muoversi, rapida e inquieta, da una stanza all'altra. Poi era tornata da
lui e gli aveva di nuovo preso le mani, obbligandolo a guardarla negli occhi. Era stato solo allora
che Thomas si era accorto di quanto fosse bella; quella sera, alla festa, era troppo distratto dai
propri pensieri; i lineamenti delicati, il naso lungo ma elegante, i lunghi capelli biondi, le
guance rosee, gli occhi profondi\dots Era stato solo allora che Thomas aveva saputo di amarla.

\plainbreak{1}

C'erano volute quasi tre settimane perché Thomas si riprendesse. Tre settimane in cui Kate, ogni
giorno, era andata a fargli visita per cucinare -- se non l'avesse fatto, ne era certa, sarebbe
morto di inedia --, per pulire e confortarlo. Aveva assistito a uno strano processo: era come se
Thomas avesse dimenticato, in pochi giorni, come muoversi, come parlare, come vivere. Era stato
compito suo insegnargli nuovamente tutto, e non sempre si era sentita all'altezza: in quelle tre
settimane Thomas aveva avuto delle ricadute, crisi improvvise di panico e rabbia: si scagliava
contro sua madre, che si era uccisa solo per punirlo, contro i suoi amici, che non si rendevano
conto di quanto stesse soffrendo, e contro la stessa Kate, per le ragioni più diverse, salvo poi
riprendere a piangere e pregarla di rimanere quando ella accennava ad andarsene, sdegnata.

Kate non sapeva perché lo stesse facendo: perché ella, così bisognosa di stabilità, stava
rinunciando a una vita tranquilla per aiutare un uomo distrutto? era forse il suo istinto materno?
l'avrebbe fatto per qualcun altro? o lo stava davvero aiutando, come Thomas la accusava qualche
volta, solo perché i sogni di lui la affascinavano? No, non poteva essere: Kate non amava l'ideatore
per amore delle idee, ma le idee per amore dell'ideatore. Avrebbe seguito Thomas in qualunque
progetto, anche il più folle, dacché vedendolo aveva subito saputo che era la persona giusta; non
avrebbe potuto essere altrimenti. E sentiva, pur conoscendolo appena, come un debito nei suoi
confronti: le aveva ridato speranza, dunque le sembrava giusto che ora ella facesse lo stesso.

Passate quelle tre settimane -- le più lunghe della sua vita -- Thomas si era come risvegliato: una
mattina Kate era andata da lui e lo aveva trovato più sereno che mai; si sarebbe forse potuto dire
che fosse radioso; pareva che ogni traccia della disperazione che lo aveva attanagliato nei giorni
precedenti fosse scomparsa senza preavviso: l'uomo correva avanti e indietro, senza interruzione,
riordinando la casa, cucinando, lavorando a certi suoi progetti\dots

Quella sera, mentre le mani di Thomas la sfioravano e un brivido di piacere la scuoteva, Kate aveva
ricordato quanto potesse essere bello lasciarsi andare.
