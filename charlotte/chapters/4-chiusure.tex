\chapter{Chiusure}
\label{ch:chiusure}

Charlotte era in un indescrivibile stato d'animo. La poveretta aveva passato le giornate addietro
passeggiando inquieta per casa, torcendosi le mani, incapace di impartire qualsivoglia lezione al
proprio studente. Se ne doleva e vergognava, certo; in primo luogo, però, nulla poteva cancellare in
lei l'eccitazione che le causava l'idea di riappropriarsi di quella parte della propria vita, che
ella aveva considerato perduta per sempre. In secondo luogo, sapeva che in pochi giorni suo padre
sarebbe riuscito a fare più progressi con Nathan di quanti ne avesse fatti ella in un mese. Il
talento di Nicholas nel suonare era comparabile solo alla sua capacità di sprigionare genuino amore
per la materia. Ella stessa, odiando tutto ciò che la musica aveva causato a suo padre, non
continuava forse a suonare?

E in fondo non le dispiaceva poi molto di poter passare più tempo parlando con Nathan lontani da un
pianoforte. Più tempo passava con lui, più soffriva all'idea di tornare sola. Continuava a ripetersi
che se non fosse stato per la sua età, se solo non fosse stato per quello allora\dots Che idea
assurda. Più assurdo e vergognoso ancora il fatto che potesse esserle venuta in mente. Ma così tanti
uomini erano passati per quella casa, nella speranza che potessero in qualche modo\dots chiudere il
cerchio. Così tante aspettative deluse. Per anni ella aveva potuto avere chiunque volesse, e ora
l'unica persona che volesse davvero, l'unica che la facesse sentire al sicuro\dots quella persona,
forse la \emph{sua} persona, le era irraggiungibile.

Era forse condannata a essere sola? A toccare mani dopo mani, a baciare labbra dopo labbra, e
restare sempre delusa dal loro incommensurabile, cieco, freddo egoismo?

Qualche pomeriggio prima, durante una lezione particolarmente frustrante e infruttuosa, Nathan era
quasi in lacrime. A suo dire, non lo era per il fallimento di fronte al padre, ma per il fallimento
di fronte a lei. Charlotte fu mossa da quel sentimento a infinita dolcezza e compassione; le era
venuto perfettamente naturale abbracciarlo. Sebbene egli non avesse ricambiato, il solo contatto le
fece provare un brivido, qualcosa che non le era mia capitato prima, con nessuno. Allora forse per
la prima volta — o forse le piaceva pensare che quella fosse la prima — le era venuta in mente
quell'assurda idea, subito rifiutata, una lettera scarlatta a segnare la sua coscienza dignitosa.

Ma più vi pensava e vi si perdeva, più si convinceva che non ci fosse nulla di così assurdo o
vergognoso nell'attrazione; vergognoso sarebbe stato agire. Quella sarebbe stata un'onta di cui non
avrebbe potuto cancellare la traccia.

Eppure\dots

No, quello certamente non avrebbe mai e poi mai potuto farlo.

\plainbreak{1}

Era stato difficile per Braxton avere l'indirizzo della signorina Barnes. Sapeva, perché era stata a
suo tempo una gran notizia, che la figlia di Nicholas Barnes si era trasferita a Ebensburg. Così,
quando era stato incaricato di scrivere un articolo su Nathan Westford, che aveva scoperto prendere
lezioni da Charlotte, vi si era precipitato. Ma riguardo l'indirizzo preciso vigeva il più totale
riserbo nella comunità locale: Charlotte non amava chi veniva dalle grandi città e tutti a Ebensburg
lo sapevano. Braxton veniva da una grande città, e tutti a Ebensburg lo sapevano.

Ma Braxton doveva scrivere un pezzo sul rampollo di uno dei uomini più ricchi del Paese e la sua
talentuosa, misteriosa insegnante, e non intendeva arrendersi: aveva iniziato a frequentare luoghi
affollati — non ve n'erano molti in quella città — sapendo che era solo questione di tempo prima che
trovasse la donna o il ragazzo, e si era rivelata una tattica vincente: dopo un paio di giorni si
era imbattuto nei due, mentre passeggiavano in piazza, e li aveva seguiti. Li voleva intervistare,
ma non era ancora il momento. C'era\dots qualcosa, in come si guardavano e in come si parlavano.
Sapeva che, se avesse aspettato, sarebbe probabilmente andato via da quel buco con molto più di
un'intervista da terza pagina.

Prolungò il suo pernottamento in hotel e avvisò in redazione che aveva qualcosa di grosso tra le
mani.

Nessuno gli avrebbe creduto: non era che un giornalista in declino che, a guardare bene, non aveva
realizzato nulla che fosse degno di nota. Non gli importava.

Pensava fosse la sua occasione, e non voleva sprecarla.

\plainbreak{1}

Basta.

Kate non pensava di averlo detto ad alta voce, ma pareva fosse proprio così.

Aveva davvero, infine, detto basta?

Per esserne sicura, lo ripeté a Thomas, che le aveva chiesto distrattamente, tra una riunione e
l'altra, di chiarire. Gli aveva chiesto — no, lo aveva pregato — di passare un giorno, almeno uno, a
casa con lei. Egli aveva acconsentito solo dopo un litigio in cui non aveva mancato di ricordare
quanto infantile ella fosse per una simile pretesa, e come la compagnia non potesse sopravvivere
senza di lui. Ovviamente, aveva concluso ricordandole che lo faceva per loro due e Nathan. Per chi
altri, se no Non avrebbe lavorato così tanto, se fosse stato solo per lui.

Kate sapeva che non era vero.

A ogni modo, anche così aveva lavorato tutto il giorno. Solo da casa.

Dunque basta.

Basta cosa, aveva chiesto lui.

Basta tutto. Basta loro due. Basta quella vita di sacrifici. Basta guardare il successo di lui alla
televisione. Basta avere paura di disturbarlo tutte le volte che gli rivolgeva la parola, e perché
quando non lavorava pensava al lavoro e quando non pensava al lavoro voleva tornare al lavoro.

Gli aveva detto tutto questo senza alcuna particolare intonazione nella voce, e in quest'ordine.

Notò con soddisfazione — ma non era davvero soddisfazione — Thomas resistere all'impulso di
rispondere a una telefonata, mentre ella gli elencava tutte le ragioni per cui aveva deciso di
chiedere il divorzio.

\plainbreak{1}

Nicholas era a Ebensburg da una settimana ormai. Il suo arrivo, come era stato per quello di Nathan,
non era passato inosservato, e il giorno dopo i giornali già titolavano ``Stella del piano torna a
casa''. Charlotte temeva che Nathan non si trovasse bene con il fare autoritario del padre, ma ora
si rendeva conto di quanto fosse stata sciocca: il ragazzo sembrava avere una naturale inclinazione
per quel tipo di insegnamento, e iniziava finalmente a sbloccarsi.

Nonostante l'insistenza di Charlotte, Nicholas aveva rifiutato la sua offerta di dormire con lei e
aveva preferito invece un albergo non lontano. Anche così, comunque, passavano insieme ogni
giornata. Solitamente si incontravano per la colazione, studiavano l'intera giornata -- Nathan
alternava Nicholas e Charlotte come insegnanti -- e, verso sera, uscivano tutti e tre. Passeggiavano
per Ebensburg come se fossero una famiglia, come se fosse una tradizione da sempre rispettata.
Parlavano di musica, perlopiù; un argomento a cui Charlotte vedeva Nathan interessarsi sempre di
più.

Poi, quando era giunto il momento per Nicholas di tornare in hotel, Charlotte lo accompagnava da
sola, e allora parlavano delle cose davvero importanti: della mamma, e di come egli non fosse
riuscito a starle accanto, di come si fosse rinchiuso nell'unica cosa che conosceva davvero, la
musica, dimenticandosi del resto. Charlotte sentiva il bisogno di perdono nella voce di lui, di
solito così ferma e ora graffiata dalla colpa, eppure non riusciva a farlo\dots Non riusciva a
dirgli che non importava, che era passato tanto tempo e andava bene così, che nonostante tutte le
sue mancanze era pur sempre suo padre, e preferiva mille volte averlo al proprio fianco piuttosto
che perderlo per errori di due decenni prima. Non ci riusciva forse perché porvi fine sarebbe stato
porre fine al ricordo della madre, e condonare tutto ed esserne complice, e questo non poteva
accettarlo.

Eppure ella sentiva un infinito bisogno di normalità. Voleva essere di nuovo parte di una famiglia,
e condividere di nuovo il proprio tempo con qualcuno, e sentirsi in pace e non dover fuggire da
tutto quello che non voleva assolutamente diventare. E adesso che tutto questo era così vicino,
valeva davvero la pena sacrificarlo per restare fedele a un ricordo? Fino a che punto si poteva
spingere, e di quanto ancora voleva privarsi?

\plainbreak{1}

Nicholas aveva informato i due che sarebbe rimasto a Ebensburg per il lungo termine, e Nathan doveva
ammettere di provare una certa gelosia. Era geloso di quello speciale e unico rapporto con Charlotte
che egli non avrebbe mai avuto, del modo in cui ella guardava il padre, come se fosse il centro
della propria felicità, delle parole e delle frasi e dei discorsi a cui si abbeveravano, assetati
com'erano da anni di silenzio. In un certo strano modo, a Nathan sembrava che i due fossero
perdutamente innamorati, come solo padre e figlia possono esserlo.

E allo stesso tempo, Nicholas gli era simpatico. In primo luogo, gli aveva reso sopportabile,
gradevole addirittura, il pianoforte. Charlotte era dolcissima e comprensiva e paziente oltre
l'immaginabile, ma c'era qualcosa di rassicurante nella serena autorevolezza con cui Nicholas
insegnava: come se dalle sue parole trasparisse il senso ultimo della musica e il significato in sé
dello strumento, e nulla di tutto questo potesse essere diverso da com'egli lo diceva. E nonostante
questa fermezza, egli era infinitamente paziente, e permetteva a Nathan di sbagliare e sbagliare e
sbagliare ancora, fino alla frustrazione, fino al successo.

Questo conflitto di sentimenti lo confondeva e lo imbarazzava. Sapeva di non essere nessuno, di non
poter avanzare alcuna pretesa nella vita di Charlotte, ma in un certo modo gli pareva di conoscerla
meglio di quanto l'avesse conosciuta Nicholas in quegli ultimi anni, e la sua improvvisa smania di
di tornare nella vita di lei faceva arrabbiare Nathan. Gli pareva una concessione immeritata, un
dono indirizzato alla persona sbagliata.

Che sentimento sciocco e vergognoso! Pensare di conoscere Charlotte meglio del padre, di poter
decidere cosa fosse meglio per lei.
