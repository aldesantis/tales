\chapter{Ripensamenti}
\label{ch:ripensamenti}

Al suo arrivo, Nathan non aveva avuto modo di ammirare pienamente la bellezza modesta, raccolta e
serena di Ebensburg: era allora accecato dalla rabbia, dalla tristezza, dalla paura, e non aveva
goduto di quella vista. Ora però gli era più facile dimenticare il perché e concentrarsi invece sul
dove, sul cosa, sul chi, giacché il suo animo era alleggerito dalla dolcezza di Charlotte; così,
egli si guardava adesso intorno con bambinesca curiosità, ubriacandosi di tutti i nuovi odori e di
quella calma misurata, felice: qui un uomo sulla sessantina curava il proprio giardino con
un'energia che Nathan aveva visto solo nel padre; qui una madre giovane e fiorente portava a spasso
il proprio neonato, come se fosse la sola cosa al mondo che dovesse fare, come se non avesse altre
preoccupazioni, altre incombenze, altri pensieri.

Era davvero possibile vivere così, come se al di fuori di quel microscopico universo non esistesse
nulla? A Nathan, abituato all'ossessione newyorkese per il lavoro e la produttività, alla ricerca
continua e spasmodica, sembrava tutto assurdo, così come agli abitanti di Ebensburg -- egli era
certo -- sarebbe sembrata assurda la frenesia della grande città, la tensione palpabile. Si trattava
di due mondi inconciliabili, che del resto non avevano alcun interesse a conciliarsi, dacché non ne
poteva venire nulla a nessuno dei due. Nathan stesso si sentiva un intruso, come se potesse con la
sua sola presenza contaminare la purezza mai scalfita di Ebensburg; allo stesso modo si era sentita
Charlotte, quando era arrivata molti anni prima.

Ella camminava ora al fianco del ragazzo, assorta, bellissima: portava un abito rosso a decorazioni
floreali, con un'elegante semplicità che Nathan non aveva mai visto in nessun'altra donna. Ed egli
non poteva fare a meno di guardarla, di ammirarne la grazia dei movimenti, la gentilezza dei modi,
la meraviglia delle forme; non poteva fare a meno di chiedersi come sarebbe stato annusare quei
capelli, o sfiorare quella pelle, eppure ne provava subito una colpa e una vergogna indescrivibili,
come se Charlotte fosse fin troppo pura, fin troppo angelica per essere abbassata a oggetto di
quelle attenzioni terrene.

«Dovrai trovare una ragione». La voce di Charlotte interruppe i suoi pensieri, e quasi ella potesse
indovinarli, Nathan si sentì avvampare le guance. Le chiese di spiegarsi meglio, evitando di
guardarla negli occhi. «Ieri, a cena\dots ti ho chiesto perché sei venuto a studiare da me. Non mi
hai saputo rispondere» chiarì allora ella.

Erano giunti sulla sponda di un piccolo fiume che attraversava Ebensburg e di fatto la divideva in
due metà decisamente asimmetriche: una parte, quella dove si trovavano Nathan e Charlotte, urbana;
l'altra rurale, destinata forse alla coltivazione. Sull'altra sponda, infatti, Nathan scorgeva campi
e null'altro; campi a perdita d'occhio, qualche casupola e in fondo una fila d'alberi altissimi.
Sulla banchina, insieme a loro, stavano alcuni pescatori, altri passanti, una o due coppie
d'innamorati sedute sulle panchine in legno. Qui sedette Charlotte, sfiorando il posto accanto a sé,
invitandolo dolcemente a raggiungerla.

Nathan obbedì; poteva sentire il profumo inebriante di lei, e improvvisamente fu colto da un
rispetto reverenziale, da un desiderio cieco, da un amore fulminante e insopprimibile. Sentì di
nuovo e più forte di prima il bisogno di dirle tutto, di raccontarle della propria vita infelice,
dell'infanzia diretta come da un burattinaio onnipresente e invincibile; voleva dirle che non aveva
alcun amore per il pianoforte, e non perché ci fosse qualcosa nello strumento che non lo attirava,
ma perché avevano tentato di imporgli la passione ed egli aveva finito per odiarlo; voleva scusarsi,
perché stava perdendo il preziosissimo tempo di lei.

E lo fece: le disse tutto, mentre Charlotte ascoltava con un'attenzione di cui non lo avevano mai
onorato, interrompendolo di quando in quando per porre qualche domanda, per chiarire un suo
sentimento, per mostrargli comprensione e compassione. Quando le disse del suo disprezzo verso
quello studio, in cui pure ella cercava di aiutarlo, non lesse dolore sul volto della donna; solo
amorevolissima, attentissima pietà. Una pietà che gonfiava il cuore di lui, che lo faceva esplodere
dalla voglia di dirle come si sentisse al suo fianco; ma si trattenne, perché gli pareva improbabile
che un sedicenne potesse bruciare d'amore sincero per una donna; e anche se fosse stato possibile,
gli pareva assurdo che siffatto amore potesse essere ricambiato; e se pure fosse stato ricambiato,
gli pareva impossibile, per un miliardo di ragioni tutte diverse, che potesse compiersi.

Quand'egli ebbe finito, Charlotte stette un poco a pensare; perfino ella sembrava turbata dal
racconto. Infine gli chiese: «Guarda il fiume; cosa vedi?», e accennò al corso d'acqua col bel capo.
Ivi Nathan pose lo sguardo, e in un primo momento non gli sembrò che ci fosse nulla degno di nota:
come ogni cosa e ogni persona a Ebensburg, il fiume era placido e sereno. Poi un dettaglio catturò
la sua attenzione; ma poteva mai essere?!\dots Ma sì! Era, era! Il fiume scorreva in due direzioni:
verso la loro sinistra la metà più vicina a Charlotte e Nathan, e quella più lontana verso la loro
destra. Il contrasto era appena percettibile: bisognava porre molta attenzione per notarlo.

Lo disse a Charlotte, ed ella rispose che realmente non era che un'illusione ottica; in assenza di
metafore migliori, gli disse, avrebbe dovuto farsi bastare quella. Ma ancora Nathan non capiva, e si
vergognò ad ammetterlo; allora Charlotte si spiegò meglio: egli e il padre, gli disse, erano come le
due metà del fiume, «inseparabili, ma in contrasto». Eppure le due metà coesistevano
tranquillamente, senza che una cercasse di deviare l'altra. Così il ragazzo avrebbe dovuto fare con
Thomas, almeno per qualche anno ancora, finché non fosse stato in grado di andarsene per la propria
strada.

Nathan ringraziò di cuore Charlotte, ma obiettò che il padre, influenzando la sua educazione,
avrebbe senza dubbio anche pregiudicato la sua strada; che egli faceva sentire la propria autorità
in maniera così subdola, quasi inconsciamente, che nemmeno Nathan si sarebbe accorto di tutte le
diverse maniere in cui le sue opinioni erano state manipolate. Allora i due stettero a parlare un
pezzo del ruolo che la società aveva sul destino del singolo, su come liberarsi dai pregiudizi del
pensiero, sul miglior modo per trovare il proprio percorso e su come convenisse seguirlo\dots

Quando gli parve di aver finito gli argomenti, già da un pezzo il cielo era scuro, e l'animo di
Nathan più leggero. Praticamente, non avevano risolto molto: non appena fosse tornato nella casa
paterna, egli si sarebbe trovato a lottare con gli stessi soliti problemi, a soffrire la stessa
solita presunzione, a sentire la stessa solita rabbia; eppure, ai problemi poteva opporre la
prospettiva, alla presunzione la fermezza, alla rabbia la pazienza. Ora sentiva di avere un'arma;
ora sentiva di avere un'amica. Non importava che vivessero nella stessa casa o a centinaia di
chilometri di distanza: quella sola consapevolezza, che qualcuno da qualche parte nel mondo lo
capisse, era sufficiente a sostentarlo.

Tornarono a casa prendendo una strada diversa; stavolta Nathan non vedeva negozi, solo i balconi di
tanti piccoli appartamenti, fiori colorati ai parapetti, disegni vivaci -- di bambini? -- alle
finestre. Charlotte sembrava ancora più pensierosa che all'andata, e melanconica nel suo silenzio,
ma mai Nathan si sarebbe azzardato a chiederle cos'avesse, per quanto si sentisse in debito: ogni
volta che ella s'interessava a lui, gli pareva un'invasione giusta, considerata, amorevole; il
contrario, a suo parere, sarebbe invece stato odioso. D'altronde, che consigli avrebbe mai potuto
dare a \emph{lei}, realizzata, perfetta, imperturbabile? Sarebbe stata una bestialità: l'alunno che
rimprovera il maestro, o il reo che condanna il giudice, o il peccatore che predica al santo.

\plainbreak{1}

Quel posto, la riva del fiume, divenne il \emph{loro} posto, tanto che Charlotte, almeno una volta
al giorno, prendeva il cappotto e chiedeva solo con quel suo dolcissimo sorriso: «Andiamo?», senza
che servisse aggiungere altro, spesso quando il cielo si colorava di rosa e arancio e si rifletteva
nell'acqua. Allora Nathan s'alzava, felice di potersi sottrarre al pianoforte, e la seguiva
silenziosamente. A volte parlavano, altre preferivano ascoltare l'acqua scorrere, le cicale cantare,
il vento frusciare; guardare i nonni giocare coi nipoti, i padri e le madri coi figli; ammirare le
barche passare. Charlotte aveva sempre amato quel momento della giornata, ma poterlo condividere con
qualcuno era\dots bello.

Charlotte non sapeva, non comprendeva bene cosa provasse per Nathan: mai le era capitato di
affezionarsi in tal modo a uno studente, di gioire per i suoi successi, di struggersi per le sue
disgrazie, di godere della sua compagnia. Naturalmente affettuosa, pure erano quelle sensazioni
nuovissime e spaventose, perché sapeva che prima o poi tutto sarebbe finito, che Nathan sarebbe
tornato nel suo mondo, New York, così diverso dal suo; e all'improvviso la fretta, l'ambizione, la
brama da cui era fuggita per tutta la vita le erano parsi così incredibilmente affascinanti e
appetibili.

E così come temeva e desiderava la competizione, così fuggiva dal padre e lo inseguiva, incalzata
dai discorsi fatti con Nathan il primo pomeriggio al fiume e quelli successivi. A Charlotte pareva
di essere incapace di allontanarlo, di dimenticarlo e ugualmente incapace di riavvicinarglisi, di
ripristinare i rapporti infranti. Di questo incolpava e ringraziava Nathan, perché fino a quel
momento ella aveva tenuto Nicholas a distanza sicura, quanto bastava perché il padre, o quantomeno
il ricordo, ché solo quello le restava, non potesse più arrecarle danno. Ora, invece, le era
impossibile ignorarlo, ed era una tortura; ma una a cui non le dispiaceva poi tanto sottoporsi,
dacché portava con sé l'eccitazione dell'avvenire, del mistero; dacché portava con sé la speranza.

Ed ella aveva un bel da fare dicendosi che dieci anni prima aveva fatto bene ad andarsene di casa,
che la sua non era stata una reazione esagerata ma giustissima, giustificatissima, meritatissima da
quel padre che le aveva voltato le spalle quand'ella ne aveva più bisogno; tutto ora sembrava
suggerirle che quella fuga non era stata che un capriccio, un colpo basso, una ferita inflitta al
padre per il puro e semplice gusto di fargli male. Chissà, se fosse rimasta, se avessero
parlato!\dots ma adesso che importava? L'affare era concluso, ella era rimasta senza un padre, il
padre senza una figlia; ne soffrivano e infantilmente fingevano che non gliene importasse nulla.

\plainbreak{1}

Nicholas ricordava con una certa simpatia -- e un po' di compassione -- i suoi primi concerti: le
mani gli sudavano tanto che temeva di non poter suonare; eppure, in un modo o nell'altro, era sempre
riuscito a strappare al pubblico uno scroscio di applausi, e il suo terrore non era riuscito a
mettere in ombra l'incommensurabile talento di cui era dotato. Ora, dopo trent'anni dalla sua prima
volta, non si sentiva affatto intimorito: saliva sul palco, si siedeva e suonava; era presente in
ogni momento, preciso, metodico, rigoroso; non c'era un solo gesto che non fosse pianificato, non
c'era spazio per l'ansia, non poteva nascere l'errore: solo la musica esisteva, e tramite la musica
il musicista, critico e distaccato.

Per questo, Nicholas non si pentiva delle rinunce fatte per raggiungere la propria fama: se non era
la musica, egli semplicemente non era. Fermamente convinto che ciascun uomo avesse uno scopo, era
altrettanto convinto che il suo scopo fosse suonare; e per suonare aveva messo tutto da parte: gli
amici, gli svaghi\dots e la famiglia. Non si pentiva di nulla, non rimpiangeva nulla: aveva
realizzato la propria missione e tanto gli bastava; stava agli altri regolarsi di conseguenza, o
quantomeno, laddove non fossero in grado di supportarlo, non intralciarlo.

Eppure, perfino Nicholas doveva riconoscere che non tutto lo spazio della sua vita poteva essere
occupato dalla musica; non perché non si convenisse, ma perché non era tecnicamente possibile
suonare in continuazione. Allora, come occupare il vuoto tra un momento e l'altro? Quello era sempre
stato il problema di Nicholas, dacché agli uomini comuni lo svago, la società, i piaceri mondani
riuscivano naturali e quasi istintivi, mentre a lui erano artificiali, inutili, d'intralcio; gli
pesavano come peccati inconfessabili, colpe da espiare, quasi che lo studio matto, la pratica
infinita e la critica feroce non gli valessero un pomeriggio di dolce far nulla.

Così, dopo la fine di ogni concerto lo prendeva ora un'ansia simile a quella che aveva tormentato
l'inizio del primo; egli non vedeva l'ora di ritrovarsi lì, seduto dinanzi al pianoforte, con o
senza un pubblico, perché adesso non sapeva che fare. Anche stavolta si trovava nell'auto, in
parcheggio, con le mani sul volante e il motore spento. Cosa fare? Andare a casa, e poi? Mangiare,
bere, respirare; dormire addirittura, anche se con l'inarrestabile avanzata degli anni tale
necessità si faceva sempre meno pressante\dots e poi? Non aveva amici, solo collaboratori. Non aveva
passatempo. Non aveva una famiglia; l'aveva avuta e l'aveva lasciata andare, per una negligenza in
cui s'era poi intestardito.

Lo squillo del telefono lo fece sussultare. Indossò gli occhiali, ma non riconosceva il numero sullo
schermo; che fosse la stampa? Era raro oramai, passato l'entusiasmo iniziale, che lo chiamassero o
che gli chiedessero interviste, e a lui andava bene così: nonostante il suo mestiere, non aveva mai
amato trovarsi sotto i riflettori. Gli pareva di rubare tempo prezioso allo strumento, di fare un
torto a chi lo ascoltava; e non era neanche un grande oratore. Chi lo chiamava, allora, qualche
collaboratore?

Rispose all'ultimo momento; per lo meno non avrebbe dovuto lottare per decidere come passare i
successivi minuti della propria vita.

«Pronto?»

«Papà, sono io.»

Un passante che si fosse trovato lì in quel momento avrebbe notato Nicholas Barnes, il musicista
abilissimo, imperturbabile e disciplinato impallidire quanto la sua già chiara carnagione
ulteriormente gli permettesse.

\plainbreak{1}

A due anni di distanza dal matrimonio, la vita di Kate non era così felice come la donna aveva
immaginato. Era accaduto infatti che, dopo la depressione che lo aveva attanagliato e da cui Kate lo
aveva aiutato a uscire, Thomas si fosse gettato anima e corpo nel lavoro, tanto da dimenticare i
propri doveri di marito, o più in generale di animale sociale: lavorava di giorno, di notte, durante
le feste, i compleanni, gli anniversari; \emph{gli anniversari!} Aveva dimenticato quello del suo
matrimonio, stando al lavoro tutto il giorno, e quand'era tornato Kate non aveva avuto il coraggio
di farglielo notare.

Non era stato un cambiamento repentino, però; piuttosto si era trattato di una trasformazione
graduale ma inarrestabile e -- Kate temeva -- irreversibile. Da che si dedicava completamente a lei,
ritenendo il lavoro funzionale a garantirle un regime di vita accettabile, il lavoro era diventato
il fine ultimo, e Kate un mezzo, una collaboratrice: ella aveva infatti, per un certo tempo e con
immenso piacere, aiutato Thomas in alcune questioni finanziarie che riguardavano la nuova azienda.
Si era sentita artefice della propria fortuna, fulgido esempio del successo imprenditoriale
costruito su una buona idea, esecutori intelligenti e tanta, tanta fortuna. Le sembrava tutto
meraviglioso, perfetto, indistruttibile.

Poi l'azienda, e con essa ella e Thomas, si era arricchita. Il marito aveva assunto un pugno di
persone per sostituire Kate, così che una mattina, mentre ella si stava vestendo, le aveva detto:
«Non serve che tu venga», e aveva continuato a ripeterlo per diversi giorni, finché Kate non vi si
era abituata e aveva smesso anche solo di pensarci. Quello era stato solamente il primo di una lunga
serie di rifiuti che Thomas aveva mosso a Kate, e il meno doloroso di tutti; anche se, sul momento,
non sembrava affatto così. Ma col tempo Thomas aveva preso a trascurare del tutto la moglie, tanto
che oramai raramente le rivolgeva la parola, se non per questioni essenziali e pratiche. Avevano
smesso di parlare, di discutere, di far l'amore; era come se Kate non esistesse più, rimpiazzata da
una manica di avvocati.

Quando la poveretta s'azzardava ad avanzare qualche obiezione parlando al marito di come si sentisse
inutile, Thomas scrollava le spalle. «Lavoro per noi,» diceva, «perché possiamo permetterci questa
vita». Poco importava che i loro conti in banca potessero mantenere quattro generazioni. Per Thomas,
nulla di ciò che avevano era abbastanza: c'era sempre, \emph{sempre!}, qualcosa da migliorare, da
aggiungere, da rifinire. E bisognava assolutamente farlo per non perdere il fasto acquisito. La
faceva sentire colpevole di quella mole di lavoro, perché tutte sulle spalle di Kate -- e di chi
altri, se no? -- gravavano le spese del proprio mantenimento; non s'arrischiava più a chiedergli un
centesimo, prendendo solo ciò che le era dato dal marito di propria volontà e temendo di dilapidare
la fortuna tanto faticosamente guadagnata.

Così, senza che Thomas le avesse mai detto nulla in tal senso, si sentiva un parassita e un'ingrata:
che contributo portava oramai alla vita del marito? I rapporti tra loro erano pressoché inesistenti,
e Thomas non si lasciava neanche più aiutare al lavoro, convinto com'era che fosse l'unico in grado
di far le cose come conveniva farle. Kate non esisteva, e di quest'inesisteza soffriva immensamente:
il suo eroe si era trasformato in carceriere, e si sentiva stupida perché di questa prigionia si
sarebbe potuta liberare con una parola, una chiamata, una firma; eppure restava, pazientava,
attendeva\dots che cosa? Che Thomas rinsavisse? Era fuori discussione. Allora di abituarsi a quella
situazione? Il giorno in cui né l'indifferenza di suo marito, né la mancanza di uno scopo sarebbero
stati più motivo di tristezza? A tanto si era ridotta?

\plainbreak{1}

La sua voce era stata sufficiente perché tutto quanto tornasse in ordine, perché un decennio di
parole taciute, colpa e astio venissero cancellati d'un tratto. Nicholas non aveva idea che sarebbe
stato così semplice. Se solo avesse saputo! Se gli avessero detto che bastava sollevare il telefono
e chiamare la figlia -- certo, avrebbe prima dovuto rintracciarne il numero -- perché entrambi si
rendessero conto degli sbagli commessi, delle imprudenze, delle risoluzioni avventate!\dots Com'era
dolce il perdono, tanto più dolce perché era un perdono inespresso, silenzioso, implicito nel
semplice gesto del parlare!

Certo, oltre al perdono c'era altro di non detto: pensieri, idee, sensazioni meno nobili\dots
l'ombra di una fuga vigliacca, di un'ignobile assenza che erano durate fin troppo per entrambi, e
senza che questi se n'accorgessero, e una timidezza d'intenti, come se né Nicholas, né Charlotte
fossero veramente sicuri di voler rimarginare il rapporto distrutto anni prima, come se temessero
che tutto quanto si ripetesse, come se non si fossero dimenticati, nemmeno per un istante, di quella
ferita mal rimarginata. Ma ora la accettavano; accettavano quel baratro che li separava ed
esprimevano la volontà di superarlo.

Non ora, però! C'era tempo per riconcilirsi formalmente, per parlare con onestà delle proprie
divergenze; ora era tempo di una tregua, ora Nicholas era ebbro e bruciava dal desiderio di sapere
tutto -- tutto! -- ciò che la figlia aveva fatto fin dall'abbandono della casa paterna. Voleva
saperlo e soprattutto voleva vederla; si rese conto di non saper bene nemmeno più che volto avesse
il sangue del suo sangue: aveva una o due fotografie in casa, ma le guardava raramente, perché non
voleva avere nulla a che fare con Charlotte e, segretamente, perché la vista di lei gli era troppo
dolorosa, un ricordo di ciò che aveva perso per via della codardia di cui egli -- ed egli solo --
era colpevole.

Così Charlotte gli disse tutto: del suo arrivo a Ebensburg, dei molti lavori che aveva fatto, delle
piccole battaglie di ogni giorno e, infine, della decisione di mettersi a insegnare; e qui Nicholas
si rese conto che nessuna lite li avrebbe mai separati tanto da sopprimere l'amore di Charlotte per
lo strumento: potevano odiarsi, disprezzarsi e insultarsi fino a morirne, ma Charlotte, alla fine di
una giornata faticosa, avrebbe comunque provato il desiderio di sedersi, comporre e suonare, di
esprimersi attraverso la musica proprio come il padre, solo in maniera più intima e leggera. Il suo
cuore si gonfiò di orgoglio sapendo dell'ultimo studente della figlia, e non perché egli tenesse in
alcuna considerazione il valore del denaro, qunato piuttosto perché si rendeva conto che Thomas
Westford avrebbe assegnato solo a un conclamato talento l'istruzione del proprio erede.

Sebbene Charlotte lo avesse chiamato nel tardo pomeriggio, i due si salutarono solo a notte fonda,
con una promessa: il prima possibile, Nicholas sarebbe andato a trovarla a Ebensburg -- giacché ella
era impossibilitata a muoversi nei mesi successivi per via di Nathan -- per vedere lo studente
recalcitrante e cercare di inculcargli un poco della materia. Questo almeno era lo scopo dichiarato:
in verità i due morivano dalla smania di vedersi.

\plainbreak{1}

Quando Nathan vide scendere Charlotte, la mattina successiva, rimase assolutamente senza parole. Non
era solo la cura che aveva messo nella scelta del vestito, che metteva così meravigliosamente in
risalto gli occhi azzurri; non era nemmeno l'attenzione con cui si era truccata, in modo che la
naturale bellezza fosse ancora più chiara, ammesso che fosse possibile. Piuttosto, Charlotte aveva
quella mattina qualcosa di\dots diverso. Si sarebbe potuto dire che fosse radiosa, che la solita
aura di malinconia avesse deciso, per un giorno soltanto, di abbandonare il suo animo.

«Sei bella.»

«Come?»

Non riusciva a credere di averlo detto ad alta voce. Ma come diavolo gli era venuto in mente? Che
cosa aveva in quella testa?

Charlotte lo guardava, con aria interrogativa. Indubbiamente stava pensando al suo commento
inappropriato. Nathan si sentì arrossire, e il pensiero di arrossire lo fece arrossire ancora di
più.

«Volevo dire\dots»

«Grazie.»

Charlotte sembrava pensierosa, come se stesse decidendo se dirgli o meno qualcosa. E se lo avesse
rimproverato? Oh, Dio, non avrebbe mai e poi mai potuto sopportare la vergogna.

«Oggi verrà mio padre. È tanto tempo che non ci vediamo\dots» disse, più a se stessa che a lui.

«Non preoccuparti, mi leverò di mezzo. Penso che andrò a fare una passeggiata.»

La donna esitò ancora, stavolta più a lungo.

«Io\dots vorrei che restassi.»

Quale gioia e quale paura! Anche se non lo avrebbe mai ammesso, Nathan aveva provato una fitta di
gelosia sapendo che qualcun altro avrebbe avuto accesso all'intimità di Charlotte. Era tutto così
assurda. Non aveva nessun diritto, non poteva avanzare nessuna pretesa\dots Eppure le si sentiva
così\dots vicino. Quasi che lei fosse sua\dots

«Resta, Nathan» ripetè ella, stavolta decisa. «Per piacere.»

Nathan quasi le parlò sopra.

«D'accordo.»
