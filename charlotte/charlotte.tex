\documentclass[a4paper,oneside,11pt]{memoir}

\usepackage[italian]{babel}
\usepackage[utf8]{inputenc}
\usepackage[T1]{fontenc}

\title{Charlotte}
\author{Alessandro Desantis}
\date{}

\chapterstyle{dash}
\pagestyle{plain}
\frenchspacing

\begin{document}

\begin{titlingpage}
\maketitle
\end{titlingpage}

\chapter{Fantasmi}

L'ufficio di Thomas Westford si trovava all'ultimo piano di un grattacielo
completamente occupato dalla sua compagnia. La Westford Dynamics era impegnata
in una moltitudine di settori, ma la percentuale maggiore dei guadagni era
ricavata dall'ingegneria aerospaziale, grazie soprattutto ai numerosi contratti
con il Dipartimento della Difesa. Ogni volta che guardava fuori dall'immensa
vetrata dietro la propria scrivania e vedeva l'intera New York, Thomas non
poteva fare a meno di pensare che avesse il mondo in pugno.

La sua carriera era incominciata quasi trent'anni prima, al liceo. Thomas non
era nessuno: suo padre passava le proprie giornate steso sul divano, ubriaco, e
sua madre lavorava in un supermercato per pagare i vizi del marito. In cambio ne
riceveva ogni giorno una sostanziosa dose di lividi e cicatrici. L'uomo
ricordava ancora il giorno in cui il padre le aveva spento una sigaretta su un
braccio senza alcun motivo.

Con il figlio non se l'era mai presa, ma solo perché era troppo vigliacco per
farlo, e Thomas l'aveva detestato per quella discriminazione: avrebbe preferito
che quella tortura venisse riservata entrambi; almeno così avrebbe potuto
combattere il senso di impotenza che lo attanagliava, smettere di essere uno
spettatore complice e diventare, invece, una vittima. Ma ogni volta che
assisteva all'ennesima violenza senza avere il coraggio di reagire, la sua
insicurezza aumentava. Si sentiva debole, fragile e codardo per non essere in
grado di difendere la madre.

Aveva passato gli anni del liceo tentando di dimostrare agli altri e a se stesso
che non era un rammollito, che sarebbe potuto diventare qualcuno. Voleva
fargliela vedere. Grazie alla sua mente brillante era riuscito a ottenere una
borsa di studio per Harvard, dove si era laureato con il massimo dei voti.

Infine suo padre era morto, portato via dal cancro ai polmoni. Nessuno lo aveva
pianto, se non qualche lontano e ipocrita conoscente che lo ricordava come
\emph{una bravissima persona}. Thomas pensava con sollievo che per la madre
fosse finalmente giunta l'occasione di voltare pagina, ma una settimana più
tardi la donna aveva deciso di mischiare qualche sonnifero di troppo con un
bicchiere di vino scadente, cadendo in un sonno liberatorio e senza risveglio.

Uno psicologo gli aveva spiegato che sua madre, per quanto ne ricevesse solo
dolore, era indissolubilmente legata al marito. Con la sua morte le era sembrato
di perdere l'unica ragione che aveva di esistere. Doveva essersi sentita vuota,
priva di qualsiasi scopo e significato. Era andata incontro al suicidio come una
macchina obsoleta andava allo smantellamento.

Allora Thomas si era lanciato anima e corpo nel suo progetto e in pochi mesi,
senza l'aiuto di nessuno, era riuscito a fondare la Westford Dynamics. Ci erano
voluti tre anni per mettere a punto il primo prototipo di un componente ormai
ampiamente utilizzato dalla {\scshape Nasa}. Tre anni in cui Thomas si era
venduto agli investitori, aveva rilasciato interviste, aveva lavorato nei fine
settimana. Tutto per quella dimostrazione di forza che gli avrebbe permesso di
elevarsi dalla massa e di vincere i pregiudizi che lui stesso nutriva nei propri
riguardi.

Ora, a distanza di quindici anni, la Westford Dynamics era una delle aziende più
potenti del mondo. Nonostante il tempo passato, Thomas si trovava spesso a
pensare alle stesse parole che gli erano venute in mente quella sera, quando
aveva personalmente firmato il primo contratto per la produzione in larga scala
della sua invenzione.

\emph{Ce l'ho fatta. Non sono come mio padre.}

\plainbreak{1}

Ebensburg è una microscopica cittadina di nemmeno quattromila abitanti nella
Pennsylvania, uno di quei luoghi piccoli, accoglienti e dignitosi dove tutti
conoscono tutti. Mentre l'elegante Mercedes nera attraversava il centro urbano,
Nathan vide molti negozianti aguzzare la vista per cercare di capire chi fossero
i passeggeri dell'auto. Se ne accorse anche Gates.

«Meno male che i vetri sono oscurati, eh?» disse nervoso.

No, non era meno male. Nathan non avrebbe voluto altro che entrare in città come
una persona comune, invece di ostentare così schifosamente la propria ricchezza
e la propria estraneità a quel luogo. Invece suo padre aveva fatto in modo che
sapessero immediatamente con chi avevano a che fare. Si sentì colpevole verso
quegli uomini e quelle donne che li osservavano con un misto di meraviglia e
sospetto.

Allo stesso tempo seppe che, non appena l'uomo irritante che lo accompagnava si
fosse dileguato come era previsto, si sarebbe sentito a casa: Ebensburg, dopo
nemmeno cinque minuti, lo faceva sentire il benvenuto. Nathan guardava i prati
verdi e le minuscole, deliziose villette a schiera e sentiva sprigionarsi dentro
di sé un nuovo, caloroso affetto.

L'auto si fermò davanti a una villa un poco più lontana delle altre. Allora,
rendendosi conto che quell'allegra gita era terminata e ricordandosi dello scopo
della sua visita, Nathan non poté che rabbuiarsi.

Un anno prima suo padre aveva deciso, per chissà quale ragione, che dovesse
imparare a suonare il pianoforte. Per sua sfortuna, Nathan non nutriva alcun
amore per lo strumento, e trovava le lezioni che gli venivano impartite noiose e
prive di significato: perché mai avrebbe dovuto imparare? Suo padre parlava di
disciplina, i suoi insegnanti di arte, ma entrambe richiedevano una passione che
lui non possedeva.

Di tutto questo non aveva mai parlato con nessuno, nemmeno con Thomas. Gli
pareva che, se solo avesse voluto, il padre avrebbe percepito il suo astio verso
il pianoforte. Credeva che il proprio doloroso silenzio al riguardo fosse un
messaggio sufficientemente chiaro, si aspettava che i genitori intuissero il suo
stato d'animo con una sola occhiata. Ma non avevano capito, o fingevano di non
capire, e Nathan era infuriato per la loro indifferenza.

Volendo punirli, faceva di tutto per essere un pessimo studente. Dopo un anno di
studio era quasi del tutto fermo al punto di partenza. Di fronte ai suoi
fallimenti, al padre non era mai venuto il sospetto che Nathan non avesse alcun
interesse a proseguire quel percorso. Pensando piuttosto che fossero gli
insegnanti a essere degli incapaci, aveva deciso di cercare qualcuno che fosse
adatto per le esigenze del figlio. Quel qualcuno era Charlotte.

\plainbreak{1}

Charlotte Barnes era la figlia di Nicholas Barnes, uno dei pianisti più famosi
del suo tempo. La donna ricordava quando, da bambina, ascoltava il padre suonare
per ore intere senza mai annoiarsi. Si rannicchiava sul pavimento vicino a lui e
vedeva le sue dita muoversi elegantemente sulla tastiera. Era stato allora,
notando come con uno sforzo così piccolo fosse possibile realizzare cose tanto
belle, che aveva deciso di imparare a suonare.

Il suo era sempre stato un padre affettuoso, ma Charlotte sospettava che amasse
la propria musica più di lei. Gli unici momenti che potevano condividere erano
quelli delle lezioni; nel resto del tempo il pianista era in giro per il mondo a
dare concerti, troppo impegnato a perfezionarsi per vedere la figlia crescere.

Le sembrava che questo aspetto del suo carattere fosse peggiorato dopo la morte
della madre in un incidente d'auto, quando lei aveva sedici anni. Al funerale
Nicholas non aveva avuto il coraggio di dire una sola parola, distrutto dal
dolore. Quando negli occhi della figlia aveva visto il bisogno di rifugio e
protezione, l'uomo si era paralizzato. Non era all'altezza del compito, lo
sapeva. Non avrebbe mai potuto crescere quella ragazza e, terrorizzato dalla
possibilità di sbagliare, aveva deciso di rifugiarsi nell'unica attività che
sapesse praticare veramente bene: suonare.

Charlotte aveva proseguito gli studi da sola, dimostrandosi presto all'altezza
del padre. A differenza sua, però, aveva scelto di tenere quel dono
esclusivamente per sé. Era raro che suonasse per qualcuno; tramite la musica
confessava l'inconfessabile, dava sfogo a pensieri e sensazioni di cui non
avrebbe osato parlare neanche a se stessa. Non voleva che nessuno varcasse quel
confine, infrangendo l'ultima barriera della sua intimità.

Con Nicholas aveva avuto una moltitudine di discussioni al riguardo: lui
sosteneva che l'artista vivesse per servire la società e, ai suoi occhi, la
decisione della figlia era uno spreco di talento. Ne avevano parlato più volte,
ma l'uomo aveva ceduto quando, dopo averle chiesto per l'ennesima volta il
motivo della sua scelta, Charlotte non era riuscita a trattenersi ed era stata
costretta a rivelargli la verità.

«Non voglio diventare come te!» gli aveva urlato, con una furia di cui non
l'avrebbe mai creduta capace. Subito dopo era diventata paonazza e aveva
mormorato alcune parole di scusa, ma Nicholas si era chiuso in un silenzio
profondo, reso ancora più penoso dal senso di colpa che sentiva aleggiare su di
sé come un macigno sospeso nel vuoto.

L'indomani, senza che nessuno glielo chiedesse, Charlotte aveva fatto le valigie
e si era trasferita, abbandonando la casa di famiglia a Washington. Aveva scelto
Ebensburg perché era la città di sua madre e perché era piccola: lì non avrebbe
corso il rischio di diventare famosa o di costruirsi una carriera. Poteva vivere
nell'anonimato, proprio come desiderava.

Presto aveva incominciato a insegnare pianoforte e, con sua sorpresa, aveva
scoperto che si trattava di un'attività abbastanza redditizia da poterne vivere.
A volte i suoi studenti le chiedevano di suonare, ma lei rifiutava sempre. Gli
unici che avevano l'onore di sentirla all'opera erano coloro che, passando
vicino alla sua casa di sera, sentivano l'eco di alcune note fuggire dalle
finestre illuminate.

\plainbreak{1}

Vedendo l'elegante facciata della villa, Nathan era riuscito a farsi un'idea
abbastanza precisa di quella che sarebbe stata la sua nuova insegnante: nella
sua mente si era profilata l'immagine di una giovane zitella che, di fronte al
rifiuto del mondo, aveva deciso di rinchiudersi nello studio, senza trarne
realmente alcun piacere, ma pronta a tiranneggiare i propri studenti finché non
avessero raggiunto la perfezione.

Per questo rimase disarmato quando Charlotte li raggiunse nel vialetto, con il
suo sorriso sincero e i suoi modi dolci. Si trovava di fronte a una donna sui
trent'anni la cui pelle bianchissima suggeriva origini inglesi. Si muoveva con
un'energica allegria. Se odiava il mondo, era brava a nasconderlo.

Charlotte strinse la mano a entrambi e, nel farlo, guardò il ragazzo dritto
negli occhi. Quelli di lei erano verdi, profondi e felici, quelli di lui
sfuggenti per l'imbarazzo.

«Tu devi essere Nathan. Ho letto di tuo padre qualche volta» gli disse, ma
vedendo l'ombra che passò sul suo volto a quell'ultima frase, fu ben attenta a
lasciar cadere l'argomento.

Scambiò qualche parola di circostanza con Russell, ma a Nathan parve che si
fosse stancata presto e in cuor suo fu felice, sebbene provasse un po' di pena
per Gates, che era evidentemente rimasto impressionato da Charlotte. Per far
colpo, l'uomo prese l'immensa valigia per portarla in casa, ma la valigia doveva
essere pesante o lui fuori allenamento, perché arrancava comicamente. Tentò di
assumere un'andatura naturale, nonostante il suo volto fosse rosso per lo
sforzo.

Charlotte lo seguì e Nathan chiudeva la fila. Poteva vedere, appena sotto i
lunghi e mossi capelli ramati della donna, le curve dei suoi fianchi muoversi a
ogni passo, coperte dal vestito bianco che aveva indossato. Si impose di non
fissarla ma, per quanto si sforzasse, di quando in quando i suoi occhi tornavano
a posarsi su quelle forme attraenti. Era una sciocchezza, una naturale e
comprensibile reazione alla grazia della donna, ma non poteva non sentirsi in
colpa per quei pensieri così terreni.

Giunti all'ingresso, Charlotte e Russell parlarono ancora un po' del più e del
meno, quindi l'uomo si congedò, ancora a disagio per la bellezza della sua
interlocutrice e per la magra figura che aveva fatto, e si diresse verso l'auto.
Il giorno dopo sarebbe tornato alle dipendenze del signor Westford, con tutte le
spiacevoli incombenze che ciò comportava.

Nathan si trovò a dispiacersi del suo destino.

\plainbreak{1}

Quando Thomas Westford in persona l'aveva chiamata per chiederle se potesse
insegnare a suo figlio, Charlotte non aveva potuto fare a meno di chiedergli se
fosse \emph{quel} Thomas Westford. L'uomo aveva risposto di sì, e lei aveva
potuto immaginarlo nel suo immenso ufficio che sorrideva per quel riconoscimento
della sua fama. Si era data della stupida: cosa importava chi fosse? Doveva
concentrarsi sul ragazzo, non sul padre.

Si era fatta raccontare la storia di Nathan, e stavolta era stata lei a
sorridere. Aveva subito capito cosa gli fosse successo perché era una storia
nota; sapeva che, volendogliela inculcare a forza, genitori e insegnanti avevano
finito per fargli odiare la musica. Probabilmente si sentiva tradito,
abbandonato, condannato da chi lo circondava.

Infine, quando Westford le aveva parlato della difficoltà di trovare un alloggio
decente in città, Charlotte si era offerta di ospitare Nathan. La sua modesta
villa non era sicuramente all'altezza degli standard a cui era abituato, però
era senza alcun dubbio più confortevole di una pensione. Inoltre quella
vicinanza sarebbe stato d'aiuto per entrambi, permettendo di stabilire più
facilmente un rapporto.

Gli avrebbe dato la camera grande, quella che lei non aveva mai usato. Per
l'occasione l'aveva spolverata e aveva lavato per terra finché non si era potuta
specchiare nel pavimento, augurandosi ironicamente che Westford non le facesse
causa per le terribili condizioni igieniche in cui obbligava a vivere il suo
prediletto.

Poi c'era stato il dramma della cena. Era andata al supermercato ed era rimasta
a fissare il frigo dei surgelati per almeno cinque minuti. Più il tempo passava
e più si convinceva che la pizza fosse un piatto così abusato e banale da
sfiorare la volgarità, senza contare che quella venduta lì, a quanto ricordava,
aveva la consistenza di uno pneumatico. Optando per qualcosa di più genuino,
aveva scelto di affidarsi a Frank, il macellaio.

L'uomo era dietro il banco, sporco di sangue, e brandiva un enorme coltello.
Nonostante il suo mestiere era magro e spigoloso.

Vedendo Charlotte il suo volto si illuminò sotto la barba bianca e ispida. Le
era molto affezionato e si raccontava che, prima di sposare Nicholas, sua madre
avesse avuto una breve relazione con lui. Quando Charlotte era arrivata in
città, ancora provata dalla conversazione col padre, era stato Frank ad
aiutarla. L'aveva ospitata finché non era stata autonoma e l'aveva presentata ai
sospettosi cittadini di Ebensburg senza mai chiederle nulla in cambio. L'aveva
accolta come una di famiglia, e Dio sapeva quanto Charlotte ne avesse bisogno in
quel momento.

«Che posso fare per te, dolcezza?» le aveva chiesto con la sua voce ruvida.

«Cosa posso far mangiare a un quindicenne?»

Frank aveva inarcato un sopracciglio.

«Hamburger» le aveva detto sicuro. Con quelli non sbagli mai.»

Charlotte aveva sospirato, indecisa su cosa fosse peggio, se la pizza o
l'hamburger. Poi un dubbio l'aveva colta all'improvviso.

«E se fosse vegetariano?»

Frank aveva agitato il coltello in aria, fingendosi minaccioso.

«Allora rispediscilo nella metropoli di elegantoni da cui è venuto.»

Ora, davanti alla padella, Charlotte sperava solo che i gusti di Nathan non
fossero meno convenzionali di quelli che immaginava. Aveva continuato a girare
la carne ogni trenta secondi per paura di bruciarla, con il risultato che gli
hamburger si erano quasi del tutto smontati. Riuscì miracolosamente a metterli
nel pane e aggiunse una foglia d'insalata.

Voltandosi trovò davanti a sé il ragazzo ed ebbe un sussulto. Dopo un istante di
esitazione gli presentò il piatto, con un sorriso di scuse già dipinto sul
volto.

«Spero che vada bene. Non sono una grande cuoca.»

Nathan osservò il panino per qualche secondo.

«Va benissimo» decretò infine.

\plainbreak{1}

Mangiarono in silenzio seduti al piccolo tavolo nella cucina, concentrati
esclusivamente sul ticchettio dell'orologio. Nathan avrebbe voluto dire
qualcosa, complimentarsi per la cena, ma le parole gli morivano in gola. Non
sapeva cosa lo atterrisse e attirasse tanto in Charlotte; probabilmente era
quella tranquilla serenità con cui affrontava la vita, come se non potesse
capitare nulla di abbastanza grave da giustificare la sua preoccupazione.

Continuava a spiare di sottecchi ogni suo più piccolo gesto: l'eleganza con cui
addentava il panino, l'attenzione con cui si portava alla bocca il bicchiere
dell'acqua dopo aver pulito gli angoli della bocca con il fazzoletto\dots La
osservava in modo discreto eppure avido, quasi volesse memorizzare la precisa
sequenza dei suoi movimenti, e a ogni secondo era sempre più affascinato dalla
sua grazia naturale, non ostentata.

Stava viaggiando con la mente. La voce di Charlotte lo riportò alla realtà.

«Allora, cosa porta un famoso newyorkese come te in questa città dimenticata da
tutti?» gli chiese. Nathan sapeva che l'attenzione di Charlotte era
completamente per lui e, nonostante il brivido che gli provocò, questo lo fece
sentire a disagio: non era un argomento che volesse affrontare. Non con lei, non
in quel momento.

\emph{Ti prego}, pensò, \emph{non rovinare tutto}.

«Tu insegni pianoforte\dots» rispose, ma subito si pentì per l'idiozia della
sua constatazione. Allora aggiunse deciso: «Voglio imparare.»

Si stupì di quanto gli fu semplice mentire e se ne vergognò, perché gli sembrava
di aver tradito la fiducia di Charlotte.

«Sì, così dicono» affermò lei con un sorriso ironico, e per un istante Nathan
pensò che sarebbe finita lì. Subito però la donna tornò alla carica: «Ma perché
hai sentito il bisogno di percorrere centocinquanta miglia per venire fin qui?
Perché io e non qualunque altro insegnante?»

Quando gli parlava, gli occhi scintillavano di curiosità e tutto il suo corpo si
sporgeva sul tavolo, quasi a voler sentire meglio la risposta. Nathan seppe di
aver trovato una persona che voleva comprenderlo, capire cosa stesse passando.
Era tutta la vita che cercava qualcuno come Charlotte; qualcuno a cui importasse
di lui solo in quanto essere umano, e non perché era il figlio di un
miliardario.

Avrebbe potuto e voluto parlare di tante cose con Charlotte, e l'ultima di
queste era il rapporto col padre. Sentiva il bisogno di dirle come spesso si
sentisse inadatto, convinto che fosse lui il problema, e di come i suoi stessi
desideri e le sue stesse aspirazioni gli sembrassero stupide, perché gli avevano
insegnato a pensare con la testa di altri piuttosto che con la sua.

\emph{Non sono io che ho deciso, è stato mio padre, perché è lui che decide
tutto. Ma per una volta sono felice che abbia scelto al posto mio: se non mi
avesse spedito qui non ti avrei incontrata.}

Sentiva che lei poteva aiutarlo, che la salvezza era dietro l'angolo. Se solo
avesse avuto il coraggio di afferrarla, forse qualcosa sarebbe potuta cambiare.
Invece preferì proseguire dritto per la sua strada. Lo fece perché era stanco di
combattere e perché aveva paura. Non per sé, ma per Charlotte: temeva che
sarebbe rimasta stritolata sotto il peso dei suoi fantasmi. Non voleva che
quella diventasse una sua battaglia.

«Ho cambiato un paio di insegnanti ma non mi sono mai trovato bene» disse,
stringendosi nelle spalle. Sperava che lei non avrebbe avuto il coraggio di
continuare.

«E cosa ti fa credere che con me sarà diverso?»

Calò il silenzio. Ancora una volta, a Nathan non mancavano le parole, ma il
coraggio di pronunciarle. Come spiegare qualcosa che neanche lui era certo di
capire completamente? Come dirle che in fondo ai suoi occhi gli era sembrato di
vedere un barlume di speranza? Come poteva, a cuor leggero, assegnarle un
compito tanto gravoso?

Avevano entrambi finito di cenare e, di fronte al suo mutismo, Charlotte decise
di alzarsi per sparecchiare, sollevandolo dall'onere di rispondere. Nathan fece
lo stesso e le loro mani si incontrarono al centro del tavolo. Il ragazzo
ritrasse la sua troppo in fretta e nel farlo colpì un bicchiere, rovesciandolo.

Sentì le guance avvampare. Si offrì di pulire, ma ovviamente Charlotte non
glielo permise. Mentre lei asciugava l'acqua finita sulla tovaglia, facendogli
notare come non le fosse mai piaciuta, poté ammirare ancora una volta la pacata
allegria dei suoi movimenti, così lontana dal frenetico nervosismo in cui
vivevano coloro che conosceva.

Qualche minuto dopo si diedero la buonanotte. Nathan osservò la donna
allontanarsi silenziosamente nella penombra del corridoio, i lunghi capelli
rossi che ondeggiavano a un ritmo regolare. Gli sarebbe mancata in quelle
lunghissime ore che separavano la sera dalla mattina.

\plainbreak{1}

Charlotte scivolò nella vasca, sospirando mentre l'acqua bollente la avvolgeva
lentamente. Nemmeno un bagno caldo avrebbe potuto lavare via i pensieri che la
tormentavano, ma la sensazione del vapore che le si condensava sul seno e sul
viso era una piacevole distrazione da quel turbine di consapevolezze a cui
tentava di sfuggire.

Quando, quasi dieci anni prima, aveva deciso di andarsene di casa, i rapporti
con suo padre non si erano interrotti di colpo; era stato un processo graduale e
sempre meno doloroso. Per i primi tempi si erano telefonati, parlando di
Ebensburg, delle rispettive vite, e talvolta perfino di sua madre. Non
discussero della loro lite, forse perché nessuno dei due ne capiva pienamente la
causa né il significato.

Con lo scorrere dei mesi, però, quelle conversazioni si erano fatte più banali e
sporadiche, finché non erano cessate del tutto. Erano passati cinque anni
dall'ultima volta che avevano avuto un contatto di qualunque tipo. Ogni tanto la
donna riceveva notizie di suo padre tramite qualche conoscente, ma si limitava
ad accoglierle con una freddezza che aveva smesso di stupirla molto prima,
quando si era convinta che il tempo riuscisse a guarire tutto.

Nicholas era ormai una presenza marginale nella sua vita, un satellite in orbita
a debita distanza. Non pensava a lui da così tanto che persino il suo volto era
difficile da ricordare, e insieme a suo padre era caduta nell'oblio anche la
sofferenza che le aveva causato.

Ma quando Nathan era venuto da lei e, nonostante i suoi sforzi per nasconderlo,
Charlotte era riuscita a vedere nei suoi occhi lo stesso senso di abbandono e la
stessa rabbia che aveva provato alla sua età, le era sembrato all'improvviso di
non essere mai riuscita a superare nulla, di non aver fatto un solo passo dalla
soglia della casa paterna. Stava riconsiderando, ora, tutte le scelte che aveva
fatto negli ultimi anni, scoprendo con orrore di essere stata avventata e
capricciosa. Si era trovata a chiedersi dove sarebbe stata se non avesse
pronunciato quell'unica frase. Le sembrava di essere tornata al punto di
partenza, e questo l'aveva gettata nel totale sconforto.

Senza che ne provasse realmente alcun desiderio, immerse una mano nell'acqua
tiepida e la posò tra le gambe, lottando per resistere al torpore iniziale.
Chiuse gli occhi, cercando di non pensare a nulla in particolare, ma più si
impegnava in quell'esercizio e più le immagini si sovrapponevano, fastidiose e
insistenti. Sprofondò ancora di più nella vasca e la sua mano spinse sul sesso
con maggior decisione.

Dopo diversi minuti inarcò la schiena, offrendo il proprio seno a un immaginario
amante. Abbastanza lucida da rendersi conto della presenza di Nathan, soffocò un
gemito mentre l'orgasmo attraversava il suo corpo e le permetteva, finalmente,
di affogare nel piacere tutti i pensieri.

Si beò di quell'estasi ancora per qualche istante, le labbra schiuse e il
respiro tremante, tutto il suo corpo scosso dai fremiti, finché non ebbe freddo
e si accorse di avere il collo indolenzito per lo sforzo. Uscì dalla vasca e
indossò un accappatoio nero. Lasciò che l'acqua le si asciugasse addosso, quindi
si infilò in una vestaglia di seta.

Passando davanti alla stanza di Nathan esitò, ricordandosi all'improvviso di
aver solo temporaneamente allontanato le preoccupazioni; sapeva che, non appena
avesse guardato nuovamente il suo viso, tutto le sarebbe riaffiorato alla mente.
Nonostante il suo timore fu tentata di entrare. Sfiorò la maniglia ma, per
qualche motivo, il contatto con il metallo freddo la scoraggiò.

Proseguì verso la sua stanza, chiuse la porta dietro di sé e si lasciò cadere
sul letto. Tutte le idee che avevano preso forma nella sua testa durante quel
pomeriggio divennero via via più nebulose, finché non furono una massa
indistinta, un groviglio senza capo né coda.

\chapter{Nascite}

Era stato il caso a farli incontrare molti anni addietro, quando Thomas aveva
appena fondato la sua compagnia e ancora viveva con un amico in attesa di
potersi permettere un appartamento. Entrambi erano stati trascinati da qualche
loro conoscente a una festa, ed entrambi avevano di meglio da fare. Per Thomas
era dedicarsi ai suoi progetti, ovviamente; per Kate, che aveva alle spalle una
relazione disastrata, era restare a casa a rimuginare su quale menzogna fosse
l'amore e quale valle di lacrime fosse il mondo.

Tuttora, se le avessero chiesto che cosa le fosse piaciuto quella sera in
Thomas, avrebbe risposto: il silenzio. Mentre tutti intorno a lei conversavano,
urlavano le inutilità che credevano indispensabile comunicare al resto del
mondo, egli taceva. Ogni tanto rivolgeva qualche sorriso di circostanza a un
passante, o rispondeva con frasi di cortesia a domande di cortesia, ma per lo
più si limitava a osservare, taciturno, pensieroso. Era come se fosse invisibile
a tutti tranne che a lei.

Per la prima volta nella propria vita, Kate aveva avvertito l'urgenza. L'urgenza
di agire, di osare, di vivere. Non poteva perderlo; sapeva che non ci sarebbe
stato rimedio. Si compiaceva e si vergognava di quell'ambiziosa intraprendenza.
Si sentiva bloccata dalla paura di un'altra delusione, ma mossa dalla volontà di
dimostrare a se stessa che la felicità non le era preclusa, che una vita serena
era ancora possibile. Per tutta la sera aveva danzato intorno a Thomas, senza
però osarsi mai avvicinare troppo. Aveva riscoperto la meraviglia di essere
stretta in una morsa e non volerne uscire.

Se non fosse stato per una sua conoscente, una di quelle fastidiosissime donne
che sembrano voler mettere ordine nella vita di tutti tranne che nella propria,
e si sentono per questo in dovere di dirigere e dare consigli a tutti,
probabilmente Kate non avrebbe mai avuto modo di parlargli. Questa ragazza,
dunque, con cui Kate aveva conversato in rarissime occasioni -- traendone ben
poco piacere -- le si era avvicinata.

«C'è una persona che devi \emph{assolutamente} conoscere!» aveva squittito,
trascinandola verso un punto tra la folla. Kate aveva tentato di opporre
resistenza, urlando per coprire le altre voci, ma ogni protesta era cessata non
appena si era accorta di essere condotta verso Thomas. Benché fosse
terrorizzata, non avrebbe mai osato rifiutare quell'occasione.

Pochi minuti più tardi conversavano come vecchi amici, ogni traccia di imbarazzo
sparita dal volto di lei, il carattere taciturno di lui sostituito da quello di
un intelligente e spiritoso osservatore. Kate non riusciva a capire molto del
progetto di Thomas, ma sapeva, anche ora, anche senza averci mai parlato prima,
che era destinato a un grande futuro. Quell'uomo le dava speranza, e la speranza
era esattamente ciò di cui aveva bisogno. Si vedeva accanto a lui, che sfidava
le intemperie della sorte, provata ma felice, consapevole di essere guidata
dalla mano ferma e precisa di una persona geniale.

Era tornata a casa con uno strano ma piacevole peso sul petto. In mano stringeva
un biglietto su cui egli aveva scritto il proprio numero di telefono. Thomas
Westford. Persino il nome suonava importante. Senza dubbio si sarebbe sentito
molte volte negli anni a seguire. Lo sapeva. Credeva in lui. Lo amava.

Per circa due settimane, ogni sera, aveva preso quel biglietto e lo aveva
guardato con ansia, tenendo il telefono in mano. Proprio per via dell'importanza
che l'incontro con Thomas aveva avuto, era ora attanagliata da un oscuro e
irritante terrore: che egli non si rivelasse all'altezza delle sue aspettative.
Si rendeva conto di come la sua fantasia avesse contribuito a proiettare su
Thomas il suo uomo ideale. Se si fosse sbagliata, non sarebbe sopravvissuta a
un'altra delusione.

C'era poi un'altra possibilità, che la preoccupava ancora di più: quella di
rovinare tutto. Non erano passati nemmeno sei mesi dalla sua ultima relazione,
ma già le sembravano così lontani i tempi in cui era ancora capace di amare.
Odiandosi per il modo in cui si era lasciata plasmare da un uomo, aveva promesso
a se stessa che non sarebbe mai più stata così ingenua. E proprio ora che le era
indispensabile, le pareva di aver perso quella pericolosa capacità di lasciarsi
completamente andare, di affidarsi a un altro essere umano.

Infine ronzava nella sua testa, senza che potesse trovare risposta, questa
domanda: perché egli non l'aveva ancora chiamata? Aveva il suo numero, cosa lo
tratteneva? Non era plausibile che l'avesse smarrito: ella teneva quel
foglietto, ormai indecentemente stropicciato, come una reliquia, e dunque
l'unica ragione poteva essere che non gli fosse piaciuta; che tutti quei
sorrisi, quelle battute, quella complicità fossero solo una farsa, magari un
modo per passare il tempo; che ella fosse stata una stupida a farsi tante
illusioni, e che fosse destinata a\dots

Il telefono squillava da almeno dieci secondi, ma Kate non se n'era accorta,
persa nei propri angosciosi pensieri. Quando finalmente aveva realizzato, lo
stupore era stato tale che un sussulto aveva scosso tutto il suo corpo. Aveva
messo a fuoco il numero sullo schermo e le era parso famigliare: i suoi occhi
erano andati velocemente al biglietto, sul tavolo lì accanto; leggendo cifra
dopo cifra, quasi ad alta voce, sentiva uno strano sentimento occupare il posto
della precedente sorpresa. Avrebbe voluto piangere, ma non sapeva bene il
motivo. Rendendosi improvvisamente conto che stava per perdere la telefonata,
aveva premuto il tasto di risposta con molta più foga del dovuto.

\plainbreak{1}

Quando l'uomo aveva aperto la porta, il suo grande e nervoso sorriso si era
spento. Aveva sgranato gli occhi, cercando di capire se quello che le stava
davanti fosse Thomas o il suo coinquilino. Il viso era pallidissimo, tranne per
le occhiaie gonfie e violacee che cerchiavano gli occhi rossi, forse per il
pianto. Nemmeno un capello sembrava voler stare al proprio posto, e sulla fronte
questi si appiccicavano per via del sudore che la imperlava. Le palpebre erano
socchiuse, come se fosse infastidito dalla luce. Indossava una felpa pesante,
decisamente troppo per la stagione, pantaloni e scarpe da ginnastica.

«Ciao» aveva detto, con una voce quasi impercettibile. Aveva chiuso la porta
dietro di lei, con lentezza esasperante. Quando gli era passata accanto, un
terribile odore di sudore e disperazione era salito su per le narici di Kate,
dove si era posato e contribuiva, piano piano, a convincerla che tutto quello
non fosse che un sogno. No, non poteva essere Thomas. Non era possibile che
l'uomo appassionato ed elegante che aveva incontrato meno di venti giorni prima
si fosse ridotto in quello stato. Non poteva esserci evento che lo abbattesse in
quella maniera.

Lo aveva guardato ancora: le spalle ricurve, lo sguardo spento, il naso che
gocciolava come quello di un neonato\dots La casa era schifosamente sporca, il
pavimento macchiato e coperto di fazzoletti usati, e impregnata dello stesso
odore di Thomas. Era troppo: non poteva sopportarlo; Kate era tornata sui propri
passi con una furia che aveva stupito anche lei, quasi travolgendo Thomas. Aveva
aperto la porta ed era uscita sul pianerottolo, decisa a fuggire da
quell'appartamento per non tornarvi mai più.

«Che fai?» le aveva gridato l'uomo. «Aspetta!»

Aveva cercato di fermarla, trattenendola per un braccio, ma ella si era
divincolata con tanta forza da farlo inciampare e cadere a terra.

«Kate, non andare! Ti prego!» continuava a urlare.

Aveva già sceso una rampa di scale quando i singhiozzi l'avevano raggiunta.
Thomas piangeva sommessamente, quasi avesse paura di disturbare. Piangeva
sdraiato a terra, proprio dove l'aveva lasciato. Piangeva tenendo la testa fra
le braccia, come i bambini puniti dai genitori. E anche Thomas era stato punito:
punito dal fato, dalla crudele ironia di un destino che sembrava prendersi gioco
di lui, che lo privava, in pochi giorni, del suo più grande male e del più
grande bene.

Kate si era voltata e fissava l'uomo, interdetta e spaventata, incapace di
distinguere tra incubo e realtà.

\plainbreak{1}

Questa era stata la reazione di Thomas alla morte della madre. Dopo aver riso
della debolezza di quella donna, incapace di vivere senza il proprio aguzzino,
era tornato a casa per confidarsi con l'amico con cui viveva: voleva
raccontargli come realmente si sentisse al riguardo, parlargli dell'ingiustizia
e della tristezza dell'esistenza; ma il ragazzo era partito per una vacanza in
Europa, e sarebbe tornato solamente il mese successivo. Per qualche giorno
Thomas era stato in grado di mantenere il controllo, ma la depressione lo
assaliva lentamente, e sentiva di esserne sempre più schiavo ogni secondo che
passava.

Aveva smesso di mangiare, di lavarsi, di vestirsi, di respirare. Aveva
accarezzato l'ipotesi del suicidio; non gli sembrava poi così male l'idea del
sonno eterno. Non ci sarebbe stata più sofferenza, né morte, né ingiustizia.
Solo il nulla. Ma soprattutto, avrebbe pagato la propria colpa: l'indifferenza.
Per tutti quegli anni aveva ignorato la situazione della madre: l'aveva
considerata una debole perché, pur avendo avuto più volte l'occasione di
sottrarsi a quella tortura, aveva scelto di non farlo; quale persona sana di
mente si sarebbe mai comportata così? Ora  Thomas sentiva che la propria morte
sarebbe stata una giusta pena, un modo adeguato per espiare il proprio immenso,
imperdonabile peccato.

Ma non aveva la forza nemmeno per quell'ultimo, estremo gesto; o forse era il
coraggio a mancargli. Così, piuttosto che farla finita in un breve istante,
aveva scelto di morire giorno dopo giorno, lasciando che le forze lo
abbandonassero lentamente ma irrimediabilmente. Aveva continuato a vagare per la
casa, e quando anche spostarsi gli era sembrato troppo difficoltoso, si era
seduto in un angolo senza più muoversi. Oramai aveva perso ogni sensibilità: il
suo corpo non era più suo, ed era meraviglioso lasciarsi cullare dalla morte,
sentire di non avere più il controllo, di non avere più responsabilità. Ormai
nulla dipendeva più da lui: si affidava al dio sulla cui esistenza era sempre
stato scettico, e che ora gli sussurrava all'orecchio.

Ma quel dio non voleva che morisse, o forse era il suo istinto di sopravvivenza
a ingannarlo. E così una scintilla si era accesa in lui: la morte aveva perso
tutto il suo fascino, portava con sé solo la prospettiva dell'annientamento
totale, l'incapacità di agire, e dunque di rimediare. La morte gli era preclusa,
riservata a chi aveva vissuto la vita dei giusti. Non ci sarebbe stata alcuna
dignità nella \emph{sua} morte. E poi, maledizione, aveva paura: non voleva
morire! Sapeva di poter ancora fare la differenza, di poter migliorare la vita
di altre persone per riscattare quella di sua madre, e l'idea di privarsi di
quell'opportunità gli era assolutamente odiosa: non era da lui rinunciare.

Ma non poteva farcela da solo. Aveva chiamato a raccolta le poche forze che
ancora gli rimanevano e aveva chiamato Kate, cercando di suonare il più naturale
e sano possibile. Sapeva di non poterla ingannare a lungo, ma non avrebbe mai
ammesso la propria debolezza al telefono. Quando era arrivata e Thomas aveva
visto il suo viso, aveva potuto intravedere, lontana, una pallida luce di
speranza. Ma poi aveva notato anche lo schifo dipingersi sul volto della donna,

Così aveva pianto. Di rabbia o di tristezza o di vergogna, neanch'egli lo sapeva
bene. Non riusciva nemmeno a immaginare quanto piccolo, debole e patetico
potesse apparire agli occhi di lei. Era finita, lo sentiva: quella era l'ultima
occasione che aveva, ed era riuscito a sprecarla; oltre, non ci sarebbe stato
più nulla. Tanto valeva tornare dentro casa e tagliarsi le vene, o impiccarsi, o
cadere in un sonno senza risveglio come aveva scelto di fare sua madre. Era
quello il suo destino, il progetto che c'era per lui; era sempre stato quello, e
pensare di sfuggirgli era solamente l'illusione di un folle incapace di
accettare la morte. Non avrebbe più pregato, non avrebbe più lottato: non
sarebbe servito a niente.

Ma Kate si era fermata. Aveva sentito il suo sguardo su di sé per qualche
minuto, poi i suoi passi e infine le sue mani, delle mani così delicate, candide
e sincere che incontravano le mani sporche di un codardo e di un assassino.
Quelle mani l'avevano costretto a tirarsi su, l'avevano portato dentro casa.
Thomas non si muoveva, non parlava, non protestava. Osava a malapena respirare:
era completamente in balìa di lei e delle sue cure amorevoli. Perché lo stava
aiutando? Non c'era motivo: non era nessuno, non meritava niente.

Kate lo aveva fatto sedere e, mentre attendeva pazientemente che smettesse di
singhiozzare, aveva pulito la casa al meglio delle proprie possibilità. Thomas
non si azzardava a incrociare il suo sguardo, ma poteva sentirla muoversi,
rapida e inquieta, da una stanza all'altra. Poi era tornata da lui e gli aveva
di nuovo preso le mani, obbligandolo a guardarla negli occhi. Era stato solo
allora che Thomas si era accorto di quanto fosse bella; quella sera, alla festa,
era troppo distratto dai propri pensieri; i lineamenti delicati, il naso lungo
ma elegante, i lunghi capelli biondi, le guance rosee, gli occhi profondi\dots
Era stato solo allora che Thomas aveva saputo di amarla.

\plainbreak{1}

C'erano volute quasi tre settimane perché Thomas si riprendesse. Tre settimane
in cui Kate, ogni giorno, era andata a fargli visita per cucinare -- se non
l'avesse fatto, ne era certa, sarebbe morto di inedia --, per pulire e
confortarlo. Aveva assistito a uno strano processo: era come se Thomas avesse
dimenticato, in pochi giorni, come muoversi, come parlare, come vivere. Era
stato compito suo insegnargli nuovamente tutto, e non sempre si era sentita
all'altezza: in quelle tre settimane Thomas aveva avuto delle ricadute, crisi
improvvise di panico e rabbia: si scagliava contro sua madre, che si era uccisa
solo per punirlo, contro i suoi amici, che non si rendevano conto di quanto
stesse soffrendo, e contro la stessa Kate, per le ragioni più diverse, salvo poi
riprendere a piangere e pregarla di rimanere quando ella accennava ad andarsene,
sdegnata.

Kate non sapeva perché lo stesse facendo: perché ella, così bisognosa di
stabilità, stava rinunciando a una vita tranquilla per aiutare un uomo
distrutto? era forse il suo istinto materno? l'avrebbe fatto per qualcun altro?
o lo stava davvero aiutando, come Thomas la accusava qualche volta, solo perché
i sogni di lui la affascinavano? No, non poteva essere: Kate non amava
l'ideatore per amore delle idee, ma le idee per amore dell'ideatore. Avrebbe
seguito Thomas in qualunque progetto, anche il più folle, dacché vedendolo aveva
subito saputo che era la persona giusta; non avrebbe potuto essere altrimenti. E
sentiva, pur conoscendolo appena, come un debito nei suoi confronti: le aveva
ridato speranza, dunque le sembrava giusto che ora ella facesse lo stesso.

Passate quelle tre settimane -- le più lunghe della sua vita -- Thomas si era
come risvegliato: una mattina Kate era andata da lui e lo aveva trovato più
sereno che mai; si sarebbe forse potuto dire che fosse radioso; pareva che ogni
traccia della disperazione che lo aveva attanagliato nei giorni precedenti fosse
scomparsa senza preavviso: l'uomo correva avanti e indietro, senza interruzione,
riordinando la casa, cucinando, lavorando a certi suoi progetti\dots e chiedendo
a Kate di sposarlo.

Kate non credeva affatto che un matrimonio fosse prematuro: negli ultimi giorni
si erano conosciuti meglio di quanto marito e moglie si conoscano dopo un
decennio; e del resto, non era Kate a pensare che quello fosse il suo grande
amore? cosa la tratteneva, ora? nulla più che sciocche convenzioni sociali!
Quell'esitazione non le faceva onore, così, qualche ora dopo, Kate aveva detto
di sì; e quella sera, mentre le mani di Thomas la sfioravano e un brivido di
piacere la scuoteva, aveva ricordato quanto potesse essere bello lasciarsi
andare.

\chapter{Ripensamenti}

Al suo arrivo, Nathan non aveva avuto modo di ammirare pienamente la bellezza
modesta, raccolta e serena di Ebensburg: era allora accecato dalla rabbia, dalla
tristezza, dalla paura, e non aveva goduto di quella vista. Ora però gli era più
facile dimenticare il perché e concentrarsi invece sul dove, sul cosa, sul chi,
giacché il suo animo era alleggerito dalla dolcezza di Charlotte; così, egli si
guardava adesso intorno con bambinesca curiosità, ubriacandosi di tutti i nuovi
odori e di quella calma misurata, felice: qui un uomo sulla sessantina curava il
proprio giardino con un'energia che Nathan aveva visto solo nel padre; qui una
madre giovane e fiorente portava a spasso il proprio neonato, come se fosse la
sola cosa al mondo che dovesse fare, come se non avesse altre preoccupazioni,
altre incombenze, altri pensieri.

Era davvero possibile vivere così, come se al di fuori di quel microscopico
universo non esistesse nulla? A Nathan, abituato all'ossessione newyorkese per
il lavoro e la produttività, alla ricerca continua e spasmodica, sembrava tutto
assurdo, così come agli abitanti di Ebensburg -- egli era certo -- sarebbe
sembrata assurda la frenesia della grande città, la tensione palpabile. Si
trattava di due mondi inconciliabili, che del resto non avevano alcun interesse
a conciliarsi, dacché non ne poteva venire nulla a nessuno dei due. Nathan
stesso si sentiva un intruso, come se potesse con la sua sola presenza
contaminare la purezza mai scalfita di Ebensburg; allo stesso modo si era
sentita Charlotte, quando era arrivata molti anni prima.

Ella camminava ora al fianco del ragazzo, assorta, bellissima: portava un abito
rosso a decorazioni floreali, con un'elegante semplicità che Nathan non aveva
mai visto in nessun'altra donna. Ed egli non poteva fare a meno di guardarla, di
ammirarne la grazia dei movimenti, la gentilezza dei modi, la meraviglia delle
forme; non poteva fare a meno di chiedersi come sarebbe stato annusare quei
capelli, o sfiorare quella pelle, eppure ne provava subito una colpa e una
vergogna indescrivibili, come se Charlotte fosse fin troppo pura, fin troppo
angelica per essere abbassata a oggetto di quelle attenzioni terrene.

«Dovrai trovare una ragione». La voce di Charlotte interruppe i suoi pensieri, e
quasi ella potesse indovinarli, Nathan si sentì avvampare le guance. Le chiese
di spiegarsi meglio, evitando di guardarla negli occhi. «Ieri, a cena\dots ti ho
chiesto perché sei venuto a studiare da me. Non mi hai saputo rispondere» chiarì
allora ella.

Erano giunti sulla sponda di un piccolo fiume che attraversava Ebensburg e di
fatto la divideva in due metà decisamente asimmetriche: una parte, quella dove
si trovavano Nathan e Charlotte, urbana; l'altra rurale, destinata forse alla
coltivazione. Sull'altra sponda, infatti, Nathan scorgeva campi e null'altro;
campi a perdita d'occhio, qualche casupola e in fondo una fila d'alberi
altissimi. Sulla banchina, insieme a loro, stavano alcuni pescatori, altri
passanti, una o due coppie d'innamorati sedutedecisamente sulle panchine in
legno. Qui sedette Charlotte, sfiorando il posto accanto a sé, invitandolo
dolcemente a raggiungerla.

Nathan obbedì; poteva sentire il profumo inebriante di lei, e improvvisamente fu
colto da un rispetto reverenziale, da un desiderio cieco, da un amore fulminante
e insopprimibile. Sentì di nuovo e più forte di prima il bisogno di dirle tutto,
di raccontarle della propria vita infelice, dell'infanzia diretta come da un
burattinaio onnipresente e invincibile; voleva dirle che non aveva alcun amore
per il pianoforte, e non perché ci fosse qualcosa nello strumento che non lo
attirava, ma perché avevano tentato di imporgli la passione ed egli aveva finito
per odiarlo; voleva scusarsi, perché stava perdendo il preziosissimo tempo di
lei.

E lo fece: le disse tutto, mentre Charlotte ascoltava con un'attenzione di cui
non lo avevano mai onorato, interrompendolo di quando in quando per porre
qualche domanda, per chiarire un suo sentimento, per mostrargli comprensione e
compassione. Quando le disse del suo disprezzo verso quello studio, in cui pure
ella cercava di aiutarlo, non lesse dolore sul volto della donna; solo
amorevolissima, attentissima pietà. Una pietà che gonfiava il cuore di lui, che
lo faceva esplodere dalla voglia di dirle come si sentisse al suo fianco; ma si
trattenne, perché gli pareva improbabile che un sedicenne potesse bruciare
d'amore sincero per una donna; e anche se fosse stato possibile, gli pareva
assurdo che siffatto amore potesse essere ricambiato; e se pure fosse stato
ricambiato, gli pareva impossibile, per un miliardo di ragioni tutte diverse,
che potesse compiersi.

Quand'egli ebbe finito, Charlotte stette un poco a pensare; perfino ella
sembrava turbata dal racconto. Infine gli chiese: «Guarda il fiume; cosa vedi?»,
e accennò al corso d'acqua col bel capo. Ivi Nathan pose lo sguardo, e in un
primo momento non gli sembrò che ci fosse nulla degno di nota: come ogni cosa e
ogni persona a Ebensburg, il fiume era placido e sereno. Poi un dettaglio
catturò la sua attenzione; ma poteva mai essere?!\dots Ma sì! Era, era! Il fiume
scorreva in due direzioni: verso la loro sinistra la metà più vicina a Charlotte
e Nathan, e quella più lontana verso la loro destra. Il contrasto era appena
percettibile: bisognava porre molta attenzione per notarlo.

Lo disse a Charlotte, ed ella rispose che realmente non era che un'illusione
ottica; in assenza di metafore migliori, gli disse, avrebbe dovuto farsi bastare
quella. Ma ancora Nathan non capiva, e si vergognò ad ammetterlo; allora
Charlotte si spiegò meglio: egli e il padre, gli disse, erano come le due metà
del fiume, «inseparabili, ma in contrasto». Eppure le due metà coesistevano
tranquillamente, senza che una cercasse di deviare l'altra. Così il ragazzo
avrebbe dovuto fare con Thomas, almeno per qualche anno ancora, finché non fosse
stato in grado di andarsene per la propria strada.

Nathan ringraziò di cuore Charlotte, ma obiettò che il padre, influenzando la
sua educazione, avrebbe senza dubbio anche pregiudicato la sua strada; che egli
faceva sentire la propria autorità in maniera così subdola, quasi
inconsciamente, che nemmeno Nathan si sarebbe accorto di tutte le diverse
maniere in cui le sue opinioni erano state manipolate. Allora i due stettero a
parlare un pezzo del ruolo che la società aveva sul destino del singolo, su come
liberarsi dai pregiudizi del pensiero, sul miglior modo per trovare il proprio
percorso e su come convenisse seguirlo\dots

Quando gli parve di aver finito gli argomenti, già da un pezzo il cielo era
scuro, e l'animo di Nathan più leggero. Praticamente, non avevano risolto molto:
non appena fosse tornato nella casa paterna, egli si sarebbe trovato a lottare
con gli stessi soliti problemi, a soffrire la stessa solita presunzione, a
sentire la stessa solita rabbia; eppure, ai problemi poteva opporre la
prospettiva, alla presunzione la fermezza, alla rabbia la pazienza. Ora sentiva
di avere un'arma; ora sentiva di avere un'amica. Non importava che vivessero
nella stessa casa o a centinaia di chilometri di distanza: quella sola
consapevolezza, che qualcuno da qualche parte nel mondo lo capisse, era
sufficiente a sostentarlo.

Tornarono a casa prendendo una strada diversa; stavolta Nathan non vedeva
negozi, solo i balconi di tanti piccoli appartamenti, fiori colorati ai
parapetti, disegni vivaci -- di bambini? -- alle finestre. Charlotte sembrava
ancora più pensierosa che all'andata, e melanconica nel suo silenzio, ma mai
Nathan si sarebbe azzardato a chiederle cos'avesse, per quanto si sentisse in
debito: ogni volta che ella s'interessava a lui, gli pareva un'invasione giusta,
considerata, amorevole; il contrario, a suo parere, sarebbe invece stato odioso.
D'altronde, che consigli avrebbe mai potuto dare a \emph{lei}, realizzata,
perfetta, imperturbabile? Sarebbe stata una bestialità: l'alunno che rimprovera
il maestro, o il reo che condanna il giudice, o il peccatore che predica al
santo.

\plainbreak{1}

Quel posto, la riva del fiume, divenne il \emph{loro} posto, tanto che
Charlotte, almeno una volta al giorno, prendeva il cappotto e chiedeva solo con
quel suo dolcissimo sorriso: «Andiamo?», senza che servisse aggiungere altro,
spesso quando il cielo si colorava di rosa e arancio e si rifletteva nell'acqua.
Allora Nathan s'alzava, felice di potersi sottrarre al pianoforte, e la seguiva
silenziosamente. A volte parlavano, altre preferivano ascoltare l'acqua
scorrere, le cicale cantare, il vento frusciare; guardare i nonni giocare coi
nipoti, i padri e le madri coi figli; ammirare le barche passare. Charlotte
aveva sempre amato quel momento della giornata, ma poterlo condividere con
qualcuno era\dots bello.

Charlotte non sapeva, non comprendeva bene cosa provasse per Nathan: mai le era
capitato di affezionarsi in tal modo a uno studente, di gioire per i suoi
successi, di struggersi per le sue disgrazie, di godere della sua compagnia.
Naturalmente affettuosa, pure erano quelle sensazioni nuovissime e spaventose,
perché sapeva che prima o poi tutto sarebbe finito, che Nathan sarebbe tornato
nel suo mondo, New York, così diverso dal suo; e all'improvviso la fretta,
l'ambizione, la brama da cui era fuggita per tutta la vita le erano parsi così
incredibilmente affascinanti e appetibili.

E così come temeva e desiderava la competizione, così fuggiva dal padre e lo
inseguiva, incalzata dai discorsi fatti con Nathan il primo pomeriggio al fiume
e quelli successivi. A Charlotte pareva di essere incapace di allontanarlo, di
dimenticarlo e ugualmente incapace di riavvicinarglisi, di ripristinare i
rapporti infranti. Di questo incolpava e ringraziava Nathan, perché fino a quel
momento ella aveva tenuto Nicholas a distanza sicura, quanto bastava perché il
padre, o quantomeno il ricordo, ché solo quello le restava, non potesse più
arrecarle danno. Ora, invece, le era impossibile ignorarlo, ed era una tortura;
ma una a cui non le dispiaceva poi tanto sottoporsi, dacché portava con sé
l'eccitazione dell'avvenire, del mistero; dacché portava con sé la speranza.

Ed ella aveva un bel da fare dicendosi che dieci anni prima aveva fatto bene ad
andarsene di casa, che la sua non era stata una reazione esagerata ma
giustissima, giustificatissima, meritatissima da quel padre che le aveva voltato
le spalle quand'ella ne aveva più bisogno; tutto ora sembrava suggerirle che
quella fuga non era stata che un capriccio, un colpo basso, una ferita inflitta
al padre per il puro e semplice gusto di fargli male. Chissà, se fosse rimasta,
se avessero parlato!\dots ma adesso che importava? L'affare era concluso, ella
era rimasta senza un padre, il padre senza una figlia; ne soffrivano e
infantilmente fingevano che non gliene importasse nulla.

\plainbreak{1}

Nicholas ricordava con una certa simpatia -- e un po' di compassione -- i suoi
primi concerti: le mani gli sudavano tanto che temeva di non poter suonare;
eppure, in un modo o nell'altro, era sempre riuscito a strappare al pubblico uno
scroscio di applausi, e il suo terrore non era riuscito a mettere in ombra
l'incommensurabile talento di cui era dotato. Ora, dopo trent'anni dalla sua
prima volta, non si sentiva affatto intimorito: saliva sul palco, si siedeva e
suonava; era presente in ogni momento, preciso, metodico, rigoroso; non c'era un
solo gesto che non fosse pianificato, non c'era spazio per l'ansia, non poteva
nascere l'errore: solo la musica esisteva, e tramite la musica il musicista,
critico e distaccato.

Per questo, Nicholas non si pentiva delle rinunce fatte per raggiungere la
propria fama: se non era la musica, egli semplicemente non era. Fermamente
convinto che ciascun uomo avesse uno scopo, era altrettanto convinto che il suo
scopo fosse suonare; e per suonare aveva messo tutto da parte: gli amici, gli
svaghi\dots e la famiglia. Non si pentiva di nulla, non rimpiangeva nulla: aveva
realizzato la propria missione e tanto gli bastava; stava agli altri regolarsi
di conseguenza, o quantomeno, laddove non fossero in grado di supportarlo, non
intralciarlo.

Eppure, perfino Nicholas doveva riconoscere che non tutto lo spazio della sua
vita poteva essere occupato dalla musica; non perché non si convenisse, ma
perché non era tecnicamente possibile suonare in continuazione. Allora, come
occupare il vuoto tra un momento e l'altro? Quello era sempre stato il problema
di Nicholas, dacché agli uomini comuni lo svago, la società, i piaceri mondani
riuscivano naturali e quasi istintivi, mentre a lui erano artificiali, inutili,
d'intralcio; gli pesavano come peccati inconfessabili, colpe da espiare, quasi
che lo studio matto, la pratica infinita e la critica feroce non gli valessero
un pomeriggio di dolce far nulla.

Così, dopo la fine di ogni concerto lo prendeva ora un'ansia simile a quella che
aveva tormentato l'inizio del primo; egli non vedeva l'ora di ritrovarsi lì,
seduto dinanzi al pianoforte, con o senza un pubblico, perché adesso non sapeva
che fare. Anche stavolta si trovava nell'auto, in parcheggio, con le mani sul
volante e il motore spento. Cosa fare? Andare a casa, e poi? Mangiare, bere,
respirare; dormire addirittura, anche se con l'inarrestabile avanzata degli anni
tale necessità si faceva sempre meno pressante\dots e poi? Non aveva amici, solo
collaboratori. Non aveva passatempo. Non aveva una famiglia; l'aveva avuta e
l'aveva lasciata andare, per una negligenza in cui s'era poi intestardito.

Lo squillo del telefono lo fece sussultare. Indossò gli occhiali, ma non
riconosceva il numero sullo schermo; che fosse la stampa? Era raro oramai,
passato l'entusiasmo iniziale, che lo chiamassero o che gli chiedessero
interviste, e a lui andava bene così: nonostante il suo mestiere, non aveva mai
amato trovarsi sotto i riflettori. Gli pareva di rubare tempo prezioso allo
strumento, di fare un torto a chi lo ascoltava; e non era neanche un grande
oratore. Chi lo chiamava, allora, qualche collaboratore?

Rispose all'ultimo momento; per lo meno non avrebbe dovuto lottare per decidere
come passare i successivi minuti della propria vita.

«Pronto?»

«Papà, sono io.»

Un passante che si fosse trovato lì in quel momento avrebbe notato Nicholas
Barnes, il musicista abilissimo, imperturbabile e disciplinato impallidire
quanto la sua già chiara carnagione ulteriormente gli permettesse.

\plainbreak{1}

A due anni di distanza dal matrimonio, la vita di Kate non era così felice come
la donna aveva immaginato. Era accaduto infatti che, dopo la depressione che lo
aveva attanagliato e da cui Kate lo aveva aiutato a uscire, Thomas si fosse
gettato anima e corpo nel lavoro, tanto da dimenticare i propri doveri di
marito, o più in generale di animale sociale: lavorava di giorno, di notte,
durante le feste, i compleanni, gli anniversari; \emph{gli anniversari!} Aveva
dimenticato quello del suo matrimonio, stando al lavoro tutto il giorno, e
quand'era tornato Kate non aveva avuto il coraggio di farglielo notare.

Non era stato un cambiamento repentino, però; piuttosto si era trattato di una
trasformazione graduale ma inarrestabile e -- Kate temeva -- irreversibile. Da
che si dedicava completamente a lei, ritenendo il lavoro funzionale a garantirle
un regime di vita accettabile, il lavoro era diventato il fine ultimo, e Kate un
mezzo, una collaboratrice: ella aveva infatti, per un certo tempo e con immenso
piacere, aiutato Thomas in alcune questioni finanziarie che riguardavano la
nuova azienda. Si era sentita artefice della propria fortuna, fulgido esempio
del successo imprenditoriale costruito su una buona idea, esecutori intelligenti
e tanta, tanta fortuna. Le sembrava tutto meraviglioso, perfetto,
indistruttibile.

Poi l'azienda, e con essa ella e Thomas, si era arricchita. Il marito aveva
assunto un pugno di persone per sostituire Kate, così che una mattina, mentre
ella si stava vestendo, le aveva detto: «Non serve che tu venga», e aveva
continuato a ripeterlo per diversi giorni, finché Kate non vi si era abituata e
aveva smesso anche solo di pensarci. Quello era stato solamente il primo di una
lunga serie di rifiuti che Thomas aveva mosso a Kate, e il meno doloroso di
tutti; anche se, sul momento, non sembrava affatto così. Ma col tempo Thomas
aveva preso a trascurare del tutto la moglie, tanto che oramai raramente le
rivolgeva la parola, se non per questioni essenziali e pratiche. Avevano smesso
di parlare, di discutere, di far l'amore; era come se Kate non esistesse più,
rimpiazzata da una manica di avvocati.

Quando la poveretta s'azzardava ad avanzare qualche obiezione parlando al marito
di come si sentisse inutile, Thomas scrollava le spalle. «Lavoro per noi,»
diceva, «perché possiamo permetterci questa vita». Poco importava che i loro
conti in banca potessero mantenere quattro generazioni. Per Thomas, nulla di ciò
che avevano era abbastanza: c'era sempre, \emph{sempre!}, qualcosa da
migliorare, da aggiungere, da rifinire. E bisognava assolutamente farlo per non
perdere il fasto acquisito. La faceva sentire colpevole di quella mole di
lavoro, perché tutte sulle spalle di Kate -- e di chi altri, se no? -- gravavano
le spese del proprio mantenimento; non s'arrischiava più a chiedergli un
centesimo, prendendo solo ciò che le era dato dal marito di propria volontà e
temendo di dilapidare la fortuna tanto faticosamente guadagnata.

Così, senza che Thomas le avesse mai detto nulla in tal senso, si sentiva un
parassita e un'ingrata: che contributo portava oramai alla vita del marito? I
rapporti tra loro erano pressoché inesistenti, e Thomas non si lasciava neanche
più aiutare al lavoro, convinto com'era che fosse l'unico in grado di far le
cose come conveniva farle. Kate non esisteva, e di quest'inesisteza soffriva
immensamente: il suo eroe si era trasformato in carceriere, e si sentiva stupida
perché di questa prigionia si sarebbe potuta liberare con una parola, una
chiamata, una firma; eppure restava, pazientava, attendeva\dots che cosa? Che
Thomas rinsavisse? Era fuori discussione. Allora di abituarsi a quella
situazione? Il giorno in cui né l'indifferenza di suo marito, né la mancanza di
uno scopo sarebbero stati più motivo di tristezza? A tanto si era ridotta?

\plainbreak{1}

La sua voce era stata sufficiente perché tutto quanto tornasse in ordine, perché
un decennio di parole taciute, colpa e astio venissero cancellati d'un tratto.
Nicholas non aveva idea che sarebbe stato così semplice. Se solo avesse saputo!
Se gli avessero detto che bastava sollevare il telefono e chiamare la figlia --
certo, avrebbe prima dovuto rintracciarne il numero -- perché entrambi si
rendessero conto degli sbagli commessi, delle imprudenze, delle risoluzioni
avventate!\dots Com'era dolce il perdono, tanto più dolce perché era un perdono
inespresso, silenzioso, implicito nel semplice gesto del parlare!

Certo, oltre al perdono c'era altro di non detto: pensieri, idee, sensazioni
meno nobili\dots l'ombra di una fuga vigliacca, di un'ignobile assenza che
erano durate fin troppo per entrambi, e senza che questi se n'accorgessero, e
una timidezza d'intenti, come se né Nicholas, né Charlotte fossero veramente
sicuri di voler rimarginare il rapporto distrutto anni prima, come se temessero
che tutto quanto si ripetesse, come se non si fossero dimenticati, nemmeno per
un istante, di quella ferita mal rimarginata. Ma ora la accettavano; accettavano
quel baratro che li separava ed esprimevano la volontà di superarlo.

Non ora, però! C'era tempo per riconcilirsi formalmente, per parlare con onestà
delle proprie divergenze; ora era tempo di una tregua, ora Nicholas era ebbro e
bruciava dal desiderio di sapere tutto -- tutto! -- ciò che la figlia aveva
fatto fin dall'abbandono della casa paterna. Voleva saperlo e soprattutto voleva
vederla; si rese conto di non saper bene nemmeno più che volto avesse il sangue
del suo sangue: aveva una o due fotografie in casa, ma le guardava raramente,
perché non voleva avere nulla a che fare con Charlotte e, segretamente, perché
la vista di lei gli era troppo dolorosa, un ricordo di ciò che aveva perso per
via della codardia di cui egli -- ed egli solo -- era colpevole.

Così Charlotte gli disse tutto: del suo arrivo a Ebensburg, dei molti lavori che
aveva fatto, delle piccole battaglie di ogni giorno e, infine, della decisione
di mettersi a insegnare; e qui Nicholas si rese conto che nessuna lite li
avrebbe mai separati tanto da sopprimere l'amore di Charlotte per lo strumento:
potevano odiarsi, disprezzarsi e insultarsi fino a morirne, ma Charlotte, alla
fine di una giornata faticosa, avrebbe comunque provato il desiderio di sedersi,
comporre e suonare, di esprimersi attraverso la musica proprio come il padre,
solo in maniera più intima e leggera. Il suo cuore si gonfiò di orgoglio sapendo
dell'ultimo studente della figlia, e non perché egli tenesse in alcuna
considerazione il valore del denaro, qunato piuttosto perché si rendeva conto
che Thomas Westford avrebbe assegnato solo a un conclamato talento l'istruzione
del proprio erede.

Sebbene Charlotte lo avesse chiamato nel tardo pomeriggio, i due si salutarono
solo a notte fonda, con una promessa: il prima possibile, Nicholas sarebbe
andato a trovarla a Ebensburg -- giacché ella era impossibilitata a muoversi nei
mesi successivi per via di Nathan -- per vedere lo studente recalcitrante e
cercare di inculcargli un poco della materia. Questo almeno era lo scopo
dichiarato: in verità i due morivano dalla smania di vedersi.

\plainbreak{1}

Quando Nathan vide scendere Charlotte, la mattina successiva, rimase assolutamente senza parole. Non
era solo la cura che aveva messo nella scelta del vestito, che metteva così meravigliosamente in
risalto gli occhi azzurri; non era nemmeno l'attenzione con cui si era truccata, in modo che la
naturale bellezza fosse ancora più chiara, ammesso che fosse possibile. Piuttosto, Charlotte aveva
quella mattina qualcosa di\dots diverso. Si sarebbe potuto dire che fosse radiosa, che la solita aura di
malinconia avesse deciso, per un giorno soltanto, di abbandonare il suo animo.

«Sei bella.»

«Come?»

Non riusciva a credere di averlo detto ad alta voce. Ma come diavolo gli era venuto in mente? Che
cosa aveva in quella testa?

Charlotte lo guardava, con aria interrogativa. Indubbiamente stava pensando al suo commento
inappropriato. Nathan si sentì arrossire, e il pensiero di arrossire lo fece arrossire ancora di
più.

«Volevo dire\dots»

«Grazie.»

Charlotte sembrava pensierosa, come se stesse decidendo se dirgli o meno qualcosa. E se lo avesse
rimproverato? Oh, Dio, non avrebbe mai e poi mai potuto sopportare la vergogna.

«Oggi verrà mio padre. È tanto tempo che non ci vediamo\dots» disse, più a se stessa che a lui.

«Non preoccuparti, mi leverò di mezzo. Penso che andrò a fare una passeggiata.»

La donna esitò ancora, stavolta più a lungo.

«Io\dots vorrei che restassi.»

Quale gioia e quale paura! Anche se non lo avrebbe mai ammesso, Nathan aveva provato una fitta di
gelosia sapendo che qualcun altro avrebbe avuto accesso all'intimità di Charlotte. Era tutto così
assurda. Non aveva nessun diritto, non poteva avanzare nessuna pretesa\dots. Eppure le si sentiva
così\dots vicino. Quasi che lei fosse sua\dots

«Resta, Nathan» ripetè ella, stavolta decisa. «Per piacere.»

Nathan quasi le parlò sopra.

«D'accordo.»

\chapter{Chiusure}

Charlotte era in un indescrivibile stato d'animo. La poveretta aveva passato le giornate addietro
passeggiando inquieta per casa, torcendosi le mani, incapace di impartire qualsivoglia lezione al
proprio studente. Se ne doleva e vergognava, certo; in primo luogo, però, nulla poteva cancellare in
lei l'eccitazione che le causava l'idea di riappropriarsi di quella parte della propria vita, che
ella aveva considerato perduta per sempre. In secondo luogo, sapeva che in pochi giorni suo padre
sarebbe riuscito a fare più progressi con Nathan di quanti ne avesse fatti ella in un mese. Il
talento di Nicholas nel suonare era comparabile solo alla sua capacità di sprigionare genuino amore
per la materia. Ella stessa, odiando tutto ciò che la musica aveva causato a suo padre, non
continuava forse a suonare?!

E in fondo non le dispiaceva poi molto di poter passare più tempo parlando con Nathan lontani da un
pianoforte. Più tempo passava con lui, più soffriva all'idea di tornare sola. Continuava a ripetersi
che se non fosse stato per la sua età, se solo non fosse stato per quello allora\dots Che idea
assurda. Più assurdo e vergognoso ancora il fatto che potesse esserle venuta in mente. Ma così tanti
uomini erano passati per quella casa, nella speranza che potessero in qualche modo\dots chiudere il
cerchio. Così tante aspettative deluse. Per anni ella aveva potuto avere chiunque volesse, e ora
l'unica persona che volesse davvero, l'unica che la facesse sentire al sicuro\dots quella persona,
forse la \emph{sua} persona, le era irraggiungibile.

Era forse condannata a essere sola? A toccare mani dopo mani, a baciare labbra dopo labbra, e
restare sempre delusa dal loro incommensurabile, cieco, freddo egoismo?

Qualche pomeriggio prima, durante una lezione particolarmente frustrante e infruttuosa, Nathan era
quasi in lacrime. A suo dire, non lo era per il fallimento di fronte al padre, ma per il fallimento
di fronte a lei. Charlotte fu mossa da quel sentimento a infinita dolcezza e compassione; le era
venuto perfettamente naturale abbracciarlo. Sebbene egli non avesse ricambiato, il solo contatto le
fece provare un brivido, qualcosa che non le era mia capitato prima, con nessuno. Allora forse per
la prima volta — o forse le piaceva pensare che quella fosse la prima — le era venuta in mente
quell'assurda idea, subito rifiutata, una lettera scarlatta a segnare la sua coscienza dignitosa.

Ma più vi pensava e vi si perdeva, più si convinceva che non ci fosse nulla di così assurdo o
vergognoso nell'attrazione; vergognoso sarebbe stato agire. Quella sarebbe stata un'onta di cui non
avrebbe potuto cancellare la traccia.

Eppure\dots

No, quello certamente non avrebbe mai e poi mai potuto farlo.

\plainbreak{1}

Era stato difficile per Braxton avere l'indirizzo della signorina Barnes. Sapeva, perché era stata a
suo tempo una gran notizia, che la figlia di Nicholas Barnes si era trasferita a Ebensburg. Così,
quando era stato incaricato di scrivere un articolo su Nathan Westford, che aveva scoperto prendere
lezioni da Charlotte, vi si era precipitato. Ma riguardo l'indirizzo preciso vigeva il più totale
riserbo nella comunità locale: Charlotte non amava chi veniva dalle grandi città e tutti a Ebensburg
lo sapevano. Braxton veniva da una grande città, e tutti a Ebensburg lo sapevano.

Ma Braxton doveva scrivere un pezzo sul rampollo di uno dei uomini più ricchi del Paese e la sua
talentuosa, misteriosa insegnante, e non intendeva arrendersi: aveva iniziato a frequentare luoghi
affollati — non venderanno molti in quella città — sapendo che era solo questione di tempo prima che
trovasse la donna o il ragazzo, e si era rivelata una tattica vincente: dopo un paio di giorni si
era imbattuto nei due, mentre passeggiavano in piazza, e li aveva seguiti. Li voleva intervistare,
ma non era ancora il momento. C'era\dots qualcosa, in come si guardavano e in come si parlavano.
Sapeva che, se avesse aspettato, sarebbe probabilmente andato via da quel buco con molto più di
un'intervista da terza pagina.

Prolungò il suo pernottamento in hotel e avvisò in redazione che aveva qualcosa di grosso tra le
mani.

Nessuno gli avrebbe creduto: non era che un giornalista in declino che, a guardare bene, non aveva
realizzato nulla che fosse degno di nota. Non gli importava.

Pensava fosse la sua occasione, e non voleva sprecarla.

\plainbreak{1}

Basta.

Kate non pensava di averlo detto ad alta voce, ma pareva fosse proprio così.

Aveva davvero, infine, detto basta?

Per esserne sicura, lo ripeté a Thomas, che le aveva chiesto distrattamente, tra una riunione e
l'altra, di chiarire. Gli aveva chiesto — no, lo aveva pregato — di passare un giorno, almeno uno, a
casa con lei. Egli aveva acconsentito solo dopo un litigio in cui non aveva mancato di ricordare
quanto infantile ella fosse per una simile pretesa, e come la compagnia non potesse sopravvivere
senza di lui. Ovviamente, aveva concluso ricordandole che lo faceva per loro due e Nathan. Per chi
altri, se no Non avrebbe lavorato così tanto, se fosse stato solo per lui.

Kate sapeva che non era vero.

A ogni modo, anche così aveva lavorato tutto il giorno. Solo da casa.

Dunque basta.

Basta cosa, aveva chiesto lui.

Basta tutto. Basta loro due. Basta quella vita di sacrifici. Basta guardare il successo di lui alla
televisione. Basta avere paura di disturbarlo tutte le volte che gli rivolgeva la parola, e perché
quando non lavorava pensava al lavoro e quando non pensava al lavoro voleva tornare al lavoro.

Gli aveva detto tutto questo senza alcuna particolare intonazione nella voce, e in quest'ordine.

Notò con soddisfazione — ma non era davvero soddisfazione — Thomas resistere all'impulso di
rispondere a una telefonata, mentre ella gli elencava tutte le ragioni per cui aveva deciso di
chiedere il divorzio.

\end{document}
