\documentclass[a4paper,oneside,10pt]{memoir}

\usepackage[italian]{babel}
\usepackage[utf8]{inputenc}

\title{Il terzo principio}
\author{Alessandro Desantis}
\date{}

\chapterstyle{dash}
\pagestyle{plain}
\frenchspacing

\begin{document}

\begin{titlingpage}
\maketitle
\end{titlingpage}

\chapter{Fantasmi}

Nathan Westford guardava annoiato fuori dal finestrino dell'aeroplano, non sapendo ancora se odiasse di più quel
viaggio o la sua destinazione. A novemila metri di quota il paesaggio non offriva alcuno svago: fuori si potevano vedere
solamente un'immensa quanto monotona distesa azzurra e, ogni tanto, qualche nuvola. Aveva già letto la rivista
distribuita dalla compagnia aerea due volte e non era dell'umore adatto per ascoltare musica.

Sospirò e appoggiò la testa allo schienale, tentando di dormire. L'impresa si rivelò impossibile: non solo l'aria
condizionata gli aveva completamente prosciugato le mucose, provocandogli un brutto mal di gola e una fastidiosa
sensazione alle narici, ma proprio nel momento in cui credeva di essersi appisolato fu svegliato dall'odore del caffè
che un'assistente di volo aveva portato a un passeggero vicino.

Passandogli affianco, la donna, una giovane e graziosa bruna di nome Wendy, gli sorrise e chiese se avesse bisogno di
qualcosa. Declinò con un cenno del capo ed emise un altro sospiro, ancora più doloroso del primo. Wendy annuì e
proseguì per la propria strada, muovendosi sinuosamente.

Vedendola così dolce e premurosa, Nathan ricordò quanto gli piacesse viaggiare con i suoi genitori quando era piccolo.
Si sentiva un impavido esploratore, ma allo stesso tempo sapeva di essere al sicuro. Era una sensazione quasi
commovente: gli sembrava che Wendy e le sue colleghe fossero angeli, sempre gentili e disponibili, mai infuriate o
spazientite.

Adesso, all'età di quindici anni, quei viaggi gli erano venuti a noia, e si rendeva conto di come quella civettuola
attenzione che Wendy rivolgeva in egual misura a ogni passeggero fosse una farsa. Una volta tornata a casa, distrutta
dal lavoro, non sarebbe stata diversa da lui, tranne che per un fatto: quando era sola, libera dagli impegni
professionali, era libera. Nessuno le imponeva le proprie scelte o le impediva di sbagliare.

Accanto a Nathan sedeva Russell Gates, il collaboratore del padre incaricato di assicurarsi che il ragazzo arrivasse a
destinazione tutto intero. Nathan aveva insistito per andare da solo, ma suo padre si era mostrato inamovibile. Ora,
vedendo il rivolo di saliva che gli scendeva da un angolo della bocca mentre dormiva, ebbe la sua conferma: Gates, oltre
a essere una persona sgradevole, era totalmente inutile. Lo chiamava \emph{giovanotto} e aveva il quoziente intellettivo
di una sedia, motivo per cui, probabilmente, gli era stato assegnato quel compito. Il padre doveva essere ben felice di
potersi liberare di un simile imbecille, anche se solo per un paio di giorni.

\plainbreak{1}

L'ufficio di Thomas Westford si trovava all'ultimo piano di un grattacielo completamente occupato dalla sua compagnia.
La Westford Dynamics era impegnata in una moltitudine di settori, ma la percentuale maggiore dei guadagni era ricavata
dall'ingegneria aerospaziale, grazie soprattutto ai numerosi contratti con il Dipartimento della Difesa. Ogni volta che
guardava fuori dall'immensa vetrata dietro la propria scrivania e vedeva l'intera New York, Thomas non poteva fare a
meno di pensare che avesse il mondo in pugno.

La sua carriera era incominciata quasi trent'anni prima, al liceo. Thomas non era nessuno: suo padre passava le proprie
giornate steso sul divano, ubriaco, e sua madre lavorava in un supermercato per pagare i vizi del marito. In cambio ne
riceveva ogni giorno una sostanziosa dose di lividi e cicatrici. L'uomo ricordava ancora il giorno in cui il padre le
aveva spento una sigaretta su un braccio così, tanto per ridere.

Con il figlio non se l'era mai presa, ma solo perché era troppo vigliacco per farlo, e Thomas l'aveva detetstato per
quella discriminazione: avrebbe preferito che quella tortura venisse riservata entrambi; almeno così avrebbe potuto
combattere il senso di impotenza che lo attanagliava, smettere di essere uno spettatore complice e diventare, invece,
una vittima. Ma ogni volta che assisteva all'ennesima violenza senza avere il coraggio di reagire, la sua insicurezza
aumentava. Si sentiva debole, fragile e codardo per non essere in grado di difendere la madre.

Aveva passato gli anni del liceo tentando di dimostrare agli altri e a se stesso che non era un rammollito, che sarebbe
potuto diventare qualcuno. Voleva fargliela vedere. Grazie alla sua mente brillante era riuscito a ottenere una borsa di
studio per Harvard, dove si era laureato con il massimo dei voti.

Infine suo padre era morto, portato via dal cancro ai polmoni. Nessuno lo aveva pianto, se non qualche lontano e
ipocrita conoscente che lo ricordava come \emph{una bravissima persona}. Thomas pensava con sollievo che per la madre
fosse finalmente giunta l'occasione di voltare pagina, ma una settimana più tardi la donna aveva deciso di mischiare
qualche sonnifero di troppo con un bicchiere di vino scadente, cadendo in un sonno liberatorio e senza risveglio.

Uno psicologo gli aveva spiegato che sua madre, per quanto ne ricevesse solo dolore, era indissolubilmente legata al
marito. Con la sua morte le era sembrato di perdere l'unica ragione che aveva di esistere. Doveva essersi sentita vuota,
priva di qualsiasi scopo e significato. Era andata incontro al suicidio come una macchina obsoleta andava allo
smantellamento.

Thomas aveva riso a gran voce sentendo quel ragionamento assurdo, turbando l'analista, eppure quella notte aveva pianto
per ore intere, sperando che al mattino il senso di colpa sarebbe svanito. Ma quando si era svegliato lo aveva trovato
ancora lì, a serrargli il cuore in una morsa. Allora, rendendosi conto che non l'avrebbe mai più abbandonato, aveva
scelto di conviverci.

Si era lanciato anima e corpo nel suo progetto ed entro pochi mesi, senza l'aiuto di nessuno, era riuscito a fondare la
Westford Dynamics. Ci erano voluti tre anni per mettere a punto il primo prototipo di un componente ormai ampiamente
utilizzato dalla {\scshape Nasa}. Tre anni in cui Thomas si era venduto agli investitori, aveva rilasciato interviste,
aveva lavorato nei fine settimana. Tutto per quella dimostrazione di forza che gli avrebbe permesso di elevarsi dalla
massa e di vincere i pregiudizi che lui stesso nutriva nei propri riguardi.

Ora, a distanza di quindici anni, la Westford Dynamics era una delle aziende più potenti del mondo. Nonostante il tempo
passato, Thomas si trovava spesso a pensare alle stesse parole che gli erano venute in mente quella sera, quando aveva
personalmente firmato il primo contratto per la produzione in larga scala della sua invenzione.

\emph{Ce l'ho fatta. Non sono come mio padre.}

\plainbreak{1}

Ebensburg è una microscopica cittadina di nemmeno quattromila abitanti nella Pennsylvania, uno di quei luoghi piccoli,
accoglienti e dignitosi dove tutti conoscono tutti. Mentre l'elegante Mercedes nera attraversava il centro urbano,
Nathan vide molti negozianti aguzzare la vista per cercare di capire chi fossero i passeggeri dell'auto. Se ne accorse
anche Gates.

«Meno male che i vetri sono oscurati, eh?» disse nervoso.

No, non era meno male. Nathan non avrebbe voluto altro che entrare in città come una persona comune, invece di
ostentare così schifosamente la propria ricchezza e la propria estraneità a quel luogo. Invece suo padre aveva fatto
in modo che sapessero immediatamente con chi avevano a che fare. Si sentì colpevole verso quegli uomini e quelle donne
che li osservavano con un misto di meraviglia e sospetto.

Allo stesso tempo seppe che, non appena l'uomo irritante che lo accompagnava si fosse dileguato come era previsto, si
sarebbe sentito a casa: Ebensburg, dopo nemmeno cinque minuti, lo faceva sentire il benvenuto. Nathan guardava i prati
verdi e le minuscole, deliziose villette a schiera e sentiva sprigionarsi dentro di sé un nuovo, caloroso affetto.
Respirò a pieni polmoni l'odore di terra bagnata, che non ricordava di aver mai sentito a New York.

L'auto si fermò davanti a una villa un poco più lontana delle altre. Allora, rendendosi conto che quell'allegra gita
era terminata e ricordandosi dello scopo della sua visita, Nathan non poté che rabbuiarsi.

Un anno prima suo padre aveva deciso, per chissà quale ragione, che dovesse imparare a suonare il pianoforte. Per sua
sfortuna, Nathan non nutriva alcun amore per lo strumento, e trovava le lezioni che gli venivano impartite noiose e
prive di significato: perché mai avrebbe dovuto imparare? Suo padre parlava di disciplina, i suoi insegnanti di arte,
ma entrambe richiedevano una passione che lui non possedeva.

Di tutto questo non aveva mai parlato con nessuno, nemmeno con Thomas. Gli pareva che, se solo avesse voluto, il padre
avrebbe percepito il suo astio verso il pianoforte. Credeva che il proprio doloroso silenzio al riguardo fosse un
messaggio sufficientemente chiaro, si aspettava che i genitori intuissero il suo stato d'animo con una sola occhiata. Ma
non avevano capito, o fingevano di non capire, e Nathan era infuriato per la loro indifferenza.

Volendo punirli, faceva di tutto per essere un pessimo studente. Dopo un anno di studio era quasi del tutto fermo al
punto di partenza. Di fronte ai suoi fallimenti, al padre non era mai venuto il sospetto che Nathan non avesse alcun
interesse a proseguire quel percorso. Pensando piuttosto che fossero gli insegnanti a essere degli incapaci, aveva
deciso di cercare qualcuno che fosse adatto per le esigenze del figlio. Quel qualcuno era Charlotte.

\plainbreak{1}

Charlotte Barnes era la figlia di Nicholas Barnes, uno dei pianisti più famosi del suo tempo. La donna ricordava
quando, da bambina, ascoltava il padre suonare per ore intere senza mai annoiarsi. Si rannicchiava sul pavimento vicino
a lui e vedeva le sue dita muoversi elegantemente sulla tastiera, creando suoni meravigliosi e suggestivi. Era stato
allora, notando come con uno sforzo così piccolo fosse possibile realizzare cose tanto belle, che aveva deciso di
imparare a suonare.

Il suo era sempre stato un padre affettuoso, ma Charlotte sospettava che amasse la propria musica più di lei. Gli unici
momenti che potevano condividere erano quelli delle lezioni; nel resto del tempo il pianista era in giro per il mondo a
dare concerti, troppo impegnato a perfezionarsi per vedere la figlia crescere.

Le sembrava che questo aspetto del suo carattere fosse peggiorato dopo la morte della madre in un incidente d'auto,
quando lei aveva sedici anni. Al funerale Nicholas non aveva avuto il coraggio di dire una sola parola, distrutto dal
dolore. Quando negli occhi della figlia aveva visto il bisogno di rifugio e protezione, l'uomo si era paralizzato. Non
era all'altezza del compito, lo sapeva. Non avrebbe mai potuto crescere quella ragazza e, terrorizzato dalla
possibilità di sbagliare, aveva deciso di rifugiarsi nell'unica attività che sapesse praticare veramente bene:
suonare.

Charlotte aveva proseguito gli studi da sola, dimostrandosi presto all'altezza del padre. A differenza sua, però, aveva
scelto di tenere quel dono esclusivamente per sé. Era raro che suonasse per qualcuno; tramite la musica confessava
l'inconfessabile, dava sfogo a pensieri e sensazioni di cui non avrebbe osato parlare neanche a se stessa. Non voleva
che nessuno varcasse quel confine, infrangendo l'ultima barriera della sua intimità.

Con Nicholas aveva avuto una moltitudine di discussioni al riguardo: lui sosteneva che l'artista vivesse per servire la
società e, ai suoi occhi, la decisione della figlia era uno spreco di talento. Ne avevano parlato più volte, ma l'uomo
aveva ceduto quando, dopo averle chiesto per l'ennesima volta il motivo della sua scelta, Charlotte non era riuscita a
trattenersi ed era stata costretta a rivelargli la verità.

«Non voglio diventare come te!» gli aveva urlato, con una furia di cui non l'avrebbe mai creduta capace. Subito dopo
era diventata paonazza e aveva mormorato alcune parole di scusa, ma Nicholas si era chiuso in un silenzio profondo, reso
ancora più penoso dal senso di colpa che sentiva aleggiare su di sé come un macigno sospeso nel vuoto.

L'indomani, senza che nessuno glielo chiedesse, Charlotte aveva fatto le valigie e si era trasferita, abbandonando la
casa di famiglia a Washington. Aveva scelto Ebensburg perché era la città di sua madre e perché era piccola: lì non
avrebbe corso il rischio di diventare famosa o di costruirsi una carriera. Poteva vivere nell'anonimato, proprio come
desiderava.

Presto aveva incominciato a insegnare pianoforte e, con sua sorpresa, aveva scoperto che si trattava di un'attività
abbastanza redditizia da poterne vivere. A volte i suoi studenti le chiedevano di suonare, ma lei rifiutava sempre. Gli
unici che avevano l'onore di sentirla all'opera erano coloro che, passando vicino alla sua casa di sera, sentivano l'eco
di alcune note fuggire dalle finestre illuminate.

\plainbreak{1}

Vedendo l'elegante facciata della villa, Nathan era riuscito a farsi un'idea abbastanza precisa di quella che sarebbe
stata la sua nuova insegnante: nella sua mente si era profilata l'immagine di una giovane zitella che, di fronte al
rifiuto del mondo, aveva deciso di rinchiudersi nello studio, senza trarne realmente alcun piacere, ma pronta a
tiranneggiare i propri studenti finché non avessero raggiunto la perfezione.

Per questo rimase disarmato quando Charlotte li raggiunse nel vialetto, con il suo sorriso sincero e i suoi modi dolci.
Si trovava di fronte a una donna sui trent'anni la cui pelle bianchissima tradiva, probabilmente, origini inglesi. Si
muoveva con un'energica allegria che si ripercosse, dopo gli iniziali sospetti, anche su di lui: aveva ancora paura che
dietro quella grazia si celasse la zitella, ma era impossibile non lasciarsi coinvolgere dalla sua apparente serenità.
Se odiava il mondo, era brava a nasconderlo.

Charlotte strinse la mano a entrambi e, nel farlo, guardò il ragazzo dritto negli occhi. Quelli di lei erano verdi,
profondi e felici, quelli di lui sfuggenti per l'imbarazzo.

«Tu devi essere Nathan. Ho letto di tuo padre qualche volta» gli disse, ma vedendo l'ombra che passò sul suo volto a
quell'ultima frase, fu ben attenta a lasciar cadere l'argomento.

Scambiò qualche parola di circostanza con Russell, ma a Nathan parve che si fosse stancata presto e in cuor suo fu
felice, sebbene provasse un po' di pena per Gates, che era evidentemente rimasto impressionato da Charlotte. Per far
colpo, l'uomo prese l'immensa valigia per portarla in casa, ma la valigia doveva essere pesante o lui fuori allenamento,
perché arrancava comicamente. Tentò ugualmente di assumere un'andatura naturale, nonostante il suo volto fosse
inconfondibilmente rosso.

Charlotte lo seguì e Nathan chiudeva la fila. Poteva vedere, appena sotto i lunghi e mossi capelli ramati della donna,
le curve dei suoi fianchi muoversi a ogni passo, coperte dal vestito bianco che aveva indossato. Si impose di non
fissarla ma, per quanto si sforzasse, di quando in quando i suoi occhi tornavano a posarsi su quelle forme perfette. Era
una sciocchezza, una naturale e comprensibile reazione alla grazia della donna, ma non poteva non sentirsi in colpa per
quei pensieri così terreni.

Giunti all'ingresso, Charlotte e Russell parlarono ancora un po' del più e del meno, quindi l'uomo si congedò, ancora
a disagio per la bellezza della sua interlocutrice e per la magra figura che aveva fatto, e si diresse verso l'auto. Il
giorno dopo sarebbe tornato alle dipendenze del signor Westford, con tutte le spiacevoli incombenze che ciò comportava.
Nathan si trovò a dispiacersi del suo destino.

Ciononostante non poté trattenere un sorriso quando, dopo che la Mercedes fu scomparsa dalla vista, Charlotte guardò
Nathan e spalancando i begli occhi disse col suo elegante accento: «Quell'uomo è veramente fastidioso».

\plainbreak{1}

Quando Thomas Westford in persona l'aveva chiamata per chiederle se potesse insegnare a suo figlio, Charlotte non aveva
potuto fare a meno di chiedergli se fosse \emph{quel} Thomas Westford. L'uomo aveva risposto di sì, e aveva potuto
immaginarlo nel suo immenso ufficio che sorrideva per quel riconoscimento della sua fama. Si era data della stupida:
cosa importava chi fosse? Doveva concentrarsi sul ragazzo, non sul padre.

Si era fatta raccontare la storia di Nathan, e stavolta era stata lei a sorridere. Aveva subito capito cosa gli fosse
successo perché era una storia nota; sapeva che, volendogliela inculcare a forza, genitori e insegnanti avevano finito
per fargli odiare la musica. Probabilmente si sentiva tradito, abbandonato, condannato da chi lo circondava.

Infine, quando Westford le aveva parlato della difficoltà di trovare un alloggio decente in città, Charlotte si era
offerta di ospitare Nathan. La sua modesta villa non era sicuramente all'altezza degli standard a cui era abituato,
però era senza alcun dubbio più confortevole di una pensione. Inoltre quella vicinanza sarebbe stato d'aiuto per
entrambi, permettendo di stabilire un rapporto più velocemente.

Gli avrebbe dato la camera grande, quella che lei aveva smesso di usare due anni prima perché dormire sola in un letto
matrimoniale le era ormai insopportabile. Per l'occasione l'aveva spolverata e aveva lavato per terra finché non si era
potuta specchiare nel pavimento, augurandosi ironicamente che Westford non le facesse causa per le terribili condizioni
igieniche in cui obbligava a vivere il suo prediletto.

Poi c'era stato il dramma della cena. Era andata al supermercato ed era rimasta a fissare il frigo dei surgelati per
almeno cinque minuti. Più il tempo passava e più si convinceva che la pizza fosse un piatto così abusato e banale da
sfiorare la volgarità, senza contare che quella venduta lì, a quanto ricordava, aveva la consistenza di un copertone.
Optando per qualcosa di più genuino, aveva scelto di affidarsi a Frank, il macellaio.

L'uomo era dietro il banco, sporco di sangue, e brandiva un enorme coltello. Nonostante il suo mestiere era magro e
spigoloso.

Vedendo Charlotte il suo volto si illuminò sotto la barba bianca e ispida. Le era molto affezionato e si raccontava
che, prima di sposare Nicholas, sua madre avesse avuto una breve relazione con lui. Quando Charlotte era arrivata in
città, ancora provata dalla conversazione col padre, era stato Frank ad aiutarla. L'aveva ospitata finché non era
stata autonoma e l'aveva presentata ai sospettosi cittadini di Ebensburg senza mai chiederle nulla in cambio. L'aveva
accolta come una di famiglia, e Dio sapeva quanto Charlotte ne avesse bisogno in quel momento.

«Che posso fare per te, dolcezza?» le aveva chiesto con la sua voce ruvida.

«Cosa posso far mangiare a un quindicenne?»

Frank aveva inarcato un sopracciglio.

«Hamburger» le aveva detto sicuro. «Con quelli non sbagli mai.»

Charlotte aveva sospirato, indecisa su cosa fosse peggio, se la pizza o l'hamburger. Poi un dubbio l'aveva colta
all'improvviso.

«E se fosse vegetariano?»

Frank aveva agitato il coltello in aria, fingendosi minaccioso.

«Allora rispediscilo nella metropoli di elegantoni da cui è venuto.»

Ora, davanti alla padella, Charlotte sperava solo che i gusti di Nathan non fossero meno convenzionali di quelli che
immaginava. Aveva continuato a girare la carne ogni trenta secondi per paura di bruciarla, con il risultato che gli
hamburger si erano quasi del tutto smontati. Riuscì miracolosamente a metterli nel pane e aggiunse una foglia
d'insalata.

Voltandosi trovò davanti a sé il ragazzo ed ebbe un sussulto. Dopo un istante di esitazione gli presentò il piatto,
con un sorriso di scuse già dipinto sul volto.

«Spero che vada bene. Non sono una grande cuoca.»

Nathan osservò il panino per qualche secondo.

«Va benissimo» decretò infine.

\plainbreak{1}

Mangiarono in silenzio seduti al piccolo tavolo nella cucina, concentrati esclusivamente sul ticchettio dell'orologio.
Nathan avrebbe voluto dire qualcosa, complimentarsi per la cena, ma le parole gli morivano in gola. Non sapeva cosa lo
atterrisse e attirasse tanto in Charlotte; probabilmente era quella tranquilla serenità con cui affrontava la vita,
come se non potesse capitare nulla di abbastanza grave da giustificare la sua preoccupazione.

Continuava a spiare di sottecchi ogni suo più piccolo gesto: l'eleganza con cui addentava il panino, l'attenzione con
cui si portava alla bocca il bicchiere dell'acqua dopo aver pulito gli angoli della bocca con il fazzoletto\dots{} La
osservava in modo discreto eppure avido, quasi volesse memorizzare la precisa sequenza dei suoi movimenti, e a ogni
secondo era sempre più affascinato dalla sua grazia naturale, non ostentata.

Stava viaggiando con la mente. La voce di Charlotte lo riportò alla realtà.

«Allora, cosa porta un famoso newyorkese come te in questa città dimenticata da tutti?» gli chiese con tono
distratto, quasi non le importasse veramente della risposta. Nathan sapeva che, in realtà, l'attenzione di Charlotte
era completamente per lui e, nonostante il brivido che gli provocò, questo lo fece sentire a disagio: non era un
argomento che volesse affrontare. Non con lei, non in quel momento.

\emph{Ti prego}, pensò, \emph{non rovinare tutto}.

«Tu insegni pianoforte\dots{}» rispose, ma subito si pentì per l'idiozia della sua constatazione. Allora aggiunse
deciso: «Voglio imparare.»

Si stupì di quanto gli fu semplice mentire e se ne vergognò, perché gli sembrava di aver tradito la fiducia di
Charlotte.

«Sì, così dicono» affermò lei con un sorriso ironico, e per un istante Nathan pensò che sarebbe finita lì. Subito
però la donna tornò alla carica: «Ma perché hai sentito il bisogno di percorrere centocinquanta miglia per venire
fin qui? Perché io e non un'altra persona?»

Quando gli parlava, gli occhi scintillavano di curiosità e tutto il suo corpo si sporgeva sul tavolo, quasi a voler
sentire meglio la risposta. Nathan seppe di aver trovato una persona che voleva comprenderlo, capire cosa stesse
passando. Era tutta la vita che cercava qualcuno come Charlotte; qualcuno a cui importasse di lui solo in quanto essere
umano, e non perché era il figlio di un miliardario. Qualcuno che gli volesse bene e gli permettesse di esprimersi
invece di coltivarlo per il futuro come se non fosse altro che una risorsa.

Avrebbe potuto e voluto parlare di tante cose con Charlotte, e l'ultima di queste era il rapporto col padre. Sentiva il
bisogno di dirle come spesso si sentisse inadatto, convinto che fosse lui il problema, e di come i suoi stessi desideri
e le sue stesse aspirazioni gli sembrassero stupide, perché gli avevano insegnato a pensare con la testa di altri
piuttosto che con la sua.

\emph{Non sono io che ho deciso, è stato mio padre, perché è lui che decide tutto. Ma per una volta sono felice che
abbia scelto al posto mio: se non mi avesse spedito qui non ti avrei incontrata.}

Sentiva che lei poteva aiutarlo, che la salvezza era dietro l'angolo. Se solo avesse avuto il coraggio di afferrarla,
forse qualcosa sarebbe potuta cambiare. Invece preferì proseguire dritto per la sua strada. Lo fece perché era stanco
di combattere e perché aveva paura. Non per sé, ma per Charlotte: temeva che sarebbe rimasta stritolata sotto il peso
dei suoi fantasmi. Non voleva che quella diventasse una sua battaglia.

«Ho cambiato un paio di insegnanti ma non mi sono mai trovato bene» disse, stringendosi nelle spalle. Sperava che lei
non avrebbe avuto il coraggio di continuare.

«E cosa ti fa credere che con me sarà diverso?»

Calò il silenzio. Ancora una volta, a Nathan non mancavano le parole, ma il coraggio di pronunciarle. Come spiegare
qualcosa che neanche lui era certo di capire completamente? Come dirle che in fondo ai suoi occhi gli era sembrato di
vedere un barlume di speranza? Come poteva, a cuor leggero, assegnarle un compito tanto gravoso?

Avevano entrambi finito di cenare e, di fronte al suo mutismo, Charlotte decise di alzarsi per sparecchiare,
sollevandolo dall'onere di rispondere. Nathan fece lo stesso e le loro mani si incontrarono al centro del tavolo. Il
ragazzo ritrasse la sua troppo in fretta e nel farlo colpì un bicchiere, rovesciandolo.

Sentì le guance avvampare. Si offrì di pulire, ma ovviamente Charlotte non glielo permise. Mentre lei asciugava
l'acqua finita sulla tovaglia, facendogli notare come non le fosse mai piaciuta, poté ammirare ancora una volta la
pacata allegria dei suoi movimenti, così lontana dal frenetico nervosismo in cui vivevano coloro che conosceva.

Qualche minuto dopo si diedero la buonanotte. Nathan osservò la donna allontanarsi silenziosamente nella penombra del
corridoio, i lunghi capelli rossi che ondeggiavano a un ritmo regolare. Pensò che la sua voce, per quanto insistente,
gli sarebbe mancata in quelle lunghissime ore che separavano la sera dalla mattina.

\plainbreak{1}

Charlotte scivolò nella vasca, sospirando mentre l'acqua bollente la avvolgeva lentamente. Nemmeno un bagno caldo
avrebbe potuto lavare via i pensieri che la tormentavano, ma la sensazione del vapore che le si condensava sul seno e
sul viso era una piacevole distrazione da quel turbine di consapevolezze a cui tentava di sfuggire.

Quando, quasi dieci anni prima, aveva deciso di andarsene di casa, i rapporti con suo padre non si erano interrotti di
colpo; era stato un processo graduale e sempre meno doloroso. Per i primi tempi si erano telefonati, parlando di
Ebensburg, delle rispettive vite, e talvolta perfino di sua madre. Non discussero della loro lite, forse perché nessuno
dei due ne capiva pienamente la causa né il significato.

Con lo scorrere dei mesi, però, quelle conversazioni si erano fatte più banali e sporadiche, finché non erano cessate
del tutto. Erano passati cinque anni dall'ultima volta che avevano avuto un contatto di qualunque tipo. Ogni tanto la
donna riceveva notizie di suo padre tramite qualche conoscente, ma si limitava ad accoglierle con una freddezza che
aveva smesso di stupirla molto prima, quando si era convinta che il tempo riuscisse a guarire tutto.

Nicholas era ormai una presenza marginale nella sua vita, un satellite in orbita a debita distanza. Non pensava a lui da
così tanto che persino il suo volto era difficile da ricordare, e insieme a suo padre era caduta nell'oblio anche la
sofferenza che le aveva causato.

Ma quando Nathan era venuto da lei e, nonostante i suoi sforzi per nasconderlo, Charlotte era riuscita a vedere nei suoi
occhi lo stesso senso di abbandono e la stessa rabbia che aveva provato alla sua età, le era sembrato all'improvviso di
non essere mai riuscita a superare nulla, di non aver fatto un solo passo dalla soglia della casa paterna. Stava
riconsiderando, ora, tutte le scelte che aveva fatto negli ultimi anni, scoprendo con orrore di essere stata avventata e
capricciosa. Si era trovata a chiedersi dove sarebbe stata se non avesse pronunciato quell'unica frase. Le sembrava di
essere tornata al punto di partenza, e questo l'aveva gettata nel totale sconforto.

Senza che ne provasse realmente alcun desiderio, immerse una mano nell'acqua tiepida e la posò tra le gambe, lottando
per resistere al torpore iniziale. Chiuse gli occhi, cercando di non pensare a nulla in particolare, ma più si
impegnava in quell'esercizio e più le immagini si sovrapponevano, fastidiose e insistenti. Sprofondò ancora di più
nella vasca e la sua mano spinse sul sesso con maggior decisione.

Dopo diversi minuti inarcò la schiena, offrendo il proprio seno a un immaginario amante. Abbastanza lucida da rendersi
conto della presenza di Nathan, soffocò un gemito mentre l'orgasmo attraversava il suo corpo e le permetteva,
finalmente, di affogare nel piacere tutti i pensieri.

Si beò di quell'estasi ancora per qualche istante, le labbra schiuse e il respiro tremante, tutto il suo corpo scosso
dai fremiti, finché non ebbe freddo e si accorse di avere il collo indolenzito per lo sforzo. Uscì dalla vasca e
indossò un accappatoio nero. Lasciò che l'acqua le si asciugasse addosso, quindi si infilò in una vestaglia di seta.

Passando davanti alla stanza di Nathan esitò, ricordandosi all'improvviso di aver solo temporaneamente allontanato le
preoccupazioni; sapeva che, non appena avesse guardato nuovamente il suo viso, tutto le sarebbe riaffiorato alla mente.
Nonostante il suo timore fu tentata di entrare. Sfiorò la maniglia ma, per qualche motivo, il contatto con il metallo
freddo la scoraggiò.

Proseguì verso la sua stanza, chiuse la porta dietro di sé e si lasciò cadere sul letto. Tutte le idee che avevano
preso forma nella sua testa durante quel pomeriggio divennero via via più nebulose, finché non furono una massa
indistinta, un groviglio senza capo né coda.

\chapter{Errori}

Era stato il caso a farli incontrare molti anni addietro, quando Thomas aveva appena fondato la sua compagnia e ancora
viveva con un amico in attesa di potersi permettere un appartamento. Entrambi erano stati trascinati da qualche loro
conoscente a una festa, ed entrambi avevano di meglio da fare. Per Thomas era dedicarsi ai suoi progetti, ovviamente;
per Kate, che aveva alle spalle una relazione disastrata, era restare a casa a rimuginare su quale menzogna fosse
l'amore e quale valle di lacrime fosse il mondo.

Tuttora, se le avessero chiesto che cosa le fosse piaciuto quella sera in Thomas, avrebbe risposto: il silenzio. Mentre
tutti intorno a lei conversavano, urlavano le inutilità che credevano indispensabile comunicare al resto del mondo, lui
taceva. Ogni tanto rivolgeva qualche sorriso di circostanza a un passante, o rispondeva con frasi di cortesia a
domande di cortesia, ma perlopiù si limitava a osservare, taciturno, pensieroso. Era come se fosse invisibile a tutti
tranne che a lei; come se qualcuno o qualcosa lo avesse messo lì solo perché lei lo potesse incontrare.

Per la prima volta nella sua vita, Kate aveva avvertito l'urgenza. L'urgenza di agire, di osare, di vivere. Non poteva
perderlo; sapeva che non ci sarebbe stato rimedio. Si compiaceva e si vergognava di quell'ambiziosa intraprendenza. Si
sentiva bloccata dalla paura di un'altra delusione, ma mossa dalla volontà di dimostrare a se stessa che la felicità non
le era preclusa, che una vita serena era ancora possibile. Per tutta la sera aveva danzato intorno a Thomas, senza però
osarsi mai avvicinare troppo. Aveva riscoperto la meraviglia di essere stretta in una morsa e, tuttavia, non volerne
uscire.

Se non fosse stato per una sua conoscente, una di quelle fastidiosissime donne che sembrano voler mettere ordine nella
vita di tutti tranne che nella propria, e si sentono per questo in dovere di dirigere e dare consigli a chiunque non
viva all'altezza dei \emph{loro} standard di perfezione, probabilmente Kate non avrebbe mai avuto modo di parlargli.
Questa ragazza, dunque, con cui Kate aveva conversato in rarissime occasioni, traendone ben poco piacere, le si
era avvicinata.

«C'è una persona che devi assolutamente conoscere!» aveva squittito, trascinandola verso un punto tra la folla. Kate
aveva tentato di opporre resistenza, urlando per coprire le altre voci, ma ogni protesta era cessata non appena si era
accorta di essere condotta verso Thomas. Benché fosse terrorizzata, non avrebbe mai osato rifiutare quell'occasione.

Pochi minuti più tardi conversavano come vecchi amici, ogni traccia di imbarazzo sparita dal volto di lei, il carattere
taciturno di lui sostituito da quello di un intelligente e spiritoso osservatore. Kate non riusciva a capire molto del
progetto di Thomas, ma sapeva, anche ora, anche senza averci mai parlato prima, che era destinato a un grande futuro.
Quell'uomo le dava speranza, e la speranza era esattamente ciò di cui aveva bisogno. Si vedeva accanto a lui, che
sfidava le intemperie della sorte, provata ma felice, consapevole di essere guidata dalla mano ferma e precisa di una
persona geniale. Che davvero il loro incontro fosse stato voluto dal destino?

Era tornata a casa con uno strano ma piacevole peso nel petto. In mano stringeva un biglietto su cui lui aveva scritto
il suo numero di telefono. Thomas Westford. Persino il nome suonava importante. Senza dubbio lo avrebbero ripetuto molto
negli anni a seguire. Lo sapeva. Credeva in lui. Lo amava.

Per circa due settimane, ogni sera, aveva preso quel biglietto e lo aveva guardato con ansia, tenendo il telefono in
mano. Proprio per via dell'importanza che l'incontro con Thomas aveva avuto, era ora attanagliata da un oscuro e
irritante terrore: che lui non si rivelasse all'altezza delle sue aspettative. Si rendeva conto di come la sua fantasia
avesse contribuito a proiettare su Thomas il suo uomo ideale. Se si fosse sbagliata, non sarebbe sopravvissuta a
un'altra delusione.

C'era poi un'altra evenienza, che la preoccupava ancora di più: quella di rovinare tutto. Non erano passati nemmeno sei
mesi dalla sua ultima relazione, ma già le sembravano così lontani i tempi in cui era ancora capace di amare. Odiandosi
per il modo in cui si era lasciata plasmare da un uomo, aveva promesso a se stessa che non sarebbe mai più stata così
ingenua. E proprio ora che le era indispensabile, le pareva di aver perso quella pericolosa capacità di lasciarsi
completamente andare, di affidarsi a un altro essere umano. Tutti i sogni sul suo principe azzurro si infrangevano
contro il muro della realtà, che l'aveva ferita gravemente in passato e contro il quale, di questo era certa, non
sarebbe mai più andata a sbattere.

Infine ronzava nella sua testa, senza che potesse trovare risposta, questa domanda: perché \emph{lui} non l'aveva ancora
chiamata? Aveva il suo numero, cosa lo tratteneva? Certo, un uomo così impegnato avrebbe potuto perderlo
facilmente\dots{} chissà a quante cose doveva pensare. Chissà a quante avrebbero dovuto pensare insieme! No, non era
plausibile che l'avesse smarrito: lei teneva quel foglietto, ormai indecentemente stropicciato, come una reliquia. E
dunque l'unica ragione poteva essere che non gli fosse piaciuta. Che tutti quei sorrisi, quelle battute, quella
complicità fossero solo una farsa, magari un modo per passare il tempo. Che lei fosse stata una stupida a farsi tante
illusioni, e che fosse destinata a\dots{}

Il telefono squillava da almeno dieci secondi, ma Kate non se n'era accorta, persa nei propri angosciosi pensieri.
Quando finalmente aveva realizzato, lo stupore era stato tale che un sussulto aveva scosso tutto il suo corpo. Aveva
messo a fuoco il numero sullo schermo e le era parso famigliare. I suoi occhi erano andati velocemente al biglietto, sul
tavolo lì accanto. Leggendo cifra dopo cifra, quasi ad alta voce, sentiva uno strano sentimento occupare il posto della
precedente sorpresa. Avrebbe voluto piangere, ma non sapeva bene il motivo. Rendendosi improvvisamente conto che stava
per perdere la telefonata, aveva premuto il tasto di risposta con molta più foga del dovuto.

\plainbreak{1}

Il dialogo in sé era durato pochi secondi, ma i silenzi erano stati eterni. Kate soppesava ogni parola dell'uomo,
cercando di leggervi le tracce di un amore dal quale, ormai, dipendeva la sua esistenza. Ma persa nelle sue analisi
finiva per dimenticarsi di rispondere: un paio di volte, addirittura, Thomas era stato sul punto di riagganciare,
convinto che ci fosse un problema di collegamento. Anche lui sembrava distante, come se qualcosa lo preoccupasse; questo
non era certo sfuggito all'orecchio attento di Kate.

Voleva rivederla. Gli avrebbe fatto molto piacere. Così aveva detto. Sentendo ciò, la mano di lei aveva preso a tremare
leggermente. Poteva avvertire distintamente, nel petto e nelle tempie, ogni battito del proprio cuore. Quant'era stata
stupida, poco prima! Come aveva potuto temere di non aver fatto colpo? Quale infondata ragione l'aveva spinta a dubitare
del loro futuro? Era tutto così ovvio e naturale, adesso! Oh, per quella sua esitazione non meritava l'amore di Thomas,
ma era così bello esserne ugualmente l'oggetto!

Dopo aver posato il telefono --- il che le era costato gran fatica --- aveva passato i successivi quaranta minuti
cercando un abito adatto, quindi era salita in auto per dirigersi verso l'indirizzo che Thomas le aveva dettato poco
prima. Si trattava di un blocco di appartamenti per studenti, non molto diverso da quello in cui viveva Kate. Fatto un
profondo sospiro, si era incamminata verso l'ingresso. Sfortunatamente per lei, Thomas abitava all'ultimo piano, così,
una volta giunta in cima, tutto il lavoro fatto per sembrare impeccabile era andato in fumo.

Quando l'uomo aveva aperto la porta, il suo grande e nervoso sorriso si era spento. Aveva sgranato gli occhi, cercando
di capire se quello che le stava davanti fosse Thomas o il suo coinquilino. Il viso era pallidissimo, tranne per le
occhiaie gonfie e violacee che cerchiavano gli occhi rossi, forse per il pianto. Nemmeno un capello sembrava voler stare
al suo posto, e sulla fronte questi si appiccicavano per via del sudore che la imperlava. Le palpebre erano socchiuse,
come se fosse infastidito dalla luce. Indossava una felpa pesante, decisamente troppo per la stagione, pantaloni e
scarpe da ginnastica.

«Ciao» aveva detto, con una voce quasi impercettibile. Aveva chiuso la porta dietro di lei, con lentezza esasperante.
Quando gli era passata accanto, un terribile odore di sudore e disperazione era salito su per le narici di Kate, fino al
cervello, dove si era posato e contribuiva, piano piano, a convincerla che tutto quello non fosse che un sogno. No, non
poteva essere Thomas. Non era possibile che l'uomo appassionato ed elegante che aveva incontrato meno di venti giorni
prima si fosse ridotto in quello stato. Non poteva esserci evento che lo abbattesse in quella maniera, rendendolo poco
più di una bestia ferita.

Lo aveva guardato ancora: le spalle ricurve, lo sguardo spento, il naso che gocciolava come quello di un neonato\dots{}
La casa era schifosamente sporca, il pavimento macchiato e coperto di fazzoletti usati, e impregnata dello stesso odore
di Thomas. Era troppo: non poteva sopportarlo. Kate era tornata sui propri passi con una furia che aveva stupito anche
lei, quasi travolgendo Thomas. Aveva aperto la porta ed era uscita sul pianerottolo, decisa a fuggire da
quell'appartamento per non tornare mai più. Non era in grado di affrontare un altro fallimento.

«Che fai?» le aveva gridato l'uomo. «Aspetta!»

Aveva cercato di fermarla, trattenendola per un braccio, ma si era divincolata con tanta forza da farlo inciampare e
cadere a terra.

«Kate, non andartene! Ti prego!» continuava a urlare.

Aveva già sceso una rampa di scale quando i singhiozzi l'avevano raggiunta. Thomas piangeva sommessamente, quasi avesse
paura di disturbare. Piangeva sdraiato a terra, proprio dove lei l'aveva lasciato. Piangeva tenendo la testa fra le
braccia, come i bambini puniti dai genitori. E anche Thomas era stato punito: punito dal fato, dalla crudele ironia di
un destino che sembrava prendersi gioco di lui, che lo privava, in pochi giorni, del suo più grande male e del più
grande bene.

Kate si era voltata e fissava l'uomo, interdetta e spaventata, incapace di distinguere tra incubo e realtà.

\plainbreak{1}

Questa era stata la reazione di Thomas alla morte della madre. Dopo aver riso della debolezza di quella donna, incapace
di vivere senza il suo aguzzino, era tornato a casa, per confidarsi con l'amico con cui viveva: voleva raccontargli come
realmente si sentisse al riguardo, parlargli dell'ingiustizia e della tristezza dell'esistenza. Ma il ragazzo era
partito per una vacanza in Europa, e sarebbe tornato solamente il mese successivo. Per qualche giorno Thomas era stato
in grado di mantenere il controllo, ma la depressione lo assaliva lentamente, e sentiva di esserne sempre più schiavo
ogni secondo che passava.

Aveva smesso di mangiare, di lavarsi, di vestirsi, di respirare. Aveva accarezzato l'ipotesi del suicidio; non gli
sembrava poi così male l'idea del sonno eterno. Non ci sarebbe stata più sofferenza, né morte, né ingiustizia. Solo il
nulla. Ma soprattutto, avrebbe pagato la sua colpa: l'indifferenza. Per tutti quegli anni aveva ignorato la situazione
che la madre stava vivendo: l'aveva considerata una debole perché, pur avendo avuto più volte l'occasione di sottrarsi a
quella tortura, aveva scelto di non farlo. Quale persona sana di mente si sarebbe mai comportata così? Ora che era
troppo tardi per rimediare, Thomas sentiva che la sua morte sarebbe stata una giusta pena, un modo adeguato per espiare
il suo immenso, imperdonabile peccato.

Ma non aveva la forza nemmeno per quell'ultimo, estremo gesto. O forse era il coraggio a mancargli. Così, piuttosto che
farla finita in un breve istante, aveva scelto di morire giorno dopo giorno, lasciando che le forze lo abbandonassero
lentamente ma irrimediabilmente. Aveva continuato a vagare per la casa, e quando anche spostarsi gli era sembrato troppo
difficoltoso, si era seduto in un angolo senza più muoversi. Ormai aveva perso ogni sensibilità: il suo corpo non era
più suo, ed era meraviglioso lasciarsi cullare dalla morte, sentire di non avere più il controllo, di non avere più
responsabilità. Ormai nulla dipendeva più da lui: si affidava al dio sulla cui esistenza era sempre stato scettico, e
che ora gli sussurrava all'orecchio.

Ma quel dio non voleva che morisse, o forse era il suo istinto di sopravvivenza a ingannarlo. E così una scintilla si
era accesa in lui: la morte aveva perso tutto il suo fascino, portava con sé solo la prospettiva dell'annientamento
totale, l'incapacità di agire e, dunque, di rimediare. La morte gli era preclusa, riservata a chi aveva vissuto la vita
dei giusti. Non ci sarebbe stata alcuna dignità nella \emph{sua} morte. E poi, maledizione, aveva paura: non voleva
morire. Sapeva di poter ancora fare la differenza, di poter migliorare la vita di altre persone per riscattare quella di
sua madre, e l'idea di privarsi di quell'opportunità gli era assolutamente odiosa: non era da lui rinunciare.

Tuttavia, non poteva farcela da solo. Aveva chiamato a raccolta le poche forze che ancora gli rimanevano e, dopo aver
trovato il numero di Kate, l'aveva chiamata, cercando di suonare il più naturale e sano possibile. Sapeva di non poterla
ingannare a lungo, ma non avrebbe mai ammesso la propria debolezza al telefono. Quando era arrivata e Thomas aveva visto
il suo viso, gli era sembrato chiaro che la Provvidenza l'avesse messa lì per salvarlo. Finalmente poteva intravedere,
lontana, una pallida luce di speranza. Ma poi aveva notato anche lo schifo dipingersi sul volto della donna, e si era
reso conto che stava per perderla. Quella luce si era spenta. Non \emph{poteva} perderla: perdere Kate avrebbe
significato perdere se stesso, di nuovo, e non era sicuro di riuscire a sopravvivere a un altro crollo.

Così aveva pianto. Di rabbia o di tristezza o di vergogna, non lo sapeva bene neanche lui. Non riusciva nemmeno a
immaginare quanto piccolo, debole e patetico potesse apparire agli occhi di lei. Era finita, lo sentiva: quella era
l'ultima occasione che aveva, ed era riuscito a sprecarla. Non ci sarebbe stato più niente, oltre. Tanto valeva
tornare dentro casa e tagliarsi le vene, o impiccarsi, o prendere qualche sonnifero di troppo come aveva fatto sua
madre. Era quello il suo destino, il progetto che c'era per lui. Era sempre stato quello, e pensare di sfuggirgli era
stata solamente l'illusione di un folle incapace di accettare la morte. Non avrebbe più pregato, non avrebbe più
lottato. Non sarebbe servito a niente.

Ma Kate si era fermata. Aveva sentito il suo sguardo su di sé per qualche minuto, poi i suoi passi, che giungevano
ovattati alle orecchie di Thomas, e infine le sue mani, delle mani così delicate, candide e sincere che incontravano
le mani sporche di un codardo e di un assassino. Quelle mani l'avevano costretto a tirarsi su, l'avevano portato dentro
casa, mentre una voce materna e dolce rassicurava i vicini. Thomas non si muoveva, non parlava, non protestava. Osava a
malapena respirare: era completamente in balìa di lei e delle sue cure amorevoli. Perché lo stava aiutando? Non c'era
motivo: lui non era nessuno, non meritava niente.

Kate lo aveva fatto sedere su un divano e, mentre attendeva pazientemente che smettesse di singhiozzare, aveva pulito la
casa al meglio delle proprie possibilità. Thomas non si azzardava a incrociare il suo sguardo, ma poteva sentirla
muoversi, rapida e inquieta, da una stanza all'altra. Poi era tornata da lui e gli aveva di nuovo preso le mani,
obbligandolo a guardarla negli occhi. Era stato solo allora che Thomas si era accorto di quanto fosse bella; quella
sera, alla festa, era troppo distratto dai propri pensieri. I lineamenti delicati, il naso lungo ma elegante, i lunghi
capelli biondi, le guance rosee, e gli occhi verdi e splendenti come smeraldi\dots{}

Era stato solo allora che Thomas aveva saputo di amarla.

\end{document}
