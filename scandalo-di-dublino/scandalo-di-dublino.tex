\documentclass[a4paper,11pt,oneside,openright,final]{memoir}

\usepackage[italian]{babel}
\usepackage[T1]{fontenc}
\usepackage[utf8]{inputenc}

\title{Lo scandalo di Dublino}
\author{Alessandro Desantis}

\makeatletter
\def\thickhrulefill{\leavevmode \leaders \hrule height 1pt\hfill \kern \z@}
\renewcommand{\maketitle}{\begin{titlingpage}%
    \let\footnotesize\small
    \let\footnoterule\relax
    \parindent \z@
    \reset@font
    \null\vfil
    \begin{flushleft}
      \huge \@title
    \end{flushleft}
    \par
    \hrule height 1pt
    \par
    \begin{flushright}
      \LARGE \@author \par
    \end{flushright}
    \vskip 60\p@
    \vfil\null
  \end{titlingpage}%
  \setcounter{footnote}{0}%
}
\makeatother

\medievalpage
\chapterstyle{verville}
\pagestyle{plain}

\begin{document}

\maketitle

\frontmatter

\chapter{Prologo}
Chi mi legge avrà quasi certamente sentito la mia storia in televisione, o
l'avrà letta sui giornali, o l'avrà ascoltata raccontare da qualche amico, che
non ricordava i dettagli e ne ha aggiunti alcuni, in modo da rendere tutto il
più piccante e scandaloso possibile. Dato che gli avvenimenti sono stati, molto
spesso, storpiati anche dalle fonti che sono considerate più affidabili, ho
deciso di redarre questo mio documento, che è da considerarsi l'unica
testimonianza attendibile dell'accaduto.

Immagino che vi chiediate come sia possibile essere così sicuri che io stesso
non abbia alterato la realtà, al fine di giustificare azioni che, a detta di
molti, furono dannose, sconsiderate e orrende. Il fatto è che io, a differenza
di altri, non ho alcun interesse a mentire riguardo ciò che è successo, perché
non c'è assolutamente nulla da nascondere. Lascio all'intelligenza del lettore
il compito di distinguere ciò che è giusto da ciò che è sbagliato, e spero che
comprenda quello che, per interesse o stupidità, molti non hanno compreso, e
cioè che nessuna delle parti coinvolte ha mai recato un torto all'altro, né
intenzionalmente, né tantomeno senza avvedersene.

Ho cercato di mantenere questo racconto imparziale, ma in alcuni passaggi mi è
stato veramente difficile, e credo di aver fallito nel mio intento. Ricordando
visi, atteggiamenti e parole, infatti, tornavano dal passato, come fantasmi,
anche le sensazioni e i pensieri, spesso non molto ortodossi, che all'epoca dei
fatti narrati li avevano accompagnati. Ho poi riflettuto molto se dovessi
sentirmi obbligato a rimuovere tali osservazioni dal manoscritto, e ho deciso
che no, non andava fatto, per due motivi che ritengo validissimi: il primo è
che, essendo mia questa opera, è giusto che essa porti il \emph{mio} pensiero, e
non sia una fredda e distaccata analisi storica; inoltre, poiché ho visto le
reazioni di chi ha seguito le mie avventure unicamente tramite i mezzi ufficiali
(e cioè tutti, a parte me e l'unica altra persona direttamente coinvolta), ho
pensato fosse meglio rendere il lettore partecipe del mio punto di vista, così
che non si lasciasse influenzare da quelli che sono stati svelti --- ma che
dico, subitanei --- a giudicare, etichettare e dimenticare, perché una volta
presa una certa posizione non avevano alcuna intenzione di tornare indietro.

\mainmatter

\chapter{}
Charlotte era la mia insegnante di piano. Quando la conobbi avevo quattordici
anni, mentre ora ne ho venti. Era una giornata umida e fredda, il che non
aiutava certo il mio umore, già pessimo; non mi piaceva, infatti, il pianoforte.
Non solo: tutti gli insegnanti che avevo avuto fino allora non avevano fatto che
accrescere il mio odio verso la musica, con il loro atteggiamento superiore e
autoritario, come se mi stessero facendo un gran favore nonostante il lauto
compenso che percepivano.

Ma Charlotte era diversa. A distinguerla dai miei altri mentori era la passione:
Charlotte era giovane, aveva compiuto da poco i trent'anni, e faceva parte delle
poche persone che ancora credono in ciò che fanno. Il suo entusiasmo era
massimo, così come la felicità con cui affrontava ogni nuova sfida, come se non
temesse alcun ostacolo, né --- figurarsi! --- il fallimento.

Quel giorno, dicevo, davvero non potevo sopportare un'altra lezione. Speravo
che, vedendo la mia espressione svogliata, lei avrebbe presto smesso di tentare
d'indottrinarmi, limitandosi a prendere i soldi di mio padre, che così sarebbe
stato soddisfatto perché suo figlio stava imparnado a suonare uno strumento. Era
una soluzione che rendeva tutti contenti. Quello che non sapevo, e non mi sarei
mai aspettato, però, era che per Charlotte il denaro era una questione di
secondo conto, tanto che, visti i prezzi irrisori delle sue lezioni, papà fu
inizialmente tentato di rivolgersi a qualcun altro, credendola un'incapace.

Quando bussammo al suo appartamento su Aungier Street, venne ad aprire la porta
una donna a piedi nudi, che indossava una camicia azzurra su dei jeans scuri. Ci
accolse con quel sorriso che divenne, per me, il suo tratto distintivo: non un
sorriso a bocca serrata, come di chi la sa lunga ed è pericoloso, o timido, ma
un sorriso divertente e cordiale, che lasciava in mostra i suoi denti
bianchissimi, tanto in contrasto con i capelli rossicci, che si arrotolovano in
piccoli ricci, fermandosi poco prima delle spalle minute.

Quel sorriso mi lasciò interdetto. Non ero pronto: mi aspettavo di trovare alla
porta una megera decrepita che non arrivava a fine mese, svogliata tanto quanto
me, in ciabatte e col naso che colava senza che se ne accorgesse. Invece
Charlotte ci salutò con tono affettuoso, come fossimo dei vecchi amici. Ci fu un
breve dialogo tra lei e mio padre, quindi lui andò via, e rimanemmo soli.
Charlotte fece di tutto per farmi sentire a mio agio, ma non ci riuscì: ero
spiazzato dal suo carattere allegro e spensierato, tanto che sembrava un delitto
essere tristi in sua compagnia.

Prima di iniziare la lezione, però, volle farmi una domanda: «Perché vuoi
imparare a suonare il pianoforte?».

Cercai una risposta esauriente, che esprimesse una gran voglia di fare. Nulla.

Decisi allora di dire la verità: «Perché lo vuole mio padre».

Charlotte spalancò i suoi enormi occhi azzurri e li fissò nei miei.

«Dici sul serio? Tu sei qui perché lo vuole tuo padre?».

Io arrossii, giacché mi accorsi di quanto stupida suonasse quella frase.

«Be', allora mi rifiuto di insegnarti» disse risoluta. «Chiama tuo padre, può
anche venire a riprendersi i suoi soldi... e te».

«Perché?» domandai agitato. «A te cosa cambia?».

«Non posso insegnare a qualcuno che non vuole imparare. Farei una violenza a me
stessa e anche all'allievo. Quindi se hai intenzione di trovare un motivo per
suonare il piano, allora sarò felicissima di aiutarti, altrimenti dovrai
rivolgerti a qualcun altro».

«Un motivo? E quale motivo?».

«Questo non spetta a me dirlo, e neanche a tuo padre. Quello che mi interessa è
che tu sia mosso da un motivo sincero, non voglio sapere quale. Non prendermi in
giro, però, perché sono brava a capire le persone. Sembro molto dolce, vero? In
verità quando mi arrabbio sono terrificante!».

Disse le ultime parole sorridendo, ed era chiaro che scherzava.

«Va bene, troverò un motivo» mi arresi.

«Giuralo».

Feci come aveva detto.

«Bene» concluse, e riprese la lezione.

Tutto quello che spiegò quel giorno mi era familiare, ma, allo stesso tempo,
completamente nuovo per via del suo modo di parlare e di affrontare le
questioni. Con lei l'insegnamento non era unidirezionale: si trattava invece di
uno scambio di opinioni tra persone sullo stesso livello. Credo fosse per questo
che, a differenza degli altri, riusciva a farmi imparare le cose senza che le
dimenticassi dopo una settimana. Volle ricominciare dalle basi, concetti che
avrei già dovuto conoscere perfettamente dopo quasi cinque anni. Non mancava,
ogni tanto, di ridere e farmi ridere, o fare qualche domanda personale a cui
rispondevo volentieri, anche se con moderato imbarazzo.

Così piacevole fu quell'ora, e anche quelle successive, che mi scoprii triste di
dovermene andare.

\plainbreak{1}

Tutto accadde in modo così banale...

Litigai con i miei genitori per un motivo che non ricordo, ma che allora mi
sembrava il più importante del mondo. In uno scatto d'ira, uscii di casa
sbattendo la porta. Me ne pentii non appena sentii il freddo invernale pungermi
le guance, ma non sarei mai tornato sui miei passi. Così, invec di rientrare in
casa e chiedere scusa, iniziai a vagare rabbioso per le strade di una Dublino
infreddolita.

Dopo mezz'ora, però, decisi che dovevo trovare un posto dove passare la notte
se volevo evitare di morire assiderato. Se solo non fossero state le dieci di
sera sarei andato da un amico. Un mio zio abitava nele vicinanze, ma
probabilmente avrebbe chiamato i miei non appena mi avesse visto nel suo
giardino. Mi rimaneva un'unica opzione: Charlotte. Non so nemmeno come poté
venirmi in mente di andare da lei in una situazione simile, ma so che fu
l'istinto di sopravvivenza ad aiutarmi a mettere da parte la mia naturale
timidezza.

Per tutto il tragitto cercai di trovare una scusa convincente, ma poiché
sembrava --- probabilmente lo era --- impossibile, e considerando che ella non
meritava una menzogna, stabilii di dirle la pura verità.

Indugiai almeno dieci minuti di fronte a quella porta, chiedendomi come avrebbe
reagito. ``C'è un solo modo per scoprirlo'', pensai infine, e bussai con
decisione. Ascoltai i suoi passi dall'altra parte, e pochi istanti dopo fu
davanti a me, con una faccia il cui stupore era pari solo alla mia vergogna.

«Ma che ci fai qui?» chiese sgomenta.

Tenevo lo sguardo a terra, e la mia voce era quasi impercettibile nel silenzio
del pianerottolo.

«Io... ho litigato con i miei».

«Oh...».

Non accennava a lasciarmi passare, così decisi di essere più diretto. Con quel
pochissimo coraggio che mi restava, proseguii: «...e pensavo che solo stanotte
avrei potuto dormire qui».

«Ah...».

Ci furono interminabili attimi di silenzio prima che pronunciasse la frase
successiva.

«Be', d'accordo».

Non so se ero più contento di aver trovato un riparo per quella notte, oppure
a disagio perché, seppure per poche ore, avremmo abitato insieme. Forse entrambe
in eguale misura.

La ringraziai, ed ella annuì con aria assente.

\plainbreak{1}

Probabilmente state già pensando che fu una notte di sesso sfrenato, durante la
quale Charlotte diede il meglio --- o, direbbero altri che delle cose del mondo
non hanno capito granché, il peggio --- di sé. Immaginate che sia accaduta chissà
quale disgrazia, che io abbia subito chissà quale trauma irreparabile, che ora mi
spinge a scrivere sì assurdi ragionamenti.

Siete fuori strada. D'altronde il solo fatto di essere così vicino a quella donna
che tanto ammiravo mi rendeva così nervoso da farmi trovare difficile anche la
semplice respirazione.

Per questo, probabilmente, non ricordo perfettamente come andarono le cose quella
sera; tuttavia sono rimasti impressi nella mia mente alcuni momenti, e
insignificanti dettagli che non saprei collocare temporalmente. Per esempio, non
scorderò mai con quale premura Charlotte ascoltò quello che avevo da dire. Questo
non significa che fosse d'accordo con me su tutto: a volte si dimostrò contraria,
e lo disse sinceramente, mostrandomi dove sbagliavo senza farmene una colpa.

Io non so cosa cercassi in lei: una madre? un'amica? un mentore? Forse tutto
insieme, e nel tempo altro e meno di questo. Ma fu la capacità di Charlotte di
adattarsi alle mie necessità e saperle capire, offrendomi in quel preciso momento
proprio ciò di cui avevo bisogno.

\clearpage

\tableofcontents

\end{document}
