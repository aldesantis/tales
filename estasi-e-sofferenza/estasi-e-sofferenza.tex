\documentclass[a4paper,11pt,oneside,openright,final]{memoir}

\usepackage[italian]{babel}
\usepackage[T1]{fontenc}
\usepackage[utf8]{inputenc}

\title{Estasi e sofferenza}
\author{Alessandro Desantis}

\makeatletter
\def\thickhrulefill{\leavevmode \leaders \hrule height 1pt\hfill \kern \z@}
\renewcommand{\maketitle}{\begin{titlingpage}%
    \let\footnotesize\small
    \let\footnoterule\relax
    \parindent \z@
    \reset@font
    \null\vfil
    \begin{flushleft}
      \huge \@title
    \end{flushleft}
    \par
    \hrule height 1pt
    \par
    \begin{flushright}
      \LARGE \@author \par
    \end{flushright}
    \vskip 60\p@
    \vfil\null
  \end{titlingpage}%
  \setcounter{footnote}{0}%
}
\makeatother

\begin{document}

\chapterstyle{companion}
\pagestyle{plain}

\maketitle

\plainbreak{1}

Accadde che un giorno, durante il suo cammino, Akram capitò in un povero e
minuscolo villaggio nel sud della Spagna, non segnato sulle carte. Un luogo
delizioso, anche se, a dir la verità, un poco inquietante. Denso fumo nero
usciva dai comignoli di alcune abitazioni, e nonostante ciò il posto sembrava
completamente deserto.

L'unica persona intorno era una donna sulla spiaggia, in piedi, impettita.
Scrutava l'orizzonte senza distogliere un attimo lo sguardo. Indossava un lungo
vestito nero. Il ragazzo si chiese cosa ci facesse lì. Non sapeva perché, ma
quella fanciulla lo affascinava più di quanto ogni altra avesse mai fatto.
C'era qualcosa di mistico e misterioso in lei. Qualcosa che valeva la pena di
essere scoperto.

Si promise di avvicinarla prima di andarsene. Ma ora aveva un disperato bisogno
di riposare. Aveva percorso molta strada per arrivare fin lì, quasi non sentiva
più le gambe, e non mangiava dalla mattina precedente.

\plainbreak{1}

Egli bussò a molte porte, ma nessuno voleva accogliere quello che sembrava un
mendicante, vestito di stracci, sporco e puzzolente.

Giunse al limitare del villaggio, e lì, finalmente, una coppia di anziani lo
accolse in casa. L'uomo era piuttosto gentile e gioviale, nonostante, immaginò
Akram, avesse superato i settant'anni. Molti avrebbero detto che era una
benedizione che fosse ancora vivo, ma il viaggiatore sapeva che si muore solo
quando non si ha più nulla per cui vivere. Quel vecchio trasudava vitalità. La
lunga barba bianca era tenuta in perfetta cura, così come i capelli che gli
restavano. Ogni movimento era eseguito con controllata energia. Lei invece era
più taciturna e riservata, si teneva in disparte, ma, come il marito, non
dimostrava affatto la sua età.

Entrambi erano vestiti in maniera semplice, ma ostentavano una certa dignità.
Anche la loro abitazione, per quanto modesta, era pulita e ordinata. Akram li
ammirava immensamente.

Gli diedero pane, formaggio e vino, lo lavarono e lo vestirono. Gli permisero di
riposare in un letto, e non sdraiato sul fieno come aveva fatto da quando era
partito.

\plainbreak{1}

Dormì tutto il giorno, fino all'alba. Quando si svegliò, chiese della donna
che aveva visto.

«Oh, lasciala stare, ti prego» disse l'uomo. «Da anni è lì. Non mangia, non
beve, non dorme. Eppure non vacilla. Molti pensano sia una strega».

«Ma è molto bella, non trovate?»

Fu allora che, per la prima volta, anche l'anziana signora parlò. Con una voce
debole ma decisa. Spaventata.

«Di una bellezza pericolosa. Chiunque abbia provato ad avvicinarla non è più
tornato indietro».

A quelle parole, Akram impallidì.

«Io... voglio provarci. E, se non riuscirò a conquistarla, tanto peggio».

I due vecchi si guardarono, e lei scoppiò in lacrime.

«Tanto tempo fa,» disse «avevamo un figlio della tua età. Era un ragazzo
forte e vigoroso. Anche lui, come te, si era innamorato della strega. Ma lei non
ha avuto pietà».

Il ragazzo posò le mani sulle spalle della donna.

«Voi siete stati buoni con me. Mi avete accolto nella vostra casa, mi avete
sfamato, mi avete lavato e fatto riposare. Ma è giunto il momento che io
continui il mio cammino».

E, detto ciò, si congedò dalla coppia, sapendo che, comunque fosse andata, non
li avrebbe mai più rivisti.

\plainbreak{1}

La spiaggia non era lontana, dopotutto quello era un piccolo villaggio, e Akram
trovò il tragitto più breve di quanto ricordasse. Improvvisamente il coraggio
gli venne meno. Tuttavia era ormai tardi per tirarsi indietro. Era convinto che
quella fosse una sfida messa sulla sua strada per testare la sua fede.

In pochi attimi quella donna fu davanti a lui. Sembrava avere non più di
trent'anni. Più che bella, si sarebbe detto di lei che era affascinante. Lunghi
capelli dorati cadevano lungo la schiena, fin quasi sotto la vita. Indossava un
abito nero che arrivava a terra, finemente ricamato. I piedi erano nudi. Teneva
la mano destra dentro la sinistra, dietro la schiena.

Akram mosse qualche passo verso di lei, che sembrò non notare la sua presenza.
Poi pronunciò le tre parole che, a molti, costano mesi, spesso anni di fatica.
A lui era invece stato insegnato a pronunciarle a cuor leggero.

«Io ti amo».

Quella si voltò a guardarlo, le labbra si incurvarono in un amaro sorriso.

E in quell'attimo il ragazzo provò una sofferenza indescrivibile. Non fisica,
ma dell'anima. Lo colse una disperazione tale che avrebbe voluto morire affogato
nel mare, ma lottò per trattenersi.

``Ecco cos'è successo a tutti gli altri uomini che sono venuti qui'' pensò,
osservando le onde infrangersi con potenza sugli scogli acuminati.

Ma il ragazzo non cedette. Dopo un tempo che a lui parve infinito, il dolore
cessò e la vista, prima annebbiata, si fece più nitida. Si accorse che la
donna aveva le guance rosee rigate di lacrime.

«Grazie al cielo» esclamò, abbracciando il ragazzo. «Tu non sei come gli
altri. Tu hai lottato».

«Perché mi hai fatto tutto questo?» chiese Akram, sfinito.

«Perché tu capissi che l'Amore è fatto di estasi e sofferenza. Tu hai
affrontato la sofferenza. È giunto il tempo dell'estasi».

I due si allontanarono, la testa di lei appoggiata sulla spalla di Akram, con la
dolcezza di cui solo una donna innamorata è capace.

\plainbreak{1}

Mezz'ora dopo, una coppia di anziani passeggiava nello stesso modo in riva alla
spiaggia. Ma le onde cancellavano le ultime impronte di Akram e della bella
strega.

\end{document}
