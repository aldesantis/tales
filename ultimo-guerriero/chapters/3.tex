\chapter{}
\label{ch:3}

Méav non ricordava di essere mai stata così abbattuta. Le sembrava che intorno fosse tutto
rigoglioso, e lei fosse l'unica erbaccia arida da estirpare.

Quella sera avrebbe dovuto essere con i suoi amici e la sua famiglia, a cantare e danzare alla luce
e al calore del fuoco. Invece aveva deciso di allontanarsene, perché non tollerava il modo in cui la
trattavano, il modo in cui evitavano di incrociare il suo sguardo.

Per loro era una povera disgraziata.

Con questi pensieri nella testa la trentenne passeggiava, lontana da tutto e tutti. L'erba
accarezzava le sue caviglie nude e una brezza muoveva il lungo abito.

Osservava il bellissimo cielo scozzese, riconoscendo le costellazioni.

Evidentemente poteva solo continuare a sognare, e fantasticare su quel bambino che era venuto a
visitare il villaggio. Se ne vergognava, come se fosse una perversa ossessione guardare ciò che le
era stato negato.

Sotto la notte stellata passeggiava.

\plainbreak{1}

Era ormai passata una settimana dall'inizio del loro viaggio, ed era ora di tornare a casa. Benjamin
si faceva sempre più silenzioso e distante. Martha capiva la sua sofferenza, ma cos'altro poteva
fare?

Si girava e rigirava nel sacco a pelo senza riuscire a prendere sonno.

Era scomparso il ricordo di quanto di bello era accaduto in quei giorni. Le risate, i giochi e la
complicità col figlio non c'erano mai stati. Rimaneva solo un grande senso di vuoto, e quella
familiare sensazione di inadeguatezza.

Sarebbe tornata tra le braccia di Richard che l'avrebbe fatta sentire un'imbecille. Sarebbero
ricominciati i piccoli e insignificanti problemi: le bollette da pagare, il rubinetto che perde...

Piangeva.

Le tornò in mente anche il suo odio per quella donna che ammiccava al figlio, al villaggio. Si sentì
una stupida: cosa aveva fatto di male se non dimostrare affetto a un bambino? Eppure quel gesto le
era sembrato così fuori luogo...

Era lei a essere fuori luogo. Era lei quella incapace di amare, incapace di godersi i piccoli
momenti.

``Sono io a vivere per un rubinetto che perde''.

\plainbreak{1}

Benjamin sgattaiolò fuori dalla tenda, in silenzio. Prima però, diede un'ultima occhiata alla madre;
non era sicuro di ciò che stava facendo, e non sapeva come sarebbe andata a finire. Aveva una sola
certezza: di non voler tornare a casa.

Il cuore gli batteva così forte che temeva qualcuno lo potesse sentire. Nel buio totale attraversò
il campo addormentato. Anch'egli aveva sonno. Pensò di lasciar perdere quella follia, tornare nel
sacco a pelo, al caldo, abbracciare Martha e sentirsi protetto da ogni male.

Cosa ci spinge ad allontanarci dai nostri cari, le persone che più al mondo ci amano? È il desiderio
di scoprire nuovi orizzonti e vedere se possiamo trovare qualcuno che ci faccia star meglio? O lo
facciamo perché vogliamo mettere alla prova il loro amore?

Sicuramente il bambino non si poneva quelle domande. Tutto ciò che voleva era restare con i Celti,
con la bella signora che cantava. Avrebbe voluto portare anche Martha ma non era possibile: a
Benjamin pareva che la madre non appartenesse nemmeno a quel mondo meraviglioso.

\plainbreak{1}

Il bambino camminava con il naso per aria, pervaso da felicità mista a un certo timore. Non lo
preoccupava cosa sarebbe successo se avesse trovato la donna che cercava, ma ciò che sarebbe
successo se non l'avesse trovata. Avrebbe saputo riconoscere la strada per tornare indietro? E cosa
avrebbe detto Martha? Si sarebbe arrabbiata?

I piedi sembravano dotati di volontà propria. Intorno a sé vedeva solo la notte e le stelle
lontanissime. La luna invece era molto vicina, tanto che credeva di poterla toccare senza doversi
nemmeno alzare sulle punte. Si era sempre chiesto perché il satellite lo seguisse in continuazione;
il padre aveva tentato di spiegarglielo, ma egli si era limitato ad annuire poco convinto per paura
di deluderlo.

Gli parve all'improvviso che l'aria si facesse molto più fredda. Era stanco e stava per cadere
addormentato, ma quel gelo lo obbligò a spalancare gli occhi.

Poi rapida com'era venuta la ventata passò.

\emph{Tonf!}

Finì a sbattere contro qualcosa, e si trovò con il viso immerso in un cumulo di seta. Indietreggiò e
si rese conto che non si trattava di qualcosa, ma di qualcuno. A due metri da lui, il suo angelo lo
guardava con quei profondi occhi azzurri. Ben non aveva previsto di incontrarla così presto.

La donna sembrava piuttosto sorpresa di vederlo. Non lo disse esplicitamente ma il bambino sapeva di
essere stato riconosciuto. Lo aveva capito dal suo sguardo.

«Che ci fai qui?» chiese.

Mentì. Disse che si era perso. Che non riusciva a trovare la strada per tornare dalla madre.

Capì immediatamente che non era stata una mossa molto astuta: l'angelo avrebbe potuto portarlo
subito indietro. Di certo lei sapeva come arrivare al campo. Ma forse il destino era dalla sua
parte.

«Allora ascolta: è tardi per ritrovare la strada. Ora andiamo al villaggio, dove potrai riposare.
Domattina ti riporterò al campo. D'accordo?» propose sorridendo.

Annuì energicamente per timore di lasciarsi sfuggire l'occasione.

I due si incamminarono, mano nella mano.

\plainbreak{1}

Il villaggio era piuttosto distante, così che Benjamin ebbe il tempo di porre alla donna molte
domande. La prima era senz'altro la più semplice importante.

«Come ti chiami?»

«Méav» rispose. E tu?»

«Benjamin. Che nome strano Méav» commentò il bambino.

Lei ne rise.

«Per le mie genti è il \emph{tuo} nome a essere strano».

Andò avanti così per tutto il tempo, ma lei non diede mai segno di essere seccata. Ben pensò che
Martha si sarebbe stufata già da molto e avrebbe iniziato a dare risposte brevi e secche; lo faceva
sempre quando parlava troppo.

\plainbreak{1}

Mezz'ora -- e almeno una ventina di domande -- più tardi, giunsero all'agognata destinazione.
L'odore di paglia e fumo impregnava l'aria, ma senza dare fastidio.

Méav finalmente lasciò la mano del ragazzo. Egli vedeva nell'oscurità solo i contorni delle tende.
Il silenzio era assordante.

Benjamin seguì la donna finché non furono vicino a un fuoco ormai prossimo a spegnersi. Era quello
intorno al quale erano riuniti, fino a poco prima, amici e parenti della principessa per discutere
della sua tremenda situazione.

Il bambino le chiedeva solo di tradizioni che non aveva mai seguito e antiche battaglie che non
aveva mai vissuto. Non volendo deluderlo ella inventava. Gli raccontò le storie che aveva sentito
dal nonno quando era ancora una bambina, e capì cosa trovasse di tanto bello nel raccontarle: la
soddisfazione di avere qualcuno che ascolti ciò che si ha da dire, senza interessi né fini nascosti,
qualcuno che possieda l'innocenza di cui solo i bambini sono dotati.

Per la prima volta dopo molto, troppo tempo Méav si sentiva libera e completa.

Le sembrava che l'Universo intero si concentrasse per carpire ogni dettaglio della sua avventura.

Poco dopo Benjamin dormiva, sdraiato per metà per terra e per metà su una scomoda panca di legno.

Passò una notte dolce e senza sogni.

\plainbreak{1}

Méav invece non riusciva a prendere sonno: troppi pensieri le affollavano la mente e le impedivano
di dormire. Aveva portato il bambino nella tenda e si era coricata accanto a lui, carezzandolo con
dolcezza come aveva desiderato fare fin dal primo momento che lo aveva visto.

``Diventerà un bell'uomo'' pensò. Una ciocca dei folti capelli castani gli copriva la fronte pallida
e lei la scostò sorridendo. Suo padre avrebbe detto che era esile e debole ma si sbagliava: aveva un
animo forte. Lei lo sapeva.

Uscì dalla tenda perché voleva che il freddo della notte la tenesse sveglia, in modo da potersi
godere ogni secondo che le restava con Benjamin.

Guardando la luna, pensò per un attimo di tenerlo solo per sé. Avrebbe potuto dirgli che la madre
ormai non c'era più perché, non trovandolo, era ripartita.

Non sarebbe stato così difficile abituarsi alla nuova vita.

Si rese subito conto di quanto fosse ridicola e crudele quell'idea e si vergognò di se stessa.

E forse per la vergogna, oppure per l'emozione, pianse a lungo. Non era stato infrequente in quei
mesi che le lacrime rigassero il suo volto. Ma quello era un pianto diverso, che la svuotava di ogni
sensazione negativa e lasciava solo una certa gioia, e con essa l'ispirazione. Era un pianto d'amore
per gli amici e i nemici, e per coloro che non aveva mai incontrato.

Era un pianto di speranza.
