\chapter{Epilogo}
\label{ch:epilogo}

{\itshape Sono tutti sul divano. Ormai vi sprofondano, perché il povero nonno passa le giornate
incollato davanti alla televisione, immobile. Sì, ci sono proprio tutti: la mamma, il papà,
Nathalie, di dieci anni, e il suo fratellino di otto.

E sono tutti riuniti per seguire quel discorso, ascoltare la vicenda che ha commosso l’intera
Scozia.

«Ti rendi conto, Nathalie? Tredici anni! Questo ragazzo è stato per tredici anni lontano dalla
madre! E tu che segui la tua anche quando va in bagno!»

«Tredici anni? Più di quanti ne ho io?»

«Sì, tre anni di più. Zitta, zitta, sta iniziando!»

Il solito monologo pieno di suspense del giornalista, e poi finalmente in primo piano Benjamin.
Stanco. Non guarda la telecamera, ma fissa il pavimento.

«Papà, perché sembra così triste?»

«Hanno detto che stava male. Ssh, fammi sentire, Nathalie!»

Nathalie non coglie a pieno le parole del ragazzo; le sembrano così complicate, e le frasi così
lunghe. Non capisce se è allegro o addolorato.

Vede poi all’improvviso il vetro della stanza andare in mille pezzi, infranto dalla sedia che
Benjamin aveva accanto. Ora almeno sembra che nessuno capisca.

Nella sala si leva un mormorio di stupore e la telecamera trema. Nathalie pensa che chi la tiene non
deve essere molto bravo: il suo papà due anni prima aveva fatto un video della recita scolastica e
la telecamera non si muoveva mai. Le aveva detto di avere la mano così ferma perché era un
supereroe, ma poi aveva confessato ridendo di aver usato un treppiede. A lei non era sembrato tanto
divertente essere presa in giro.

Ora il ragazzo guarda giù dalla finestra, inquieto. Le persone nella sala strillano. Due donne, una
bionda e bella, un’altra più brutta con i capelli rossi, si avvicinano. Sono le due madri di Ben,
questo Nathalie lo sa. Quella con i capelli rossi piange e urla qualcosa mentre la bionda guarda
stupita.

Qualche interminabile istante, poi il ragazzo si gira verso la gente, e si lascia cadere
all’indietro. Nathalie non capisce, ma ora i grandi a quanto pare sì, perché le coprono gli occhi.

Non abbastanza in fretta, però, perché Nathalie ha visto il lampo negli occhi del ragazzo, prima che
sparisse dall’inquadratura. \/}
