\chapter{}
\label{ch:2}

«In Scozia?»

«Sì, un viaggio in Scozia! Hai sempre voluto farlo, ti lamentavi di non esserci mai stato, no?
Dicevi che volevi conoscere quel popolo antico... i Celti!» disse Martha, terrorizzata di aver speso
i suoi risparmi in due biglietti aerei per la destinazione sbagliata.

Ma Benjamin dissolse le sue preoccupazioni: «Quando partiamo?»

La donna si chinò ad arruffargli i capelli, sorridendo sollevata.

«Domani mattina».

Martha aveva organizzato quella gita improvvisa perché voleva stare lontana da Richard dopo il loro
litigio. Lui aveva fatto le valigie ed era andato dal fratello; non si sarebbe neanche accorto della
loro assenza.

Non si accorgeva mai di niente.

\plainbreak{1}

In aereo Benjamin non la smetteva di fare domande. Voleva sapere dove sarebbero stati, cosa
avrebbero fatto.

Martha sapeva di essere finalmente riuscita a catturare la sua attenzione.

Gli spiegò che sarebbero andati in un campeggio. D'altronde, era l'unica sistemazione che si era
potuta permettere senza dover chiedere aiuto al marito.

«Ho parlato col proprietario e sembra saperne molto sui Celti, sai? Ha detto che ne conosce alcuni!»

Il bambino pensò che stesse esagerando.

Poi fece una domanda che Martha avrebbe voluto evitare a tutti i costi.

«E papà dov'è?»

Tentò di ignorarlo ma Benjamin si dimostrò insistente.

«Ha avuto un problema al lavoro. Prenderà il prossimo aereo» rispose, e subito si rese conto del suo
«errore.

Perché aveva mentito? Non lo sapeva neanche lei. Dopotutto il figlio si sarebbe presto accorto che
non era vero. Voleva confessare subito, ma quando vide il lampo di gioia nei suoi occhi non ne ebbe
il coraggio.

``In un altro momento''.

\plainbreak{1}

L’alloggio non era male; un po' rustico, ma almeno Benjamin aveva l'occasione di stare insieme a
persone che condividevano la sua stessa passione.

Martha non capiva cosa potesse piacergli così tanto in una manica di ubriaconi vecchi di qualche
millennio che andavano in giro a urlare e staccare teste con le loro spade. Nonostante le ultime
generazioni di Celti fossero state costrette a modernizzarsi per sopravvivere, la loro indole
guerriera era rimasta intatta.

Aveva appena finito di montare la tenda -- il che le aveva fatto rimpiangere di aver intrapreso il
viaggio -- quando una donna le posò una mano sulla spalla e senza dire niente le indicò il panorama
tutt'intorno.

Era davvero uno spettacolo impressionante: a parte i monti a ovest, per diversi chilometri si
stendeva un'enorme prateria. Lontano, molto lontano, il fumo di un fuoco e qualche capanna.

«In questo posto veniamo per riposare e meditare» disse Celine, che Martha aveva conosciuto in
aereo. «Se penserai solo ai tuoi doveri di madre,» aggiunse «te ne andrai più nervosa di prima».

«Può darsi» rispose lei con un sorriso.

\plainbreak{1}

Arrivò la notte e con essa uno stupendo cielo stellato, quasi del tutto libero dall'inquinamento
luminoso. Se si ascoltava attentamente si potevano udire anche gli allegri canti che venivano dal
villaggio.

«Gli uomini bevono come spugne, le donne cantano come dee» commentò Celine, dietro Martha.

«Pare che tu li conosca bene».

Quella annuì. «Tutti noi che siamo qui abbiamo dedicato, chi solo una parte di essa, chi la vita
intera, allo studio della cultura celtica».

«E perché lo avete fatto?» chiese Martha incuriosita.

«Difficile dirlo. Forse perché ci affascinano quei canti,» e accennò con la testa al villaggio «o
forse perché non ne possiamo più della società moderna e vogliamo allontanarcene. Sai qual è la
differenza tra noi e loro, Martha? Sai cosa ci distingue?»

Quella scosse la testa.

«Anche loro hanno gli ignoranti, gli ipocriti e gli stupidi. Quelli sono ovunque. Anche loro hanno
uomini politici che mettono i propri interessi davanti a quelli del popolo. Anche loro hanno i
bugiardi e gli intolleranti. Ma per loro queste persone costituiscono delle eccezioni, e hanno
quello che si meritano: niente. Cosa insegniamo, invece, noi ai nostri figli? Che vince il più
furbo, il più malvagio, il più prepotente. Che il fine giustifica sempre i mezzi». E si fermò,
guardando oltre i monti. «Che devono avere paura delle proprie emozioni, perché sono sbagliate».

\plainbreak{1}

«Allora Ben, sei pronto?»

In realtà era Martha a doversi ancora vestire quella mattina. Benjamin, per via dell'eccitazione --
era il suo primo contatto con i Celti! -- non aveva nemmeno dormito.

Avrebbero visitato un villaggio poco distante, i cui abitanti erano famosi per essere perfino più
abili dei loro antenati nell'arte canora.

Dopo mezz'ora si misero in marcia.

David, la guida, parlò loro delle origini di quel popolo. Ne ripercorse la nascita, le avventure e
le sventure, i successi e i fallimenti. Nella mente di Benjamin si andavano pian piano disegnando
immagini sempre più nitide e fantastiche: uomini che combattevano, facendo roteare le spade
scintillanti, duelli per donne meravigliose, e così via.

La guida aveva preso a cuore madre e figlio, forse perché intuiva quale fosse la loro situazione.
Inoltre l'interesse del bambino per la storia era ammirevole in qualcuno della sua età: era come se
cercasse nei Celti ciò che gli mancava.

David sperava ardentemente che lo trovasse.

\plainbreak{1}

Gli abitanti del villaggio non aspettavano la visita dei campeggiatori quella mattina.

Non faceva comunque differenza, dato che anche quando venivano avvisati si limitavano a portare
avanti la loro vita. Ignorare la presenza degli estranei era il regalo più grande che potessero fare
a questi ultimi.

Sì, il patto era quello: esibire le proprie tradizioni, dimostrarsi una valida risorsa culturale. In
cambio i Celti avevano ottenuto di poter abitare le terre in cui erano nati.

La vita per loro non era però così terribile: potevano mantenere la propria identità, e a volte si
stabilivano relazioni sincere con gli stranieri. Méav stessa era una buona amica di David, che aveva
dedicato la vita allo studio delle loro usanze.

``Chi cerca trova'' dice un vecchio detto.

I Celti avevano trovato negli stranieri, e gli stranieri nei Celti, amici, amanti, sposi e spose.

Méav cercava un figlio.

Chi cerca trova.

\plainbreak{1}

Martha cercava di seguire gli altri, tuttavia era difficile: Ben continuava a fermarsi ogni dieci
passi; voleva vedere e conoscere i più insignificanti dettagli di ogni filo di paglia su cui metteva
i piedi. Quando poi passarono davanti a un gruppo di persone che cantavano il bambino ne fu
totalmente meravigliato. A Martha non piaceva particolarmente la musica, perché non l'aveva potuta
mai davvero apprezzare. Le casalinghe non possono permettersi certi lussi.

Tuttavia la donna non poté fare a meno di fermarsi e ascoltare in contemplazione quando sentì quelle
voci così dolci e belle. Era come se i canti la trasportassero lontano, in altri luoghi e tempi. Non
riusciva a capirne le parole perché erano in una qualche lingua antica, ma sapeva perfettamente di
quali argomenti trattavano.

Dopo qualche istante notò che Ben era attratto da una donna bionda al centro del gruppo. Era la
solista, e certamente la più brava. Senz'altro una donna affascinante.

Anch'ella, ogni tanto, lanciava sguardi languidi al bambino. Sguardi che a Martha non piacevano. Li
trovava falsi e malvagi. O forse ne era semplicemente gelosa, perché al figlio piaceva quella donna
più di quanto non gli piacesse sua madre.

«Vieni, Ben» disse, e lo trascinò lontano.

La bionda lanciò loro un'ultima occhiata.

\plainbreak{1}

Quella sera Martha era agitata sia per via della donna che avevano incontrato, sia perché Benjamin
continuava a chiederle del padre. Ormai doveva aver capito che qualcosa non andava. Quando le ripeté
la stessa domanda per la decima volta in un'ora non poté più far finta di niente.

«Papà non verrà» gli disse, sospirando.

Quella frase era al tempo stesso una liberazione e una tortura.

«Ma avevi detto che..».

«Lo so cosa ho detto» tagliò corto.

«E allora perché non viene?»

Lo guardò a lungo prima di rispondere. Soppesava le parole.

«Non era vero che avrebbe preso il prossimo aereo. In realtà non sa nemmeno che siamo qui».

Vide gli occhi di Ben spalancarsi mentre lo diceva.

«E perché non glielo hai detto?»

«Perché avevamo litigato. Le persone litigano e quando succede non si parlano per un po'».

«Sì, lo so» e pronunciò quelle parole come se volesse in realtà dire: ``Avete discusso così tante
volte da rendermi un esperto in materia''. «Bene, ora ci siamo chiariti, vero?» chiese speranzosa
Martha.

«Ma quando torno lo posso dire a papà?»

``Oddio, ti prego, no''.

«No che non puoi!»

«E perché?»

«Perché no e basta!» strillò infine, irritata.

Forse aveva esagerato.

Non riuscì a dire altro e del resto non ne ebbe bisogno, perché Benjamin si fece silenzioso e non le
rivolse la parola per tutta la sera.
