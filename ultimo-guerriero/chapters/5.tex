\chapter{}
\label{ch:5}

Con stupore di tutti Benjamin si abituò molto velocemente alla sua nuova vita.

Méav faceva di tutto per viziarlo per quanto le fosse possibile, ma anche se non si fosse impegnata
non sarebbe cambiato niente: quel ragazzo era ormai un celta. Dormiva su letti di tela e paglia,
stava imparando le tradizioni di quel popolo e si allenava nell'arte della spada che non avrebbe mai
avuto occasione di mettere in pratica.

Nonostante vivesse lontano dalla società dell'educazione, prendeva lezioni dagli abitanti del
villaggio, soprattutto riguardanti l'astronomia, la geografia e le lingue (stava lentamente
imparando il Gaelico, e continuava a esercitarsi con l'inglese, perché così aveva voluto Méav).

In poco meno di sei mesi aveva già assimilato tutto quello che c'era da sapere sulla sua nuova
famiglia. La parte che più gli piaceva rimanevano i canti e in questo Méav eccelleva. Così passava
le giornate ad ascoltare e pregare di poter ascoltare, e ogni sera si addormentava al suono della
voce di sua madre.

La sua nuova madre.

\plainbreak{1}

Méav aveva finalmente raggiunto quella felicità che tanto desiderava. Aveva un figlio. Quando la
notte si svegliava, a volte, si assicurava che Benjamin fosse ancora lì accanto a lei: le sembrava
di poterlo perdere da un momento all'altro.

Il villaggio era diviso riguardo ciò che le era capitato. La maggioranza pensava che la donna avesse
agito per il meglio, e che dovesse essere il bambino a scegliere la propria famiglia. Questo
rispecchiava la filosofia del suo popolo: ognuno era libero di disporre della propria vita, e non
era compito degli uomini indirizzarlo verso una scelta.

Altri invece vedevano Méav come una volgare criminale e chiedevano che fosse esiliata. Tenevano se
stessi -- e i propri figli, ovviamente -- il più lontano possibile da lei e dalla sua famiglia. Il
padre della donna, capo-villaggio, la poteva proteggere grazie alla sua autorità, ma non sarebbe
stato sempre lì e ciò lo preoccupava. Si chiedeva cosa sarebbe successo a sua figlia e a quello che
si era ormai rassegnato a considerare suo nipote.

L'unica a cui non interessasse nulla di tutto questo era, paradossalmente, Méav. Ella non si curava
delle opinioni altrui e si preoccupava unicamente di far sì che a Ben non mancasse nulla.

Voleva solo renderlo felice, e per riflesso esserlo anche lei.

\plainbreak{1}

{\itshape Incontrava Benjamin durante una passeggiata serale. Era vestita proprio come quella sera,
ma c'era qualcosa di diverso nell'aria.

«Scusi» le diceva il bambino, che le era venuto addosso e subito si voltava per andarsene.

«Non preoccuparti» rispondeva lei sorridente, cercando di apparire il più possibile buona e
inoffensiva. «Aspetta!» lo fermava poi prima che scappasse.

«Vuoi restare un attimo con me?»

Quello però non poteva: doveva tornare dalla madre, e correva verso Martha, che osservava la scena
poco distante, con aria grave.

«Aspetta!» ripeteva Méav e tentava di fermarlo, ma senza riuscirci.

Benjamin raggiungeva Martha che lo accarezzava dolcemente.

«Non è tuo figlio». \/}

Quel sogno, che si ripeteva quasi identico da ormai una settimana, turbava profondamente Méav. Era
forse dovuto al fatto che Ben aveva incominciato a riferirsi a lei come ``mamma''. Ella aveva
insistito per essere chiamata col proprio nome ma quello faceva finta di non sentire.

Si rese conto che non sarebbe mai riuscita a godersi appieno la sua nuova vita: c'era sempre quella
fastidiosa sensazione, più lieve di un senso di colpa e meno dell'indifferenza dalla quale aveva
sperato di essere pervasa. Il tempo sembrava peggiorare le cose invece di migliorarle.

``Non è mio figlio''.

\plainbreak{1}

Erano ormai passati due anni dalla scomparsa di Benjamin e da quando Martha lo aveva detto a
Richard. Quest'ultimo aveva attraversato una fase d'ira, fatta di minacce e cattiverie nei confronti
della moglie e poi si era calmato, ma la donna avrebbe preferito che ciò non fosse successo.

Ormai non si parlavano più. Anzi, si parlavano, ma era come se non si parlassero. Compra l'acqua, è
finita la frutta, fai il pieno alla macchina. La loro vita coniugale si limitava a questo. Quando
c'era Benjamin almeno avevano ancora la forza di litigare, tenevano ancora l'uno all'altra. Ora
vivevano sotto lo stesso tetto soltanto perché un divorzio sarebbe stato troppo faticoso e costoso.

E quella sera Richard aveva deciso di rompere il silenzio. Mentre mangiavano un piatto di pasta
scondita aveva bofonchiato, tra una forchettata e l'altra: «Mi vedo con una donna». Nonostante la
loro disastrosa situazione l'avesse portata a immaginarlo, Martha rimase ferita, ma cercò di non
darlo a vedere. Che senso aveva esprimere emozioni se nessuno poteva accoglierle? Così aveva fatto
finta di non aver sentito.

Richard non si azzardò a ripetere la frase.

\plainbreak{1}

Pochi mesi più tardi Richard era andato a vivere con la sua nuova compagna. Non c'era stato bisogno
di dire niente: quando aveva smesso di tornare a casa la sera, era stato tutto chiaro.

Lei invece aveva deciso di trasferirsi in Scozia. Non si illudeva di trovare Benjamin né -- giammai!
-- una nuova vita, ma sentiva che quella era la cosa giusta da fare. Si sarebbe sentita più vicina a
suo figlio, l'avrebbe aiutata ad andare avanti. Così aveva chiesto alla madre i soldi per il
biglietto aereo ed era partita senza indugiare oltre.

Durante il viaggio aveva pensato a lungo di andare a trovare David, ma alla fine si era convinta che
non sarebbe stata una buona idea. Dopotutto di cosa potevano parlare? Dei ``bei vecchi tempi''?

Per qualche tempo aveva alloggiato a casa di una compagna di liceo, poi aveva trovato un lavoro come
cameriera in un bar e aveva deciso di trasferirsi in un microscopico appartamento poco lontano dal
centro. Le pareti erano scrostate e ammuffite e il posto puzzava di birra, ma Martha voleva restare
lontana da qualunque essere umano.

Ora nella sua nuova casa, durante l'unico giorno libero, guardava le goccie di pioggia cadere
obliquamente sul vetro della finestra e pensò che la sua vita negli ultimi anni era scorsa con la
stessa velocità.
