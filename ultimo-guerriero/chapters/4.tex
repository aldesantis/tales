\chapter{}
\label{ch:4}

Il sole non era ancora sorto quando Méav svegliò il bambino.

«Benjamin, dobbiamo andare» disse scuotendolo con delicatezza.

Quello si mosse leggermente. Allora la principessa lo scosse con più forza, e stavolta Ben si alzò e
si stropicciò gli occhi. Lo prese per mano e lo condusse rapidamente fuori dal villaggio. Non voleva
che qualche suo mattiniero conoscente li vedesse e lei si trovasse costretta a dare spiegazioni.

Stavolta Benjamin fu silenzioso per tutto il viaggio. Méav pensò che fosse per via del sonno; aveva
l'aria di non sapere neanche perché si trovasse lì invece che con la madre.

A metà della strada Ben la strattonò e sussurrò qualcosa. La donna si chinò.

«Come hai detto?» chiese.

«Non voglio tornare da mia madre. Voglio stare con te».

Lei non sapeva cosa dire. Come aveva potuto non capirlo? Benjamin non si era mai perso, l'aveva
cercata! Trovava la cosa tristemente dolce. Si sentì sciocca per non averlo intuito prima.

Cercò di farlo ragionare.

«Non puoi... non si può fare!»

«Lo so» rispose quello, costernato. Ma non puoi fare finta di niente, stavolta? Fare finta di essere
«mia mamma?»

La principessa sorrise.

«Non sarebbe giusto nei confronti della tua \emph{vera} mamma» cercò di spiegargli.

«Pensa a come sarebbe triste!»

«Ma così sono triste io!»

La cosa assurda di quella conversazione era che la donna iniziava a pensare che il bambino avesse
ragione. Chi era la madre per privarlo di quella felicità che egli cercava tanto ardentemente?

Fece un ultimo, disperato tentativo.

«Ascoltami. Se deciderai di rimanere con me la tua vita cambierà per sempre: diventerai un
«guerriero».

Quelle parole, che volevano suonare come terribili presagi, erano per Benjamin l'immagine del futuro
che aveva sempre desiderato. I suoi occhi si riempirono di lacrime.

«Perché non mi vuoi portare con te?» domandò sconsolato.

Era troppo per Méav. Si morse il labbro inferiore, indecisa sul da farsi, nervosa come mai prima
d'allora. Guardò il campo, poi il ragazzo. Sospirò. «D'accordo».

La abbracciò, e lei ne fu quasi imbarazzata.

Per l'ultima volta percorsero quel tragitto che dal campo portava al villaggio.

Il campo dove dopo qualche ora Martha si sarebbe svegliata.

\plainbreak{1}

«Tu hai fatto cosa?!»

Solamh era scioccato. Non neanche cosa dire, tanto era allucinato. Non gli erano mai molto piaciuti
i contatti con le altre generazioni, ma ciò che aveva fatto sua figlia esulava da ogni ragionevole
limite. Pensava agli effetti che avrebbe avuto su Méav, su Benjamin, sulla loro famiglia e infine
sull'intero villaggio.

Il novenne si teneva in disparte, strusciava i piedi a terra e teneva lo sguardo basso. Quell'uomo
baffuto gli incuteva un certo timore: sembrava che potesse esplodere da un momento all'altro.

Méav neanche sapeva cosa dire perché si rendeva conto dell'assurdità di quella situazione. Così
padre e figlia si scrutavano, lei lo sguardo fuggente, lui fisso, gli occhi sbarrati.

Fu sua moglie a interrompere il silenzio.

«Se lei vuole bene al bambino, e il bambino a lei, non vedo quale sia il problema».

Parlavano in Gaelico, perciò Benjamin non capiva una parola. Ciò nonostante sapeva che c'era in
gioco il suo futuro.

«Éibhleann, sei impazzita anche tu dunque? Ti rendi conto che ha rapito un bambino?» disse infuriato
e guardò Benjamin, che arrossì.

«Non l'ho rapito. Lui ha voluto venire con me» si difese Méav.

«Esatto. E se dovesse cambiare idea c'è sempre tempo, no?»

«E invece no, non ce n'è! Se non lo portiamo indietro ora, capiranno che non è arrivato qui per
caso! Le autorità non vedono l'ora di esiliarci: aspettano solo un'occasione come questa!»

Éibhleann gli posò le mani sulle spalle.

«Tu permetteresti che un tuo abitante -- tua figlia! -- sia triste per paura di ciò che potrebbe
succedere se non lo fosse? È questo l'uomo che ho sposato?

Sei così codardo?!»

Solamh si rassegnò. Capì che non avrebbe potuto fare molto per allontanare Benjamin da Méav.

«D'accordo» disse, «ma il bambino sarà libero di andarsene quando vorrà, e lo consegneremo se
qualcuno verrà a cercarlo».

Méav sospirò di sollievo. La madre la guardò. Anche se la stava appoggiando, non era completamente
d'accordo con ciò che aveva fatto.

«Va bene».

\plainbreak{1}

«Ben, non è divertente. Vieni fuori!»

Martha era uscita dalla tenda, e ora si stava aggirando per il campo. Chiese a Celine, appena
sveglia e già pronta per partire, a una coppia, moglie e marito che avevano entrambi gli occhi gonfi
dal sonno, a un giovane ragazzo forse in visita con la sua scuola, ma nessuno lo aveva visto.

Le sembrava di trovarsi in un film dell'orrore di quarta categoria. Non le erano mai piaciute le
scene in cui qualcuno scompariva, e ancora meno ora che quel qualcuno era suo figlio.

Vagabondava da ormai venti minuti quando David la raggiunse, probabilmente avvisato da qualcuno. Con
la solita calma le chiese cosa non andasse.

«Non trovo più mio figlio, ecco cosa!» sbraitò.

Si rese conto di essere stata troppo dura con qualcuno che cercava solamente di aiutarla, ma non
c'era tempo per le scuse.

Sentiva i battiti del cuore e il mal di testa aumentare ogni secondo che passava senza vedere
Benjamin. In preda al panico iniziò a correre in lungo e in largo, con David che la seguiva a
fatica. La fermò, col fiatone.

«Calma, Martha. Adesso lo cerchiamo fuori dal campo e se non riusciamo a trovarlo avvisiamo la
«polizia».

Ma lei era irritata da quel tono pacato e tranquillo con cui affrontava qualunque situazione. Come
poteva non rendersene conto? Si trattava di suo figlio!

La polizia, poi! Quando David lo disse a Martha si seccò la gola. Cosa avrebbero detto di lei? E
peggio ancora: cosa avrebbe pensato Richard se non fosse riuscita più a trovare loro figlio?

Ormai terrorizzata sradicò da terra la sua stessa tenda che volò, sospinta dal vento, per qualche
metro prima di fermarsi in mezzo all'erba.

Benjamin non c'era.

Si levarono mormorii di disapprovazione e si accorse che tutti la guardavano.

Benjamin, dov'era Benjamin?!

\plainbreak{1}

«Mi dispiace... Eri l'unica persona di cui mi fidassi».

«Lo capisco, ma io sono quasi sempre stato accanto alla madre in questi due giorni: è distrutta. Non
può più andare avanti così».

«Ha già avvisato la polizia?»

«No... Assurdo, vero? Voleva che prima venissi a parlare con voi, perché pensa che potreste aiutarla
meglio delle autorità. Ora che so la verità, però...».

«Non penserai di dirglielo, vero?»

«È ovvio che devo dirglielo! Non puoi tenerlo tu, non è giusto! Tu non hai sentito Martha piangere e
disperarsi a notte fonda! Lo sai che ha avvisato suo marito, e lui sta pensando di chiedere il
divorzio?»

«Non meritava suo figlio. Lo sai anche tu».

«Può darsi. Chi siamo noi per giudicare, però?»

«Io voglio stare con lei..».

«Benjamin, non è una tua decisione: sei troppo, davvero troppo piccolo per una simile cosa. Pensi
che sia una bella avventura, questa, ma tra poco tempo -- meno di quanto immagini -- te ne pentirai,
e allora sarà troppo tardi».

«Invece è lui che deve decidere. Si tratta della sua vita, non di quella della madre».

«Piantala! Sei solo un'egoista! Che fine ha fatto la donna meravigliosa e sensibile che conoscevo?
Colei a cui avrei affidato la mia vita? Cosa ne è stato della persona che amavo?»

«Io sono felice, lui è felice. È tutto quello che conta».

«Per te, forse. Non per noialtri. Ma perché sto ancora discutendo? So cos'è giusto fare».

«Aspetta un attimo! Non vorrai dirglielo, vero? Aspetta, ti prego!»

Il sole era alto nel cielo. La figura dell'uomo in cammino si stagliava all'orizzonte.

\plainbreak{1}

Martha non si aspettava una tale modernità in un ufficio di polizia.

Soprattutto non si aspettava una tale cordialità: almeno dieci agenti cercavano di metterla a
proprio agio da quando era entrata. Fallendo miseramente. Nulla può calmare una madre il cui figlio
è scomparso.

Al suo fianco stava David, che aveva maggiore dimestichezza con la giustizia scozzese.

L'unica persona sgradevole lì era l'impiegata che stava prendendo i suoi dati.

Una donna bionda, grassoccia, che doveva avere poco meno di cinquant'anni e l'aria di una persona
che una volta era bella e buona ma è stata inacidita dal tempo. Continuava a porle domande con un
tono di arrogante superiorità, sforzandosi di mostarsi visibilmente infastidita quando Martha
chiedeva spiegazioni.

«Signora, perché non ci ha avvisati subito? Quarantotto ore sono un'eternità!»

Iniziò a balbettare. Cosa avrebbe dovuto dire? Si aspettava un aiuto di David, ma lui sembrava
concentrato su un filo che spuntava dai pantaloni. Lo odiò in quell'unico momento.

«Allora, signora?» si innervosì quella. Nulla da dire?»

Fu invece proprio uno dei colleghi dell'arpia a salvare Martha.

«Suvvia, Ailis, non vedi com'è tesa? Non peggiorare le cose».

Doveva essere un superiore, perché Ailis cambiò subito tono.

«Certamente... Io cercavo solo di capire cosa fosse successo».

«Suo figlio è scomparso. Concentriamoci su questo».

La poliziotta annuì e appena quello si voltò roteò gli occhi con aria seccata.

Dopo un altro quarto d'ora di domande incessanti Martha fu liquidata da un «le faremo sapere», che
le ricordava tanto un colloquio di lavoro andato male.

Uscì di lì ancora più affranta.

Si ricordò all'improvviso della presenza di David. Era stato come un fantasma in quelle ore.

«Tu sei sicuro di non sapere dove possa essere finito mio figlio?»

«Sì».

«Al villaggio non ti hanno detto niente? Quando l'abbiamo visitato c'era una donna che cantava e
«Benjamin..».

«No».
