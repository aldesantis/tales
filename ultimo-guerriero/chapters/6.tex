\chapter{}
\label{ch:6}

Tredici anni erano passati da quando Méav aveva preso Benjamin con sé. Così tanto che ormai lui
aveva quasi dimenticato che la donna non era sua madre. Non la madre biologica almeno, perché ella
non gli aveva mai fatto mancare nulla.

D'altronde egli non bisognava più di nulla. Aveva ventuno anni ed era diventato più alto -- e molto
più forte -- di Méav. Eccelleva tra i Celti per l'abilità nei lavori manuali, la fedeltà alle
tradizioni e l'integrità morale. Era divenuto tutto quello che sua madre, entrambe le sue madri
speravano di vederlo diventare. Parlava perfettamente il gaelico e terribilmente l'inglese. Tutti
col tempo lo avevano accettato come proprio pari.

Méav era contenta di vederlo così ben integrato, ma a volte le pareva che nella sua ostinazione
volesse liberarsi del passato, e ci stava senz'altro riuscendo. Non era contenta di ciò: fare in
modo che Benjamin mantenesse i contatti con la sua precedente vita era l'unico modo che le rimaneva
di liberarsi dei sensi di colpa.

L'unico modo per chiedere scusa.

\plainbreak{1}

La Malattia devastava il villaggio da ormai un anno. Credevano di poterla controllare ma non era
così. I sintomi erano quelli di un comune raffreddore che portava lentamente alla morte per febbre,
o più raramente per asfissia.

Alcuni erano andati in ospedale e lì erano riusciti a curarli. Altri, troppo orgogliosi, non ce
l'avevano fatta.

Méav per qualche motivo era rimasta immune. Si era però ammalato Benjamin, e i sintomi peggioravano
ogni settimana. La notte tossiva così forte da svegliare chi gli stava vicino, e si lamentava per il
dolore a ogni respiro. Per la donna era una sofferenza vederlo ridotto in quello stato. Così un
giorno gli propose di andare in ospedale, anche se già sapeva quale sarebbe stata la sua reazione.

«Non se ne parla neanche» rispose seccamente Ben.

«Ma perché? Loro potrebbero aiutarti!»

«Non voglio avere nulla a che fare con quella gente».

«Lo capisco, Benjamin, davvero, ma non puoi fare di tutta l'erba un fascio.

Solo perché hai avuto un'esperienza negativa con i tuoi genitori..».

«Tu non capisci» la interruppe. «Io non ho avuto un'esperienza negativa con i miei genitori, ma con
il mio mondo. Non era solo il rapporto tra Martha e Richard a non piacermi, ma il rapporto tra
ognuno e il suo prossimo: sono tutti così disonesti, e malvagi, e ipocriti».

«Per questo non ti sto chiedendo di tornare a vivere tra loro. Andiamo solo a farti visitare, poi
torneremo al villaggio e continueremo la nostra vita».

«Ho detto di no. Non intendo discuterne oltre. Piuttosto che avere a che fare di nuovo con quelle
bestie preferirei morire».

La discussione finì lì. Cos'altro poteva dire? Non c'era modo di convincere suo figlio.

\plainbreak{1}

Una notte però Méav non poté sopportare la vista di Ben che nel sonno delirava e tossiva,
macchiandosi di sangue. Non era terrorizzata all'idea di perderlo -- da molto tempo non lo guardava
più come quel bambino spaventato che l'aveva cercata per vivere con i Celti -- ma all'idea che egli
stesso perdesse la vita.

Quattro uomini -- tra cui il padre di Méav -- si offrirono volontari per portare fino in ospedale
quello che ormai era un amico. Lo caricarono su una rudimentale barella, costruita tempo addietro e
si incamminarono. Tutti guardarono il triste corteo uscire dal villaggio.

Méav li seguiva e dai suoi occhi sgorgavano copiose lacrime. Per strada non fece che ripensare a
com'era stata bizzarra la sua vita: gli dei le avevano donato un figlio quando tutto sembrava
perduto, e ora che tutto andava per il meglio gli dei volevano toglierglielo.

Fin da piccola le era stato insegnato che i disegni divini sono complessi e indecifrabili, ma sempre
a favore dell'uomo. Eppure lei non riusciva a trovare proprio nulla -- neanche pensandoci fino a
farsi venire il mal di testa -- che fosse a vantaggio suo o di Benjamin, in ciò che stava accadendo.

Nulla.

\plainbreak{1}

La donna all'accettazione sbadigliò pesantemente, permettendo ai pazienti di ammirare la sua ugola.

``Eccone un altro'' pensò guardando l'uomo disteso sulla lettiga. Non aveva una bella cera: era
ormai quasi trasparente e aveva sudato tanto che ci si poteva specchiare nella sua fronte.

«Quarta porta a sinistra, l'ultima in fondo al corridoio».

Quelli la guardarono per un po', con gli occhi sbarrati. Quando stava per indicare nuovamente la
strada a gesti -- forse non parlavano l'inglese? -- il gruppo si avviò verso la sala d'attesa.

I pazienti osservarono la scena, alcuni per nulla sorpresi, altri esterrefatti.

\plainbreak{1}

Benjamin tentò di capire dove si trovasse, ma non riusciva a distinguere chiaramente le forme. Gli
sembrava di vedere Méav, la riconosceva dal colore dei capelli, così puri e sfavillanti nonostante
l'età. Altre figure si stagliavano intorno a lui. C'era un pesante odore nell'aria, che era
insopportabilmente calda e stantia; sembrava difficile afferrare le poche molecole d'ossigeno.

«Dove sono?» sussurrò, e non era certo di essere stato udito perché non gli giunse risposta per
interi minuti, o forse erano secondi.

«Stavi male, molto male,» diceva la madre con la sua dolce voce, ora coperta da un velo di tristezza
«così non ho avuto scelta. Ho dovuto farlo».

«Fare cosa?» «Oh, no..». aggiunse, quando i suoi sensi si acuirono ed egli si rese conto che
quell'odore nell'aria era disinfettante. «Non dovevi, non dovevi...».

Sentì Méav singhiozzare.

«Mi dispiace, ma non potevo lasciarti morire. Sei mio figlio. Non potevo...».

«Non più! Ora che siamo tornati qui non sono più tuo figlio».

La donna singhiozzava in silenzio. Vide qualcuno avvicinarsi per abbracciarla.

Chiuse gli occhi e cercò di dormire, ma il pianto della principessa continuava a tormentarlo.

\plainbreak{1}

Quella sera, Méav apprese che sarebbero dovuti rimanere in ospedale almeno per un mese: la malattia
-- per la quale, scoprì, era distribuito un vaccino da tempo -- aveva devastato il corpo di
Benjamin, che doveva recuperare forze.

Chiese agli altri di tornare a casa: non c'era motivo perché restassero. Suo padre non voleva ma lei
insistette.

Uscì fuori, nel cortile dell'ospedale, e per un'ora non fece altro che guardare il cielo pensando a
come avrebbe potuto dirlo a suo figlio.

\plainbreak{1}

«Un mese... un lunghissimo mese».

Benjamin si alzò dal letto per andarsene, facendo oscillare pericolosamente la flebo. Méav si mosse
per fermarlo.

«Non sforzarti. Ti prego, dammi ascolto una sola volta. Non l’hai fatto tredici anni fa, fallo ora!»

Quello non disse niente: sospirò e si sdraiò nuovamente, in un gesto che sembrava una rinuncia a
combattere.
