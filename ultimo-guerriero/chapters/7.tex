\chapter{}
\label{ch:7}

Le dieci. Le dieci e mezza. Un battito di ciglia. Mezzogiorno. Sì, signora, è quasi il suo turno.
Che numero ha? Ottantanove? Siamo al sessantatré. Un attimo, guardi quanta gente c’è. Aria calda,
aria pesante, vapori di alcol.

Quella fastidiosa, pressante premura di una mamma troppo protettiva.

Bip. Ottantanove. Finalmente!

Si aspettava tanto per un pizzico. Un pizzico ed era tutto finito, non si rischiava più quella
brutta influenza. Martha ancora non si era vaccinata e non sapeva bene perché. In verità non le
interessava più di tanto cosa succedesse: anche se non lo avrebbe mai ammesso, vivere o morire non
faceva troppa differenza. Si vive per un principio, un obiettivo o una persona. Quali ideali aveva
lei? La famiglia? Neanche a pensarci. L’obiettivo? Rendere il mondo un posto migliore? Dopo quello
che aveva fatto? Ma per carità! E la persona? Quella una volta l’aveva... Ora non più.

Un’amica però l’aveva convinta ad andare in ospedale. «Quell’influenza è davvero brutta» aveva
detto. Non più brutta di qualunque cosa fosse accaduta a Martha.

Si avviò verso l’uscita, guardando distrattamente i malati nelle camere. Loro certamente avevano un
motivo per continuare a lottare. I volti fuggivano via veloci, alcuni giravano lentamente il collo
nella sua direzione, doloranti, e tentavano di catturarne lo sguardo, ma lei era già lontana.

«Mamma?»

Era quasi un sussurro ma a Martha sembrò urlato con quanto fiato un essere umano ha nei polmoni. Le
pareva che il cuore volesse liberarsi da quella gabbia che è la gabbia toracica, per giungere più
velocemente di lei verso l’origine di quel suono. Era certa che non fosse niente: non poteva essere,
ormai aveva perso la sua occasione; eppure non avrebbe potuto dormire quella notte se non avesse
prima controllato.

Quando entrò nella stanza lasciò cadere il braccio vaccinato che teneva piegato, cosa che le provocò
un improvviso quanto impercettibile dolore.

I lineamenti del volto, i capelli, ma soprattutto gli occhi! Gli occhi erano l’unica cosa che
sembrava non essere invecchiata in tutti quegli anni.

«Benjamin?»

La donna credette di morire. Il suo fu davvero un urlo, e fece accorrere diversi infermieri.
Intimarono a Martha di abbassare la voce, ma ormai non c’era più bisogno, perché era passata dalle
urla al pianto. In ginocchio accanto al letto continuava a toccare e baciare il figlio, come per
accertarsi che fosse ancora lì.

Benjamin piangeva tredici lunghi anni di lacrime.

E Méav, anch’ella piangeva, e non osava muoversi né parlare.

In quel momento il pianto di Martha era l’unico a essere certamente un pianto di gioia: per gli
altri due si trattava di un misto di gioia, rabbia e stupore.

Qualunque cosa fosse però era terribilmente liberatoria.

\plainbreak{1}

Tutti quanti parlarono a lungo, l’intera notte. Benjamin non osò confessare alla madre che era
scappato perché non voleva più stare con lei: la vedeva troppo felice. Così le raccontò che si era
perso cercandola e per una settimana aveva vissuto di stenti nel bosco finché Méav non l’aveva
trovato.

Avevano poi tentato di tornare al campo ma non erano riusciti a scovarlo. Quando il ragazzo si
addormentò, provato dalle emozioni e dalla malattia, Martha uscì con Méav nel cortile. Passeggiarono
per un po’ in silenzio, poi prese le sue mani e si fermò, guardandola negli occhi.

«Volevo dirti una cosa, ma sono stata così occupata a parlare con Ben che non ho avuto il tempo.
Grazie. Tantissimo. Se non fosse stato per te mio figlio sarebbe morto. Sei sua madre almeno quanto
me, forse di più».

Méav esplose in un nuovo pianto, stavolta di certa disperazione. Sapeva benissimo di non meritare
quelle parole ma non poteva confessare la verità a Martha. Avrebbe scoperto che Benjamin la odiava!

E soprattutto avrebbe scoperto come era stata crudelmente privata di suo figlio per un puro
interesse personale.

La abbracciò e pianse più forte mentre Martha, sorpresa, tentava di consolarla.

«Dai, adesso va tutto bene» diceva sorridente.

``No. Adesso non c’è una sola cosa che vada bene'' pensò Méav.

Quello fu l’episodio che segnò la ricomparsa di Benjamin nel mondo tecnologico, quello in cui le
notizie viaggiano alla velocità della luce. E così qualcuno che, chissà come, aveva intuito cosa
stesse succedendo aveva avvertito la stampa, che la mattina dopo faceva pressioni per poter entrare
nella stanza del ragazzo. Non era sicuramente storia di tutti i giorni quella di un bambino che si
perde, vive con i Celti per tredici anni e come se non bastasse ritrova poi la madre!

Per due settimane li tennero lontani perché non volevano che il ragazzo si affaticasse. Quando
Benjamin fu quasi guarito Martha e Méav gli chiesero se gli sarebbe piaciuto raccontare la sua
storia. Egli accettò. Quale delle due avrebbe raccontato, l’amara verità o la dolce bugia, non era
ancora chiaro a nessuno, neanche al ragazzo stesso.

Impiegò tre giorni per prepararsi, rimuginando sul suo passato. Pensò a cosa avrebbe potuto dire, ma
ancor di più a cosa avrebbe potuto fare. Perché Benjamin non sopportava assolutamente la situazione
che si era creata. Era combattuto: da una parte felice perché aveva ritrovato la sua madre
biologica, dall’altra disperato perché aveva perso quella che lo aveva davvero accudito per tredici
anni.

La sera del terzo giorno infine prese la sua decisione. No, non poteva farcela.

Doveva agire. Uscì fuori dalla sua stanza. Martha e Méav lo seguirono mentre saliva fino al quarto
piano dell’ospedale, dove c’era la sala riunioni in cui i giornalisti si davano il cambio aspettando
il loro scoop.

«Sono pronto» disse loro.

Un enorme orso pigro sembrò svegliarsi dal letargo per aggredirlo di domande, ma lui salì sulla
pedana riservata agli oratori. I riflettori furono accesi, e gli procurarono un caldo infernale,
insopportabile, mentre calava il silenzio.

Agire una volta per non agire più.
