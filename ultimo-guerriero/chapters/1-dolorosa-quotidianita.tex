\chapter{Dolorosa quotidianità}
\label{ch:dolorosa-quotidianita}

«Non azzardarti mai più a ridicolizzarmi in pubblico, chiaro?! Non ti ho sposato per questo!»

Benjamin chinò la testa sul suo disegno, ma non riusciva a concentrarsi con quelle urla nelle
orecchie. Aveva dimenticato cosa stesse ritraendo con tanto impegno. Cercò di occupare la mente
pensando a qualcosa che non fossero i suoi genitori che si sbranavano giù in salone.

La camera era grande per un bambino di nove anni. Qualche poster era attaccato alle pareti, ma a lui
non piacevano sul serio: era stato il padre a metterli, quando aveva visto quanto spoglia fosse la
stanza del figlio.

«Tutti hanno un idolo, tu no?» aveva chiesto.

No, Benjamin non l’aveva.

Allora Richard aveva fatto sì che il suo idolo, un cantante che ascoltava quando era ancora al
liceo, divenisse quello di Ben.

Era una bella giornata e dalla finestra arrivava la luce tiepida e rassicurante del sole. Seguì con
lo sguardo un raggio che terminava la sua corsa sulla sua mano chiusa nell’impugnare una matita
azzurra.

La porta di casa si chiuse pesantemente.

\plainbreak{1}

«Mi dispiace tanto, figlia mia».

Seduta sul suo trono ricoperto di pelli di volpe Méav ascoltava la madre parlare con la voce rotta
dalla tristezza. Suo padre accanto alla moglie non aveva il coraggio di dire nulla.

Era la terza gravidanza che non riusciva a portare a termine. La prima volta aveva pensato che fosse
un caso, la seconda sfortuna, ma ormai non poteva più far finta di niente.

Nel villaggio s'era sparsa la voce che fosse maledetta e tutti, per quanto fossero obbligati a
mostrarle un certo rispetto -- dopotutto era la figlia del capovillaggio, una principessa -- se ne
tenevano alla larga il più possibile.

Desiderava così tanto un bambino. Voleva guidarlo, aiutarlo e consigliarlo...

Voleva vederlo crescere, vederlo diventare un uomo. Invecchiare mentre lui invigoriva. Invece era
destinata a rimanere sola. Sì, avrebbe sposato un uomo bellissimo, ma cosa sarebbe rimasto della
loro unione?

Quei pensieri la distruggevano. Durante il giorno Méav riusciva a scacciarli dalla testa occupandosi
di altre faccende. Ma quando dopo essere stati davanti al fuoco fino a notte inoltrata ognuno si
rintanava nella sua dimora lei, sentendo i pianti dei neonati che si svegliavano all'improvviso, non
riusciva a prender sonno, e non perché quei rumori la infastidissero: erano come una dolce melodia
per le sue orecchie.

Una melodia che pensava non avrebbe mai ascoltato da vicino.

\plainbreak{1}

Stava ultimando il suo disegno.

«Benjamin, vieni, è pronto!» urlava Martha.

Dopo che lo ebbe ripetuto tre o quattro volte il bambino sentì i passi della donna per le scale. La
porta si aprì di scatto e una trentottenne dall'aria stravolta si affacciò. Aveva gli occhi rossi,
probabilmente per via della sua allergia; sì, Benjamin sapeva che era allergica a qualcosa.

«Vieni a mangiare o no? La zuppa si fredda».

Lui scese le scale, raggiunse la cucina e si sedette a tavola. Non gli piaceva pranzare con la
madre: non faceva altro che riscaldare zuppe nel microonde, non importava se fosse gennaio o agosto.

Quel giorno, Benjamin non prese nemmeno in mano il cucchiaio; restò a fissare il piatto.

«Allora, non mangi?» chiese Martha.

«Non mi piace la zuppa» disse lui con voce flebile, aspettandosi un ceffone.

Invece non accadde niente. La madre lo guardò. I suoi occhi non tradivano alcuna emozione.

«Non si può sempre avere quello che si vuole, sai?» si limitò a ribattere avvicinandogli il piatto.

Allora Benjamin incominciò a mangiare. Se la madre si fosse arrabbiata avrebbe potuto sbattere i
piedi e frignare, se si fosse rassegnata gli avrebbe cucinato qualcos'altro.

Ma quell'apatia lo spaventava più di ogni altra reazione.

\plainbreak{1}

Mentre il figlio pranzava, Martha decise di dare una sistemata alla casa. Era già tirata a lustro,
ma aveva un disperato bisogno di sentirsi utile.

In bagno vide il proprio riflesso nello specchio. Era ancora una bella donna. I capelli rossi, un
poco sbiaditi, le cadevano dolcemente sulle spalle. Era esile e smunta, ma aveva l'aspetto di una
grande sognatrice brutalmente gettata nella realtà. Una volta era in cerca del principe azzurro per
mettere su famiglia e vivere felice per sempre. Non aveva mai smesso di desiderare quell'incontro;
ora però sapeva che il primo uomo di cui ci si innamora non è necessariamente quello perfetto.

Quell’uomo era troppo furbo e vigliacco per toccarla, ma abbastanza sfrontato da distruggerla
psicologicamente, giorno dopo giorno, in modo da non lasciare prove. Martha sapeva che Richard non
era sempre stato così: un tempo era anch'egli una persona sensibile e romantica, ma il tempo, il
lavoro, e soprattutto la nascita di Ben lo avevano cambiato.

La sua unica consolazione era quel figlio tanto desiderato, che tuttavia non si godeva come avrebbe
voluto: a volte le sembrava di trascurarlo, a volte di arrabbiarsi per niente, altre ancora di non
riprenderlo quando doveva. Si sentiva sempre inadeguata. Solo in pochi momenti era se stessa e
allora era al settimo cielo: dimenticava tutti i problemi e i guai che le erano capitati.

Ma la cosa peggiore di quella situazione era la compassione altrui. Non sopportava quei finti
sorrisi pieni di amarezza e comprensione. Sua madre, sua sorella, le sue amiche, tutti la facevano
sentire un'idiota.

Reagisci, dicevano, combatti.

Cosa ne potevano sapere loro che non ci erano mai passate. Cosa ne potevano sapere di tutte le notti
passate a piangere, a pensare di prendere Ben, fuggire e ricominciare una nuova vita accanto a
qualcuno che la amasse sul serio. Poi il coraggio le veniva meno, e allora le davano della
smidollata, sospettavano che in fondo le piacesse quella vita.

E dopo tanto tempo lo sospettava anche lei.
