\documentclass[a4paper,12pt]{book}

\usepackage[italian]{babel}
\usepackage[utf8]{inputenc}

\frenchspacing

\title{L'ultimo guerriero}
\author{Alessandro Desantis}
\date{}

\begin{document}

\pagenumbering{roman}

\maketitle

\begin{flushright}
A Méav, che mi ha insegnato a vedere la luce.
\end{flushright}

\chapter{Dolorosa quotidianità}

\pagenumbering{arabic}

\paragraph{}
«Non azzardarti mai più a ridicolizzarmi in pubblico, chiaro?! Non ti ho
sposato per questo!»

Benjamin chinò la testa sul suo disegno, ma non riusciva a concentrarsi con
quelle urla nelle orecchie. Aveva dimenticato cosa stesse ritraendo con tanto
impegno. Cercò di occupare la mente pensando a qualcosa che non fossero i suoi
genitori che si sbranavano giù in salone.

La camera era grande per un bambino di nove anni. Qualche poster era attaccato
alle pareti, ma a lui non piacevano sul serio: era stato il padre a metterli,
quando aveva visto quanto spoglia fosse la stanza del figlio.

«Tutti hanno un idolo, tu no?» aveva chiesto.

No, Benjamin non l’aveva.

Allora Richard aveva fatto sì che il suo idolo, un cantante che ascoltava
quando era ancora al liceo, divenisse quello di Ben.

Era una bella giornata e dalla finestra arrivava la luce tiepida e rassicurante
del sole. Seguì con lo sguardo un raggio che terminava la sua corsa sulla sua
mano chiusa nell’impugnare una matita azzurra.

La porta di casa si chiuse pesantemente.

\paragraph{}
«Mi dispiace tanto, figlia mia».

Seduta sul suo trono ricoperto di pelli di volpe Méav ascoltava la madre
parlare con la voce rotta dalla tristezza. Suo padre accanto alla moglie non
aveva il coraggio di dire nulla.

Era la terza gravidanza che non riusciva a portare a termine. La prima volta
aveva pensato che fosse un caso, la seconda sfortuna, ma ormai non poteva più
far finta di niente.

Nel villaggio s'era sparsa la voce che fosse maledetta e tutti, per quanto
fossero obbligati a mostrarle un certo rispetto -- dopotutto era la figlia del
capovillaggio, una principessa -- se ne tenevano alla larga il più possibile.

Desiderava così tanto un bambino. Voleva guidarlo, aiutarlo e consigliarlo...

Voleva vederlo crescere, vederlo diventare un uomo. Invecchiare mentre lui
invigoriva. Invece era destinata a rimanere sola. Sì, avrebbe sposato un uomo
bellissimo, ma cosa sarebbe rimasto della loro unione?

Quei pensieri la distruggevano. Durante il giorno Méav riusciva a scacciarli
dalla testa occupandosi di altre faccende. Ma quando dopo essere stati davanti
al fuoco fino a notte inoltrata ognuno si rintanava nella sua dimora lei,
sentendo i pianti dei neonati che si svegliavano all'improvviso, non riusciva a
prender sonno, e non perché quei rumori la infastidissero: erano come una dolce
melodia per le sue orecchie.

Una melodia che pensava non avrebbe mai ascoltato da vicino.

\paragraph{}
Stava ultimando il suo disegno.

«Benjamin, vieni, è pronto!» urlava Martha.

Dopo che lo ebbe ripetuto tre o quattro volte il bambino sentì i passi della
donna per le scale. La porta si aprì di scatto e una trentottenne dall'aria
stravolta si affacciò. Aveva gli occhi rossi, probabilmente per via della sua
allergia; sì, Benjamin sapeva che era allergica a qualcosa.

«Vieni a mangiare o no? La zuppa si fredda».

Lui scese le scale, raggiunse la cucina e si sedette a tavola. Non gli piaceva
pranzare con la madre: non faceva altro che riscaldare zuppe nel microonde, non
importava se fosse gennaio o agosto.

Quel giorno, Benjamin non prese nemmeno in mano il cucchiaio; restò a fissare
il piatto.

«Allora, non mangi?» chiese Martha.

«Non mi piace la zuppa» disse lui con voce flebile, aspettandosi un ceffone.

Invece non accadde niente. La madre lo guardò. I suoi occhi non tradivano
alcuna emozione.

«Non si può sempre avere quello che si vuole, sai?» si limitò a ribattere
avvicinandogli il piatto.

Allora Benjamin incominciò a mangiare. Se la madre si fosse arrabbiata avrebbe
potuto sbattere i piedi e frignare, se si fosse rassegnata gli avrebbe cucinato
qualcos'altro.

Ma quell'apatia lo spaventava più di ogni altra reazione.

\paragraph{}
Mentre il figlio pranzava, Martha decise di dare una sistemata alla casa. Era
già tirata a lustro, ma aveva un disperato bisogno di sentirsi utile.

In bagno vide il proprio riflesso nello specchio. Era ancora una bella donna. I
capelli rossi, un poco sbiaditi, le cadevano dolcemente sulle spalle. Era esile
e smunta, ma aveva l'aspetto di una grande sognatrice brutalmente gettata nella
realtà. Una volta era in cerca del principe azzurro per mettere su famiglia e
vivere felice per sempre. Non aveva mai smesso di desiderare quell'incontro; ora
però sapeva che il primo uomo di cui ci si innamora non è necessariamente
quello perfetto.

Quell’uomo era troppo furbo e vigliacco per toccarla, ma abbastanza sfrontato
da distruggerla psicologicamente, giorno dopo giorno, in modo da non lasciare
prove. Martha sapeva che Richard non era sempre stato così: un tempo era
anch'egli una persona sensibile e romantica, ma il tempo, il lavoro, e
soprattutto la nascita di Ben lo avevano cambiato.

La sua unica consolazione era quel figlio tanto desiderato, che tuttavia non si
godeva come avrebbe voluto: a volte le sembrava di trascurarlo, a volte di
arrabbiarsi per niente, altre ancora di non riprenderlo quando doveva. Si
sentiva sempre inadeguata. Solo in pochi momenti era se stessa e allora era al
settimo cielo: dimenticava tutti i problemi e i guai che le erano capitati.

Ma la cosa peggiore di quella situazione era la compassione altrui. Non
sopportava quei finti sorrisi pieni di amarezza e comprensione. Sua madre, sua
sorella, le sue amiche, tutti la facevano sentire un'idiota.

Reagisci, dicevano, combatti.

Cosa ne potevano sapere loro che non ci erano mai passate. Cosa ne potevano
sapere di tutte le notti passate a piangere, a pensare di prendere Ben, fuggire
e ricominciare una nuova vita accanto a qualcuno che la amasse sul serio. Poi il
coraggio le veniva meno, e allora le davano della smidollata, sospettavano che
in fondo le piacesse quella vita.

E dopo tanto tempo lo sospettava anche lei.

\chapter{Amare non basta}

\paragraph{}
«In Scozia?»

«Sì, un viaggio in Scozia! Hai sempre voluto farlo, ti lamentavi di non
esserci mai stato, no? Dicevi che volevi conoscere quel popolo antico... i
Celti!» disse Martha, terrorizzata di aver speso i suoi risparmi in due
biglietti aerei per la destinazione sbagliata.

Ma Benjamin dissolse le sue preoccupazioni: «Quando partiamo?»

La donna si chinò ad arruffargli i capelli, sorridendo sollevata.

«Domani mattina».

Martha aveva organizzato quella gita improvvisa perché voleva stare lontana da
Richard dopo il loro litigio. Lui aveva fatto le valigie ed era andato dal
fratello; non si sarebbe neanche accorto della loro assenza.

Non si accorgeva mai di niente.

\paragraph{}
In aereo Benjamin non la smetteva di fare domande. Voleva sapere dove sarebbero
stati, cosa avrebbero fatto.

Martha sapeva di essere finalmente riuscita a catturare la sua attenzione.

Gli spiegò che sarebbero andati in un campeggio. D'altronde, era l'unica
sistemazione che si era potuta permettere senza dover chiedere aiuto al marito.

«Ho parlato col proprietario e sembra saperne molto sui Celti, sai? Ha detto
che ne conosce alcuni!»

Il bambino pensò che stesse esagerando.

Poi fece una domanda che Martha avrebbe voluto evitare a tutti i costi.

«E papà dov'è?»

Tentò di ignorarlo ma Benjamin si dimostrò insistente.

«Ha avuto un problema al lavoro. Prenderà il prossimo aereo» rispose, e
subito si rese conto del suo errore.

Perché aveva mentito? Non lo sapeva neanche lei. Dopotutto il figlio si sarebbe
presto accorto che non era vero. Voleva confessare subito, ma quando vide il
lampo di gioia nei suoi occhi non ne ebbe il coraggio.

``In un altro momento''.
\paragraph{}
L’alloggio non era male; un po' rustico, ma almeno Benjamin aveva l'occasione
di stare insieme a persone che condividevano la sua stessa passione.

Martha non capiva cosa potesse piacergli così tanto in una manica di ubriaconi
vecchi di qualche millennio che andavano in giro a urlare e staccare teste con
le loro spade. Nonostante le ultime generazioni di Celti fossero state costrette
a modernizzarsi per sopravvivere, la loro indole guerriera era rimasta intatta.

Aveva appena finito di montare la tenda -- il che le aveva fatto rimpiangere di
aver intrapreso il viaggio -- quando una donna le posò una mano sulla spalla e
senza dire niente le indicò il panorama tutt'intorno.

Era davvero uno spettacolo impressionante: a parte i monti a ovest, per diversi
chilometri si stendeva un'enorme prateria. Lontano, molto lontano, il fumo di un
fuoco e qualche capanna.

«In questo posto veniamo per riposare e meditare» disse Celine, che Martha
aveva conosciuto in aereo. «Se penserai solo ai tuoi doveri di madre,»
aggiunse «te ne andrai più nervosa di prima».

«Può darsi» rispose lei con un sorriso.

\paragraph{}
Arrivò la notte e con essa uno stupendo cielo stellato, quasi del tutto libero
dall'inquinamento luminoso. Se si ascoltava attentamente si potevano udire anche
gli allegri canti che venivano dal villaggio.

«Gli uomini bevono come spugne, le donne cantano come dee» commentò Celine,
dietro Martha.

«Pare che tu li conosca bene».

Quella annuì. «Tutti noi che siamo qui abbiamo dedicato, chi solo una parte di
essa, chi la vita intera, allo studio della cultura celtica».

«E perché lo avete fatto?» chiese Martha incuriosita.

«Difficile dirlo. Forse perché ci affascinano quei canti,» e accennò con la
testa al villaggio «o forse perché non ne possiamo più della società moderna
e vogliamo allontanarcene. Sai qual è la differenza tra noi e loro, Martha? Sai
cosa ci distingue?»

Quella scosse la testa.

«Anche loro hanno gli ignoranti, gli ipocriti e gli stupidi. Quelli sono
ovunque. Anche loro hanno uomini politici che mettono i propri interessi davanti
a quelli del popolo. Anche loro hanno i bugiardi e gli intolleranti. Ma per loro
queste persone costituiscono delle eccezioni, e hanno quello che si meritano:
niente. Cosa insegniamo, invece, noi ai nostri figli? Che vince il più furbo,
il più malvagio, il più prepotente. Che il fine giustifica sempre i mezzi». E
si fermò, guardando oltre i monti. «Che devono avere paura delle proprie
emozioni, perché sono sbagliate».

\paragraph{}
«Allora Ben, sei pronto?»

In realtà era Martha a doversi ancora vestire quella mattina. Benjamin, per via
dell'eccitazione -- era il suo primo contatto con i Celti! -- non aveva
nemmeno dormito.

Avrebbero visitato un villaggio poco distante, i cui abitanti erano famosi per
essere perfino più abili dei loro antenati nell'arte canora.

Dopo mezz'ora si misero in marcia.

David, la guida, parlò loro delle origini di quel popolo. Ne ripercorse la
nascita, le avventure e le sventure, i successi e i fallimenti. Nella mente di
Benjamin si andavano pian piano disegnando immagini sempre più nitide e
fantastiche: uomini che combattevano, facendo roteare le spade scintillanti,
duelli per donne meravigliose, e così via.

La guida aveva preso a cuore madre e figlio, forse perché intuiva quale fosse
la loro situazione. Inoltre l'interesse del bambino per la storia era ammirevole
in qualcuno della sua età: era come se cercasse nei Celti ciò che gli mancava.

David sperava ardentemente che lo trovasse.

\paragraph{}
Gli abitanti del villaggio non aspettava la visita dei campeggiatori quella
mattina.

Non faceva comunque differenza, dato che anche quando venivano avvisati si
limitavano a portare avanti la loro vita. Ignorare la presenza degli estranei
era il regalo più grande che potessero fare a questi ultimi.

Sì, il patto era quello: esibire le proprie tradizioni, dimostrarsi una valida
risorsa culturale. In cambio i Celti avevano ottenuto di poter abitare le terre
in cui erano nati.

La vita per loro non era però così terribile: potevano mantenere la propria
identità, e a volte si stabilivano relazioni sincere con gli stranieri. Méav
stessa era una buona amica di David, che aveva dedicato la vita allo studio
delle loro usanze.

``Chi cerca trova'' dice un vecchio detto.

I Celti avevano trovato negli stranieri, e gli stranieri nei Celti, amici,
amanti, sposi e spose.

Méav cercava un figlio.

Chi cerca trova.

\paragraph{}
Martha cercava di seguire gli altri, tuttavia era difficile: Ben continuava a
fermarsi ogni dieci passi; voleva vedere e conoscere i più insignificanti
dettagli di ogni filo di paglia su cui metteva i piedi. Quando poi passarono
davanti a un gruppo di persone che cantavano il bambino ne fu totalmente
meravigliato. A Martha non piaceva particolarmente la musica, perché non
l'aveva potuta mai davvero apprezzare. Le casalinghe non possono permettersi
certi lussi.

Tuttavia la donna non poté fare a meno di fermarsi e ascoltare in
contemplazione quando sentì quelle voci così dolci e belle. Era come se i
canti la trasportassero lontano, in altri luoghi e tempi. Non riusciva a capirne
le parole perché erano in una qualche lingua antica, ma sapeva perfettamente di
quali argomenti trattavano.

Dopo qualche istante notò che Ben era attratto da una donna bionda al centro
del gruppo. Era la solista, e certamente la più brava. Senz'altro una donna
affascinante.

Anch'ella, ogni tanto, lanciava sguardi languidi al bambino. Sguardi che a
Martha non piacevano. Li trovava falsi e malvagi. O forse ne era semplicemente
gelosa, perché al figlio piaceva quella donna più di quanto non gli piacesse
sua madre.

«Vieni, Ben» disse, e lo trascinò lontano.

La bionda lanciò loro un'ultima occhiata.

\paragraph{}
Quella sera Martha era agitata sia per via della donna che avevano incontrato,
sia perché Benjamin continuava a chiederle del padre. Ormai doveva aver capito
che qualcosa non andava. Quando le ripeté la stessa domanda per la decima volta
in un'ora non poté più far finta di niente.

«Papà non verrà» gli disse, sospirando.

Quella frase era al tempo stesso una liberazione e una tortura.

«Ma avevi detto che..».

«Lo so cosa ho detto» tagliò corto.

«E allora perché non viene?»

Lo guardò a lungo prima di rispondere. Soppesava le parole.

«Non era vero che avrebbe preso il prossimo aereo. In realtà non sa nemmeno
che siamo qui».

Vide gli occhi di Ben spalancarsi mentre lo diceva.

«E perché non glielo hai detto?»

«Perché avevamo litigato. Le persone litigano e quando succede non si parlano
per un po'».

«Sì, lo so» e pronunciò quelle parole come se volesse in realtà dire:
``Avete discusso così tante volte da rendermi un esperto in materia''.
«Bene, ora ci siamo chiariti, vero?» chiese speranzosa Martha.

«Ma quando torno lo posso dire a papà?»

``Oddio, ti prego, no''.

«No che non puoi!»

«E perché?»

«Perché no e basta!» strillò infine, irritata.

Forse aveva esagerato.

Non riuscì a dire altro e del resto non ne ebbe bisogno, perché Benjamin si
fece silenzioso e non le rivolse la parola per tutta la sera.

\chapter{Le stesse stelle}

\paragraph{}
Méav non ricordava di essere mai stata così abbattuta. Le sembrava che intorno
fosse tutto rigoglioso, e lei fosse l'unica erbaccia arida da estirpare.

Quella sera avrebbe dovuto essere con i suoi amici e la sua famiglia, a cantare
e danzare alla luce e al calore del fuoco. Invece aveva deciso di
allontanarsene, perché non tollerava il modo in cui la trattavano, il modo in
cui evitavano di incrociare il suo sguardo.

Per loro era una povera disgraziata.

Con questi pensieri nella testa la trentenne passeggiava, lontana da tutto e
tutti. L'erba accarezzava le sue caviglie nude e una brezza muoveva il lungo
abito.

Osservava il bellissimo cielo scozzese, riconoscendo le costellazioni.

Evidentemente poteva solo continuare a sognare, e fantasticare su quel bambino
che era venuto a visitare il villaggio. Se ne vergognava, come se fosse una
perversa ossessione guardare ciò che le era stato negato.

Sotto la notte stellata passeggiava.

\paragraph{}
Era ormai passata una settimana dall'inizio del loro viaggio, ed era ora di
tornare a casa. Benjamin si faceva sempre più silenzioso e distante. Martha
capiva la sua sofferenza, ma cos'altro poteva fare?

Si girava e rigirava nel sacco a pelo senza riuscire a prendere sonno.

Era scomparso il ricordo di quanto di bello era accaduto in quei giorni. Le
risate, i giochi e la complicità col figlio non c'erano mai stati. Rimaneva
solo un grande senso di vuoto, e quella familiare sensazione di inadeguatezza.

Sarebbe tornata tra le braccia di Richard che l'avrebbe fatta sentire
un'imbecille. Sarebbero ricominciati i piccoli e insignificanti problemi: le
bollette da pagare, il rubinetto che perde...

Piangeva.

Le tornò in mente anche il suo odio per quella donna che ammiccava al figlio,
al villaggio. Si sentì una stupida: cosa aveva fatto di male se non dimostrare
affetto a un bambino? Eppure quel gesto le era sembrato così fuori luogo...

Era lei a essere fuori luogo. Era lei quella incapace di amare, incapace di
godersi i piccoli momenti.

``Sono io a vivere per un rubinetto che perde''.
\paragraph{}
Benjamin sgattaiolò fuori dalla tenda, in silenzio. Prima però, diede
un'ultima occhiata alla madre; non era sicuro di ciò che stava facendo, e non
sapeva come sarebbe andata a finire. Aveva una sola certezza: di non voler
tornare a casa.

Il cuore gli batteva così forte che temeva qualcuno lo potesse sentire. Nel
buio totale attraversò il campo addormentato. Anch'egli aveva sonno. Pensò di
lasciar perdere quella follia, tornare nel sacco a pelo, al caldo, abbracciare
Martha e sentirsi protetto da ogni male.

Cosa ci spinge ad allontanarci dai nostri cari, le persone che più al mondo ci
amano? È il desiderio di scoprire nuovi orizzonti e vedere se possiamo trovare
qualcuno che ci faccia star meglio? O lo facciamo perché vogliamo mettere alla
prova il loro amore?

Sicuramente il bambino non si poneva quelle domande. Tutto ciò che voleva era
restare con i Celti, con la bella signora che cantava. Avrebbe voluto portare
anche Martha ma non era possibile: a Benjamin pareva che la madre non
appartenesse nemmeno a quel mondo meraviglioso.

\paragraph{}
Il bambino camminava con il naso per aria, pervaso da felicità mista a un certo
timore. Non lo preoccupava cosa sarebbe successo se avesse trovato la donna che
cercava, ma ciò che sarebbe successo se non l'avesse trovata. Avrebbe saputo
riconoscere la strada per tornare indietro? E cosa avrebbe detto Martha? Si
sarebbe arrabbiata?

I piedi sembravano dotati di volontà propria. Intorno a sé vedeva solo la
notte e le stelle lontanissime. La luna invece era molto vicina, tanto che
credeva di poterla toccare senza doversi nemmeno alzare sulle punte. Si era
sempre chiesto perché il satellite lo seguisse in continuazione; il padre aveva
tentato di spiegarglielo, ma egli si era limitato ad annuire poco convinto per
paura di deluderlo.

Gli parve all'improvviso che l'aria si facesse molto più fredda. Era stanco e
stava per cadere addormentato, ma quel gelo lo obbligò a spalancare gli occhi.

Poi rapida com'era venuta la ventata passò.

\emph{Tonf!}

Finì a sbattere contro qualcosa, e si trovò con il viso immerso in un cumulo
di seta. Indietreggiò e si rese conto che non si trattava di qualcosa, ma di
qualcuno. A due metri da lui, il suo angelo lo guardava con quei profondi occhi
azzurri. Ben non aveva previsto di incontrarla così presto.

La donna sembrava piuttosto sorpresa di vederlo. Non lo disse esplicitamente ma
il bambino sapeva di essere stato riconosciuto. Lo aveva capito dal suo sguardo.

«Che ci fai qui?» chiese.

Mentì. Disse che si era perso. Che non riusciva a trovare la strada per tornare
dalla madre.

Capì immediatamente che non era stata una mossa molto astuta: l'angelo avrebbe
potuto portarlo subito indietro. Di certo lei sapeva come arrivare al campo. Ma
forse il destino era dalla sua parte.

«Allora ascolta: è tardi per ritrovare la strada. Ora andiamo al villaggio,
dove potrai riposare. Domattina ti riporterò al campo. D'accordo?» propose
sorridendo.

Annuì energicamente per timore di lasciarsi sfuggire l'occasione.

I due si incamminarono, mano nella mano.

\paragraph{}
Il villaggio era piuttosto distante, così che Benjamin ebbe il tempo di porre
alla donna molte domande. La prima era senz'altro la più semplice importante.

«Come ti chiami?»

«Méav» rispose. «E tu?»

«Benjamin. Che nome strano Méav» commentò il bambino.

Lei ne rise.

«Per le mie genti è il \emph{tuo} nome a essere strano».

Andò avanti così per tutto il tempo, ma lei non diede mai segno di essere
seccata. Ben pensò che Martha si sarebbe stufata già da molto e avrebbe
iniziato a dare risposte brevi e secche; lo faceva sempre quando parlava troppo.

\paragraph{}
Mezz'ora -- e almeno una ventina di domande -- più tardi, giunsero
all'agognata destinazione. L'odore di paglia e fumo impregnava l'aria, ma senza
dare fastidio.

Méav finalmente lasciò la mano del ragazzo. Egli vedeva nell'oscurità solo i
contorni delle tende. Il silenzio era assordante.

Benjamin seguì la donna finché non furono vicino a un fuoco ormai prossimo a
spegnersi. Era quello intorno al quale erano riuniti, fino a poco prima, amici e
parenti della principessa per discutere della sua tremenda situazione.

Il bambino le chiedeva solo di tradizioni che non aveva mai seguito e antiche
battaglie che non aveva mai vissuto. Non volendo deluderlo ella inventava. Gli
raccontò le storie che aveva sentito dal nonno quando era ancora una bambina, e
capì cosa trovasse di tanto bello nel raccontarle: la soddisfazione di avere
qualcuno che ascolti ciò che si ha da dire, senza interessi né fini nascosti,
qualcuno che possieda l'innocenza di cui solo i bambini sono dotati.

Per la prima volta dopo molto, troppo tempo Méav si sentiva libera e completa.

Le sembrava che l'Universo intero si concentrasse per carpire ogni dettaglio
della sua avventura.

Poco dopo Benjamin dormiva, sdraiato per metà per terra e per metà su una
scomoda panca di legno.

Passò una notte dolce e senza sogni.

\paragraph{}
Méav invece non riusciva a prendere sonno: troppi pensieri le affollavano la
mente e le impedivano di dormire. Aveva portato il bambino nella tenda e si era
coricata accanto a lui, carezzandolo con dolcezza come aveva desiderato fare fin
dal primo momento che lo aveva visto.

``Diventerà un bell'uomo'' pensò. Una ciocca dei folti capelli castani gli
copriva la fronte pallida e lei la scostò sorridendo. Suo padre avrebbe detto
che era esile e debole ma si sbagliava: aveva un animo forte. Lei lo sapeva.

Uscì dalla tenda perché voleva che il freddo della notte la tenesse sveglia,
in modo da potersi godere ogni secondo che le restava con Benjamin.

Guardando la luna, pensò per un attimo di tenerlo solo per sé. Avrebbe potuto
dirgli che la madre ormai non c'era più perché, non trovandolo, era ripartita.

Non sarebbe stato così difficile abituarsi alla nuova vita.

Si rese subito conto di quanto fosse ridicola e crudele quell'idea e si
vergognò di se stessa.

E forse per la vergogna, oppure per l'emozione, pianse a lungo. Non era stato
infrequente in quei mesi che le lacrime rigassero il suo volto. Ma quello era un
pianto diverso, che la svuotava di ogni sensazione negativa e lasciava solo una
certa gioia, e con essa l'ispirazione. Era un pianto d'amore per gli amici e i
nemici, e per coloro che non aveva mai incontrato.

Era un pianto di speranza.

\chapter{Diciassette secoli}

\paragraph{}
Il sole non era ancora sorto quando Méav svegliò il bambino.

«Benjamin, dobbiamo andare» disse scuotendolo con delicatezza.

Quello si mosse leggermente. Allora la principessa lo scosse con più forza, e
stavolta Ben si alzò e si stropicciò gli occhi. Lo prese per mano e lo
condusse rapidamente fuori dal villaggio. Non voleva che qualche suo mattiniero
conoscente li vedesse e lei si trovasse costretta a dare spiegazioni.

Stavolta Benjamin fu silenzioso per tutto il viaggio. Méav pensò che fosse per
via del sonno; aveva l'aria di non sapere neanche perché si trovasse lì invece
che con la madre.

A metà della strada Ben la strattonò e sussurrò qualcosa. La donna si chinò.

«Come hai detto?» chiese.

«Non voglio tornare da mia madre. Voglio stare con te».

Lei non sapeva cosa dire. Come aveva potuto non capirlo? Benjamin non si era mai
perso, l'aveva cercata! Trovava la cosa tristemente dolce. Si sentì sciocca per
non averlo intuito prima.

Cercò di farlo ragionare.

«Non puoi... non si può fare!»

«Lo so» rispose quello, costernato. «Ma non puoi fare finta di niente,
stavolta? Fare finta di essere mia mamma?»

La principessa sorrise.

«Non sarebbe giusto nei confronti della tua \emph{vera} mamma» cercò di
spiegargli.

«Pensa a come sarebbe triste!»

«Ma così sono triste io!»

La cosa assurda di quella conversazione era che la donna iniziava a pensare che
il bambino avesse ragione. Chi era la madre per privarlo di quella felicità che
egli cercava tanto ardentemente?

Fece un ultimo, disperato tentativo.

«Ascoltami. Se deciderai di rimanere con me la tua vita cambierà per sempre:
diventerai un guerriero».

Quelle parole, che volevano suonare come terribili presagi, erano per Benjamin
l'immagine del futuro che aveva sempre desiderato. I suoi occhi si riempirono di
lacrime.

«Perché non mi vuoi portare con te?» domandò sconsolato.

Era troppo per Méav. Si morse il labbro inferiore, indecisa sul da farsi,
nervosa come mai prima d'allora. Guardò il campo, poi il ragazzo. Sospirò.

«D'accordo».

La abbracciò, e lei ne fu quasi imbarazzata.

Per l'ultima volta percorsero quel tragitto che dal campo portava al villaggio.

Il campo dove dopo qualche ora Martha si sarebbe svegliata.

\paragraph{}
«Tu hai fatto cosa?!»

Solamh era scioccato. Non neanche cosa dire, tanto era allucinato. Non gli erano
mai molto piaciuti i contatti con le altre generazioni, ma ciò che aveva fatto
sua figlia esulava da ogni ragionevole limite. Pensava agli effetti che avrebbe
avuto su Méav, su Benjamin, sulla loro famiglia e infine sull'intero villaggio.

Il novenne si teneva in disparte, strusciava i piedi a terra e teneva lo sguardo
basso. Quell'uomo baffuto gli incuteva un certo timore: sembrava che potesse
esplodere da un momento all'altro.

Méav neanche sapeva cosa dire perché si rendeva conto dell'assurdità di
quella situazione. Così padre e figlia si scrutavano, lei lo sguardo fuggente,
lui fisso, gli occhi sbarrati.

Fu sua moglie a interrompere il silenzio.

«Se lei vuole bene al bambino, e il bambino a lei, non vedo quale sia il
problema».

Parlavano in Gaelico, perciò Benjamin non capiva una parola. Ciò nonostante
sapeva che c'era in gioco il suo futuro.

«Éibhleann, sei impazzita anche tu dunque? Ti rendi conto che ha rapito un
bambino?» disse infuriato e guardò Benjamin, che arrossì.

«Non l'ho rapito. Lui ha voluto venire con me» si difese Méav.

«Esatto. E se dovesse cambiare idea c'è sempre tempo, no?»

«E invece no, non ce n'è! Se non lo portiamo indietro ora, capiranno che non
è arrivato qui per caso! Le autorità non vedono l'ora di esiliarci: aspettano
solo un'occasione come questa!»

Éibhleann gli posò le mani sulle spalle.

«Tu permetteresti che un tuo abitante -- tua figlia! -- sia triste per paura
di ciò che potrebbe succedere se non lo fosse? È questo l'uomo che ho sposato?

Sei così codardo?!»

Solamh si rassegnò. Capì che non avrebbe potuto fare molto per allontanare
Benjamin da Méav.

«D'accordo» disse, «ma il bambino sarà libero di andarsene quando vorrà, e
lo consegneremo se qualcuno verrà a cercarlo».

Méav sospirò di sollievo. La madre la guardò. Anche se la stava appoggiando,
non era completamente d'accordo con ciò che aveva fatto.

«Va bene».

\paragraph{}
«Ben, non è divertente. Vieni fuori!»

Martha era uscita dalla tenda, e ora si stava aggirando per il campo. Chiese a
Celine, appena sveglia e già pronta per partire, a una coppia, moglie e marito
che avevano entrambi gli occhi gonfi dal sonno, a un giovane ragazzo forse in
visita con la sua scuola, ma nessuno lo aveva visto.

Le sembrava di trovarsi in un film dell'orrore di quarta categoria. Non le erano
mai piaciute le scene in cui qualcuno scompariva, e ancora meno ora che quel
qualcuno era suo figlio.

Vagabondava da ormai venti minuti quando David la raggiunse, probabilmente
avvisato da qualcuno. Con la solita calma le chiese cosa non andasse.

«Non trovo più mio figlio, ecco cosa!» sbraitò.

Si rese conto di essere stata troppo dura con qualcuno che cercava solamente di
aiutarla, ma non c'era tempo per le scuse.

Sentiva i battiti del cuore e il mal di testa aumentare ogni secondo che passava
senza vedere Benjamin. In preda al panico iniziò a correre in lungo e in largo,
con David che la seguiva a fatica. La fermò, col fiatone.

«Calma, Martha. Adesso lo cerchiamo fuori dal campo e se non riusciamo a
trovarlo avvisiamo la polizia».

Ma lei era irritata da quel tono pacato e tranquillo con cui affrontava
qualunque situazione. Come poteva non rendersene conto? Si trattava di suo
figlio!

La polizia, poi! Quando David lo disse a Martha si seccò la gola. Cosa
avrebbero detto di lei? E peggio ancora: cosa avrebbe pensato Richard se non
fosse riuscita più a trovare loro figlio?

Ormai terrorizzata sradicò da terra la sua stessa tenda che volò, sospinta dal
vento, per qualche metro prima di fermarsi in mezzo all'erba.

Benjamin non c'era.

Si levarono mormorii di disapprovazione e si accorse che tutti la guardavano.

Benjamin, dov'era Benjamin?!


\paragraph{}
«Mi dispiace... Eri l'unica persona di cui mi fidassi».

«Lo capisco, ma io sono quasi sempre stato accanto alla madre in questi due
giorni: è distrutta. Non può più andare avanti così».

«Ha già avvisato la polizia?»

«No... Assurdo, vero? Voleva che prima venissi a parlare con voi, perché pensa
che potreste aiutarla meglio delle autorità. Ora che so la verità, però..».

«Non penserai di dirglielo, vero?»

«È ovvio che devo dirglielo! Non puoi tenerlo tu, non è giusto! Tu non hai
sentito Martha piangere e disperarsi a notte fonda! Lo sai che ha avvisato suo
marito, e lui sta pensando di chiedere il divorzio?»

«Non meritava suo figlio. Lo sai anche tu».

«Può darsi. Chi siamo noi per giudicare, però?»

«Io voglio stare con lei..».

«Benjamin, non è una tua decisione: sei troppo, davvero troppo piccolo per una
simile cosa. Pensi che sia una bella avventura, questa, ma tra poco tempo --
meno di quanto immagini -- te ne pentirai, e allora sarà troppo tardi».

«Invece è lui che deve decidere. Si tratta della sua vita, non di quella della
madre».

«Piantala! Sei solo un'egoista! Che fine ha fatto la donna meravigliosa e
sensibile che conoscevo? Colei a cui avrei affidato la mia vita? Cosa ne è
stato della persona che amavo?»

«Io sono felice, lui è felice. È tutto quello che conta».

«Per te, forse. Non per noialtri. Ma perché sto ancora discutendo? So cos'è
giusto fare».

«Aspetta un attimo! Non vorrai dirglielo, vero? Aspetta, ti prego!»

Il sole era alto nel cielo. La figura dell'uomo in cammino si stagliava
all'orizzonte.

\paragraph{}
Martha non si aspettava una tale modernità in un ufficio di polizia.

Soprattutto non si aspettava una tale cordialità: almeno dieci agenti cercavano
di metterla a proprio agio da quando era entrata. Fallendo miseramente. Nulla
può calmare una madre il cui figlio è scomparso.

Al suo fianco stava David, che aveva maggiore dimestichezza con la giustizia
scozzese.

L'unica persona sgradevole lì era l'impiegata che stava prendendo i suoi dati.

Una donna bionda, grassoccia, che doveva avere poco meno di cinquant'anni e
l'aria di una persona che una volta era bella e buona ma è stata inacidita dal
tempo. Continuava a porle domande con un tono di arrogante superiorità,
sforzandosi di mostarsi visibilmente infastidita quando Martha chiedeva
spiegazioni.

«Signora, perché non ci ha avvisati subito? Quarantotto ore sono
un'eternità!»

Iniziò a balbettare. Cosa avrebbe dovuto dire? Si aspettava un aiuto di David,
ma lui sembrava concentrato su un filo che spuntava dai pantaloni. Lo odiò in
quell'unico momento.

«Allora, signora?» si innervosì quella. «Nulla da dire?»

Fu invece proprio uno dei colleghi dell'arpia a salvare Martha.

«Suvvia, Ailis, non vedi com'è tesa? Non peggiorare le cose».

Doveva essere un superiore, perché Ailis cambiò subito tono.

«Certamente... Io cercavo solo di capire cosa fosse successo».

«Suo figlio è scomparso. Concentriamoci su questo».

La poliziotta annuì e appena quello si voltò roteò gli occhi con aria
seccata.

Dopo un altro quarto d'ora di domande incessanti Martha fu liquidata da un «le
faremo sapere», che le ricordava tanto un colloquio di lavoro andato male.

Uscì di lì ancora più affranta.

Si ricordò all'improvviso della presenza di David. Era stato come un fantasma
in quelle ore.

«Tu sei sicuro di non sapere dove possa essere finito mio figlio?»

«Sì».

«Al villaggio non ti hanno detto niente? Quando l'abbiamo visitato c'era una
donna che cantava e Benjamin..».

«No».

\chapter{La vita è una ruota}

\paragraph{}
Con stupore di tutti Benjamin si abituò molto velocemente alla sua nuova vita.

Méav faceva di tutto per viziarlo per quanto le fosse possibile, ma anche se
non si fosse impegnata non sarebbe cambiato niente: quel ragazzo era ormai un
celta. Dormiva su letti di tela e paglia, stava imparando le tradizioni di quel
popolo e si allenava nell'arte della spada che non avrebbe mai avuto occasione
di mettere in pratica.

Nonostante vivesse lontano dalla società dell'educazione, prendeva lezioni
dagli abitanti del villaggio, soprattutto riguardanti l'astronomia, la geografia
e le lingue (stava lentamente imparando il Gaelico, e continuava a esercitarsi
con l'Inglese, perché così aveva voluto Méav).

In poco meno di sei mesi aveva già assimilato tutto quello che c'era da sapere
sulla sua nuova famiglia. La parte che più gli piaceva rimanevano i canti e in
questo Méav eccelleva. Così passava le giornate ad ascoltare e pregare di
poter ascoltare, e ogni sera si addormentava al suono della voce di sua madre.

La sua nuova madre.

\paragraph{}
Méav aveva finalmente raggiunto quella felicità che tanto desiderava. Aveva un
figlio. Quando la notte si svegliava, a volte, si assicurava che Benjamin fosse
ancora lì accanto a lei: le sembrava di poterlo perdere da un momento
all'altro.

Il villaggio era diviso riguardo ciò che le era capitato. La maggioranza
pensava che la donna avesse agito per il meglio, e che dovesse essere il bambino
a scegliere la propria famiglia. Questo rispecchiava la filosofia del suo
popolo: ognuno era libero di disporre della propria vita, e non era compito
degli uomini indirizzarlo verso una scelta.

Altri invece vedevano Méav come una volgare criminale e chiedevano che fosse
esiliata. Tenevano se stessi -- e i propri figli, ovviamente -- il più
lontano possibile da lei e dalla sua famiglia. Il padre della donna,
capo-villaggio, la poteva proteggere grazie alla sua autorità, ma non sarebbe
stato sempre lì e ciò lo preoccupava. Si chiedeva cosa sarebbe successo a sua
figlia e a quello che si era ormai rassegnato a considerare suo nipote.

L'unica a cui non interessasse nulla di tutto questo era, paradossalmente,
Méav. Ella non si curava delle opinioni altrui e si preoccupava unicamente di
far sì che a Ben non mancasse nulla.

Voleva solo renderlo felice, e per riflesso esserlo anche lei.

\paragraph{}
{\itshape
Incontrava Benjamin durante una passeggiata serale. Era vestita proprio come
quella sera, ma c'era qualcosa di diverso nell'aria.

«Scusi» le diceva il bambino, che le era venuto addosso e subito si voltava
per andarsene.

«Non preoccuparti» rispondeva lei sorridente, cercando di apparire il più
possibile buona e inoffensiva. «Aspetta!» lo fermava poi prima che scappasse.

«Vuoi restare un attimo con me?»

Quello però non poteva: doveva tornare dalla madre, e correva verso Martha, che
osservava la scena poco distante, con aria grave.

«Aspetta!» ripeteva Méav e tentava di fermarlo, ma senza riuscirci.

Benjamin raggiungeva Martha che lo accarezzava dolcemente.

«Non è tuo figlio».
\/}

Quel sogno, che si ripeteva quasi identico da ormai una settimana, turbava
profondamente Méav. Era forse dovuto al fatto che Ben aveva incominciato a
riferirsi a lei come ``mamma''. Ella aveva insistito per essere chiamata col
proprio nome ma quello faceva finta di non sentire.

Si rese conto che non sarebbe mai riuscita a godersi appieno la sua nuova vita:
c'era sempre quella fastidiosa sensazione, più lieve di un senso di colpa e
meno dell'indifferenza dalla quale aveva sperato di essere pervasa. Il tempo
sembrava peggiorare le cose invece di migliorarle.

``Non è mio figlio''.

\paragraph{}
Erano ormai passati due anni dalla scomparsa di Benjamin e da quando Martha lo
aveva detto a Richard. Quest'ultimo aveva attraversato una fase d'ira, fatta di
minacce e cattiverie nei confronti della moglie e poi si era calmato, ma la
donna avrebbe preferito che ciò non fosse successo.

Ormai non si parlavano più. Anzi, si parlavano, ma era come se non si
parlassero. Compra l'acqua, è finita la frutta, fai il pieno alla macchina. La
loro vita coniugale si limitava a questo. Quando c'era Benjamin almeno avevano
ancora la forza di litigare, tenevano ancora l'uno all'altra. Ora vivevano sotto
lo stesso tetto soltanto perché un divorzio sarebbe stato troppo faticoso e
costoso.

E quella sera Richard aveva deciso di rompere il silenzio. Mentre mangiavano un
piatto di pasta scondita aveva bofonchiato, tra una forchettata e l'altra: «Mi
vedo con una donna». Nonostante la loro disastrosa situazione l'avesse portata
a immaginarlo, Martha rimase ferita, ma cercò di non darlo a vedere. Che senso
aveva esprimere emozioni se nessuno poteva accoglierle? Così aveva fatto finta
di non aver sentito.

Richard non si azzardò a ripetere la frase.

\paragraph{}
Pochi mesi più tardi Richard era andato a vivere con la sua nuova compagna. Non
c'era stato bisogno di dire niente: quando aveva smesso di tornare a casa la
sera, era stato tutto chiaro.

Lei invece aveva deciso di trasferirsi in Scozia. Non si illudeva di trovare
Benjamin né -- giammai! -- una nuova vita, ma sentiva che quella era la cosa
giusta da fare. Si sarebbe sentita più vicina a suo figlio, l'avrebbe aiutata
ad andare avanti. Così aveva chiesto alla madre i soldi per il biglietto aereo
ed era partita senza indugiare oltre.

Durante il viaggio aveva pensato a lungo di andare a trovare David, ma alla fine
si era convinta che non sarebbe stata una buona idea. Dopotutto di cosa potevano
parlare? Dei ``bei vecchi tempi''?

Per qualche tempo aveva alloggiato a casa di una compagna di liceo, poi aveva
trovato un lavoro come cameriera in un bar e aveva deciso di trasferirsi in un
microscopico appartamento poco lontano dal centro. Le pareti erano scrostate e
ammuffite e il posto puzzava di birra, ma Martha voleva restare lontana da
qualunque essere umano.

Ora nella sua nuova casa, durante l'unico giorno libero, guardava le goccie di
pioggia cadere obliquamente sul vetro della finestra e pensò che la sua vita
negli ultimi anni era scorsa con la stessa velocità.

\chapter{Chiedere scusa}

\paragraph{}
Tredici anni erano passati da quando Méav aveva preso Benjamin con sé. Così
tanto che ormai lui aveva quasi dimenticato che la donna non era sua madre. Non
la madre biologica almeno, perché ella non gli aveva mai fatto mancare nulla.

D'altronde egli non bisognava più di nulla. Aveva ventuno anni ed era diventato
più alto -- e molto più forte -- di Méav. Eccelleva tra i Celti per
l'abilità nei lavori manuali, la fedeltà alle tradizioni e l'integrità
morale. Era divenuto tutto quello che sua madre, entrambe le sue madri speravano
di vederlo diventare. Parlava perfettamente il Gaelico e terribilmente
l'Inglese. Tutti col tempo lo avevano accettato come proprio pari.

Méav era contenta di vederlo così ben integrato, ma a volte le pareva che
nella sua ostinazione volesse liberarsi del passato, e ci stava senz'altro
riuscendo. Non era contenta di ciò: fare in modo che Benjamin mantenesse i
contatti con la sua precedente vita era l'unico modo che le rimaneva di
liberarsi dei sensi di colpa.

L'unico modo per chiedere scusa.

\paragraph{}
La Malattia devastava il villaggio da ormai un anno. Credevano di poterla
controllare ma non era così. I sintomi erano quelli di un comune raffreddore
che portava lentamente alla morte per febbre, o più raramente per asfissia.

Alcuni erano andati in ospedale e lì erano riusciti a curarli. Altri, troppo
orgogliosi, non ce l'avevano fatta.

Méav per qualche motivo era rimasta immune. Si era però ammalato Benjamin, e i
sintomi peggioravano ogni settimana. La notte tossiva così forte da svegliare
chi gli stava vicino, e si lamentava per il dolore a ogni respiro. Per la donna
era una sofferenza vederlo ridotto in quello stato. Così un giorno gli propose
di andare in ospedale, anche se già sapeva quale sarebbe stata la sua reazione.

«Non se ne parla neanche» rispose seccamente Ben.

«Ma perché? Loro potrebbero aiutarti!»

«Non voglio avere nulla a che fare con quella gente».

«Lo capisco, Benjamin, davvero, ma non puoi fare di tutta l'erba un fascio.

Solo perché hai avuto un'esperienza negativa con i tuoi genitori..».

«Tu non capisci» la interruppe. «Io non ho avuto un'esperienza negativa con i
miei genitori, ma con il mio mondo. Non era solo il rapporto tra Martha e
Richard a non piacermi, ma il rapporto tra ognuno e il suo prossimo: sono tutti
così disonesti, e malvagi, e ipocriti».

«Per questo non ti sto chiedendo di tornare a vivere tra loro. Andiamo solo a
farti visitare, poi torneremo al villaggio e continueremo la nostra vita».

«Ho detto di no. Non intendo discuterne oltre. Piuttosto che avere a che fare
di nuovo con quelle bestie preferirei morire».

La discussione finì lì. Cos'altro poteva dire? Non c'era modo di convincere
suo figlio.

\paragraph{}
Una notte però Méav non poté sopportare la vista di Ben che nel sonno
delirava e tossiva, macchiandosi di sangue. Non era terrorizzata all'idea di
perderlo -- da molto tempo non lo guardava più come quel bambino spaventato
che l'aveva cercata per vivere con i Celti -- ma all'idea che egli stesso
perdesse la vita.

Quattro uomini -- tra cui il padre di Méav -- si offrirono volontari per
portare fino in ospedale quello che ormai era un amico. Lo caricarono su una
rudimentale barella, costruita tempo addietro e si incamminarono. Tutti
guardarono il triste corteo uscire dal villaggio.

Méav li seguiva e dai suoi occhi sgorgavano copiose lacrime. Per strada non
fece che ripensare a com'era stata bizzarra la sua vita: gli dei le avevano
donato un figlio quando tutto sembrava perduto, e ora che tutto andava per il
meglio gli dei volevano toglierglielo.

Fin da piccola le era stato insegnato che i disegni divini sono complessi e
indecifrabili, ma sempre a favore dell'uomo. Eppure lei non riusciva a trovare
proprio nulla -- neanche pensandoci fino a farsi venire il mal di testa -- che
fosse a vantaggio suo o di Benjamin, in ciò che stava accadendo.

Nulla.

\paragraph{}
La donna all'accettazione sbadigliò pesantemente, permettendo ai pazienti di
ammirare la sua ugola.

``Eccone un altro'' pensò guardando l'uomo disteso sulla lettiga. Non aveva
una bella cera: era ormai quasi trasparente e aveva sudato tanto che ci si
poteva specchiare nella sua fronte.

«Quarta porta a sinistra, l'ultima in fondo al corridoio».

Quelli la guardarono per un po', con gli occhi sbarrati. Quando stava per
indicare nuovamente la strada a gesti -- forse non parlavano l'Inglese? -- il
gruppo si avviò verso la sala d'attesa.

I pazienti osservarono la scena, alcuni per nulla sorpresi, altri esterrefatti.

\paragraph{}
Benjamin tentò di capire dove si trovasse, ma non riusciva a distinguere
chiaramente le forme. Gli sembrava di vedere Méav, la riconosceva dal colore
dei capelli, così puri e sfavillanti nonostante l'età. Altre figure si
stagliavano intorno a lui. C'era un pesante odore nell'aria, che era
insopportabilmente calda e stantia; sembrava difficile afferrare le poche
molecole d'ossigeno.

«Dove sono?» sussurrò, e non era certo di essere stato udito perché non gli
giunse risposta per interi minuti, o forse erano secondi.

«Stavi male, molto male,» diceva la madre con la sua dolce voce, ora coperta
da un velo di tristezza «così non ho avuto scelta. Ho dovuto farlo».

«Fare cosa?» «Oh, no..». aggiunse, quando i suoi sensi si acuirono ed egli
si rese conto che quell'odore nell'aria era disinfettante. «Non dovevi, non
dovevi..».

Sentì Méav singhiozzare.

«Mi dispiace, ma non potevo lasciarti morire. Sei mio figlio. Non potevo..».

«Non più! Ora che siamo tornati qui non sono più tuo figlio».

La donna singhiozzava in silenzio. Vide qualcuno avvicinarsi per abbracciarla.

Chiuse gli occhi e cercò di dormire, ma il pianto della principessa continuava
a tormentarlo.

\paragraph{}
Quella sera, Méav apprese che sarebbero dovuti rimanere in ospedale almeno per
un mese: la malattia -- per la quale, scoprì, era distribuito un vaccino da
tempo -- aveva devastato il corpo di Benjamin, che doveva recuperare forze.

Chiese agli altri di tornare a casa: non c'era motivo perché restassero. Suo
padre non voleva ma lei insistette.

Uscì fuori, nel cortile dell'ospedale, e per un'ora non fece altro che guardare
il cielo pensando a come avrebbe potuto dirlo a suo figlio.

\paragraph{}
«Un mese... un lunghissimo mese».

Benjamin si alzò dal letto per andarsene, facendo oscillare pericolosamente la
flebo. Méav si mosse per fermarlo.

«Non sforzarti. Ti prego, dammi ascolto una sola volta. Non l’hai fatto
tredici anni fa, fallo ora!»

Quello non disse niente: sospirò e si sdraiò nuovamente, in un gesto che
sembrava una rinuncia a combattere.

\chapter{Due madri}

\paragraph{}
Le dieci. Le dieci e mezza. Un battito di ciglia. Mezzogiorno. Sì, signora, è
quasi il suo turno. Che numero ha? Ottantanove? Siamo al sessantatré. Un
attimo, guardi quanta gente c’è. Aria calda, aria pesante, vapori di alcol.

Quella fastidiosa, pressante premura di una mamma troppo protettiva.

Bip. Ottantanove. Finalmente!

Si aspettava tanto per un pizzico. Un pizzico ed era tutto finito, non si
rischiava più quella brutta influenza. Martha ancora non si era vaccinata e non
sapeva bene perché. In verità non le interessava più di tanto cosa
succedesse: anche se non lo avrebbe mai ammesso, vivere o morire non faceva
troppa differenza. Si vive per un principio, un obiettivo o una persona. Quali
ideali aveva lei? La famiglia? Neanche a pensarci. L’obiettivo? Rendere il
mondo un posto migliore? Dopo quello che aveva fatto? Ma per carità! E la
persona? Quella una volta l’aveva... Ora non più.

Un’amica però l’aveva convinta ad andare in ospedale. «Quell’influenza
è davvero brutta» aveva detto. Non più brutta di qualunque cosa fosse
accaduta a Martha.

Si avviò verso l’uscita, guardando distrattamente i malati nelle camere. Loro
certamente avevano un motivo per continuare a lottare. I volti fuggivano via
veloci, alcuni giravano lentamente il collo nella sua direzione, doloranti, e
tentavano di catturarne lo sguardo, ma lei era già lontana.

«Mamma?»

Era quasi un sussurro ma a Martha sembrò urlato con quanto fiato un essere
umano ha nei polmoni. Le pareva che il cuore volesse liberarsi da quella gabbia
che è la gabbia toracica, per giungere più velocemente di lei verso
l’origine di quel suono. Era certa che non fosse niente: non poteva essere,
ormai aveva perso la sua occasione; eppure non avrebbe potuto dormire quella
notte se non avesse prima controllato.

Quando entrò nella stanza lasciò cadere il braccio vaccinato che teneva
piegato, cosa che le provocò un improvviso quanto impercettibile dolore.

I lineamenti del volto, i capelli, ma soprattutto gli occhi! Gli occhi erano
l’unica cosa che sembrava non essere invecchiata in tutti quegli anni.

«Benjamin?»

La donna credette di morire. Il suo fu davvero un urlo, e fece accorrere diversi
infermieri. Intimarono a Martha di abbassare la voce, ma ormai non c’era più
bisogno, perché era passata dalle urla al pianto. In ginocchio accanto al letto
continuava a toccare e baciare il figlio, come per accertarsi che fosse ancora
lì.

Benjamin piangeva tredici lunghi anni di lacrime.

E Méav, anch’ella piangeva, e non osava muoversi né parlare.

In quel momento il pianto di Martha era l’unico a essere certamente un pianto
di gioia: per gli altri due si trattava di un misto di gioia, rabbia e stupore.

Qualunque cosa fosse però era terribilmente liberatoria.

\paragraph{}
Tutti quanti parlarono a lungo, l’intera notte. Benjamin non osò confessare
alla madre che era scappato perché non voleva più stare con lei: la vedeva
troppo felice. Così le raccontò che si era perso cercandola e per una
settimana aveva vissuto di stenti nel bosco finché Méav non l’aveva trovato.

Avevano poi tentato di tornare al campo ma non erano riusciti a scovarlo. Quando
il ragazzo si addormentò, provato dalle emozioni e dalla malattia, Martha uscì
con Méav nel cortile. Passeggiarono per un po’ in silenzio, poi prese le sue
mani e si fermò, guardandola negli occhi.

«Volevo dirti una cosa, ma sono stata così occupata a parlare con Ben che non
ho avuto il tempo. Grazie. Tantissimo. Se non fosse stato per te mio figlio
sarebbe morto. Sei sua madre almeno quanto me, forse di più».

Méav esplose in un nuovo pianto, stavolta di certa disperazione. Sapeva
benissimo di non meritare quelle parole ma non poteva confessare la verità a
Martha. Avrebbe scoperto che Benjamin la odiava!

E soprattutto avrebbe scoperto come era stata crudelmente privata di suo figlio
per un puro interesse personale.

La abbracciò e pianse più forte mentre Martha, sorpresa, tentava di
consolarla.

«Dai, adesso va tutto bene» diceva sorridente.

``No. Adesso non c’è una sola cosa che vada bene'' pensò Méav.

Quello fu l’episodio che segnò la ricomparsa di Benjamin nel mondo
tecnologico, quello in cui le notizie viaggiano alla velocità della luce. E
così qualcuno che, chissà come, aveva intuito costa stesse succedendo aveva
avvertito la stampa, che la mattina dopo faceva pressioni per poter entrare
nella stanza del ragazzo. Non era sicuramente storia di tutti i giorni quella di
un bambino che si perde, vive con i Celti per tredici anni e come se non
bastasse ritrova poi la madre!

Per due settimane li tennero lontani perché non volevano che il ragazzo si
affaticasse. Quando Benjamin fu quasi guarito Martha e Méav gli chiesero se gli
sarebbe piaciuto raccontare la sua storia. Egli accettò. Quale delle due
avrebbe raccontato, l’amara verità o la dolce bugia, non era ancora chiaro a
nessuno, neanche al ragazzo stesso.

Impiegò tre giorni per prepararsi, rimuginando sul suo passato. Pensò a cosa
avrebbe potuto dire, ma ancor di più a cosa avrebbe potuto fare. Perché
Benjamin non sopportava assolutamente la situazione che si era creata. Era
combattuto: da una parte felice perché aveva ritrovato la sua madre biologica,
dall’altra disperato perché aveva perso quella che lo aveva davvero accudito
per tredici anni.

La sera del terzo giorno infine prese la sua decisione. No, non poteva farcela.

Doveva agire. Uscì fuori dalla sua stanza. Martha e Méav lo seguirono mentre
saliva fino al quarto piano dell’ospedale, dove c’era la sala riunioni in
cui i giornalisti si davano il cambio aspettando il loro scoop.

«Sono pronto» disse loro.

Un enorme orso pigro sembrò svegliarsi dal letargo per aggredirlo di domande,
ma lui salì sulla pedana riservata agli oratori. I riflettori furono accesi, e
gli procurarono un caldo infernale, insopportabile, mentre calava il silenzio.

Agire una volta per non agire più.

\paragraph{}
{\itshape
Sono tutti sul divano. Ormai vi sprofondano, perché il povero nonno passa le
giornate incollato davanti alla televisione, immobile. Sì, ci sono proprio
tutti: la mamma, il papà, Nathalie, di dieci anni, e il suo fratellino di otto.

E sono tutti riuniti per seguire quel discorso, ascoltare la vicenda che ha
commosso l’intera Scozia.

«Ti rendi conto, Nathalie? Tredici anni! Questo ragazzo è stato per tredici
anni lontano dalla madre! E tu che segui la tua anche quando va in bagno!»

«Tredici anni? Più di quanti ne ho io?»

«Sì, tre anni di più. Zitta, zitta, sta iniziando!»

Il solito monologo pieno di suspense del giornalista, e poi finalmente in primo
piano Benjamin. Stanco. Non guarda la telecamera, ma fissa il pavimento.

«Papà, perché sembra così triste?»

«Hanno detto che stava male. Ssh, fammi sentire, Nathalie!»

Nathalie non coglie a pieno le parole del ragazzo; le sembrano così complicate,
e le frasi così lunghe. Non capisce se è allegro o addolorato.

Vede poi all’improvviso il vetro della stanza andare in mille pezzi, infranto
dalla sedia che Benjamin aveva accanto. Ora almeno sembra che nessuno capisca.

Nella sala si leva un mormorio di stupore e la telecamera trema. Nathalie pensa
che chi la tiene non deve essere molto bravo: il suo papà due anni prima aveva
fatto un video della recita scolastica e la telecamera non si muoveva mai. Le
aveva detto di avere la mano così ferma perché era un supereroe, ma poi aveva
confessato ridendo di aver usato un treppiede. A lei non era sembrato tanto
divertente essere presa in giro.

Ora il ragazzo guarda giù dalla finestra, inquieto. Le persone nella sala
strillano. Due donne, una bionda e bella, un’altra più brutta con i capelli
rossi, si avvicinano. Sono le due madri di Ben, questo Nathalie lo sa. Quella
con i capelli rossi piange e urla qualcosa mentre la bionda guarda stupita.

Qualche interminabile istante, poi il ragazzo si gira verso la gente, e si
lascia cadere all’indietro. Nathalie non capisce, ma ora i grandi a quanto
pare sì, perché le coprono gli occhi.

Non abbastanza in fretta, però, perché Nathalie ha visto il lampo negli occhi
del ragazzo, prima che sparisse dall’inquadratura.
\/}

\tableofcontents

\end{document}
