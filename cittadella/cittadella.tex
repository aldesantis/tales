\documentclass[a4paper,oneside,11pt]{memoir}

\usepackage[italian]{babel}
\usepackage[utf8]{inputenc}
\usepackage[T1]{fontenc}

\title{La Cittadella}
\author{Alessandro Desantis}
\date{}

%\chapterstyle{dash}
\pagestyle{plain}
\frenchspacing

\begin{document}

\begin{titlingpage}
\maketitle
\end{titlingpage}

\chapter{Cosa lasciamo indietro}

``Signora Robertson, siamo arrivati.''

Lisa sospirò, facendo attenzione che il suo autista non lo notasse. Attraverso i
vetri oscurati del Suv vedeva il Learjet, tanto lussuoso quanto anonimo, che
attendeva sulla pista. Presto il portellone si sarebbe chiuso dietro di lei, e
non ci sarebbe stato ritorno. Ma si illudeva: nemmeno ora c'era ritorno.

Uno degli uomini che l'avrebbero scortata andò incontro alla sua auto. Quando la
mano si posò sulla maniglia, Lisa seppe di dover mettere in scena il suo sorriso
più ammaliante.

\emph{Puoi farlo.}

Lo sportello si aprì. Lisa posò un piede sull'asfalto e la prima goccia d'acqua
la colpì in pieno volto, sorprendendola: le nuvole c'erano anche quand'era
partita? Non riusciva a ricordarlo. I dettagli del prossimo reclutamento
l'avevano assorbita completamente.

Qualcuno aprì un ombrello sopra la sua testa, ma Lisa lo rifiutò. Vide dei volti
stupiti. La pioggia le scorreva sul volto e la riportava al presente, a ciò che
le stava davanti. La salutarono, ma lei non rispose.

\emph{L'hai fatto decine di volte.}

Accanto alla scaletta pilota e copilota, entrambi in uniforme, la aspettavano
con aria grave. Conosceva il primo: era un membro della Cittadella da
trent'anni e insieme avevano viaggiato diverse volte. Aveva il carattere
taciturno e professionale di chi sa qual è il proprio compito e si limita a
farlo. Il copilota aveva invece non più di venticinque anni. Sembrava che stesse
per esplodere dall'eccitazione. Quando la vide, allungò una mano sudata.

``Signora Robertson, ho sentito tanto parlare di lei!''

Lisa inarcò un sopracciglio.

``Parlare bene, spero'' rispose. Si era sforzata di suonare seria, ma il sorriso
che le decorava gli angoli della bocca la tradiva. O l'avrebbe tradita, se
l'altro non fosse stato troppo agitato per notarlo.

``Bene, sì, certo!'' si affrettò a precisare, arrossendo appena. Lisa vide il
suo collega anziano scuotere la testa in segno di disapprovazione, e questo la
fece ridere sinceramente. Il giovane, non capendo, arrossì ancora di più. Lisa
ne ebbe pena e salì sull'aereo, seguita dai due.

\emph{È l'ultima volta che lo fai.}

Per lei, quello di un jet privato era odore di casa. Si guardò intorno per
qualche secondo, indecisa su dove sedersi.

``Hai fatto i compiti?''

Lisa sussultò. ``Cristo!''

Aaron era dietro di lei. Quand'era arrivato? Lisa guardò fuori dal finestrino e
vide la sua auto. L'uomo si esibì in un profondo inchino.

``In persona.''

Era invecchiato dall'ultima volta che l'aveva visto, quattro anni prima. Era
stato in mezza Europa e il viaggio l'aveva segnato. Lisa non sapeva dire cosa
fosse di preciso; sembrava solo\dots{} stanco. Forse erano le rughe che
solcavano il volto, o le occhiaie che cerchiavano gli occhi. Forse era la
lentezza con cui si muoveva, come se la vita lo stesse abbandonando un po' alla
volta.

``Fai rumore la prossima volta che mi arrivi alle spalle.''

Aaron ignorò il suggerimento. ``Come ti sembro?''

La donna lo guardò con affetto. ``Non male. Come se qualcuno ti avesse masticato
e risputato.''

La risata di Aaron riempì l'aereo. Lisa l'aveva sempre amata: era così pura,
aperta, sincera. Era raro che qualcuno nel suo campo trovasse l'innocenza per
ridesse in quel modo.

``La mia ragazza\dots{}'' disse, lasciandosi cadere su un sedile. Lisa non era
più ufficialmente \emph{bambina} da quando aveva compiuto vent'anni. Chissà,
forse a quaranta avrebbe smesso di essere \emph{ragazza}. Solo il tempo poteva
decidere il prossimo appellativo che Aaron avrebbe trovato per lei. ``Non sei
molto lontana dalla verità.''

Lisa gli sedette di fronte. Sapeva che chiedere sarebbe stato inutile.

``Ma non hai risposto alla mia domanda'' riprese Aaron, fingendo una severità
che non aveva mai realmente mostrato. ``Hai fatto i compiti?''

In mezzo a loro stava un tavolo, e sul tavolo un dossier. Aaron lo prese, lo
aprì e iniziò a sfogliarlo. Lisa sapeva che era composto di cinquantadue pagine,
perché le aveva pressoché memorizzate una settimana prima: biografia, profilo
psicologico e referenze lavorative, accenni ai parenti e agli amici stretti. E
poi conti bancari, cassette di sicurezza, hobby, abitudini, gusti artistici e
culinari, luoghi più frequentati\dots{} Tutto ciò che si potesse sapere di un
uomo, e in particolare di \emph{quell'uomo}, era nelle mani della Cittadella.

Lisa sorrise ad Aaron con aria di sfida. ``Vuoi interrogarmi?''

Aaron guardò lei, poi il dossier, poi di nuovo lei.

``Nome, cognome, data e luogo di nascita.''

``Thomas Flikrey, 28 settembre 1978, Londra.''

Ovviamente, Aaron non era impressionato.

``Genitori?''

``Anna Lawson, insegnante, e Robert Flikrey, neurochirurgo. Ex-neurochirurgo. È
in pensione.''

Ancora niente.

``Educazione?''

% TODO: L'educazione si potrebbe rendere un po' più dettagliata.

``Saint Mary's, poi Woodside. Studente modello. Si laurea in legge a Westminster
nel 2001 con una tesi sullo stalking. Allora si stava appena iniziando a parlare
dell'argomento e il lavoro piacque moltissimo.''

Aaron fece una smorfia di disgusto. ``Immagino.''

``Vuoi che ti racconti della storia lavorativa?''

Il portellone dell'aereo si chiuse.

``Risparmiami i dettagli, morirei di noia. Ha lavorato per studi piccoli, poi
più grandi. Alla fine ha aperto il suo. Giusto?''

Lisa annuì. ``Ora lavora per i pezzi grossi.''

``E cosa fa il signor Flikrey quando non aiuta le multinazionali a rendere il
mondo un posto peggiore?''

``Non molto, veramente. Gli piace leggere, gialli per lo più, e portare la
moglie a cena in ristoranti dove si paga per prenotare.''

``Moglie? Quale moglie?'' Aaron iniziò a sfogliare il fascicolo avanti e
indietro. Lisa lo prese dalle sue mani e lo aprì con sicurezza al centro, dove
c'era una foto dei giovani sposi al loro matrimonio. I due sorridevano in mezzo
a un prato durante quella che sembrava una giornata perfetta: nemmeno una timida
nuvoletta intaccava la loro felicità.

\emph{Una coppia realizzata.}

Indicò la foto con aria trionfante. ``Questa moglie. Helen Bricks. Ha una
galleria d'arte che dà visibilità a fotografi emergenti. Si sono sposati nel
2004. Bel matrimonio.''

``Storia perfetta?''

Lisa si strinse nelle spalle. ``È una storia. Condividono l'amore per i gialli,
non quello per la fotografia. Nulla di speciale.''

Aaron sembrava insoddisfatto. ``Separarli non dovrebbe essere difficile. Ma ne
vale la pena?''

``Che vuol dire?''

``Banale. \emph{Ba-na-le}!'' scandì Aaron, con lo stesso tono che un bambino
userebbe per lamentarsi del regalo di Natale. ``È un uomo banale. Perché
dovrebbe diventare dei nostri?''

Era quello che si chiedeva da un mese. Con quali meriti Thomas Flikrey sarebbe
dovuto entrare nella Cittadella? Non era nessuno. Lisa non sapeva nemmeno come
quel dossier fosse arrivato a lei, una reclutatrice esperta, senza che vi fosse
nulla di rilevante. Di solito le venivano assegnati i soggetti più interessanti;
Flikrey era un cliché vivente, la noia fatta persona.

``Non ne ho idea. Ho letto quel fascicolo tre volte e ancora non lo so. Credo
che tornerò senza aver concluso niente.''

Il Learjet si portava sulla pista. Lisa guardò il paesaggio uniforme scorrere
fuori dal finestrino.

L'uomo annuì, assorto. ``Hai chiesto a Nadia di rilevarti dall'incarico, vero?
È il tuo ultimo reclutamento.''

``Non è come pensi. Non le ho detto di darmi un caso disperato per tornare
prima. Ero convinta che mi conoscessi meglio di così'' protestò Lisa,
oltraggiata all'idea che Aaron potesse crederla una scansafatiche. Sentiva da
qualche parte nel petto un fastidio, e si accorse che doveva essere simile a
quello che prova lo studente modello quando viene rimproverato per la prima
volta.

Sì, moriva dalla voglia di riabbracciare suo figlio. Sì, era stanca di
manipolare, di distruggere e ricreare le persone, nonostante fosse così
maledettamente brava a farlo. Ma avrebbe messo tutta se stessa in quell'ultimo
caso, proprio come aveva fatto col primo. Lo doveva alla Cittadella e lo doveva
all'uomo che ora le stava davanti.

``Pensavo tutto il contrario: se Nadia sapesse che vuoi uscirne, ti assegnerebbe
il caso più difficile che ha per tenerti lontana da casa, farti dimenticare la
tua famiglia e convincerti a continuare.''

Lisa sentì un calore irradiarle le guance. Lottò per respingerlo e, dopo aver
ingannato centinaia di persone per così tanti anni, si aspettava di riuscirci.
Eppure davanti ad Aaron si sentiva nuda: lui la leggeva con una semplicità
preclusa persino a suo marito. Era comprensibile: l'aveva cresciuta, l'aveva
coltivata, le aveva insegnato tutto\dots{} Eppure quella debolezza la irritava,
e la irritava il fatto che la irritasse.

``Credi davvero che lo farebbe?''

``Come pensi di avvicinare Flikrey?''

``Mi inventerò qualcosa a Londra'' tagliò corto lei, seccata dal cambio di
argomento. ``Credi davvero che Nadia mi terrebbe lontana dalla mia famiglia?''

``Lascia stare, parlavo a vanvera. Sicuramente sarà una passeggiata'' disse
Aaron, e il suo tono lasciava intendere che non voleva o non poteva essere più
preciso. A Lisa non serviva un decennio di esperienza per capire che stava
mentendo. Sapeva che Aaron non credeva di ingannarla.

L'aereo si alzò in volo. Lisa provò quella famigliare, piacevole leggerezza. Per
qualche istante non ebbe alcuna preoccupazione: Aaron, Nadia, Flikrey e tutte le
altre persone con cui aveva a che fare per via di quel lavoro assurdo\dots{}
Nessuna di loro significava più nulla.

Ma il Learjet era troppo lento. Dopo pochi secondi la verità la raggiunse di
nuovo, come una vecchia amica che ritrova la strada di casa.

\emph{Troppo tardi per tornare indietro.}

\plainbreak{1}

Se c'era un qualche record di velocità per l'ingresso in un albergo, Lisa era
certa di averlo battuto. Lo confermava anche l'espressione, a metà tra il
terrorizzato e il divertito, dell'impiegato al bancone del Milestone Hotel di
Londra, mentre lei compilava come una furia i moduli che le aveva passato e si
precipitava in ascensore seguita da un facchino ansimante.

Dopo averci passato così tanto tempo, aveva sviluppato un'avversione per gli
alberghi: la facevano sentire minacciata. Non sopportava più l'idea di affidare
una parte tanto importante della propria vita a dei perfetti sconosciuti. Non
sopportava più l'incertezza. Sentiva il bisogno di riprendere il controllo.
Invidiava gli uomini comuni con le loro abitudini e la loro rassicurante
ripetitività, perché nella sua vita c'era solo caos. \emph{Lei} portava solo
caos.

% TODO: Aggiungere descrizione stanza.

Quando il telefono squillò, Lisa aveva da poco chiuso dietro di sé la porta
della camera e sfogliava il fascicolo su Flikrey, cercando un modo per
avvicinarlo. Si stava rivelando più difficile del previsto.

``Pronto?'' Silenzio. ``Pronto?'' ripeté.

La voce incerta di un bambino chiese: ``Mamma?''

Lisa si affrettò a mettere via il fascicolo, come se il figlio potesse vederlo e
rimproverarla della sua distrazione. ``Ben! Come stai?''

Benjamin non rispose alla domanda. ``Quando torni?''

Poteva immaginarlo a casa mentre teneva il telefono con due mani, mostrando una
concentrazione di cui Lisa avrebbe riso se solo fosse stata lì. Quasi senza
alcuno sforzo, riusciva a vedere gli occhi verdi, le guance rosee, i capelli
chiari e folti\dots{} Ma era un esercizio doloroso.

``Presto. Torno presto'' disse, e nello stesso istante si rese conto che la sua
voce si era ridotta a un sussurro. Aveva sempre odiato mentire a suo figlio,
anche quando lo faceva per proteggerlo. Gli sembrava di tradire quella fiducia
cieca che Ben riponeva in lei, come se fosse il centro del mondo, come se non
potesse fare o pensare nulla di male.

``Cosa stai facendo?''

Lisa accarezzò la copertina nera del dossier, illuminata dalla luce pomeridiana
che filtrava dalla finestra dell'hotel. La pelle -- era davvero pelle? --
portava come segni indelebili le tracce di molte dita che avevano aperto quel
fascicolo. La vita di una persona. Cinquantadue pagine. Consultata, memorizzata,
commentata e analizzata. Cinquantadue pagine. Si può ridurre un essere umano
alla somma delle parti che lo compongono?

Esitò. ``Leggo.''

Ci fu qualche secondo di silenzio. Lisa chiuse gli occhi e ascoltò il respiro di
Benjamin. Era così regolare, così rassicurante\dots{}

\emph{Non ancora. Non è il momento.}

``\dots{}Lisa? Lisa, ci sei ancora?''

Quando, precisamente, la voce dolce di suo figlio aveva lasciato il posto a
quella profonda del marito?

``Sì, sì, sono qui'' si affrettò a rispondere.

``Ti ho chiesto come va il lavoro'' ripeté Stephan. La sua voce era un misto di
stupore, rimprovero e preoccupazione. Lisa si rendeva conto di essere stata
irriconoscibile negli ultimi mesi.

``Bene'' rispose automaticamente, perché era passato molto tempo dall'ultima
volta che un suo reclutamento era andato male. Poi ricordò. ``No, è un disastro.
Questo tizio passa tutto il tempo chiuso in casa, al lavoro o con la moglie.
Avvicinarsi è praticamente impossibile.''

Stephan rispose senza esitazione. ``Ti verrà in mente qualcosa. Sei brava a
trovare le debolezze delle persone e ancora più brava a sfruttarle. Per molti
qui sei la reclutatrice migliore che abbiamo.''

Lisa sapeva che volevano essere un complimento, eppure non riusciva a sopprimere
il senso di desolazione che le parole di suo marito le avevano suscitato. I
talenti degli uomini sono vari\dots{} ma lei poteva mai vantarsi del suo? Chi
l'avrebbe apprezzato al di fuori della Cittadella? Quale uomo onesto sarebbe
stato orgoglioso della propria abilità nell'inganno?

Poiché non rispondeva, Stephan continuò, schiarendosi la voce. ``Be', sbrigati
a tornare. Ci manchi.''

Lisa fu sorpresa. Suo marito, tanto chiuso quanto determinato, non si lasciava
andare facilmente a certe manifestazioni di affetto. Che qualcosa stesse
cambiando anche in lui?

\emph{Dovrei dirglielo. Merita almeno questo.}

Non aveva detto a Stephan che quello sarebbe stato il suo ultimo reclutamento.
Ci aveva provato più volte di quante ne riuscisse a ricordare, eppure le era
sempre mancato il coraggio. Sapeva che lui non avrebbe approvato, nonostante
l'amore che li legava: era convinto che il bene della comunità, ovvero il bene
della Cittadella, venisse prima di tutto. Prima di sé, prima di sua moglie,
prima di suo figlio. Se la Cittadella soffriva, il mondo soffriva, e
certamente il mondo era più importante dei suoi singoli abitanti.

Lisa aveva condiviso la sua stessa lealtà\dots{} fino alla nascita di Ben.
Mentre la fede di Stephan non era stata scossa dal figlio, nonostante lo
amasse come ci si aspettava da un padre, per Lisa tutto era cambiato: i viaggi
avevano iniziato a pesarle come macigni, le menzogne erano una macchia
indelebile sulla sua coscienza. Aveva cercato di resistere: per sette anni
aveva combattuto contro se stessa, reprimendo ciò che provava, quasi
vergognandosene, ma non ne poteva più. Doveva finire, e doveva finire subito.

``Davvero ti manco?'' chiese, sorridendo appena.

``Certo. Non ho scattato una foto decente da quando sei partita.'' Per chi lo
conosceva, quel semplice fatto bastava per capire quanto fosse smarrito.

Stephan era un fotografo. Ogni anno tutta la famiglia lasciava la Cittadella per
visitare una nuova località. Lei preferiva le meraviglie dell'architettura a
quelle della natura: era un animale urbano, inseparabile dalle comodità del
mondo civilizzato. L'amore di Stephan, invece, era tutto per i paesaggi esotici,
i luoghi meno esplorati, la fauna e i tramonti. Per Ben la destinazione non era
importante: solo l'eccitazione del viaggio bastava a renderlo felice. Insieme
avevano visitato tre continenti e quattordici Paesi.

% TODO: Dove di preciso in Africa? E dove mettere questo passaggio? Magari
% sarebbe meglio se fosse inserito in una conversazione con Stephan?

% Lisa si era innamorata dell'Italia, e di Roma in particolare: ogni passeggiata
% lungo il Tevere le strappava lacrime di commozione. Stephan invece non aveva
% mai potuto dimenticare l'Africa.

Anche quando il dovere lo teneva a casa, però, Stephan trovava sempre qualche
progetto a cui lavorare, come scattare il ritratto di ogni singolo membro della
Cittadella, o immortalare scene di vita quotidiana, attimi che per gli altri
erano insignificanti nella loro normalità, ma che Stephan considerava carichi di
significato. Prima che venisse reclutato aveva uno studio fotografico, modelle e
modelli che lavoravano per lui. Ora gli rimanevano solo Lisa e Ben, e di tanto
in tanto qualche volontario della Cittadella.

``Cos'è, ti manca l'ispirazione senza di me?''

``Sai come si dice: dietro ogni grande uomo c'è una grande donna.''

Lisa rise per un istante, poi divenne muta. Come aveva potuto non pensarci? Era
così semplice! Un'idiozia! Doveva solo\dots{} Sapeva già a quale pagina aprire
il fascicolo. Lesse velocemente: \emph{galleria d'arte}\dots{} \emph{fotografi
emergenti}\dots{} \emph{rapporto materno}\dots{} Sì, quella era la strada
giusta. Sembrava tutto così ovvio, adesso!

Stephan le stava raccontando della sua giornata. ``Avrò fatto trecento scatti.
Nemmeno uno sembrava decente. E sai qual è la cosa peggiore?''

``Puoi mandarmeli?''

L'uomo ammutolì, non sapendo se avesse capito bene. ``Come?''

``Le foto che hai scattato oggi. Mandami le dieci più belle.''

La sua voce prese un tono sinceramente dispiaciuto, Lisa non sapeva se più per
lei o per le foto. ``Erano orrende\dots{} Le ho buttate.''

``Mandamene altre. Le migliori! Mandami le tue foto migliori'' esclamò, in preda
a un'euforia febbrile. ``Ma niente volti! Niente di riconoscibile.''

``Cosa vuoi farci?''

Lisa si affacciò alla finestra: le riusciva difficile credere che tutti quei
passanti conducessero vite perfettamente normali. Mediocri? Forse. Ma
\emph{libere}. E lei? Non poteva abbracciare suo figlio, né fare l'amore con suo
marito. Certo, \emph{loro} potevano risparmiarla, lasciarla fare a modo suo
forse per un un anno o un decennio, ma prima o poi le avrebbero ordinato di fare
qualcosa che non voleva fare, e non avrebbe avuto altra scelta che obbedire. Era
condannata a girare il mondo, fingendo di liberare gli altri da una schiavitù
che, col senno di poi, non sembrava tanto orrenda.

\end{document}
