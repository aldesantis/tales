\documentclass[a4paper,oneside,11pt]{memoir}

\usepackage[italian]{babel}
\usepackage[utf8]{inputenc}
\usepackage[T1]{fontenc}

\title{Riflessioni semiserie}
\author{Alessandro Desantis}
\date{}

\pagestyle{plain}
\frenchspacing

\begin{document}

\begin{titlingpage}
\maketitle
\end{titlingpage}

\mainmatter

\chapter{Dio}

--- Può Dio creare un masso inamovibile? Se può, allora non è onnipotente perché
non può spostarlo; se non può, non è onnipotente perché non può crearlo. Se
non è onnipotente, non è Dio.

--- Ma se è onnipotente, non può essere costretto dalla logica. Dunque Dio
può creare e spostare un masso inamovibile.

--- Non ha alcun senso. Un masso inamovibile non può essere spostato, o non
sarebbe inamovibile.

--- Per noi è così, per un essere onnipotente le regole sono diverse.

--- Allora è inutile starne a parlare: se Dio esula dalla logica, non può essere
conosciuto o discusso dagli uomini.

--- Esattamente.

\plainbreak{1}

Le religioni: folli. I religiosi: arroganti.

Come si può credere di aver trovato Dio? e se pure Dio si manifestasse, come
sapere che non si tratti di allucinazioni collettive? e se pure Dio mostrasse il
proprio potere, come essere sicuri che sia realmente onnipotente?

La fede: rifugio dei deboli.

\plainbreak{1}

--- Dio è perfetto, dunque non Gli si può aggiungere né togliere nulla,
nemmeno l'esistenza; in conclusione, Dio esiste.

--- È assurdo pensare che l'esistenza sia propria della perfezione. Se così
fosse, ogni utopia sarebbe necessariamente vera.

--- Ma qui stiamo parlando della perfezione effetiva, non di quella che noi
percepiamo come tale!

--- E chi stabilisce qual è la perfezione?

--- Solo Dio può farlo.

--- Allora perché ne stiamo parlando?

\plainbreak{1}

Un'altra confutazione: se esiste il bene assoluto, ovvero Dio, allora deve
esistere necessariamente anche il male assoluto. Altrimenti, sulla base di quali
criteri si giudicherebbe ciò che è bene e ciò che è male?

Ora, pare sensato pensare che il male assoluto sia l'opposto del bene assoluto:
se quest'ultimo è perfetto, l'altro è dunque assolutamente imperfetto.

Se l'esistenza fosse propria della perfezione, l'inesistenza dovrebbe essere
propria dell'imperfezione assoluta; dunque il male assoluto non può esistere. Se
non esiste il male assoluto, non esiste Dio.

Tutto questo non per dimostrare che Dio non esiste, ma che l'esistenza non ha
nulla a che fare con la perfezione: un essere perfetto può non esistere e uno
imperfetto può esistere.

\plainbreak{1}

L'ateo non è diverso dal più fervente dei religiosi: arrogante nelle
convinzioni, cieco nelle osservazioni. Come si può affermare l'inesistenza di
Dio?

Se la si afferma logicamente -- ovvero si dimostra che Dio \emph{non può}
esistere, salvo incorrerre in contraddizioni logiche -- ci si scontra contro
questo semplice fatto: di un essere onnipotente non si può predicare
logicamente.

Se la si afferma empiricamente -- ovvero che Dio non esista perché non si è mai
mostrato -- ci si macchia di vanità: perché mai Dio dovrebbe mostrarsi?

L'ateismo non è che un'altra soluzione alle incertezze umane. Che Dio esista o
non esista ultimamente non fa alcuna differenza: l'uomo vuole una risposta;
quando ce l'ha, la difende fino alla morte.

\chapter{Filosofia}

Che sia il modo più semplice per complicarsi l'esistenza con quesiti senza
risposta e problemi senza soluzione.

Che gran parte di quella presocratica e parte di quella post-socratica richieda
un atto di fede non differente da quello che chiede la religione.

Che sia ridicolo passare la propria vita a cercare il senso della vita,
dimenticando di vivere.

\plainbreak{1}

Lo stoicismo può essere utile per affrontare le tragedie della vita.

Eppure, ache lo stoicismo chiede un atto di fede, parlando del \emph{logos}, che
io non sono disposto a concedergli.

Inoltre, rischia di sfociare nell'indifferenza. Se nulla può toccarci, che cosa
dovrebbe spingerci ad andare avanti? Non credo di avere un posto nel mondo. Sono
io a creare il mio destino. Perché dovrei crearlo se niente ha senso?

Come se non bastasse, la sua applicazione completa richiede il totale dominio
sulla mente, un'impresa che pochi uomini possono permettersi di tentare e che
nessuno si vanterà mai di aver realizzato.

\backmatter

\tableofcontents

\end{document}
