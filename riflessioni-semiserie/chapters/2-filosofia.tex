\chapter{Filosofia}
\label{ch:filosofia}

Che sia il modo più semplice per complicarsi l'esistenza con quesiti senza
risposta e problemi senza soluzione.

Che gran parte di quella presocratica e parte di quella post-socratica richieda
un atto di fede non differente da quello che chiede la religione.

Che sia ridicolo passare la propria vita a cercare il senso della vita,
dimenticando di vivere.

\plainbreak{1}

Lo stoicismo può essere utile per affrontare le tragedie della vita.

Eppure, ache lo stoicismo chiede un atto di fede, parlando del \emph{logos}, che
io non sono disposto a concedergli.

Inoltre, rischia di sfociare nell'indifferenza. Se nulla può toccarci, che cosa
dovrebbe spingerci ad andare avanti? Non credo di avere un posto nel mondo. Sono
io a creare il mio destino. Perché dovrei crearlo se niente ha senso?

Come se non bastasse, la sua applicazione completa richiede il totale dominio
sulla mente, un'impresa che pochi uomini possono permettersi di tentare e che
nessuno si vanterà mai di aver realizzato.
