\documentclass[12pt]{book}

\usepackage[italian]{babel}
\usepackage[T1]{fontenc}
\usepackage[utf8]{inputenc}
\usepackage{todonotes}

\begin{document}

\title{Il fiume impossibile}
\author{Alessandro Desantis}
\date{}
\maketitle

\cleardoublepage

Qualche anno fa, il quotidiano per cui lavoro ha aperto una scuola di giornalismo. All'inizio mi è sembrata una sciocchezza totale: come la maggior parte dei mestieri, il nostro è complesso e variopinto, e il corso che avevamo organizzato era così breve da rendere impossibile qualunque spiegazione o approfondimento degni di merito. Me ne sono tenuto fuori, e ho lasciato che fossero alcuni colleghi a occuparsene.

Dopo circa sei mesi ho cambiato idea. Per buona parte è stato merito di mia moglie, che mi ha sempre considerato un insegnante terribile: un po' per gioco e un po' perché mi ha sempre punto nel vivo, volevo mostrarle che si sbagliava. Il resto lo ha fatto il senso del dovere: penso che la qualità del giornalismo -- con poche, notevoli eccezioni -- sia così bassa che non mi bastava più solo farlo: volevo \emph{spiegarlo,} con buona pace del senso di inadeguatezza con cui combatto da quando ho memoria.

Alla fine, mi sono appassionato molto più di quanto mi aspettassi. Pensavo che, da giornalista, non avrei avuto difficoltà a trovare le parole per esprimere ciò che pensavo, e in effetti è stato così. Il problema principale, in realtà, è stato capire \emph{cosa pensassi:} mi sono ritrovato a combattere con me stesso per dare forma a concetti che erano vaghi e sfuggevoli nella mia testa, nodi che altrimenti non avrei mai cercato di sciogliere.

Una di queste questioni, in particolare, riguardava la differenza tra giornalismo e scrittura creativa. Spesso i nostri studenti avevano difficoltà a tracciare un confine chiaro tra questi due campi: del resto, entrambi raccontano storie con le parole. Uno racconta la realtà, l'altro racconta la finzione, ma non è in fondo una distinzione tecnica?

Io ero l'unico in redazione che avesse scritto dei romanzi -- un paio di gialli in gioventù, che in verità non hanno mai avuto un grande successo e che nessuno ricorda -- eppure ho avuto molta difficoltà a spiegare la differenza. Non che non la cogliessi; è che l'idea di raccontarla mi provocava un certo fastidio, come se stessi scoperchiando un baule in cui già sapevo non esserci nulla di buono.

Ovviamente c'è un abisso tra la responsabilità dello scrittore e quella del giornalista: il primo racconta la storia più bella che può creare, il secondo la storia più vera che può trovare. Uno scrittore che fa male il suo lavoro fa solo un torto al lettore che lo paga, mentre un giornalista che fa male il suo lavoro fa, in primo luogo, un torto alla storia che racconta. Eppure sentivo che c'era dell'altro, qualcosa che non riuscivo a spiegare neanche a me stesso.

È stato grazie alla storia di Elena, che ho iniziato a raccontare proprio in quegli anni, che sono riuscito a trovare una risposta. La mia idea per questo libro era di farne un'opera completa, un racconto di cronaca dettagliato e logico, con un inizio e una fine. Ero alla ricerca di una verità che potesse raccontare, una volta per tutte, chi fosse Elena.

Ma più mi addentravo nella sua storia, sentendo le testimonianze delle persone di cui si era circondata, più mi allontanavo dalla verità che volevo raccontare: Elena era la maestra, la figlia, l'amante, la strega, e molto di più. Ogni scelta narrativa che prendevo era un colpo di remi nella direzione sbagliata, un sisifeo tentativo di ridurre l'irriducibile.

Ecco la risposta che ho trovato: lo scrittore si muove in un contesto \emph{limitato.} Le sue storie hanno un inizio e una fine. Certo, ci sono infiniti personaggi che può creare, prospettive che può scegliere, storie che può raccontare; ma ogni scelta è totale e definitiva, ed esclude tutte le altre. Un libro è un universo completo, e al di fuori di quell'universo non esiste altro.

Il giornalista, invece, si muove in un contesto illimitato, perché sta raccontando personaggi, prospettive, storie che esistono al di fuori di lui. Per il giornalista, ogni scelta significa prendere una strada piuttosto che un'altra, che resterà per sempre inesplorata. E per quanto possa fare bene il proprio lavoro, e sforzarsi di essere diligente, non riuscirà mai a cogliere la complessità della cosa vera.

Per uno scrittore, un cielo può essere di mille colori, ma infine sarà del colore che ha scelto. Per un giornalista, il cielo \emph{è} di mille colori, che non hanno nomi, e se ce li hanno gli sono sconosciuti, e se gli sono conosciuti sono troppi da recitare, e dovrà convivere per sempre con la decisione di dirne alcuni piuttosto che altri.

Nel caso di Elena mi sono rifiutato di accettare questo compromesso, perché sentivo che le avrei fatto un torto imperdonabile. Invece di romanzare la sua vita, ho scelto semplicemente di pubblicare le mie interviste per intero, raccontandola tramite le parole di chi l'ha conosciuta e mantenendo dove possibile l'ordine cronologico dei fatti.

Il risultato è un collage di storie e analisi spesso in contraddizione tra di loro, e che allo stesso tempo mi sembrano più vere e corrette di qualunque sintesi avrei mai potuto partorire. Questa è stata l'eredità che Elena, senza avermi mai conosciuto, ha voluto lasciarmi.

Queste pagine, invece, sono il mio regalo per lei.

\cleardoublepage

\end{document}
