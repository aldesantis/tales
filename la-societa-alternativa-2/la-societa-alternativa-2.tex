\documentclass[a4paper,11pt,oneside]{memoir}

\usepackage[italian]{babel}
\usepackage[T1]{fontenc}
\usepackage[utf8]{inputenc}

\frenchspacing

\title{La Società Alternativa}
\author{Alessandro Desantis}

\begin{document}

\begin{titlingpage}
\maketitle
\end{titlingpage}

\chapter{Una dolorosa perfezione}

Prima dei trent'anni Stephan non si era mai soffermato a riflettere su quanto la
perfezione potesse essere dolorosa. Fino al compimento di quella maledetta età,
del resto, aveva ancora delle sfide da affrontare e dei risultati di cui gioire.
Prima lo studio, poi il lavoro, poi l'amore lo avevano coinvolto tanto da non
lasciargli un attimo di tregua; di questo era felice, felicissimo! Quando poi la
sua vita si era stabilizzata, era stato il tempo della celebrazione: a ogni
istante gioiva della propria fortuna.

Eppure, non molti anni dopo il matrimonio, Stephan si trovava all'apice di
un'odiosa depressione. Per troppo tempo, invece di affrontarla, aveva tentato di
annegarla. Di conseguenza, questa era riaffiorata ancora più forte di prima.
Tutto quanto gli era venuto a noia: i pazienti (egli era per l'appunto un
medico) lo annoiavano, i colleghi lo annoiavano, gli amici lo annoiavano.
Persino sua moglie, Anna, iniziava a risultargli occasionalmente fastidiosa, con
quell'affetto così ripetitivo. Nulla al mondo poteva più scatenare nel suo animo
una reazione che non fosse la nausea.

A peggiorare il tutto vi era il senso di inadeguatezza che Stephan provava verso
i propri stessi sentimenti. Si rendeva conto di quanta ingratitudine vi fosse in
quella noia con cui ora guardava ogni cosa della vita, di quanto ingiusto fosse
che lui, avendo il piatto pieno, vi sputasse invece di farsi da parte perché gli
affamati potessero servirsi. Eppure non poteva farci niente; del resto, se ci
fosse dato di controllare i nostri sentimenti, nella storia umana non sarebbe
mai accaduto nulla di significativo.

Stephan non si affaticava a dissimulare il proprio stato d'animo; quando
(raramente) ci provava gli riusciva piuttosto male. Così tutte le persone che
gli stavano a gran noia si erano accorte di quanto apatico stesse diventando.
Questo non sarebbe stato un problema (aveva smesso di interessarsi anche
dell'opinione altrui) se non avessero cercato di aiutarlo con i loro consigli.
I loro inutili, patetici consigli! Come potevano sperare di comprendere la sua
condizione, loro che erano così piccoli e insignificanti, soddisfatti delle loro
vite senza causa e senza fine?

Ecco, era quello che gli serviva: un fine. Stephan aveva realizzato, a soli
trent'anni, la vanità della vita. Ogni suo gesto, ogni suo respiro, ogni suo
risultato era, nel grande schema delle cose, inutile. Alla sua morte tutti lo
avrebbero dimenticato e nulla sarebbe rimasto di suo nell'universo. Come si
poteva vivere con una simile consapevolezza? Perché continuare a lottare se si
sa di non poter vincere, né ora né mai, indipendentemente dallo sforzo
impiegato? Che senso ha nuotare se, per ogni metro guadagnato, le onde
inclementi ne fanno perdere due?

E di fronte a un tale dramma quegli omuncoli si permettevano di suggerirgli di
trovare un passatempo. Assurdo. Inaudito. Un passatempo lo avrebbe salvato dalla
noia forse per una settimana. E poi? Sarebbe forse stato condannato a provare
tutte le attività conosciute all'uomo? Che idiozia! Avrebbe passato la propria
vita correndo da una parte all'altra come un folle per non pensare alla propria
miseria. Una vita del genere era, se possibile, addirittura peggiore della sua
attuale immobilità.

A dir la verità c'era anche un'altra ragione, oltre a una loro superficiale
ridicolaggine, per cui Stephan si rifiutava di provare i rimedi di chi lo
conosceva: perché era terrorizzato all'idea di avere torto, di scoprire che le
sue convinzioni fossero deboli ed errate. Cos'avrebbe fatto, allora? Non poteva
liberarsene con una scrollata di spalle e non poteva aggiungere al dolore in sé
anche quello causato dalla consapevolezza di essere in fallo. Così, a costo di
precludersi la strada per la felicità, aveva deciso di ignorare tutti. Del resto
nessuno avrebbe mai potuto capirlo se neanche egli stesso riusciva a decifrarsi.

Molte persone sarebbero spinte da simili idee al suicidio. Non era il caso di
Stephan. Forse per un senso di sfida, o forse perché persino il pensiero di
uccidersi richiedeva da parte sua uno sforzo che non aveva alcuna voglia di
compiere, l'ipotesi non lo aveva neanche sfiorato. Proseguiva così con la sua
vita, incapace sia di godere che di dolersi realmente della propria condizione.
Era semplicemente il modo in cui le cose procedevano. Poteva solo sperare che
tutto finisse presto, che quel poco che esisteva di lui nel mondo venisse
inghiottito nel vuoto eterno assieme a ogni ricordo; la pace, il riposo e il
silenzio a cullarlo.

\plainbreak{1}

Stephan aveva tentato di realizzarsi attraverso l'arte. Era stato, in realtà, il
primo rimedio che aveva provato. Credeva che sperimentare e creare l'arte lo
avrebbe liberato dalla sua maledizione; la sperimentazione lo distraeva quanto
bastava perché non pensasse più alla propria condizione, mentre la creazione gli
avrebbe permesso di lasciare una traccia di sé nel mondo, un'eredità perpetua e
indistruttibile.

Tentò di scrivere, ma i risultati furono disastrosi. Gli riusciva impossibile,
infatti, trattare l'arte come un fine. Poiché voleva farsi ricordare, era
necessario che scrivesse qualcosa di memorabile. Per scrivere qualcosa di
memorabile, era necessaria un'immaginazione che semplicemente non possedeva. Per
sviluppare l'immaginazione era necessario\dots che cosa? Che accadesse qualcosa
di interessante nella sua vita! Come può un uomo che non ha mai provato
interesse creare un'opera interessante?

Scrisse della depressione che lo aveva preso, ma subito si vergognò
terribilmente del proprio egocentrismo; scrisse del suo amore per la moglie, ma
gli sembrò frivolo e banale; scrisse del suo lavoro, ma era una questione troppo
tecnica. Si rifiutò di frequentare i corsi di scrittura creativa che vedeva
spuntare come funghi perché, nonostante avesse bisogno dell'arte, sentiva di
essere naturalmente incapace e temeva di fallire. Cosa gli sarebbe rimasto,
allora?

% Effetti della depressione sul fisico?

% Manifestazioni della depressione nelle attività quotidiane?

\end{document}
